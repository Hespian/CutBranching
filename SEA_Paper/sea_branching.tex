
\documentclass[a4paper,UKenglish,cleveref, autoref, thm-restate]{lipics-v2021}
%See lipics-v2021-authors-guidelines.pdf for further information.
%for A4 paper format use option "a4paper", for US-letter use option "letterpaper"
%for british hyphenation rules use option "UKenglish", for american hyphenation rules use option "USenglish"
%for section-numbered lemmas etc., use "numberwithinsect"
%for enabling cleveref support, use "cleveref"
%for enabling autoref support, use "autoref"
%for anonymousing the authors (e.g. for double-blind review), add "anonymous"
%for enabling thm-restate support, use "thm-restate"
%for enabling a two-column layout for the author/affilation part (only applicable for > 6 authors), use "authorcolumns"

\listfiles

%\graphicspath{{./graphics/}}%helpful if your graphic files are in another directory

\bibliographystyle{plainurl}% the mandatory bibstyle

\title{Faster Maximum Independent Sets using Targeted Branching} %TODO Please add

\titlerunning{Faster MIS using Targeted Branching} %TODO optional, please use if title is longer than one line

% \author{John Q. Public}{Dummy University Computing Laboratory, [optional: Address], Country \and My second affiliation, Country \and \url{http://www.myhomepage.edu} }{johnqpublic@dummyuni.org}{https://orcid.org/0000-0002-1825-0097}{(Optional) author-specific funding acknowledgements}%TODO mandatory, please use full name; only 1 author per \author macro; first two parameters are mandatory, other parameters can be empty. Please provide at least the name of the affiliation and the country. The full address is optional

\author{Demian Hespe}{Karlsruhe Institute of Technology, Institute for
  Theoretical Informatics, Germany}{hespe@kit.edu}{}{}

\author{Sebastian Lamm}{Karlsruhe Institute of Technology, Institute for
  Theoretical Informatics, Germany}{lamm@kit.edu}{}{\textcolor{red} {TODO: Add funding}}

\author{Christian Schorr}{Karlsruhe Institute of Technology, Institute for
  Theoretical Informatics, Germany}{christian.schorr@student.kit.edu}{}{}

\authorrunning{D. Hespe and S. Lamm and C. Schorr} %TODO mandatory. First: Use abbreviated first/middle names. Second (only in severe cases): Use first author plus 'et al.'

\Copyright{Demian Hespe and Sebastian Lamm and Christian Schorr} %TODO mandatory, please use full first names. LIPIcs license is "CC-BY";  http://creativecommons.org/licenses/by/3.0/
\begin{CCSXML}
  <ccs2012>
  <concept>
  <concept_id>10002950.10003624.10003633.10010917</concept_id>
  <concept_desc>Mathematics of computing~Graph algorithms</concept_desc>
  <concept_significance>500</concept_significance>
  </concept>
  <concept>
  <concept_id>10002950.10003624.10003625.10003630</concept_id>
  <concept_desc>Mathematics of computing~Combinatorial optimization</concept_desc>
  <concept_significance>500</concept_significance>
  </concept>
  </ccs2012>
\end{CCSXML}

\ccsdesc[500]{Mathematics of computing~Graph algorithms}
\ccsdesc[500]{Mathematics of computing~Combinatorial optimization}
% \ccsdesc[100]{\textcolor{red}{Replace ccsdesc macro with valid oll
% ne}} %TODO mandatory: Please choose ACM 2012 classifications from https://dl.acm.org/ccs/ccs_flat.cfm 

\keywords{Graphs, Combinatorial Optimization, Maximum Independent Set, Branch
  and Reduce, Kernelization}%TODO mandatory; please add comma-separated list of keywords

\category{} %optional, e.g. invited paper

\relatedversion{} %optional, e.g. full version hosted on arXiv, HAL, or other respository/website
%\relatedversiondetails[linktext={opt. text shown instead of the URL}, cite=DBLP:books/mk/GrayR93]{Classification (e.g. Full Version, Extended Version, Previous Version}{URL to related version} %linktext and cite are optional

%\supplement{}%optional, e.g. related research data, source code, ... hosted on a repository like zenodo, figshare, GitHub, ...
%\supplementdetails[linktext={opt. text shown instead of the URL}, cite=DBLP:books/mk/GrayR93, subcategory={Description, Subcategory}, swhid={Software Heritage Identifier}]{General Classification (e.g. Software, Dataset, Model, ...)}{URL to related version} %linktext, cite, and subcategory are optional

%\funding{(Optional) general funding statement \dots}%optional, to capture a funding statement, which applies to all authors. Please enter author specific funding statements as fifth argument of the \author macro.

\acknowledgements{}%optional

%\nolinenumbers %uncomment to disable line numbering

% \hideLIPIcs  %uncomment to remove references to LIPIcs series (logo, DOI, ...), e.g. when preparing a pre-final version to be uploaded to arXiv or another public repository

%Editor-only macros:: begin (do not touch as author)%%%%%%%%%%%%%%%%%%%%%%%%%%%%%%%%%%
\EventEditors{John Q. Open and Joan R. Access}
\EventNoEds{2}
\EventLongTitle{42nd Conference on Very Important Topics (CVIT 2016)}
\EventShortTitle{CVIT 2016}
\EventAcronym{CVIT}
\EventYear{2016}
\EventDate{December 24--27, 2016}
\EventLocation{Little Whinging, United Kingdom}
\EventLogo{}
\SeriesVolume{42}
\ArticleNo{23}
%%%%%%%%%%%%%%%%%%%%%%%%%%%%%%%%%%%%%%%%%%%%%%%%%%%%%%

% Custom
\usepackage{xspace}
\newcommand{\ie}{i.\,e.,\xspace}
\newcommand{\eg}{e.\,g.,\xspace}
\newcommand{\etal}{et~al.\xspace}
\newcommand{\Wlog}{w.\,l.\,o.\,g.\ }
\newcommand{\wrt}{w.\,r.\,t.\xspace}

\begin{document}

\maketitle

%TODO mandatory: add short abstract of the document
\begin{abstract}
  \textcolor{red}{TODO: Write Abstract}
\end{abstract}

\newpage

\section{Introduction}
% Probably cite Schorr~\cite{schorr2020improved}.
An \emph{independent set} of a graph $G = (V,E)$ is a set of vertices $I \subseteq V$ of $G$ such that no two vertices in this set are adjacent.
The problem of finding such an independent set of maximum cardinality, the \emph{maximum independent set problem}, is a fundamental NP-hard problem~\cite{Garey1974}.
Its applications cover a wide variety of fields including computer graphics \cite{CG}, network analysis \cite{NW}, route planning \cite{RP} and computational biology \cite{BIO1, BIO2}.
In computer graphics for instance, large independent sets can be used to optimize the traversal of mesh edges in a triangle mesh.
Further applications stem from its complementary problems minimum vertex cover and maximum clique.

One of the best known techniques for finding maximum independent sets, both in theory~\cite{XiaoNagamochi, ChenXiaKanj} and practice \cite{AkibaIwata}, are \emph{kernelization algorithms}.
These algorithms apply a set of reduction rules to decrease the complexity of an instance while maintaining the ability to compute an optimal solution afterwards.
A recently successful type of kernelization algorithms are so-called \emph{branch-and-reduce algorithms}~\cite{AkibaIwata,WGYC}, which exhaustively apply a set of reduction rules to compute a \emph{kernel}.
If no further rule can be applied, the algorithm branches into (at least) two subproblems of lower complexity, which are then solved recursively.
To make them more efficient in practice, these algorithms also make use of problem specific upper and lower bounds to quickly prune the search space.

Due to the practical impact of kernelization, most of the research aimed at improving the performance of branch-and-reduce algorithms so far has been focused on either proposing more practically efficient special cases of already existing rules~\cite{ChangKern,dahlum2016accelerating}, or maintaining dependencies between reduction rules to reduce unnecessary checks~\cite{alsahafy2020computing,hespe2019scalable}.
In comparison, efforts to improve other aspects of branch-and-reduce, in particular the branching strategy are still lacking.
However, the branching strategy in particular has been shown to have a significant impact on the running time~\cite{AkibaIwata}.
Up to now, the most frequently used branching strategy employed in many state-of-the-art solvers selects branching vertices solely based on their degree.
Other factors, such as the actual reduction rules used during the algorithm are rarely taken into account.

\subsection{Contribution}
In this paper, we propose and examine several novel strategies for selecting branching vertices.
These strategies follow two main approaches that are motivated by existing research: (1) Branching on vertices that decompose the graph into several connected components that can be solved independently.
This has been shown to significantly improve the performance of branch-and-reduce in practice, especially when the side of the largest component is kept small~\cite{alsahafy2020computing}.
(2) Branching on vertices whose removal leads to reduction rules becoming applicable again.
In turn, this leads to a smaller reduced graph and thus improved performance.
For each approach we present several concrete strategies that vary in their complexity.
Finally, we evaluate their performance by comparing them to the aforementioned default strategy used in the state-of-the-art solver by Akiba and Iwata~\cite{AkibaIwata}.
For this purpose we make of a wide spectrum of instances from different graph classes and applications.
Our experiments indicate \ldots.

\section{Preliminaries}
Let $G=(V,E)$ be an undirected graph, where $V = \{0, \ldots, n-1\}$ is a set of $n$ vertices and $E \subseteq  \{\{u,v\} \mid u,v \in V\}$ is a set of $m$ edges. 
We assume that $G$ is \emph{simple}, \ie it has no self loops or multi-edges.
The \emph{(open) neighborhood} of a vertex $v \in V$ is denoted by $N(v) = \{u \mid \{v,u\} \in E\}$.
Furthermore, we denote the \emph{closed neighborhood} of a vertex by $N[v]=N(v) \cup \{v\}$.
We define the open and closed neighborhood of a set of vertices $U \subseteq V$ as $N(U) = \cup_{u \in U} N(v)$ and $N[U] = N(U) \cup U$, respectively.
The \emph{degree} of a vertex $v \in V$ is the size of its neighborhood $d(v) = |N(v)|$.
For a vertex $v \in V$, we further define $N^2(v) = N(N(v))$.
Finally, for a subset of vertices $S \subseteq V$, the \emph{induced subgraph} $G[S] = (S, E_S)$ is given by restricting edges to vertices of $S$, \ie $E_S = \{\{u,v\} \in E \mid u,v \in S\}$.

An \emph{independent set} of a graph is a subset of vertices $I \subseteq V$ such that no two vertices of $I$ are adjacent. 
A \emph{maximum independent set} (MIS) is an independent set of maximum cardinality.
Closely related to independent set are vertex covers and cliques.
A \emph{vertex cover} is a set of vertices $C \subseteq V$ such that for each edge $\{u,v\} \in E$ either $u$ or $v$ is contained in $C$.
The complement of a (maximum) independent set of a graph $G$ is a \emph{(minimum) vertex cover} (MVC) of $G$.
A \emph{clique} is a subset of vertices $K \subseteq V$ such that all vertices of $K$ are adjacent to each other, \ie $\forall u,v \in K: \{u,v\} \in E$.
A (maximum) independent set of a graph $G$ is a \emph{(maximum) clique} (MC) in the complement graph $\bar{G} = (V, \bar{E})$, where $\bar{E} = \{\{u,v\} \mid \{u,v\} \not\in E\}$.

To-Do: Edge-induced Subgraphs, Separators, Components

\section{Related Work}
Selecting branching vertices is a part of every branch and reduce or branch and
bound algorithm. In this section we give an overview of the techniques used in
algorithms for MIS, MVC and MC.

The most commonly used branching strategy for MIS and MVC is to select a vertex
of maximum degree. Fomin et al.~\cite{Fomin} show that using a vertex of maximum
degree that also minimizes the number of edges in its neighborhood is optimal
with respect to their complexity measure. The Algorithm by Akiba and Iwata~\cite{AkibaIwata}
(which we use for our experiments) also uses this strategy. Akiba and Iwata also
compare this strategy to branching on a vertex of minimum degree and a random
vertex. They show that both of these perform significantly worse than branching
on a maximum degree vertex.

Xiao and Nagamochi~\cite{XiaoNagamochi} also use
this strategy in most cases. For dense subgraphs, however, they use as edge
branching strategy: They branch on an edge $\{u, v\}$ where $|N(u) \cap N(v)|$
is large enough (depending on the maximum degree of the graph) by
excluding both $u$ and $v$ in one branch and applying the alternative reduction
(see Section~\ref{sec:reduction_branching}) to $\{u\}$ and $\{v\}$ in the other branch.

Buurgeois et al.~\cite{Bourgeois} use maximum degree branching as long as the
average degree of the graph is above 4. Otherwise, if there is no vertex with degree of at least 5, their algorithm branches on vertices
contained in 3- or 4-cycles. \textcolor{red}{TODO: @Christian weißt du auch, was
passiert, wenn es Knoten mit höherem Grad gibt?}

Chen et al.~\cite{ChenXiaKanj} \textcolor{red}{TODO: Die good pairs scheinen
  hier relevant. Dazu steht aber nochts in der BA.}

Most algorithms for MC (for example San Segundo and Tapia~\cite{Color}) compute a
greedy coloring and branch on vertices with a high coloring number.
More sophisticated MC algorithms use MaxSAT encodings to prune the set of
branching vertices~\cite{LiFangXu,LiJiang,LiQuan}. Li et al.~\cite{LiMaxSat}
combine greedy coloring and MaxSAT reasoning the further reduce the number of
branching vertices.

Another approach used for MC is using the \emph{degeneracy order} $v_1 < v_2 <
\dots < v_n$ where $v_i$ is a vertex of smallest degree in $G - \{v_1, \dots
v_{i-1}\}$. Carraghan and Pardalos~\cite{CarraghanPardalos} present an algorithm
that branches in descending degeneracy order. Li et al.~\cite{LiFangXu}
introduce another vertex ordering using iterative maximum independent set
computations (which might be easier than MC on some graphs) and breaking ties
according to the degeneracy order.

\subsection{Branch and Reduce Algorithm by Akiba and Iwata~\cite{AkibaIwata}}

\section{Decomposition Branching}
Our first approach to improve the default branching strategy found in many state-of-the-art algorithms (including that of Akiba and Iwata~\cite{AkibaIwata}) is to decompose the graph into several connected components.
Subsequently processing these components individually has been shown to improve the performance of branch-and-reduce in practice~\cite{alsahafy2020computing}.
To this end, we now present three concrete strategies with varying computational complexity.

\subsection{Articulation Points}
First, we are concerned with finding single vertices that are able to decompose a graph into at least two separated components.
Such points are called \emph{articulation points} (or cut vertices).
Articulation points can be computed in linear time $\mathcal{O}(n+m)$ using a simple depth-first search (DFS) algorithm.
In particular, a vertex $v$ is an articulation point if it is either the root of the DFS tree or any non-root vertex that has a child $u$, such that no vertex in the subtree rooted at $u$ has a backwards edge to one of the ancestors of $v$.

For our first branching strategy we maintain a set of articulation points $A \subseteq V$.
When selecting a branching vertex, we first discard all invalid vertices from $A$, \ie vertices that were removed from the graph by a preceding kernelization step.
If this results in $A$ becoming empty, a new set of articulation points is computed on the current graph in linear time.
However, if no articulation points can be found, we select a vertex based on the default branching strategy.
Otherwise, a random vertex from $A$ is selected as the branching vertex.

Even tough this strategy yields only a small (linear) overhead, finding articulation points can be rare depending on the type of graph.
This results in the default branching strategy being selected rather frequently.

\subsection{Edge Cuts}
To alleviate the restrictive nature of finding articulation points, we now propose a more flexible branching strategy based on \emph{(minimal) edge cuts}.
In general, we still aim to find small vertex separators, \ie a set of vertices whose removal disconnects the graph.
However, since computing an actual minimum vertex separator is computationally expensive, we opt to use minimum edge cuts instead.

A \emph{cut} $(S,T)$ is a partitioning of $V$ into two sets $S$ and $T=V\setminus S$.
Furthermore, a cut is called minimum if its \emph{cut set} $C = \{\{u,v\} \in E \mid u \in S, v \in T\}$ has minimal cardinality.
However, in practice, finding minimum cuts often yields trivial cuts with either $S$ or $T$ only consisting of a single vertex with minimum degree. 
Thus, we are interested in finding \emph{$s$-$t$-cuts}, \ie cuts where $S$ and $T$ contain specific vertices $s,t \in V$.
Finding these cuts can be done efficiently in practice, \eg using a preflow push algorithm.
However, selecting the vertices $s$ and $t$ to ensure reasonably balanced cuts can be tricky.
Natural choices include random vertices, as well as vertices that are far apart in terms of their shortest path distance.
However, our preliminary experiments indicate that selecting the highest degree vertices for $s$ and $t$ seems to produce the best results.
Finally, to derive a vertex separator from a cut one can compute a minimum vertex cover on the bipartite graph induced by the cut set, \eg using the Hopcroft-Karp algorithm.
This separator can then be used to select branching vertices.

Overall, our second strategy works similar to the first one: We maintain a set of possible branching vertices that were selected by computing a minimum $s$-$t$-cut and turning it into a vertex separator.
Vertices that were removed by kernelization are discarded from this set and once it is empty, a new cut computation is started.
However, in contrast to the first strategy finding a set of suitable branching vertices is much more likely.
In order to avoid separators that contain too many vertices, and thus would require too many branching steps to disconnect the graph, we only keep those that do not exceed a certain size and balance threshold.
The specific values for these threshold are presented in Section~\ref{sec:algo_conf}.
Finally, if no suitable separator is found, we use the default branching strategy.
Furthermore, in this case we do not try to find a new separator for a fixed number of branching steps as finding one is both unlikely and costly.

\subsection{Nested Dissection}

\section{Reduction Branching}
\label{sec:reduction_branching}
Demian

\section{Experimental Evaluation}

\subsection{Experimental Environment}

\subsection{Algorithm Configuration}
\label{sec:algo_conf}

\subsection{Instances}

\subsection{Decomposition Branching}

\subsection{Reduction Branching}

\section{Conclusion and Future Work}

%%
%% Bibliography
%%

%% Please use bibtex, 

\bibliography{references}

\appendix

\section{Detailed Instance Information}

\end{document}
