\documentclass[a4paper,UKenglish,cleveref, autoref, thm-restate]{lipics-v2021}
%See lipics-v2021-authors-guidelines.pdf for further information.
%for A4 paper format use option "a4paper", for US-letter use option "letterpaper"
%for british hyphenation rules use option "UKenglish", for american hyphenation rules use option "USenglish"
%for section-numbered lemmas etc., use "numberwithinsect"
%for enabling cleveref support, use "cleveref"
%for enabling autoref support, use "autoref"
%for anonymousing the authors (e.g. for double-blind review), add "anonymous"
%for enabling thm-restate support, use "thm-restate"
%for enabling a two-column layout for the author/affilation part (only applicable for > 6 authors), use "authorcolumns"

\usepackage{todonotes}
\usepackage{numprint}
\npstyleenglish
\usepackage{placeins}
\usepackage{pgf}

\usepackage[ruled,vlined,linesnumbered,norelsize]{algorithm2e}
\DontPrintSemicolon
\def\NlSty#1{\textnormal{\fontsize{8}{10}\selectfont{}#1}}
\SetKwSty{texttt}
\SetCommentSty{emph}
\def\algorithmautorefname{Algorithm}

\listfiles

%\graphicspath{{./graphics/}}%helpful if your graphic files are in another directory

\bibliographystyle{plainurl}% the mandatory bibstyle

\title{Targeted Branching for the Maximum Independent Set Problem} %TODO Please add

\titlerunning{Targeted Branching for the MIS Problem} %TODO optional, please use if title is longer than one line

% \author{John Q. Public}{Dummy University Computing Laboratory, [optional: Address], Country \and My second affiliation, Country \and \url{http://www.myhomepage.edu} }{johnqpublic@dummyuni.org}{https://orcid.org/0000-0002-1825-0097}{(Optional) author-specific funding acknowledgements}%TODO mandatory, please use full name; only 1 author per \author macro; first two parameters are mandatory, other parameters can be empty. Please provide at least the name of the affiliation and the country. The full address is optional

\author{Demian Hespe}{Karlsruhe Institute of Technology, Institute for
  Theoretical Informatics,
  Germany}{hespe@kit.edu}{https://orcid.org/0000-0001-6232-2951}{\todo{Add
    funding}}

\author{Sebastian Lamm}{Karlsruhe Institute of Technology, Institute for
  Theoretical Informatics, Germany}{lamm@kit.edu}{}{\todo{Add funding}}

\author{Christian Schorr}{Karlsruhe Institute of Technology, Institute for
  Theoretical Informatics, Germany}{christian.schorr@student.kit.edu}{}{}

\authorrunning{D. Hespe and S. Lamm and C. Schorr} %TODO mandatory. First: Use abbreviated first/middle names. Second (only in severe cases): Use first author plus 'et al.'

\Copyright{Demian Hespe and Sebastian Lamm and Christian Schorr} %TODO mandatory, please use full first names. LIPIcs license is "CC-BY";  http://creativecommons.org/licenses/by/3.0/
\begin{CCSXML}
  <ccs2012>
  <concept>
  <concept_id>10002950.10003624.10003633.10010917</concept_id>
  <concept_desc>Mathematics of computing~Graph algorithms</concept_desc>
  <concept_significance>500</concept_significance>
  </concept>
  <concept>
  <concept_id>10003752.10003809.10011254.10011256</concept_id>
  <concept_desc>Theory of computation~Branch-and-bound</concept_desc>
  <concept_significance>500</concept_significance>
  </concept>
  <concept>
  <concept_id>10002950.10003624.10003625.10003630</concept_id>
  <concept_desc>Mathematics of computing~Combinatorial optimization</concept_desc>
  <concept_significance>500</concept_significance>
  </concept>
  </ccs2012>
\end{CCSXML}

\ccsdesc[500]{Mathematics of computing~Graph algorithms}
\ccsdesc[500]{Theory of computation~Branch-and-bound}
\ccsdesc[500]{Mathematics of computing~Combinatorial optimization}
% \ccsdesc[100]{\textcolor{red}{Replace ccsdesc macro with valid oll
% ne}} %TODO mandatory: Please choose ACM 2012 classifications from https://dl.acm.org/ccs/ccs_flat.cfm 

\keywords{Graphs, Combinatorial Optimization, Independent Set, Vertex Cover,
  Clique, Branch-and-Reduce, Branch-and-Bound, Data Reduction}%TODO mandatory; please add comma-separated list of keywords

\category{} %optional, e.g. invited paper

\relatedversion{} %optional, e.g. full version hosted on arXiv, HAL, or other respository/website
%\relatedversiondetails[linktext={opt. text shown instead of the URL}, cite=DBLP:books/mk/GrayR93]{Classification (e.g. Full Version, Extended Version, Previous Version}{URL to related version} %linktext and cite are optional

%\supplement{}%optional, e.g. related research data, source code, ... hosted on a repository like zenodo, figshare, GitHub, ...
%\supplementdetails[linktext={opt. text shown instead of the URL}, cite=DBLP:books/mk/GrayR93, subcategory={Description, Subcategory}, swhid={Software Heritage Identifier}]{General Classification (e.g. Software, Dataset, Model, ...)}{URL to related version} %linktext, cite, and subcategory are optional

%\funding{(Optional) general funding statement \dots}%optional, to capture a funding statement, which applies to all authors. Please enter author specific funding statements as fifth argument of the \author macro.

\acknowledgements{}%optional

%\nolinenumbers %uncomment to disable line numbering

% \hideLIPIcs  %uncomment to remove references to LIPIcs series (logo, DOI, ...), e.g. when preparing a pre-final version to be uploaded to arXiv or another public repository

%Editor-only macros:: begin (do not touch as author)%%%%%%%%%%%%%%%%%%%%%%%%%%%%%%%%%%
\EventEditors{John Q. Open and Joan R. Access}
\EventNoEds{2}
\EventLongTitle{42nd Conference on Very Important Topics (CVIT 2016)}
\EventShortTitle{CVIT 2016}
\EventAcronym{CVIT}
\EventYear{2016}
\EventDate{December 24--27, 2016}
\EventLocation{Little Whinging, United Kingdom}
\EventLogo{}
\SeriesVolume{42}
\ArticleNo{23}
%%%%%%%%%%%%%%%%%%%%%%%%%%%%%%%%%%%%%%%%%%%%%%%%%%%%%%

% Custom
\usepackage{xspace}
\newcommand{\ie}{i.\,e.,\xspace}
\newcommand{\eg}{e.\,g.,\xspace}
\newcommand{\etal}{et~al.\xspace}
\newcommand{\Wlog}{w.\,l.\,o.\,g.\ }
\newcommand{\wrt}{w.\,r.\,t.\xspace}

\begin{document}

\maketitle

%TODO mandatory: add short abstract of the document
\begin{abstract}
  Finding a maximum independent set is a fundamental NP-hard problem that is used in many real-world applications.
Given an unweighted graph, this problem asks for a maximum cardinality set of pairwise non-adjacent vertices.
In recent years, some of the most successful algorithms for solving this problem are based on the branch-and-bound or branch-and-reduce paradigms.
In particular, branch-and-reduce algorithms, which combine branch-and-bound with reduction rules, have been able to achieve substantial results, solving many previously infeasible real-world instances.
These results were to a large part achieved by developing new, more practical reduction rules.
However, other components that have been shown to have a significant impact on the performance of these algorithms have not received as much attention.
One of these is the branching strategy, which determines what vertex is included or excluded in a potential solution.
Even now, the most commonly used strategy selects vertices solely based on their degree and does not take into account other factors that contribute to the performance of the algorithm.

In this work, we develop and evaluate several novel branching strategies for both branch-and-bound and branch-and-reduce algorithms.
Our strategies are based on one of two approaches which are motivated by existing research. 
They either (1) aim to decompose the graph into two or more connected components which can then be solved independently, or (2) try to remove vertices that hinder the application of a reduction rule which can lead to smaller graphs.
Our experimental evaluation on a large set of real-world instances indicates
that our strategies are able to improve the performance of the state-of-the-art
branch-and-reduce algorithm by Akiba and Iwata by a geometric mean speedup of $8\%$. In particular, our
decomposition-based strategy achieves a geometric mean speedup of $\numprint{1.8}$ on sparse networks.
Furthermore, our reduction-rule-based strategies are able to outperform the
default branching strategy of selecting a vertex of highest degree on $78\%$ of all
($88\%$ of solved)
instances tested.

\end{abstract}

\newpage

\section{Introduction}

An \emph{independent set} of a graph $G = (V,E)$ is a set of vertices $I \subseteq V$ of $G$ such that no two vertices in this set are adjacent.
The problem of finding such an independent set of maximum cardinality, the \emph{maximum independent set problem}, is a fundamental NP-hard problem~\cite{Garey1974}.
Its applications cover a wide variety of fields including computer graphics \cite{CG}, network analysis \cite{NW}, route planning \cite{RP} and computational biology \cite{BIO1, BIO2}.
In computer graphics for instance, large independent sets can be used to optimize the traversal of mesh edges in a triangle mesh.
Further applications stem from its complementary problems minimum vertex cover and maximum clique.

One of the best known techniques for finding maximum independent sets, both in
theory~\cite{XiaoNagamochi, ChenXiaKanj} and practice \cite{AkibaIwata}, are
\emph{data reduction algorithms}.
These algorithms apply a set of reduction rules to decrease the size of an instance while maintaining the ability to compute an optimal solution afterwards.
A recently successful type of data reduction algorithm is so-called
\emph{branch-and-reduce algorithms}~\cite{AkibaIwata,WGYC}, which exhaustively
apply a set of reduction rules to compute an irreducible graph.
If no further rule can be applied, the algorithm branches into (at least) two
smaller subproblems, which are then solved recursively.
To make them more efficient in practice, these algorithms also make use of problem-specific upper and lower bounds to quickly prune the search space.

Due to the practical impact of data reduction, most of the research aimed at
improving the performance of branch-and-reduce algorithms so far has been
focused on either proposing more practically efficient special cases of already
existing rules~\cite{ChangKern,dahlum2016accelerating}, or maintaining
dependencies between reduction rules to reduce unnecessary
checks~\cite{alsahafy2020computing,hespe2019scalable}.\todo{cite SAT-and-reduce?}
In comparison, efforts to improve other aspects of branch-and-reduce are still lacking.
However, the branching strategy in particular has been shown to have a significant impact on the running time~\cite{AkibaIwata}.
Up to now, the most frequently used branching strategy employed in many state-of-the-art solvers selects branching vertices solely based on their degree.
Other factors, such as the actual reduction rules used during the algorithm are rarely taken into account.
However, recently there have been some attempts to incorporate such branching
strategies for other problems such as finding a maximum $k$-plex~\cite{gao2018exact}.

\subsection{Contribution}
In this paper, we propose and examine several novel strategies for selecting branching vertices.
These strategies follow two main approaches that are motivated by existing research: (1) Branching on vertices that decompose the graph into several connected components that can be solved independently.
This has been shown to significantly improve the performance of branch-and-reduce in practice, especially when the size of the largest component is small~\cite{alsahafy2020computing}.
(2) Branching on vertices whose removal leads to reduction rules becoming applicable again.
In turn, this leads to a smaller reduced graph and thus improved performance.
For each approach we present several concrete strategies that vary in their complexity.
Finally, we evaluate their performance by comparing them to the aforementioned default strategy used in the state-of-the-art solver by Akiba and Iwata~\cite{AkibaIwata}.
For this purpose we make of a wide spectrum of instances from different graph classes and applications.
Our experiments indicate that our strategies are able to find an optimal
solution faster than the default strategy on $84\%$ of all ($93\%$ of solved)
instances tested.\todo{make these numbers the same here and in abstract}
In particular, our strategies achieve a geometric mean speedup of $8\%$ (over the default strategy) over all instances.
A more detailed explanation of a previous version of this work
can be found in Schorr's Bachelor's thesis~\cite{schorr2020improved}.

\section{Preliminaries}
Let $G=(V,E)$ be an undirected graph, where $V = \{0, \ldots, n-1\}$ is a set of $n$ vertices and $E \subseteq  \{\{u,v\} \mid u,v \in V\}$ is a set of $m$ edges. 
We assume that $G$ is \emph{simple}, \ie it has no self loops or multi-edges.
The \emph{(open) neighborhood} of a vertex $v \in V$ is denoted by $N(v) = \{u \mid \{v,u\} \in E\}$.
Furthermore, we denote the \emph{closed neighborhood} of a vertex by $N[v]=N(v) \cup \{v\}$.
We define the open and closed neighborhood of a set of vertices $U \subseteq V$
as $N(U) = \cup_{u \in U} N(v) \setminus U$ and $N[U] = N(U) \cup U$, respectively.
The \emph{degree} of a vertex $v \in V$ is the size of its neighborhood $d(v) =
|N(v)|$ and $\Delta =
\max_{v \in V} \{d(v)\}$.
For a vertex $v \in V$, we further define $N^2(v) = N(N(v))$.

For a subset of vertices $V_S \subseteq V$, the \emph{(vertex-)induced subgraph}
$G[V_S] = (V_S, E_S)$ is given by restricting the edges of $G$ to vertices of $V_S$, \ie $E_S = \{\{u,v\} \in E \mid u,v \in V_S\}$.
Likewise, for a subset of edges $E_S \subseteq E$, the \emph{edge-induced
  subgraph} $G[E_S] = (V_S, E_S)$ is given by restricting the vertices of $G$ to the endpoints of edges in $E_S$, \ie $V_S = \{u \in V \mid \{u,v\} \in E_S\}$.
  For a subset of vertices $U \subset V$, we further define $G - U$ as the induced subgraph $G[V \setminus U]$.

A \emph{path} $P=(v_1, \ldots, v_k)$ of length $k$ is a sequence of $k$ distinct vertices in $G$ such that $\{v_i, v_{i+1}\} \in E$ for all $i \in \{1, \ldots, k-1\}$.
A subgraph of $G$ induced by a maximal subset of vertices that are connected by a path is called a \emph{connected component}.
Furthermore, a graph that only contains one connected components is called \emph{connected}.
Likewise, a graph with more than one connected component is called \emph{disconnected}.
A subset $S \subset V$ of a connected graph $G$ is called a \emph{vertex separator}, if the removal of $S$ from $G$ makes the graph disconnected.

An \emph{independent set} of a graph is a subset of vertices $I \subseteq V$ such that no two vertices of $I$ are adjacent. 
A \emph{maximum independent set} (MIS) is an independent set of maximum cardinality.
Closely related to independent set are vertex covers and cliques.
A \emph{vertex cover} is a set of vertices $C \subseteq V$ such that for each edge $\{u,v\} \in E$ either $u$ or $v$ is contained in $C$.
The complement of a (maximum) independent set of a graph $G$ is a \emph{(minimum) vertex cover} (MVC) of $G$.
A \emph{clique} is a subset of vertices $K \subseteq V$ such that all vertices of $K$ are adjacent to each other, \ie $\forall u,v \in K: \{u,v\} \in E$.
Finally, a (maximum) independent set of a graph $G$ is a \emph{(maximum) clique} (MC) in the complement graph $\overline{G} = (V, \overline{E})$, where $\overline{E} = \{\{u,v\} \mid \{u,v\} \not\in E\}$.

\section{Related Work}
Selecting branching vertices is a part of every branch-and-reduce or
branch-and-bound algorithm. In this section we give an overview of the techniques used in
algorithms for MIS, MVC and MC.

The most commonly used branching strategy for MIS and MVC is to select a vertex
of maximum degree. Fomin et al.~\cite{Fomin} show that using a vertex of maximum
degree that also minimizes the number of edges between its neighbors is optimal
with respect to their complexity measure. The algorithm by Akiba and Iwata~\cite{AkibaIwata}
(which we augment with our new branching rules) also uses this strategy. Akiba and Iwata also
compare this strategy to branching on a vertex of minimum degree and a random
vertex. They show that both of these perform significantly worse than branching
on a maximum degree vertex.

Xiao and Nagamochi~\cite{XiaoNagamochi} also use
this strategy in most cases. For dense subgraphs, however, they use an edge
branching strategy: They branch on an edge $\{u, v\}$ where $|N(u) \cap N(v)|$
is sufficiently large (depending on the maximum degree of the graph) by
excluding both $u$ and $v$ in one branch and applying the alternative reduction
(see Section~\ref{sec:almost_funnels}) to $\{u\}$ and $\{v\}$ in the other branch.

Bourgeois et al.~\cite{Bourgeois} use maximum degree branching as long as there are vertices of degree at least five. Otherwise, they utilize specialized algorithms to solve subinstance with an average degree of three or four. Those algorithms perform a rather complex case analysis to find a suitable branching vertex. The analysis is based on exploiting structures that contain 3- or 4-cycles. Branching on specific vertices in such structures often enables further reduction rules to be applied.

Chen et al.~\cite{ChenXiaKanj} use a notion of \emph{good pairs} that are advantageous
for branching. They chose these good pairs by a set of rules which are omitted
here. They combine these with so-called \emph{tuples} of a set of vertices and the
number of vertices from this set that have to be included in an MIS. This
information can be used when branching on a vertex
contained in that set to remove further vertices from the graph. Akiba and
Iwata~\cite{AkibaIwata} use the same concept in their \emph{packing} rule. Chen
et al. combine good pairs, tuples and high degree vertices for their branching strategy.

Most algorithms for MC (e.g. San Segundo and Tapia~\cite{Color}) compute a
greedy coloring and branch on vertices with a high coloring number.
More sophisticated MC algorithms use MaxSAT encodings to prune the set of
branching vertices~\cite{LiFangXu,LiJiang,LiQuan}. Li et al.~\cite{LiMaxSat}
combine greedy coloring and MaxSAT reasoning the further reduce the number of
branching vertices.\todo{cite MCS algorithm by Tomita et al?}

Another approach used for MC is using the \emph{degeneracy order} $v_1 < v_2 <
\dots < v_n$ where $v_i$ is a vertex of smallest degree in $G - \{v_1, \dots
v_{i-1}\}$. Carraghan and Pardalos~\cite{CarraghanPardalos} present an algorithm
that branches in descending degeneracy order. Li et al.~\cite{LiFangXu}
introduce another vertex ordering using iterative maximum independent set
computations (which might be easier than MC on some graphs) and breaking ties
according to the degeneracy order.

\subsection{Branch-and-Reduce Algorithm by Akiba and Iwata}

The algorithm by Akiba and Iwata~\cite{AkibaIwata} is a so-called
\emph{branch-and-reduce} algorithm: It repeatedly reduces the instance size by a
set of polynomial-time reduction rules and then branches on a vertex once no
more reduction rules can be applied. Since branching removes at least one vertex
from the graph, more reduction rules might be applicable afterwards. The
set of reductions used in their algorithm is relatively large and not covered
completely here. However, some reduction rules are explained in
Section~\ref{sec:reduction_branching} where we show how to target particular reduction
rules when branching. Akiba and Iwata apply the reduction rules by iterating over all
vertices in the graph once for each rule and checking whether the rule
can be applied to each vertex. In order to prune the search space, bounds on the largest
possible independent set of a branch are computed. They implement three different
methods for determining upper bounds: clique cover, LP relaxation and cycle
cover. Additionally, they employ special reduction rules that can be applied
during branching. Another optimization done by their algorithm is to solve
connected components separately. We utilize this in
Section~\ref{sec:decomposition_branching} where we introduce branching rules
that decompose the graph into connected components. Akiba and Iwata also
compare three different strategies for
selecting the branching vertex: Uniform random, minimum degree and maximum
degree. They conclude that choosing a vertex of maximum degree that also
minimizes the number of edges among $N(v)$ performs best. We use this algorithm
as the base implementation to test our new branching strategies.

\section{Decomposition Branching}\label{sec:decomposition_branching}
Our first approach to improve the default branching strategy found in many state-of-the-art algorithms (including that of Akiba and Iwata~\cite{AkibaIwata}) is to decompose the graph into several connected components.
Subsequently processing these components individually has been shown to improve the performance of branch-and-reduce in practice~\cite{alsahafy2020computing}.
To this end, we now present three concrete strategies with varying computational
complexity: articulation points, edge cuts and nested dissections.

\subsection{Articulation Points}
First, we are concerned with finding single vertices that are able to decompose a graph into at least two separated components.
Such points are called \emph{articulation points} (or cut vertices).
Articulation points can be computed in linear time $\mathcal{O}(n+m)$ using a simple depth-first search (DFS) algorithm.
In particular, a vertex $v$ is an articulation point if it is either the root of
the DFS tree and has at least two children or any non-root vertex that has a child $u$, such that no vertex in
the subtree rooted at $u$ has a back edge to one of the ancestors of
$v$.

For our first branching strategy we maintain a set of articulation points $A \subseteq V$.
When selecting a branching vertex, we first discard all invalid vertices from
$A$, \ie vertices that were removed from the graph by a preceding data reduction step.
If this results in $A$ becoming empty, a new set of articulation points is computed on the current graph in linear time.
However, if no articulation points exist, we select a vertex based on the default branching strategy.
Otherwise, if $A$ contains at least one vertex, a random one from $A$ selected as the branching
vertex. Figure~\ref{fig:articulation_points} illustrates branching on an
articulation point.

\begin{figure}[t]
  \centering
  \includegraphics[scale=1]{figures/articulation_points}
  \caption{Branching on an articulation point (circled vertex) the
    graph decomposes into two connected components (gray boxes) that can be
    solved independently. The graphic shows the branch on which the vertex is
    excluded from the independent set.}\label{fig:articulation_points}
\end{figure}

Even though this strategy introduces only a small (linear) overhead, finding articulation points can be rare depending on the type of graph.
This results in the default branching strategy being selected rather
frequently.
Furthermore, our preliminary experiments indicate that articulation points are rarely found at higher depth.
However, due to their low overhead, we can justify searching for them whenever
our set of branching vertices becomes empty.

\subsection{Edge Cuts}
To alleviate the restrictive nature of finding articulation points, we now propose a more flexible branching strategy based on \emph{(minimal) edge cuts}.
In general, we aim to find small vertex separators, \ie a set of vertices whose removal disconnects the graph.
We do so by making use of minimum edge cuts.

A \emph{cut} $(S,T)$ is a partitioning of $V$ into two sets $S$ and $T=V\setminus S$.
Furthermore, a cut is called minimum if its \emph{cut set} $C = \{\{u,v\} \in E \mid u \in S, v \in T\}$ has minimal cardinality.
However, in practice, finding minimum cuts often yields trivial cuts with either $S$ or $T$ only consisting of a single vertex with minimum degree. 
Thus, we are interested in finding \emph{$s$-$t$-cuts}, \ie cuts where $S$ and $T$ contain specific vertices $s,t \in V$.
Finding these cuts can be done efficiently in practice, \eg using a preflow push algorithm~\cite{goldberg1988new}.
However, selecting the vertices $s$ and $t$ to ensure reasonably balanced cuts can be tricky.
Natural choices include random vertices, as well as vertices that are far apart in terms of their shortest path distance.
However, our preliminary experiments indicate that selecting random vertices of maximum degree
 for $s$ and $t$ seems to produce the best results.
Finally, to derive a vertex separator from a cut one can compute an MVC on the bipartite graph induced by the cut set, \eg using the Hopcroft-Karp algorithm~\cite{hopcroft1973n}.
This separator can then be used to select branching vertices.

Overall, our second strategy works similar to the first one: We maintain a set of possible branching vertices that were selected by computing a minimum $s$-$t$-cut and turning it into a vertex separator.
Vertices that were removed by data reduction are discarded from this set and once it is empty, a new cut computation is started.
However, in contrast to the first strategy finding a set of suitable branching vertices is much more likely.
In order to avoid separators that contain too many vertices, and thus would require too many branching steps to disconnect the graph, we only keep those that do not exceed a certain size and balance threshold.
The specific values for these threshold are presented in Section~\ref{sec:algo_conf}.
Finally, if no suitable separator is found, we use the default branching strategy.
Furthermore, in this case we do not try to find a new separator for a fixed number of branching steps as finding one is both unlikely and costly.

\subsection{Nested Dissection}
Both of our previous strategies dynamically maintain a set of branching vertices.
Even though this comes at the advantage that most of the computed vertices remain viable candidates for some branching steps, it introduces a noticeable overhead.
To alleviate this, our last strategy uses a static ordering of possible branching vertices that is computed once at the beginning of the algorithm.
For this purpose we make use of a \emph{nested dissection ordering}~\cite{george1973nested}.

A nested dissection ordering of the vertices of a graph $G$ is obtained by recursively computing balanced bipartitions $(A,B)$ and a vertex separator $S$, that separates $A$ and $B$.
The actual ordering is then given by concatenating the orderings of $A$ and $B$ followed by the vertices of $S$.
Thus, if we select branching vertices based on the reverse of a nested dissection
ordering, we continuously branch on vertices that disconnect the graph into balanced partitions. 
We compute such an ordering once, after finishing the initial data reduction phase.

There are two main optimizations that we use when considering the nested dissection ordering.
First, we limit the number of recursive calls during the nested dissection
computation, because we noticed that vertices at the end of the ordering seldom lead to a decomposition of the graph. 
This is due to vertices being removed by data reduction which can lead to separators becoming invalid.
Furthermore, similar to the edge-cut-based strategy, we limit the size of
separators considered during branching using a threshold.
Again, this is done to ensure that we do not require too many branching steps to decompose the graph.
The specific value for this size threshold is given in Section~\ref{sec:algo_conf}.
If any separator in the nested dissection exceeds this threshold, we use the default branching strategy.

\section{Reduction Branching}\label{sec:reduction_branching}
Our second approach to selecting good branching vertices is to choose a vertex
whose removal will enable the application of new reduction rules. During every
reduction step we find a list of candidate vertices to branch on. The following
sections will demonstrate how we identify such branching candidate vertices with little
computational overhead in practice. To be self contained we will also repeat the reduction
rules used here but omit any proofs that can be found in the original
publications. Out of the candidates found we then select a vertex of
maximum degree. If the degree of all candidate vertices lies below a threshold
(defined in Section~\ref{sec:algo_conf}) or no candidate vertices were found we fall back to branching on a vertex
of maximum degree. The rational here is that a vertex of large degree changes
the structure of the graph more than a vertex of small degree even if that
vertex is guaranteed to enable the application of a reduction rule. Also, our
current strategies only enable the application of the targeted reduction rule in
the branch that excludes the vertex from the independent set, the
\emph{excluding branch}. However, in the
case that includes it into the independent set (\emph{including branch}) all neighbors are removed from
the graph as well because they already have an adjacent vertex in the solution.

We also did some preliminary experiments with storing the candidate vertices in
a priority queue without resetting after every branch and updating that list continually.
However, changes were too frequent for this approach to be faster because of
the high amount of priority queue operations compared to the relatively low overhead
caused by the current approach.

\subsection{Almost Twins}

The first reduction we target is the \emph{twin} reduction by Xiao and Nagamochi~\cite{XiaoUnconfined}:

\begin{definition}(Twins~\cite{XiaoUnconfined})
  In a graph $G=(V,E)$ two vertices $u$ and $v$ are called twins, if $N(u) = N(v)$ and $d(u) = d(v) = 3$.
\end{definition}

\begin{theorem} (Twin Reduction~\cite{XiaoUnconfined}) In a graph $G=(V,E)$ let
  vertices $u$ and $v$ be twins. If there is an edge among $N(u)$, then there is
  always an MIS that includes $\{u,v\}$ and therefore
  excludes $N(u)$. Otherwise, let $G'=(V',E')$ be the graph with $V'=(V\setminus
  N[\{u,v\}])\cup\{w\}$ where $w\notin V$ and $E'=(E\cap\binom{V'}{2})\cup
  \{\{w,x\}\;|\;x\in N^2(u)\}\}$ and let $I'$ be an MIS in $G'$. Then, 
  $
  I=\begin{cases}
    I'\cup \{u,v\} & \text{, if }w\notin I'\\
    (I'\setminus \{w\})\cup N(u) & \text{, else} %w\in I'
  \end{cases}
  $
  is an MIS in $G$.
\end{theorem}

We now define \emph{almost twins} as follows:
\begin{definition} (Almost Twins)
  In a graph $G=(V,E)$ two non adjacent vertices $u$ and $v$ are called almost twins if $d(u) = 4$, $d(v) = 3$ and $N(v)\subseteq N(u)$ (\ie $N(u) = N(v) \cup \{w\}$). 
\end{definition}

Clearly, after removing $w$, $u$ and $v$ are twins so we can apply the twin
reduction. Finding almost twins can be done while searching for twins: The
original algorithm checks for each vertex $v$ of degree-$3$ whether there is a
vertex $u \in N^2(v)$ which $d(u) = 3$ and $N(u) = N(v)$. We augment this
algorithm by simultaneously also searching for $u \in N^2(v)$ with $d(u) = 4$
and $N(v) \subset N(u)$. This induces about the same computational cost for
degree-$4$ vertices in $N^2(v)$ as for degree $3$ vertices. While there might be
instances where this causes high overhead, we expect the practical slowdown to
be small. Figure~\ref{fig:twin} illustrates branching for almost twins. 

\begin{figure}[t]
  \centering
  \includegraphics[scale=1]{figures/twin}
  \caption{Vertices $a$ and $b$ are almost twins. After branching on
  the circled vertex they become twins (in the excluding branch) and can
  be reduced.}\label{fig:twin}
\end{figure}

\subsection{Almost Funnels}\label{sec:almost_funnels}

Next we consider the \emph{funnel} reduction which is a special case of the
\emph{alternative} reduction by Xiao and Nagamochi~\cite{XiaoUnconfined}:

\begin{definition} (Alternative Sets~\cite{XiaoUnconfined})
	In a graph $G=(V,E)$ two non empty, disjoint subsets $A,B\subseteq V$ are
    called alternatives, if $|A| = |B|$ and there is an MIS $I$ in $G$ such that $I\cap(A\cup B)$ is either $A$ or $B$.
\end{definition}

\begin{theorem} (Alternative Reduction~\cite{XiaoUnconfined})
	In a graph $G=(V,E)$ let $A$ and $B$ be alternative sets. Let $G'=(V', E')$
    the graph with $V' = V\setminus(A\cup B\cup (N(A)\cap N(B)))$ and $E' =
    \{\{u,v\} \in E \mid u,v \in V'\} \cup \{ \{u,v\}\;|\; u\in
    N(A)\setminus N[B], v\in N(B)\setminus N[A] \}$ and let $I'$ be an MIS in $G'$. Then,
	$
    I=\begin{cases}
      I'\cup A & \text{, if } (N(A)\setminus N[B]) \cap I' = \emptyset% N(B)\setminus N[A]\subseteq I'\\
      \\
      I'\cup B & \text{, else}
	\end{cases}
    $
	is an MIS in $G$.
\end{theorem}

Note that the alternative reduction adds new edges between existing
vertices of the graph which might not be beneficial in every case. To counteract this, the algorithm
by Akiba and Iwata~\cite{AkibaIwata} only uses special cases, one of which is
the funnel reduction:

\begin{definition} (Funnel~\cite{XiaoUnconfined})
	In a graph $G=(V,E)$ two adjacent vertices $u$ and $v$ are called funnels, if $G_{N(v)\setminus\{u\}}$ is a complete graph, i.e, if $N(v)\setminus\{u\}$ is a clique.
\end{definition}
\begin{theorem} (Funnel Reduction~\cite{XiaoUnconfined}) In a graph $G=(V,E)$ let $u$ and $v$ be funnels. Then, $\{u\}$ and $\{v\}$ are alternative sets.	
\end{theorem}

Again, we define a structure that is covered by the funnel reduction after
removal of a single vertex:

\begin{definition}(Almost Funnel)
  In a graph $G=(V,E)$ two adjacent vertices $u$ and $v$ are called almost funnels if $u$ and $v$ are not funnels and there is a vertex $w$ such that $N(v)\setminus\{u,w\}$ induces a clique.
\end{definition}

By removing $w$, $u$ and $v$ become funnels. The original funnel algorithm
checks whether $u$ and $v$ are funnels by iterating over the vertices in $N(v) \setminus \{u\}$
and checking whether they are adjacent to \emph{all} previous vertices. Once a vertex
is found that is not adjacent to all previous vertices, the algorithm concludes
that $u$ and $v$ are not funnels and terminates. We augment this algorithm by
not immediately terminating in this case. Instead, we consider the following two
cases: Either the current vertex $w$ is not adjacent to at least two of the
previous vertices. In this case, we can check whether $N(v) \setminus \{u, w\}$
induces a clique. In the second case, $w$ is adjacent to all but one previous
vertex $w'$. In this case, both $w$ and $w'$ might be candidate branching
vertices. Thus, we check whether $N(v) \setminus \{u, w\}$ or $N(v) \setminus
\{u, w'\}$ induce a clique. This adds up to two additional clique checks (of
slightly smaller size) to the one clique check in the original algorithm.

% \subsection{Almost Dominance}
% \label{sec:almost_dominance}

% The next branching strategy is based on the dominance reduction rule by Fomin et al.~\cite{Fomin}:

% \begin{definition} (Dominance~\cite{Fomin})
%   In a graph $G=(V,E)$ a vertex $u$ is called dominated by a neighbor $v$, if $N[u]\subseteq N[v]$.
% \end{definition}

% \begin{theorem} (Dominance Reduction~\cite{Fomin}) In a graph $G=(V,E)$, if a
%   vertex $u$ is dominated by a neighbor $v$, then, there always exists an MIS that does not include $v$, i.e.~$\alpha(G)=\alpha(G-\{v\})$
% \end{theorem}

% A vertex is almost dominated by one of it has only one surplus neighbor:

% \begin{definition} (Almost Dominance)
%   In a graph $G=(V,E)$ a vertex $u$ is called almost dominated by a neighbor $v$ if $|N(u)\setminus N(v)| = 1$ and $N[v]\not\subseteq N(u)$.
% \end{definition}
%  The last condition ensures that $v$ is not dominated by $u$. The original
%  algorithm checks for each neighbor of $u$ is also a neighbor of $v$. Once it
%  finds a vertex in $N(u)$ that is not in $N(v)$, it terminates because $u$ can not
%  be dominated by $v$. We augment this algorithm by allowing a single vertex $w
%  \in N(u)$ that is not adjacent to $v$ and adding $w$ as a candidate branching
%  vertex if it exists. If $w$ is found early in the algorithm, this might induce
%  some overhead to the algorithm but assuming the position of $w$ in $u$'s
%  adjacency list to be uniformly randomly distributed, the expected number of
%  vertices to check only doubles.

%  Note that the dominance reduction is not directly used in the Algorithm by
%  Akiba and Iwata~\cite{AkibaIwata} because it is fully contained in the unconfined
%  reduction (see Section~\ref{sec:almost_unconfined}). This means that we cannot
%  simply search for almost dominated vertices during the reduction step. Instead,
%  we have to do an additional pass over all vertices.
 
\subsection{Almost Unconfined}\label{sec:almost_unconfined}

% The \emph{unconfined} reduction by Xiao and Nagamochi~\cite{XiaoNagamochi} is a generalization of the dominance reduction
% (see Section~\ref{sec:almost_dominance}).
The core idea of the \emph{unconfined} reduction by Xiao and
Nagamochi~\cite{XiaoUnconfined} is to detect vertices not required for an MIS
that can therefore can be removed from the graph by algorithmically
contradicting the assumption that every MIS contains the vertex.

\begin{definition} (Removable Vertex~\cite{XiaoUnconfined})
	In a graph $G=(V,E)$ a vertex $v$ is called removable, if 
	$\alpha(G) = \alpha(G-\{v\})$.
\end{definition}

\begin{definition} (Child, Parent~\cite{XiaoUnconfined}) In a graph $G=(V,E)$ with an \textit{independent set} $I$, a vertex $v$ is called a child of $I$, if $|N(v)\cap I| = 1$ and the unique neighbor of $v$ in $I$ is called the parent of $v$.
\end{definition}

\begin{theorem} (\cite{XiaoUnconfined})\label{theorem:unconfined}
	In a graph $G=(V,E)$ let $S$ be an independent set that is not maximal but is contained in every MIS of $G$ and let $v$ be any child of $S$. Then, every MIS includes at least one vertex from $N(v)\setminus N[S]$.
\end{theorem}

Based on Theorem \ref{theorem:unconfined} Algorithm \ref{alg:unconf} detects so called \textit{unconfined} vertices.

\begin{algorithm}[t]
	\caption{Unconfined -- Xiao and Nagamochi
      \cite{XiaoUnconfined}}\label{alg:unconf}
	\SetKwFunction{U}{Unconfined}
	\DontPrintSemicolon
	
	\KwIn{A graph $G$, a vertex $v$}
	\U{G, v}	
	\Begin{
		$S \leftarrow \{v\}$\;
		\While{$S\text{ has child }u\text{ with } |N(u)\setminus N[S]|\leq 1$}{
			\eIf{$|N(u)\setminus N[S]| = 0$}{\Return true\tcp*{contradiction to Theorem \ref{theorem:unconfined}}}{
				$\{w\}\leftarrow N(u)\setminus N[v]$\tcp*{by assumption $w$ also has to}
				$S\leftarrow S\cup\{w\}$\tcp*{be contained in every MIS}
			}
		}
		
		\Return $\text{false}$
		
	}
	
	\KwOut{true if $v$ is unconfined, false otherwise}
	
\end{algorithm}


\begin{theorem}(Unconfined Reduction~\cite{XiaoUnconfined}) In a graph $G=(V,E)$,
  if Algorithm \ref{alg:unconf} returns true for an unconfined vertex $v$, then
  there is always an MIS that does not contain~$v$.
\end{theorem}

Again, we define a vertex to be almost unconfined:

\begin{definition} (Almost Unconfined)
  In a graph $G=(V,E)$ a vertex $v$ is called almost unconfined if $v$ is not unconfined but there is a vertex $w$ such that $v$ is unconfined in $G-\{w\}$.
\end{definition}

Here, we only present an augmentation that detects \emph{some} almost
unconfined vertices. In particular, if at any point during the algorithm there
is only \emph{one} extending child, i.e. a child $u$ of $S$ with $N(u)\setminus
N[S] = \{w\}$, then removal of $w$ makes $v$ unconfined. During
Algorithm~\ref{alg:unconf} we collect all these vertices $w$ and add them to the
set of candidate branching vertices if the algorithm cannot already remove $v$.
This only adds the overhead of temporarily storing the potential candidates and
adding them to the actual candidate list if $v$ is not removed.

% Note that this special case of almost unconfined vertices might not catch all
% cases of almost dominated vertices.

\section{Experimental Evaluation}

In this section we present the results of our experimental evaluation. Tables
and figures here show aggregated results. For
detailed results for all of our algorithms across all instances, see Appendix~\ref{app:detailed_results}.

\subsection{Experimental Environment}
We augment a C++-adaptation of the algorithm by Akiba and
Iwata~\cite{AkibaIwata} with our branching strategies. \todo{add Software} Our
code is publicly available on GitHub\footnote{\url{https://github.com/Hespian/CutBranching}}. We execute all our experiments on a machine with 4 8-core Intel Xeon E5-4640 CPUs
(2.4~GHz) and 512 GiB DDR3-PC1600 RAM. To speed up our experiments we use two
identical machines and run at most 16 instances at once on the
same machine (using the same machine for all algorithms on a specific instance).
All numbers reported are arithmetic means of three runs with a timeout of ten
hours.

\subsection{Algorithm Configuration}\label{sec:algo_conf}
We use a C++ adaptation of the implementation by Akiba and
Iwata~\cite{AkibaIwata} in its default configuration as a basis for our algorithm. During preliminary experiments we found
suitable values for the parameters of
our techniques. These experiments were run on a subset of our total instance
set. We use the geometric mean over all instances of the speedup over the
default branching strategy as a basis for the following decisions: For the technique based on
edge cuts, we only use cuts that contain at most 25 vertices and where the smaller side of
the cut contains at least ten percent of the remaining vertices. If no suitable separator is found, we skip ten branching steps. For computing nested dissections, we use 
InertialFlowCutter~\cite{gottesburen2019faster}
with the KaFFPa~\cite{DBLP:conf/wea/SandersS13}
backend. The KaFFPa partitioner is configured to use the \emph{strong} preset
with a fixed seed\footnote{42}\todo{@Christian: Check seed}. For branching we use
three levels of nested dissections with a minimum balance of at least 40 percent of the vertices in the smaller part of each dissection. Furthermore, we
only use the nested dissection if separators contain at most
50 vertices. For the reduction-based
branching rules, we fall back to the default branching strategy if all
candidates have a degree of less than $\Delta - 2$. 

\subsection{Instances}
We use instances from several sources: A sample of the instances used for the
PACE 2019 Challenge on Minimum Vertex Cover~\cite{dzulfikar_et_al:LIPIcs:2019:11486}. 
Complements of Maximum Clique instances from the second DIMACS Implementation Challenge~\cite{johnson1993cliques} and sparse instances from
the Stanford Network Analysis Project (SNAP)~\cite{snapnets}, the 9th DIMACS
Implementation Challenge on Shortest Paths~\cite{demetrescu2009shortest} and the
Network Data Repository~\cite{nr}. Detailed instance information can
be found in Table~\ref{table:instance} in Appendix~\ref{app:instances}. Directed instances were converted into
undirected graphs by ignoring the direction of edges and removing duplicates.
Our original set of instances contained the first 100 public PACE instances, 41 DIMACS instances and 29 sparse networks.
From these instances, we excluded all instances that (1) required no branches, (2) on which all techniques had a running time of less than $0.1$ second, and (3) on which no technique was able to find a solution within 24 hours.
The remaining set of instances is composed of 40 PACE instances, 30 DIMACS instances and 13 sparse networks.

\subsection{Decomposition Branching}\label{sec:experiments_decomp}

Figure~\ref{fig:all_decomp_time} shows a performance plot of the running time of
our decomposition-based branching strategies: For each instance, the best
algorithm is determined. The plot shows for each algorithm $1 - (t_{best}/
t_{alg})$, where $t_{best}$ is the running time of the fastest algorithm
for that instance and $t_{alg}$ is the time of the respective algorithm.
A value of $0$ means that
the algorithm was the fastest algorithm on that instance and a value of
$\numprint{0.5}$ means that the algorithm took twice as long as the fastest
algorithm. Timeouts are shown as a value $> 1$. For each algorithm, instances
are then sorted by their value. Note that this means that the points on a
vertical line do not necessarily correspond to the same instance. Furthermore,
the plot uses a logarithmic scale for the $y$-axis to better illustrate small values.

The plot in Figure~\ref{fig:all_decomp_time} shows that for most instances, the default strategy of branching on a
vertex of maximum degree outperforms our decomposition-based approaches.
However, for instances that have suitable candidates for decomposition such as
sparse networks, significant speedups compared to the default strategy can be
seen. To be more specific, accounting a time of ten hours (our timeout threshold) for unfinished
instances, we achieve a geometric
mean speedup of $\numprint{1.65}$ to $\numprint{1.84}$ for our
decomposition-based techniques on sparse networks (see
Table~\ref{tab:summary_decomp}). In particular, there is one instance (web-stanford) that causes a timeout with the
default strategy but can be solved in $48$ (articulation points) to $164$ (nested
dissections) seconds. Table~\ref{tab:summary_decomp} shows that overall, our
technique using edge cuts seems to be the most beneficial, achieving an overall speedup of $8\%$ over maximum degree and $1\%$
when only considering instances that were solved by all decomposition-based
techniques and maximum degree branching.
Finally, most running
times are only slightly slower than the default strategy with a few instances
showing a speedup. This is mainly because
the number of branches required to solve the instances does not change in most
cases (see Figure~\ref{fig:all_decomp_branch}) and most of the running time
difference is caused by the overhead from
searching for branching vertices.\todo{address Darren's comments after figuring
  out how to average speedups}
% \todo{I would add that this behavior is not 'too' surprising}

\begin{table}[t]
  \caption{Speedup of our decomposition-based techniques over the default
    strategy (maximum degree). ``all'' are geometric means of the speedup over all
  instances of that type (timeouts are assigned a running time of
  ten hours), ``solved'' are geometric means of the
  speedup over all instances that have been solved by all techniques.}\label{tab:summary_decomp}

  \centering
  \begin{tabular}{|l|rr|rr|rr|rr|}
    \hline
    Instance type $\rightarrow$ & \multicolumn{2}{c|}{PACE} & \multicolumn{2}{c|}{DIMACS} & \multicolumn{2}{c|}{sparse nets.} & \multicolumn{2}{c|}{All Instances}                                                                                                            \\
    technique $\downarrow$      & \multicolumn{1}{c}{all}   & \multicolumn{1}{c|}{solved} & \multicolumn{1}{c}{all}   & \multicolumn{1}{c|}{solved} & \multicolumn{1}{c}{all}  & \multicolumn{1}{c|}{solved} & \multicolumn{1}{c}{all}  & \multicolumn{1}{c|}{solved} \\
    \hline
    articulation points         & \numprint{0.98}           & \numprint{0.98}             & \numprint{0.98}           & \numprint{0.98}             & \numprint{1.82}          & \numprint{0.92}             & \numprint{1.07}          & \numprint{0.97}             \\
    edge cuts                   & \textbf{\numprint{1.02}}  & \textbf{\numprint{1.02}}    & \numprint{0.97}           & \numprint{0.97}             & \textbf{\numprint{1.84}} & \textbf{\numprint{1.06}}    & \textbf{\numprint{1.08}} & \textbf{\numprint{1.01}}    \\
    nested dissections          & \numprint{0.96}           & \numprint{0.96}             & \numprint{0.78}           & \numprint{0.78}             & \numprint{1.65}          & \numprint{0.86}             & \numprint{0.95}          & \numprint{0.88}             \\
    \hline
    \end{tabular}
\end{table}

%\begin{figure}[t]
%\includegraphics[width=\textwidth]{plots/all_decomposition_based_time}
%\caption{Performance plot comparing running times of decomposition-based branching strategies.}
%\label{fig:all_decomp_time}
%\end{figure}

%\begin{figure}[t]
%  \includegraphics[width=\textwidth]{plots/all_decomposition_based_branch}
%  \caption{Performance plot comparing branches of decomposition-based branching strategies.}
%  \label{fig:all_decomp_branch}
%\end{figure}

\begin{figure}[t]
	\centering
	%% Creator: Matplotlib, PGF backend
%%
%% To include the figure in your LaTeX document, write
%%   \input{<filename>.pgf}
%%
%% Make sure the required packages are loaded in your preamble
%%   \usepackage{pgf}
%%
%% and, on pdftex
%%   \usepackage[utf8]{inputenc}\DeclareUnicodeCharacter{2212}{-}
%%
%% or, on luatex and xetex
%%   \usepackage{unicode-math}
%%
%% Figures using additional raster images can only be included by \input if
%% they are in the same directory as the main LaTeX file. For loading figures
%% from other directories you can use the `import` package
%%   \usepackage{import}
%%
%% and then include the figures with
%%   \import{<path to file>}{<filename>.pgf}
%%
%% Matplotlib used the following preamble
%%   \renewcommand{\sfdefault}{phv}
%%   \renewcommand{\rmdefault}{ptm}
%%   \renewcommand{\ttdefault}{pcr}
%%   \normalfont\selectfont
%%
\begingroup%
\makeatletter%
\begin{pgfpicture}%
\pgfpathrectangle{\pgfpointorigin}{\pgfqpoint{5.412900in}{3.307160in}}%
\pgfusepath{use as bounding box, clip}%
\begin{pgfscope}%
\pgfsetbuttcap%
\pgfsetmiterjoin%
\definecolor{currentfill}{rgb}{1.000000,1.000000,1.000000}%
\pgfsetfillcolor{currentfill}%
\pgfsetlinewidth{0.000000pt}%
\definecolor{currentstroke}{rgb}{1.000000,1.000000,1.000000}%
\pgfsetstrokecolor{currentstroke}%
\pgfsetdash{}{0pt}%
\pgfpathmoveto{\pgfqpoint{0.000000in}{0.000000in}}%
\pgfpathlineto{\pgfqpoint{5.412900in}{0.000000in}}%
\pgfpathlineto{\pgfqpoint{5.412900in}{3.307160in}}%
\pgfpathlineto{\pgfqpoint{0.000000in}{3.307160in}}%
\pgfpathclose%
\pgfusepath{fill}%
\end{pgfscope}%
\begin{pgfscope}%
\pgfsetbuttcap%
\pgfsetmiterjoin%
\definecolor{currentfill}{rgb}{1.000000,1.000000,1.000000}%
\pgfsetfillcolor{currentfill}%
\pgfsetlinewidth{0.000000pt}%
\definecolor{currentstroke}{rgb}{0.000000,0.000000,0.000000}%
\pgfsetstrokecolor{currentstroke}%
\pgfsetstrokeopacity{0.000000}%
\pgfsetdash{}{0pt}%
\pgfpathmoveto{\pgfqpoint{0.581793in}{0.460676in}}%
\pgfpathlineto{\pgfqpoint{5.312900in}{0.460676in}}%
\pgfpathlineto{\pgfqpoint{5.312900in}{3.044924in}}%
\pgfpathlineto{\pgfqpoint{0.581793in}{3.044924in}}%
\pgfpathclose%
\pgfusepath{fill}%
\end{pgfscope}%
\begin{pgfscope}%
\pgfpathrectangle{\pgfqpoint{0.581793in}{0.460676in}}{\pgfqpoint{4.731107in}{2.584248in}}%
\pgfusepath{clip}%
\pgfsetbuttcap%
\pgfsetroundjoin%
\definecolor{currentfill}{rgb}{0.121569,0.466667,0.705882}%
\pgfsetfillcolor{currentfill}%
\pgfsetlinewidth{1.003750pt}%
\definecolor{currentstroke}{rgb}{0.121569,0.466667,0.705882}%
\pgfsetstrokecolor{currentstroke}%
\pgfsetdash{}{0pt}%
\pgfsys@defobject{currentmarker}{\pgfqpoint{-0.019018in}{-0.019018in}}{\pgfqpoint{0.019018in}{0.019018in}}{%
\pgfpathmoveto{\pgfqpoint{0.000000in}{-0.019018in}}%
\pgfpathcurveto{\pgfqpoint{0.005044in}{-0.019018in}}{\pgfqpoint{0.009881in}{-0.017014in}}{\pgfqpoint{0.013448in}{-0.013448in}}%
\pgfpathcurveto{\pgfqpoint{0.017014in}{-0.009881in}}{\pgfqpoint{0.019018in}{-0.005044in}}{\pgfqpoint{0.019018in}{0.000000in}}%
\pgfpathcurveto{\pgfqpoint{0.019018in}{0.005044in}}{\pgfqpoint{0.017014in}{0.009881in}}{\pgfqpoint{0.013448in}{0.013448in}}%
\pgfpathcurveto{\pgfqpoint{0.009881in}{0.017014in}}{\pgfqpoint{0.005044in}{0.019018in}}{\pgfqpoint{0.000000in}{0.019018in}}%
\pgfpathcurveto{\pgfqpoint{-0.005044in}{0.019018in}}{\pgfqpoint{-0.009881in}{0.017014in}}{\pgfqpoint{-0.013448in}{0.013448in}}%
\pgfpathcurveto{\pgfqpoint{-0.017014in}{0.009881in}}{\pgfqpoint{-0.019018in}{0.005044in}}{\pgfqpoint{-0.019018in}{0.000000in}}%
\pgfpathcurveto{\pgfqpoint{-0.019018in}{-0.005044in}}{\pgfqpoint{-0.017014in}{-0.009881in}}{\pgfqpoint{-0.013448in}{-0.013448in}}%
\pgfpathcurveto{\pgfqpoint{-0.009881in}{-0.017014in}}{\pgfqpoint{-0.005044in}{-0.019018in}}{\pgfqpoint{0.000000in}{-0.019018in}}%
\pgfpathclose%
\pgfusepath{stroke,fill}%
}%
\begin{pgfscope}%
\pgfsys@transformshift{0.796844in}{2.888974in}%
\pgfsys@useobject{currentmarker}{}%
\end{pgfscope}%
\begin{pgfscope}%
\pgfsys@transformshift{0.849295in}{2.888974in}%
\pgfsys@useobject{currentmarker}{}%
\end{pgfscope}%
\begin{pgfscope}%
\pgfsys@transformshift{0.901746in}{2.888974in}%
\pgfsys@useobject{currentmarker}{}%
\end{pgfscope}%
\begin{pgfscope}%
\pgfsys@transformshift{0.954198in}{2.888974in}%
\pgfsys@useobject{currentmarker}{}%
\end{pgfscope}%
\begin{pgfscope}%
\pgfsys@transformshift{1.006649in}{2.888974in}%
\pgfsys@useobject{currentmarker}{}%
\end{pgfscope}%
\begin{pgfscope}%
\pgfsys@transformshift{1.059100in}{2.888974in}%
\pgfsys@useobject{currentmarker}{}%
\end{pgfscope}%
\begin{pgfscope}%
\pgfsys@transformshift{1.111551in}{2.888974in}%
\pgfsys@useobject{currentmarker}{}%
\end{pgfscope}%
\begin{pgfscope}%
\pgfsys@transformshift{1.164003in}{2.888974in}%
\pgfsys@useobject{currentmarker}{}%
\end{pgfscope}%
\begin{pgfscope}%
\pgfsys@transformshift{1.216454in}{2.888974in}%
\pgfsys@useobject{currentmarker}{}%
\end{pgfscope}%
\begin{pgfscope}%
\pgfsys@transformshift{1.268905in}{2.444901in}%
\pgfsys@useobject{currentmarker}{}%
\end{pgfscope}%
\begin{pgfscope}%
\pgfsys@transformshift{1.321357in}{1.664462in}%
\pgfsys@useobject{currentmarker}{}%
\end{pgfscope}%
\begin{pgfscope}%
\pgfsys@transformshift{1.373808in}{1.566817in}%
\pgfsys@useobject{currentmarker}{}%
\end{pgfscope}%
\begin{pgfscope}%
\pgfsys@transformshift{1.426259in}{1.556800in}%
\pgfsys@useobject{currentmarker}{}%
\end{pgfscope}%
\begin{pgfscope}%
\pgfsys@transformshift{1.478710in}{1.295703in}%
\pgfsys@useobject{currentmarker}{}%
\end{pgfscope}%
\begin{pgfscope}%
\pgfsys@transformshift{1.531162in}{1.115759in}%
\pgfsys@useobject{currentmarker}{}%
\end{pgfscope}%
\begin{pgfscope}%
\pgfsys@transformshift{1.583613in}{0.769167in}%
\pgfsys@useobject{currentmarker}{}%
\end{pgfscope}%
\begin{pgfscope}%
\pgfsys@transformshift{1.636064in}{0.746040in}%
\pgfsys@useobject{currentmarker}{}%
\end{pgfscope}%
\begin{pgfscope}%
\pgfsys@transformshift{1.688516in}{0.460676in}%
\pgfsys@useobject{currentmarker}{}%
\end{pgfscope}%
\begin{pgfscope}%
\pgfsys@transformshift{1.740967in}{0.460676in}%
\pgfsys@useobject{currentmarker}{}%
\end{pgfscope}%
\begin{pgfscope}%
\pgfsys@transformshift{1.793418in}{0.460676in}%
\pgfsys@useobject{currentmarker}{}%
\end{pgfscope}%
\begin{pgfscope}%
\pgfsys@transformshift{1.845870in}{0.460676in}%
\pgfsys@useobject{currentmarker}{}%
\end{pgfscope}%
\begin{pgfscope}%
\pgfsys@transformshift{1.898321in}{0.460676in}%
\pgfsys@useobject{currentmarker}{}%
\end{pgfscope}%
\begin{pgfscope}%
\pgfsys@transformshift{1.950772in}{0.460676in}%
\pgfsys@useobject{currentmarker}{}%
\end{pgfscope}%
\begin{pgfscope}%
\pgfsys@transformshift{2.003223in}{0.460676in}%
\pgfsys@useobject{currentmarker}{}%
\end{pgfscope}%
\begin{pgfscope}%
\pgfsys@transformshift{2.055675in}{0.460676in}%
\pgfsys@useobject{currentmarker}{}%
\end{pgfscope}%
\begin{pgfscope}%
\pgfsys@transformshift{2.108126in}{0.460676in}%
\pgfsys@useobject{currentmarker}{}%
\end{pgfscope}%
\begin{pgfscope}%
\pgfsys@transformshift{2.160577in}{0.460676in}%
\pgfsys@useobject{currentmarker}{}%
\end{pgfscope}%
\begin{pgfscope}%
\pgfsys@transformshift{2.213029in}{0.460676in}%
\pgfsys@useobject{currentmarker}{}%
\end{pgfscope}%
\begin{pgfscope}%
\pgfsys@transformshift{2.265480in}{0.460676in}%
\pgfsys@useobject{currentmarker}{}%
\end{pgfscope}%
\begin{pgfscope}%
\pgfsys@transformshift{2.317931in}{0.460676in}%
\pgfsys@useobject{currentmarker}{}%
\end{pgfscope}%
\begin{pgfscope}%
\pgfsys@transformshift{2.370382in}{0.460676in}%
\pgfsys@useobject{currentmarker}{}%
\end{pgfscope}%
\begin{pgfscope}%
\pgfsys@transformshift{2.422834in}{0.460676in}%
\pgfsys@useobject{currentmarker}{}%
\end{pgfscope}%
\begin{pgfscope}%
\pgfsys@transformshift{2.475285in}{0.460676in}%
\pgfsys@useobject{currentmarker}{}%
\end{pgfscope}%
\begin{pgfscope}%
\pgfsys@transformshift{2.527736in}{0.460676in}%
\pgfsys@useobject{currentmarker}{}%
\end{pgfscope}%
\begin{pgfscope}%
\pgfsys@transformshift{2.580188in}{0.460676in}%
\pgfsys@useobject{currentmarker}{}%
\end{pgfscope}%
\begin{pgfscope}%
\pgfsys@transformshift{2.632639in}{0.460676in}%
\pgfsys@useobject{currentmarker}{}%
\end{pgfscope}%
\begin{pgfscope}%
\pgfsys@transformshift{2.685090in}{0.460676in}%
\pgfsys@useobject{currentmarker}{}%
\end{pgfscope}%
\begin{pgfscope}%
\pgfsys@transformshift{2.737541in}{0.460676in}%
\pgfsys@useobject{currentmarker}{}%
\end{pgfscope}%
\begin{pgfscope}%
\pgfsys@transformshift{2.789993in}{0.460676in}%
\pgfsys@useobject{currentmarker}{}%
\end{pgfscope}%
\begin{pgfscope}%
\pgfsys@transformshift{2.842444in}{0.460676in}%
\pgfsys@useobject{currentmarker}{}%
\end{pgfscope}%
\begin{pgfscope}%
\pgfsys@transformshift{2.894895in}{0.460676in}%
\pgfsys@useobject{currentmarker}{}%
\end{pgfscope}%
\begin{pgfscope}%
\pgfsys@transformshift{2.947347in}{0.460676in}%
\pgfsys@useobject{currentmarker}{}%
\end{pgfscope}%
\begin{pgfscope}%
\pgfsys@transformshift{2.999798in}{0.460676in}%
\pgfsys@useobject{currentmarker}{}%
\end{pgfscope}%
\begin{pgfscope}%
\pgfsys@transformshift{3.052249in}{0.460676in}%
\pgfsys@useobject{currentmarker}{}%
\end{pgfscope}%
\begin{pgfscope}%
\pgfsys@transformshift{3.104701in}{0.460676in}%
\pgfsys@useobject{currentmarker}{}%
\end{pgfscope}%
\begin{pgfscope}%
\pgfsys@transformshift{3.157152in}{0.460676in}%
\pgfsys@useobject{currentmarker}{}%
\end{pgfscope}%
\begin{pgfscope}%
\pgfsys@transformshift{3.209603in}{0.460676in}%
\pgfsys@useobject{currentmarker}{}%
\end{pgfscope}%
\begin{pgfscope}%
\pgfsys@transformshift{3.262054in}{0.460676in}%
\pgfsys@useobject{currentmarker}{}%
\end{pgfscope}%
\begin{pgfscope}%
\pgfsys@transformshift{3.314506in}{0.460676in}%
\pgfsys@useobject{currentmarker}{}%
\end{pgfscope}%
\begin{pgfscope}%
\pgfsys@transformshift{3.366957in}{0.460676in}%
\pgfsys@useobject{currentmarker}{}%
\end{pgfscope}%
\begin{pgfscope}%
\pgfsys@transformshift{3.419408in}{0.460676in}%
\pgfsys@useobject{currentmarker}{}%
\end{pgfscope}%
\begin{pgfscope}%
\pgfsys@transformshift{3.471860in}{0.460676in}%
\pgfsys@useobject{currentmarker}{}%
\end{pgfscope}%
\begin{pgfscope}%
\pgfsys@transformshift{3.524311in}{0.460676in}%
\pgfsys@useobject{currentmarker}{}%
\end{pgfscope}%
\begin{pgfscope}%
\pgfsys@transformshift{3.576762in}{0.460676in}%
\pgfsys@useobject{currentmarker}{}%
\end{pgfscope}%
\begin{pgfscope}%
\pgfsys@transformshift{3.629213in}{0.460676in}%
\pgfsys@useobject{currentmarker}{}%
\end{pgfscope}%
\begin{pgfscope}%
\pgfsys@transformshift{3.681665in}{0.460676in}%
\pgfsys@useobject{currentmarker}{}%
\end{pgfscope}%
\begin{pgfscope}%
\pgfsys@transformshift{3.734116in}{0.460676in}%
\pgfsys@useobject{currentmarker}{}%
\end{pgfscope}%
\begin{pgfscope}%
\pgfsys@transformshift{3.786567in}{0.460676in}%
\pgfsys@useobject{currentmarker}{}%
\end{pgfscope}%
\begin{pgfscope}%
\pgfsys@transformshift{3.839019in}{0.460676in}%
\pgfsys@useobject{currentmarker}{}%
\end{pgfscope}%
\begin{pgfscope}%
\pgfsys@transformshift{3.891470in}{0.460676in}%
\pgfsys@useobject{currentmarker}{}%
\end{pgfscope}%
\begin{pgfscope}%
\pgfsys@transformshift{3.943921in}{0.460676in}%
\pgfsys@useobject{currentmarker}{}%
\end{pgfscope}%
\begin{pgfscope}%
\pgfsys@transformshift{3.996373in}{0.460676in}%
\pgfsys@useobject{currentmarker}{}%
\end{pgfscope}%
\begin{pgfscope}%
\pgfsys@transformshift{4.048824in}{0.460676in}%
\pgfsys@useobject{currentmarker}{}%
\end{pgfscope}%
\begin{pgfscope}%
\pgfsys@transformshift{4.101275in}{0.460676in}%
\pgfsys@useobject{currentmarker}{}%
\end{pgfscope}%
\begin{pgfscope}%
\pgfsys@transformshift{4.153726in}{0.460676in}%
\pgfsys@useobject{currentmarker}{}%
\end{pgfscope}%
\begin{pgfscope}%
\pgfsys@transformshift{4.206178in}{0.460676in}%
\pgfsys@useobject{currentmarker}{}%
\end{pgfscope}%
\begin{pgfscope}%
\pgfsys@transformshift{4.258629in}{0.460676in}%
\pgfsys@useobject{currentmarker}{}%
\end{pgfscope}%
\begin{pgfscope}%
\pgfsys@transformshift{4.311080in}{0.460676in}%
\pgfsys@useobject{currentmarker}{}%
\end{pgfscope}%
\begin{pgfscope}%
\pgfsys@transformshift{4.363532in}{0.460676in}%
\pgfsys@useobject{currentmarker}{}%
\end{pgfscope}%
\begin{pgfscope}%
\pgfsys@transformshift{4.415983in}{0.460676in}%
\pgfsys@useobject{currentmarker}{}%
\end{pgfscope}%
\begin{pgfscope}%
\pgfsys@transformshift{4.468434in}{0.460676in}%
\pgfsys@useobject{currentmarker}{}%
\end{pgfscope}%
\begin{pgfscope}%
\pgfsys@transformshift{4.520885in}{0.460676in}%
\pgfsys@useobject{currentmarker}{}%
\end{pgfscope}%
\begin{pgfscope}%
\pgfsys@transformshift{4.573337in}{0.460676in}%
\pgfsys@useobject{currentmarker}{}%
\end{pgfscope}%
\begin{pgfscope}%
\pgfsys@transformshift{4.625788in}{0.460676in}%
\pgfsys@useobject{currentmarker}{}%
\end{pgfscope}%
\begin{pgfscope}%
\pgfsys@transformshift{4.678239in}{0.460676in}%
\pgfsys@useobject{currentmarker}{}%
\end{pgfscope}%
\begin{pgfscope}%
\pgfsys@transformshift{4.730691in}{0.460676in}%
\pgfsys@useobject{currentmarker}{}%
\end{pgfscope}%
\begin{pgfscope}%
\pgfsys@transformshift{4.783142in}{0.460676in}%
\pgfsys@useobject{currentmarker}{}%
\end{pgfscope}%
\begin{pgfscope}%
\pgfsys@transformshift{4.835593in}{0.460676in}%
\pgfsys@useobject{currentmarker}{}%
\end{pgfscope}%
\begin{pgfscope}%
\pgfsys@transformshift{4.888045in}{0.460676in}%
\pgfsys@useobject{currentmarker}{}%
\end{pgfscope}%
\begin{pgfscope}%
\pgfsys@transformshift{4.940496in}{0.460676in}%
\pgfsys@useobject{currentmarker}{}%
\end{pgfscope}%
\begin{pgfscope}%
\pgfsys@transformshift{4.992947in}{0.460676in}%
\pgfsys@useobject{currentmarker}{}%
\end{pgfscope}%
\begin{pgfscope}%
\pgfsys@transformshift{5.045398in}{0.460676in}%
\pgfsys@useobject{currentmarker}{}%
\end{pgfscope}%
\begin{pgfscope}%
\pgfsys@transformshift{5.097850in}{0.460676in}%
\pgfsys@useobject{currentmarker}{}%
\end{pgfscope}%
\end{pgfscope}%
\begin{pgfscope}%
\pgfpathrectangle{\pgfqpoint{0.581793in}{0.460676in}}{\pgfqpoint{4.731107in}{2.584248in}}%
\pgfusepath{clip}%
\pgfsetbuttcap%
\pgfsetroundjoin%
\definecolor{currentfill}{rgb}{1.000000,0.498039,0.054902}%
\pgfsetfillcolor{currentfill}%
\pgfsetfillopacity{0.750000}%
\pgfsetlinewidth{1.003750pt}%
\definecolor{currentstroke}{rgb}{1.000000,0.498039,0.054902}%
\pgfsetstrokecolor{currentstroke}%
\pgfsetstrokeopacity{0.750000}%
\pgfsetdash{}{0pt}%
\pgfsys@defobject{currentmarker}{\pgfqpoint{-0.019018in}{-0.019018in}}{\pgfqpoint{0.019018in}{0.019018in}}{%
\pgfpathmoveto{\pgfqpoint{-0.000000in}{-0.019018in}}%
\pgfpathlineto{\pgfqpoint{0.019018in}{0.019018in}}%
\pgfpathlineto{\pgfqpoint{-0.019018in}{0.019018in}}%
\pgfpathclose%
\pgfusepath{stroke,fill}%
}%
\begin{pgfscope}%
\pgfsys@transformshift{0.796844in}{2.899987in}%
\pgfsys@useobject{currentmarker}{}%
\end{pgfscope}%
\begin{pgfscope}%
\pgfsys@transformshift{0.849295in}{2.899987in}%
\pgfsys@useobject{currentmarker}{}%
\end{pgfscope}%
\begin{pgfscope}%
\pgfsys@transformshift{0.901746in}{2.899987in}%
\pgfsys@useobject{currentmarker}{}%
\end{pgfscope}%
\begin{pgfscope}%
\pgfsys@transformshift{0.954198in}{2.899987in}%
\pgfsys@useobject{currentmarker}{}%
\end{pgfscope}%
\begin{pgfscope}%
\pgfsys@transformshift{1.006649in}{2.899987in}%
\pgfsys@useobject{currentmarker}{}%
\end{pgfscope}%
\begin{pgfscope}%
\pgfsys@transformshift{1.059100in}{2.899987in}%
\pgfsys@useobject{currentmarker}{}%
\end{pgfscope}%
\begin{pgfscope}%
\pgfsys@transformshift{1.111551in}{2.899987in}%
\pgfsys@useobject{currentmarker}{}%
\end{pgfscope}%
\begin{pgfscope}%
\pgfsys@transformshift{1.164003in}{2.899987in}%
\pgfsys@useobject{currentmarker}{}%
\end{pgfscope}%
\begin{pgfscope}%
\pgfsys@transformshift{1.216454in}{2.409060in}%
\pgfsys@useobject{currentmarker}{}%
\end{pgfscope}%
\begin{pgfscope}%
\pgfsys@transformshift{1.268905in}{2.265988in}%
\pgfsys@useobject{currentmarker}{}%
\end{pgfscope}%
\begin{pgfscope}%
\pgfsys@transformshift{1.321357in}{1.828832in}%
\pgfsys@useobject{currentmarker}{}%
\end{pgfscope}%
\begin{pgfscope}%
\pgfsys@transformshift{1.373808in}{1.422382in}%
\pgfsys@useobject{currentmarker}{}%
\end{pgfscope}%
\begin{pgfscope}%
\pgfsys@transformshift{1.426259in}{1.034343in}%
\pgfsys@useobject{currentmarker}{}%
\end{pgfscope}%
\begin{pgfscope}%
\pgfsys@transformshift{1.478710in}{1.003743in}%
\pgfsys@useobject{currentmarker}{}%
\end{pgfscope}%
\begin{pgfscope}%
\pgfsys@transformshift{1.531162in}{0.974485in}%
\pgfsys@useobject{currentmarker}{}%
\end{pgfscope}%
\begin{pgfscope}%
\pgfsys@transformshift{1.583613in}{0.969796in}%
\pgfsys@useobject{currentmarker}{}%
\end{pgfscope}%
\begin{pgfscope}%
\pgfsys@transformshift{1.636064in}{0.969785in}%
\pgfsys@useobject{currentmarker}{}%
\end{pgfscope}%
\begin{pgfscope}%
\pgfsys@transformshift{1.688516in}{0.939964in}%
\pgfsys@useobject{currentmarker}{}%
\end{pgfscope}%
\begin{pgfscope}%
\pgfsys@transformshift{1.740967in}{0.928718in}%
\pgfsys@useobject{currentmarker}{}%
\end{pgfscope}%
\begin{pgfscope}%
\pgfsys@transformshift{1.793418in}{0.895300in}%
\pgfsys@useobject{currentmarker}{}%
\end{pgfscope}%
\begin{pgfscope}%
\pgfsys@transformshift{1.845870in}{0.891596in}%
\pgfsys@useobject{currentmarker}{}%
\end{pgfscope}%
\begin{pgfscope}%
\pgfsys@transformshift{1.898321in}{0.879957in}%
\pgfsys@useobject{currentmarker}{}%
\end{pgfscope}%
\begin{pgfscope}%
\pgfsys@transformshift{1.950772in}{0.862786in}%
\pgfsys@useobject{currentmarker}{}%
\end{pgfscope}%
\begin{pgfscope}%
\pgfsys@transformshift{2.003223in}{0.821325in}%
\pgfsys@useobject{currentmarker}{}%
\end{pgfscope}%
\begin{pgfscope}%
\pgfsys@transformshift{2.055675in}{0.818911in}%
\pgfsys@useobject{currentmarker}{}%
\end{pgfscope}%
\begin{pgfscope}%
\pgfsys@transformshift{2.108126in}{0.813449in}%
\pgfsys@useobject{currentmarker}{}%
\end{pgfscope}%
\begin{pgfscope}%
\pgfsys@transformshift{2.160577in}{0.811986in}%
\pgfsys@useobject{currentmarker}{}%
\end{pgfscope}%
\begin{pgfscope}%
\pgfsys@transformshift{2.213029in}{0.811520in}%
\pgfsys@useobject{currentmarker}{}%
\end{pgfscope}%
\begin{pgfscope}%
\pgfsys@transformshift{2.265480in}{0.808836in}%
\pgfsys@useobject{currentmarker}{}%
\end{pgfscope}%
\begin{pgfscope}%
\pgfsys@transformshift{2.317931in}{0.808735in}%
\pgfsys@useobject{currentmarker}{}%
\end{pgfscope}%
\begin{pgfscope}%
\pgfsys@transformshift{2.370382in}{0.808552in}%
\pgfsys@useobject{currentmarker}{}%
\end{pgfscope}%
\begin{pgfscope}%
\pgfsys@transformshift{2.422834in}{0.807506in}%
\pgfsys@useobject{currentmarker}{}%
\end{pgfscope}%
\begin{pgfscope}%
\pgfsys@transformshift{2.475285in}{0.806781in}%
\pgfsys@useobject{currentmarker}{}%
\end{pgfscope}%
\begin{pgfscope}%
\pgfsys@transformshift{2.527736in}{0.805873in}%
\pgfsys@useobject{currentmarker}{}%
\end{pgfscope}%
\begin{pgfscope}%
\pgfsys@transformshift{2.580188in}{0.805859in}%
\pgfsys@useobject{currentmarker}{}%
\end{pgfscope}%
\begin{pgfscope}%
\pgfsys@transformshift{2.632639in}{0.804694in}%
\pgfsys@useobject{currentmarker}{}%
\end{pgfscope}%
\begin{pgfscope}%
\pgfsys@transformshift{2.685090in}{0.803244in}%
\pgfsys@useobject{currentmarker}{}%
\end{pgfscope}%
\begin{pgfscope}%
\pgfsys@transformshift{2.737541in}{0.801869in}%
\pgfsys@useobject{currentmarker}{}%
\end{pgfscope}%
\begin{pgfscope}%
\pgfsys@transformshift{2.789993in}{0.800452in}%
\pgfsys@useobject{currentmarker}{}%
\end{pgfscope}%
\begin{pgfscope}%
\pgfsys@transformshift{2.842444in}{0.798581in}%
\pgfsys@useobject{currentmarker}{}%
\end{pgfscope}%
\begin{pgfscope}%
\pgfsys@transformshift{2.894895in}{0.797657in}%
\pgfsys@useobject{currentmarker}{}%
\end{pgfscope}%
\begin{pgfscope}%
\pgfsys@transformshift{2.947347in}{0.796337in}%
\pgfsys@useobject{currentmarker}{}%
\end{pgfscope}%
\begin{pgfscope}%
\pgfsys@transformshift{2.999798in}{0.796049in}%
\pgfsys@useobject{currentmarker}{}%
\end{pgfscope}%
\begin{pgfscope}%
\pgfsys@transformshift{3.052249in}{0.794716in}%
\pgfsys@useobject{currentmarker}{}%
\end{pgfscope}%
\begin{pgfscope}%
\pgfsys@transformshift{3.104701in}{0.794050in}%
\pgfsys@useobject{currentmarker}{}%
\end{pgfscope}%
\begin{pgfscope}%
\pgfsys@transformshift{3.157152in}{0.789818in}%
\pgfsys@useobject{currentmarker}{}%
\end{pgfscope}%
\begin{pgfscope}%
\pgfsys@transformshift{3.209603in}{0.788044in}%
\pgfsys@useobject{currentmarker}{}%
\end{pgfscope}%
\begin{pgfscope}%
\pgfsys@transformshift{3.262054in}{0.786429in}%
\pgfsys@useobject{currentmarker}{}%
\end{pgfscope}%
\begin{pgfscope}%
\pgfsys@transformshift{3.314506in}{0.780324in}%
\pgfsys@useobject{currentmarker}{}%
\end{pgfscope}%
\begin{pgfscope}%
\pgfsys@transformshift{3.366957in}{0.776312in}%
\pgfsys@useobject{currentmarker}{}%
\end{pgfscope}%
\begin{pgfscope}%
\pgfsys@transformshift{3.419408in}{0.768191in}%
\pgfsys@useobject{currentmarker}{}%
\end{pgfscope}%
\begin{pgfscope}%
\pgfsys@transformshift{3.471860in}{0.763718in}%
\pgfsys@useobject{currentmarker}{}%
\end{pgfscope}%
\begin{pgfscope}%
\pgfsys@transformshift{3.524311in}{0.760533in}%
\pgfsys@useobject{currentmarker}{}%
\end{pgfscope}%
\begin{pgfscope}%
\pgfsys@transformshift{3.576762in}{0.760156in}%
\pgfsys@useobject{currentmarker}{}%
\end{pgfscope}%
\begin{pgfscope}%
\pgfsys@transformshift{3.629213in}{0.759617in}%
\pgfsys@useobject{currentmarker}{}%
\end{pgfscope}%
\begin{pgfscope}%
\pgfsys@transformshift{3.681665in}{0.758681in}%
\pgfsys@useobject{currentmarker}{}%
\end{pgfscope}%
\begin{pgfscope}%
\pgfsys@transformshift{3.734116in}{0.755990in}%
\pgfsys@useobject{currentmarker}{}%
\end{pgfscope}%
\begin{pgfscope}%
\pgfsys@transformshift{3.786567in}{0.755105in}%
\pgfsys@useobject{currentmarker}{}%
\end{pgfscope}%
\begin{pgfscope}%
\pgfsys@transformshift{3.839019in}{0.752674in}%
\pgfsys@useobject{currentmarker}{}%
\end{pgfscope}%
\begin{pgfscope}%
\pgfsys@transformshift{3.891470in}{0.752618in}%
\pgfsys@useobject{currentmarker}{}%
\end{pgfscope}%
\begin{pgfscope}%
\pgfsys@transformshift{3.943921in}{0.749755in}%
\pgfsys@useobject{currentmarker}{}%
\end{pgfscope}%
\begin{pgfscope}%
\pgfsys@transformshift{3.996373in}{0.749164in}%
\pgfsys@useobject{currentmarker}{}%
\end{pgfscope}%
\begin{pgfscope}%
\pgfsys@transformshift{4.048824in}{0.748842in}%
\pgfsys@useobject{currentmarker}{}%
\end{pgfscope}%
\begin{pgfscope}%
\pgfsys@transformshift{4.101275in}{0.748081in}%
\pgfsys@useobject{currentmarker}{}%
\end{pgfscope}%
\begin{pgfscope}%
\pgfsys@transformshift{4.153726in}{0.746504in}%
\pgfsys@useobject{currentmarker}{}%
\end{pgfscope}%
\begin{pgfscope}%
\pgfsys@transformshift{4.206178in}{0.743043in}%
\pgfsys@useobject{currentmarker}{}%
\end{pgfscope}%
\begin{pgfscope}%
\pgfsys@transformshift{4.258629in}{0.742650in}%
\pgfsys@useobject{currentmarker}{}%
\end{pgfscope}%
\begin{pgfscope}%
\pgfsys@transformshift{4.311080in}{0.736745in}%
\pgfsys@useobject{currentmarker}{}%
\end{pgfscope}%
\begin{pgfscope}%
\pgfsys@transformshift{4.363532in}{0.736630in}%
\pgfsys@useobject{currentmarker}{}%
\end{pgfscope}%
\begin{pgfscope}%
\pgfsys@transformshift{4.415983in}{0.736149in}%
\pgfsys@useobject{currentmarker}{}%
\end{pgfscope}%
\begin{pgfscope}%
\pgfsys@transformshift{4.468434in}{0.733165in}%
\pgfsys@useobject{currentmarker}{}%
\end{pgfscope}%
\begin{pgfscope}%
\pgfsys@transformshift{4.520885in}{0.732483in}%
\pgfsys@useobject{currentmarker}{}%
\end{pgfscope}%
\begin{pgfscope}%
\pgfsys@transformshift{4.573337in}{0.723812in}%
\pgfsys@useobject{currentmarker}{}%
\end{pgfscope}%
\begin{pgfscope}%
\pgfsys@transformshift{4.625788in}{0.723025in}%
\pgfsys@useobject{currentmarker}{}%
\end{pgfscope}%
\begin{pgfscope}%
\pgfsys@transformshift{4.678239in}{0.719913in}%
\pgfsys@useobject{currentmarker}{}%
\end{pgfscope}%
\begin{pgfscope}%
\pgfsys@transformshift{4.730691in}{0.714417in}%
\pgfsys@useobject{currentmarker}{}%
\end{pgfscope}%
\begin{pgfscope}%
\pgfsys@transformshift{4.783142in}{0.699631in}%
\pgfsys@useobject{currentmarker}{}%
\end{pgfscope}%
\begin{pgfscope}%
\pgfsys@transformshift{4.835593in}{0.667114in}%
\pgfsys@useobject{currentmarker}{}%
\end{pgfscope}%
\begin{pgfscope}%
\pgfsys@transformshift{4.888045in}{0.634829in}%
\pgfsys@useobject{currentmarker}{}%
\end{pgfscope}%
\begin{pgfscope}%
\pgfsys@transformshift{4.940496in}{0.596360in}%
\pgfsys@useobject{currentmarker}{}%
\end{pgfscope}%
\begin{pgfscope}%
\pgfsys@transformshift{4.992947in}{0.493031in}%
\pgfsys@useobject{currentmarker}{}%
\end{pgfscope}%
\begin{pgfscope}%
\pgfsys@transformshift{5.045398in}{0.460676in}%
\pgfsys@useobject{currentmarker}{}%
\end{pgfscope}%
\begin{pgfscope}%
\pgfsys@transformshift{5.097850in}{0.460676in}%
\pgfsys@useobject{currentmarker}{}%
\end{pgfscope}%
\end{pgfscope}%
\begin{pgfscope}%
\pgfpathrectangle{\pgfqpoint{0.581793in}{0.460676in}}{\pgfqpoint{4.731107in}{2.584248in}}%
\pgfusepath{clip}%
\pgfsetbuttcap%
\pgfsetroundjoin%
\definecolor{currentfill}{rgb}{0.172549,0.627451,0.172549}%
\pgfsetfillcolor{currentfill}%
\pgfsetfillopacity{0.750000}%
\pgfsetlinewidth{1.003750pt}%
\definecolor{currentstroke}{rgb}{0.172549,0.627451,0.172549}%
\pgfsetstrokecolor{currentstroke}%
\pgfsetstrokeopacity{0.750000}%
\pgfsetdash{}{0pt}%
\pgfsys@defobject{currentmarker}{\pgfqpoint{-0.019018in}{-0.019018in}}{\pgfqpoint{0.019018in}{0.019018in}}{%
\pgfpathmoveto{\pgfqpoint{-0.019018in}{-0.019018in}}%
\pgfpathlineto{\pgfqpoint{0.019018in}{-0.019018in}}%
\pgfpathlineto{\pgfqpoint{0.019018in}{0.019018in}}%
\pgfpathlineto{\pgfqpoint{-0.019018in}{0.019018in}}%
\pgfpathclose%
\pgfusepath{stroke,fill}%
}%
\begin{pgfscope}%
\pgfsys@transformshift{0.796844in}{2.910950in}%
\pgfsys@useobject{currentmarker}{}%
\end{pgfscope}%
\begin{pgfscope}%
\pgfsys@transformshift{0.849295in}{2.910950in}%
\pgfsys@useobject{currentmarker}{}%
\end{pgfscope}%
\begin{pgfscope}%
\pgfsys@transformshift{0.901746in}{2.910950in}%
\pgfsys@useobject{currentmarker}{}%
\end{pgfscope}%
\begin{pgfscope}%
\pgfsys@transformshift{0.954198in}{2.910950in}%
\pgfsys@useobject{currentmarker}{}%
\end{pgfscope}%
\begin{pgfscope}%
\pgfsys@transformshift{1.006649in}{2.910950in}%
\pgfsys@useobject{currentmarker}{}%
\end{pgfscope}%
\begin{pgfscope}%
\pgfsys@transformshift{1.059100in}{2.910950in}%
\pgfsys@useobject{currentmarker}{}%
\end{pgfscope}%
\begin{pgfscope}%
\pgfsys@transformshift{1.111551in}{2.910950in}%
\pgfsys@useobject{currentmarker}{}%
\end{pgfscope}%
\begin{pgfscope}%
\pgfsys@transformshift{1.164003in}{2.910950in}%
\pgfsys@useobject{currentmarker}{}%
\end{pgfscope}%
\begin{pgfscope}%
\pgfsys@transformshift{1.216454in}{2.407897in}%
\pgfsys@useobject{currentmarker}{}%
\end{pgfscope}%
\begin{pgfscope}%
\pgfsys@transformshift{1.268905in}{1.559379in}%
\pgfsys@useobject{currentmarker}{}%
\end{pgfscope}%
\begin{pgfscope}%
\pgfsys@transformshift{1.321357in}{1.386195in}%
\pgfsys@useobject{currentmarker}{}%
\end{pgfscope}%
\begin{pgfscope}%
\pgfsys@transformshift{1.373808in}{1.175051in}%
\pgfsys@useobject{currentmarker}{}%
\end{pgfscope}%
\begin{pgfscope}%
\pgfsys@transformshift{1.426259in}{0.993347in}%
\pgfsys@useobject{currentmarker}{}%
\end{pgfscope}%
\begin{pgfscope}%
\pgfsys@transformshift{1.478710in}{0.944483in}%
\pgfsys@useobject{currentmarker}{}%
\end{pgfscope}%
\begin{pgfscope}%
\pgfsys@transformshift{1.531162in}{0.901969in}%
\pgfsys@useobject{currentmarker}{}%
\end{pgfscope}%
\begin{pgfscope}%
\pgfsys@transformshift{1.583613in}{0.897787in}%
\pgfsys@useobject{currentmarker}{}%
\end{pgfscope}%
\begin{pgfscope}%
\pgfsys@transformshift{1.636064in}{0.890116in}%
\pgfsys@useobject{currentmarker}{}%
\end{pgfscope}%
\begin{pgfscope}%
\pgfsys@transformshift{1.688516in}{0.883010in}%
\pgfsys@useobject{currentmarker}{}%
\end{pgfscope}%
\begin{pgfscope}%
\pgfsys@transformshift{1.740967in}{0.881580in}%
\pgfsys@useobject{currentmarker}{}%
\end{pgfscope}%
\begin{pgfscope}%
\pgfsys@transformshift{1.793418in}{0.870388in}%
\pgfsys@useobject{currentmarker}{}%
\end{pgfscope}%
\begin{pgfscope}%
\pgfsys@transformshift{1.845870in}{0.867441in}%
\pgfsys@useobject{currentmarker}{}%
\end{pgfscope}%
\begin{pgfscope}%
\pgfsys@transformshift{1.898321in}{0.855452in}%
\pgfsys@useobject{currentmarker}{}%
\end{pgfscope}%
\begin{pgfscope}%
\pgfsys@transformshift{1.950772in}{0.851023in}%
\pgfsys@useobject{currentmarker}{}%
\end{pgfscope}%
\begin{pgfscope}%
\pgfsys@transformshift{2.003223in}{0.845839in}%
\pgfsys@useobject{currentmarker}{}%
\end{pgfscope}%
\begin{pgfscope}%
\pgfsys@transformshift{2.055675in}{0.844696in}%
\pgfsys@useobject{currentmarker}{}%
\end{pgfscope}%
\begin{pgfscope}%
\pgfsys@transformshift{2.108126in}{0.843602in}%
\pgfsys@useobject{currentmarker}{}%
\end{pgfscope}%
\begin{pgfscope}%
\pgfsys@transformshift{2.160577in}{0.827334in}%
\pgfsys@useobject{currentmarker}{}%
\end{pgfscope}%
\begin{pgfscope}%
\pgfsys@transformshift{2.213029in}{0.823382in}%
\pgfsys@useobject{currentmarker}{}%
\end{pgfscope}%
\begin{pgfscope}%
\pgfsys@transformshift{2.265480in}{0.823192in}%
\pgfsys@useobject{currentmarker}{}%
\end{pgfscope}%
\begin{pgfscope}%
\pgfsys@transformshift{2.317931in}{0.803553in}%
\pgfsys@useobject{currentmarker}{}%
\end{pgfscope}%
\begin{pgfscope}%
\pgfsys@transformshift{2.370382in}{0.802804in}%
\pgfsys@useobject{currentmarker}{}%
\end{pgfscope}%
\begin{pgfscope}%
\pgfsys@transformshift{2.422834in}{0.785097in}%
\pgfsys@useobject{currentmarker}{}%
\end{pgfscope}%
\begin{pgfscope}%
\pgfsys@transformshift{2.475285in}{0.781576in}%
\pgfsys@useobject{currentmarker}{}%
\end{pgfscope}%
\begin{pgfscope}%
\pgfsys@transformshift{2.527736in}{0.775596in}%
\pgfsys@useobject{currentmarker}{}%
\end{pgfscope}%
\begin{pgfscope}%
\pgfsys@transformshift{2.580188in}{0.774886in}%
\pgfsys@useobject{currentmarker}{}%
\end{pgfscope}%
\begin{pgfscope}%
\pgfsys@transformshift{2.632639in}{0.765943in}%
\pgfsys@useobject{currentmarker}{}%
\end{pgfscope}%
\begin{pgfscope}%
\pgfsys@transformshift{2.685090in}{0.764910in}%
\pgfsys@useobject{currentmarker}{}%
\end{pgfscope}%
\begin{pgfscope}%
\pgfsys@transformshift{2.737541in}{0.758001in}%
\pgfsys@useobject{currentmarker}{}%
\end{pgfscope}%
\begin{pgfscope}%
\pgfsys@transformshift{2.789993in}{0.757356in}%
\pgfsys@useobject{currentmarker}{}%
\end{pgfscope}%
\begin{pgfscope}%
\pgfsys@transformshift{2.842444in}{0.756421in}%
\pgfsys@useobject{currentmarker}{}%
\end{pgfscope}%
\begin{pgfscope}%
\pgfsys@transformshift{2.894895in}{0.752649in}%
\pgfsys@useobject{currentmarker}{}%
\end{pgfscope}%
\begin{pgfscope}%
\pgfsys@transformshift{2.947347in}{0.750603in}%
\pgfsys@useobject{currentmarker}{}%
\end{pgfscope}%
\begin{pgfscope}%
\pgfsys@transformshift{2.999798in}{0.746367in}%
\pgfsys@useobject{currentmarker}{}%
\end{pgfscope}%
\begin{pgfscope}%
\pgfsys@transformshift{3.052249in}{0.743217in}%
\pgfsys@useobject{currentmarker}{}%
\end{pgfscope}%
\begin{pgfscope}%
\pgfsys@transformshift{3.104701in}{0.742082in}%
\pgfsys@useobject{currentmarker}{}%
\end{pgfscope}%
\begin{pgfscope}%
\pgfsys@transformshift{3.157152in}{0.739765in}%
\pgfsys@useobject{currentmarker}{}%
\end{pgfscope}%
\begin{pgfscope}%
\pgfsys@transformshift{3.209603in}{0.739379in}%
\pgfsys@useobject{currentmarker}{}%
\end{pgfscope}%
\begin{pgfscope}%
\pgfsys@transformshift{3.262054in}{0.738927in}%
\pgfsys@useobject{currentmarker}{}%
\end{pgfscope}%
\begin{pgfscope}%
\pgfsys@transformshift{3.314506in}{0.736474in}%
\pgfsys@useobject{currentmarker}{}%
\end{pgfscope}%
\begin{pgfscope}%
\pgfsys@transformshift{3.366957in}{0.735047in}%
\pgfsys@useobject{currentmarker}{}%
\end{pgfscope}%
\begin{pgfscope}%
\pgfsys@transformshift{3.419408in}{0.731423in}%
\pgfsys@useobject{currentmarker}{}%
\end{pgfscope}%
\begin{pgfscope}%
\pgfsys@transformshift{3.471860in}{0.731386in}%
\pgfsys@useobject{currentmarker}{}%
\end{pgfscope}%
\begin{pgfscope}%
\pgfsys@transformshift{3.524311in}{0.717451in}%
\pgfsys@useobject{currentmarker}{}%
\end{pgfscope}%
\begin{pgfscope}%
\pgfsys@transformshift{3.576762in}{0.712029in}%
\pgfsys@useobject{currentmarker}{}%
\end{pgfscope}%
\begin{pgfscope}%
\pgfsys@transformshift{3.629213in}{0.711020in}%
\pgfsys@useobject{currentmarker}{}%
\end{pgfscope}%
\begin{pgfscope}%
\pgfsys@transformshift{3.681665in}{0.709682in}%
\pgfsys@useobject{currentmarker}{}%
\end{pgfscope}%
\begin{pgfscope}%
\pgfsys@transformshift{3.734116in}{0.707801in}%
\pgfsys@useobject{currentmarker}{}%
\end{pgfscope}%
\begin{pgfscope}%
\pgfsys@transformshift{3.786567in}{0.707540in}%
\pgfsys@useobject{currentmarker}{}%
\end{pgfscope}%
\begin{pgfscope}%
\pgfsys@transformshift{3.839019in}{0.705997in}%
\pgfsys@useobject{currentmarker}{}%
\end{pgfscope}%
\begin{pgfscope}%
\pgfsys@transformshift{3.891470in}{0.697859in}%
\pgfsys@useobject{currentmarker}{}%
\end{pgfscope}%
\begin{pgfscope}%
\pgfsys@transformshift{3.943921in}{0.695662in}%
\pgfsys@useobject{currentmarker}{}%
\end{pgfscope}%
\begin{pgfscope}%
\pgfsys@transformshift{3.996373in}{0.693031in}%
\pgfsys@useobject{currentmarker}{}%
\end{pgfscope}%
\begin{pgfscope}%
\pgfsys@transformshift{4.048824in}{0.692940in}%
\pgfsys@useobject{currentmarker}{}%
\end{pgfscope}%
\begin{pgfscope}%
\pgfsys@transformshift{4.101275in}{0.689535in}%
\pgfsys@useobject{currentmarker}{}%
\end{pgfscope}%
\begin{pgfscope}%
\pgfsys@transformshift{4.153726in}{0.683826in}%
\pgfsys@useobject{currentmarker}{}%
\end{pgfscope}%
\begin{pgfscope}%
\pgfsys@transformshift{4.206178in}{0.681729in}%
\pgfsys@useobject{currentmarker}{}%
\end{pgfscope}%
\begin{pgfscope}%
\pgfsys@transformshift{4.258629in}{0.681454in}%
\pgfsys@useobject{currentmarker}{}%
\end{pgfscope}%
\begin{pgfscope}%
\pgfsys@transformshift{4.311080in}{0.677740in}%
\pgfsys@useobject{currentmarker}{}%
\end{pgfscope}%
\begin{pgfscope}%
\pgfsys@transformshift{4.363532in}{0.677531in}%
\pgfsys@useobject{currentmarker}{}%
\end{pgfscope}%
\begin{pgfscope}%
\pgfsys@transformshift{4.415983in}{0.676503in}%
\pgfsys@useobject{currentmarker}{}%
\end{pgfscope}%
\begin{pgfscope}%
\pgfsys@transformshift{4.468434in}{0.675571in}%
\pgfsys@useobject{currentmarker}{}%
\end{pgfscope}%
\begin{pgfscope}%
\pgfsys@transformshift{4.520885in}{0.664676in}%
\pgfsys@useobject{currentmarker}{}%
\end{pgfscope}%
\begin{pgfscope}%
\pgfsys@transformshift{4.573337in}{0.660717in}%
\pgfsys@useobject{currentmarker}{}%
\end{pgfscope}%
\begin{pgfscope}%
\pgfsys@transformshift{4.625788in}{0.650931in}%
\pgfsys@useobject{currentmarker}{}%
\end{pgfscope}%
\begin{pgfscope}%
\pgfsys@transformshift{4.678239in}{0.649855in}%
\pgfsys@useobject{currentmarker}{}%
\end{pgfscope}%
\begin{pgfscope}%
\pgfsys@transformshift{4.730691in}{0.629452in}%
\pgfsys@useobject{currentmarker}{}%
\end{pgfscope}%
\begin{pgfscope}%
\pgfsys@transformshift{4.783142in}{0.619306in}%
\pgfsys@useobject{currentmarker}{}%
\end{pgfscope}%
\begin{pgfscope}%
\pgfsys@transformshift{4.835593in}{0.605199in}%
\pgfsys@useobject{currentmarker}{}%
\end{pgfscope}%
\begin{pgfscope}%
\pgfsys@transformshift{4.888045in}{0.460676in}%
\pgfsys@useobject{currentmarker}{}%
\end{pgfscope}%
\begin{pgfscope}%
\pgfsys@transformshift{4.940496in}{0.460676in}%
\pgfsys@useobject{currentmarker}{}%
\end{pgfscope}%
\begin{pgfscope}%
\pgfsys@transformshift{4.992947in}{0.460676in}%
\pgfsys@useobject{currentmarker}{}%
\end{pgfscope}%
\begin{pgfscope}%
\pgfsys@transformshift{5.045398in}{0.460676in}%
\pgfsys@useobject{currentmarker}{}%
\end{pgfscope}%
\begin{pgfscope}%
\pgfsys@transformshift{5.097850in}{0.460676in}%
\pgfsys@useobject{currentmarker}{}%
\end{pgfscope}%
\end{pgfscope}%
\begin{pgfscope}%
\pgfpathrectangle{\pgfqpoint{0.581793in}{0.460676in}}{\pgfqpoint{4.731107in}{2.584248in}}%
\pgfusepath{clip}%
\pgfsetbuttcap%
\pgfsetroundjoin%
\definecolor{currentfill}{rgb}{0.839216,0.152941,0.156863}%
\pgfsetfillcolor{currentfill}%
\pgfsetfillopacity{0.750000}%
\pgfsetlinewidth{1.003750pt}%
\definecolor{currentstroke}{rgb}{0.839216,0.152941,0.156863}%
\pgfsetstrokecolor{currentstroke}%
\pgfsetstrokeopacity{0.750000}%
\pgfsetdash{}{0pt}%
\pgfsys@defobject{currentmarker}{\pgfqpoint{-0.019018in}{-0.019018in}}{\pgfqpoint{0.019018in}{0.019018in}}{%
\pgfpathmoveto{\pgfqpoint{0.000000in}{0.019018in}}%
\pgfpathlineto{\pgfqpoint{-0.019018in}{-0.019018in}}%
\pgfpathlineto{\pgfqpoint{0.019018in}{-0.019018in}}%
\pgfpathclose%
\pgfusepath{stroke,fill}%
}%
\begin{pgfscope}%
\pgfsys@transformshift{0.796844in}{2.921865in}%
\pgfsys@useobject{currentmarker}{}%
\end{pgfscope}%
\begin{pgfscope}%
\pgfsys@transformshift{0.849295in}{2.921865in}%
\pgfsys@useobject{currentmarker}{}%
\end{pgfscope}%
\begin{pgfscope}%
\pgfsys@transformshift{0.901746in}{2.921865in}%
\pgfsys@useobject{currentmarker}{}%
\end{pgfscope}%
\begin{pgfscope}%
\pgfsys@transformshift{0.954198in}{2.921865in}%
\pgfsys@useobject{currentmarker}{}%
\end{pgfscope}%
\begin{pgfscope}%
\pgfsys@transformshift{1.006649in}{2.921865in}%
\pgfsys@useobject{currentmarker}{}%
\end{pgfscope}%
\begin{pgfscope}%
\pgfsys@transformshift{1.059100in}{2.921865in}%
\pgfsys@useobject{currentmarker}{}%
\end{pgfscope}%
\begin{pgfscope}%
\pgfsys@transformshift{1.111551in}{2.921865in}%
\pgfsys@useobject{currentmarker}{}%
\end{pgfscope}%
\begin{pgfscope}%
\pgfsys@transformshift{1.164003in}{2.921865in}%
\pgfsys@useobject{currentmarker}{}%
\end{pgfscope}%
\begin{pgfscope}%
\pgfsys@transformshift{1.216454in}{2.921865in}%
\pgfsys@useobject{currentmarker}{}%
\end{pgfscope}%
\begin{pgfscope}%
\pgfsys@transformshift{1.268905in}{2.601231in}%
\pgfsys@useobject{currentmarker}{}%
\end{pgfscope}%
\begin{pgfscope}%
\pgfsys@transformshift{1.321357in}{2.469726in}%
\pgfsys@useobject{currentmarker}{}%
\end{pgfscope}%
\begin{pgfscope}%
\pgfsys@transformshift{1.373808in}{2.440827in}%
\pgfsys@useobject{currentmarker}{}%
\end{pgfscope}%
\begin{pgfscope}%
\pgfsys@transformshift{1.426259in}{2.409937in}%
\pgfsys@useobject{currentmarker}{}%
\end{pgfscope}%
\begin{pgfscope}%
\pgfsys@transformshift{1.478710in}{2.380055in}%
\pgfsys@useobject{currentmarker}{}%
\end{pgfscope}%
\begin{pgfscope}%
\pgfsys@transformshift{1.531162in}{2.212739in}%
\pgfsys@useobject{currentmarker}{}%
\end{pgfscope}%
\begin{pgfscope}%
\pgfsys@transformshift{1.583613in}{2.043175in}%
\pgfsys@useobject{currentmarker}{}%
\end{pgfscope}%
\begin{pgfscope}%
\pgfsys@transformshift{1.636064in}{1.939395in}%
\pgfsys@useobject{currentmarker}{}%
\end{pgfscope}%
\begin{pgfscope}%
\pgfsys@transformshift{1.688516in}{1.771897in}%
\pgfsys@useobject{currentmarker}{}%
\end{pgfscope}%
\begin{pgfscope}%
\pgfsys@transformshift{1.740967in}{1.659515in}%
\pgfsys@useobject{currentmarker}{}%
\end{pgfscope}%
\begin{pgfscope}%
\pgfsys@transformshift{1.793418in}{1.445330in}%
\pgfsys@useobject{currentmarker}{}%
\end{pgfscope}%
\begin{pgfscope}%
\pgfsys@transformshift{1.845870in}{1.436280in}%
\pgfsys@useobject{currentmarker}{}%
\end{pgfscope}%
\begin{pgfscope}%
\pgfsys@transformshift{1.898321in}{1.350469in}%
\pgfsys@useobject{currentmarker}{}%
\end{pgfscope}%
\begin{pgfscope}%
\pgfsys@transformshift{1.950772in}{1.233831in}%
\pgfsys@useobject{currentmarker}{}%
\end{pgfscope}%
\begin{pgfscope}%
\pgfsys@transformshift{2.003223in}{1.154769in}%
\pgfsys@useobject{currentmarker}{}%
\end{pgfscope}%
\begin{pgfscope}%
\pgfsys@transformshift{2.055675in}{1.153998in}%
\pgfsys@useobject{currentmarker}{}%
\end{pgfscope}%
\begin{pgfscope}%
\pgfsys@transformshift{2.108126in}{1.011125in}%
\pgfsys@useobject{currentmarker}{}%
\end{pgfscope}%
\begin{pgfscope}%
\pgfsys@transformshift{2.160577in}{1.009180in}%
\pgfsys@useobject{currentmarker}{}%
\end{pgfscope}%
\begin{pgfscope}%
\pgfsys@transformshift{2.213029in}{0.956512in}%
\pgfsys@useobject{currentmarker}{}%
\end{pgfscope}%
\begin{pgfscope}%
\pgfsys@transformshift{2.265480in}{0.945552in}%
\pgfsys@useobject{currentmarker}{}%
\end{pgfscope}%
\begin{pgfscope}%
\pgfsys@transformshift{2.317931in}{0.940007in}%
\pgfsys@useobject{currentmarker}{}%
\end{pgfscope}%
\begin{pgfscope}%
\pgfsys@transformshift{2.370382in}{0.932694in}%
\pgfsys@useobject{currentmarker}{}%
\end{pgfscope}%
\begin{pgfscope}%
\pgfsys@transformshift{2.422834in}{0.932620in}%
\pgfsys@useobject{currentmarker}{}%
\end{pgfscope}%
\begin{pgfscope}%
\pgfsys@transformshift{2.475285in}{0.919399in}%
\pgfsys@useobject{currentmarker}{}%
\end{pgfscope}%
\begin{pgfscope}%
\pgfsys@transformshift{2.527736in}{0.918531in}%
\pgfsys@useobject{currentmarker}{}%
\end{pgfscope}%
\begin{pgfscope}%
\pgfsys@transformshift{2.580188in}{0.907942in}%
\pgfsys@useobject{currentmarker}{}%
\end{pgfscope}%
\begin{pgfscope}%
\pgfsys@transformshift{2.632639in}{0.902711in}%
\pgfsys@useobject{currentmarker}{}%
\end{pgfscope}%
\begin{pgfscope}%
\pgfsys@transformshift{2.685090in}{0.901014in}%
\pgfsys@useobject{currentmarker}{}%
\end{pgfscope}%
\begin{pgfscope}%
\pgfsys@transformshift{2.737541in}{0.893081in}%
\pgfsys@useobject{currentmarker}{}%
\end{pgfscope}%
\begin{pgfscope}%
\pgfsys@transformshift{2.789993in}{0.891512in}%
\pgfsys@useobject{currentmarker}{}%
\end{pgfscope}%
\begin{pgfscope}%
\pgfsys@transformshift{2.842444in}{0.891139in}%
\pgfsys@useobject{currentmarker}{}%
\end{pgfscope}%
\begin{pgfscope}%
\pgfsys@transformshift{2.894895in}{0.889362in}%
\pgfsys@useobject{currentmarker}{}%
\end{pgfscope}%
\begin{pgfscope}%
\pgfsys@transformshift{2.947347in}{0.888067in}%
\pgfsys@useobject{currentmarker}{}%
\end{pgfscope}%
\begin{pgfscope}%
\pgfsys@transformshift{2.999798in}{0.884072in}%
\pgfsys@useobject{currentmarker}{}%
\end{pgfscope}%
\begin{pgfscope}%
\pgfsys@transformshift{3.052249in}{0.883065in}%
\pgfsys@useobject{currentmarker}{}%
\end{pgfscope}%
\begin{pgfscope}%
\pgfsys@transformshift{3.104701in}{0.881757in}%
\pgfsys@useobject{currentmarker}{}%
\end{pgfscope}%
\begin{pgfscope}%
\pgfsys@transformshift{3.157152in}{0.881115in}%
\pgfsys@useobject{currentmarker}{}%
\end{pgfscope}%
\begin{pgfscope}%
\pgfsys@transformshift{3.209603in}{0.880859in}%
\pgfsys@useobject{currentmarker}{}%
\end{pgfscope}%
\begin{pgfscope}%
\pgfsys@transformshift{3.262054in}{0.880595in}%
\pgfsys@useobject{currentmarker}{}%
\end{pgfscope}%
\begin{pgfscope}%
\pgfsys@transformshift{3.314506in}{0.879677in}%
\pgfsys@useobject{currentmarker}{}%
\end{pgfscope}%
\begin{pgfscope}%
\pgfsys@transformshift{3.366957in}{0.879218in}%
\pgfsys@useobject{currentmarker}{}%
\end{pgfscope}%
\begin{pgfscope}%
\pgfsys@transformshift{3.419408in}{0.879014in}%
\pgfsys@useobject{currentmarker}{}%
\end{pgfscope}%
\begin{pgfscope}%
\pgfsys@transformshift{3.471860in}{0.877769in}%
\pgfsys@useobject{currentmarker}{}%
\end{pgfscope}%
\begin{pgfscope}%
\pgfsys@transformshift{3.524311in}{0.877706in}%
\pgfsys@useobject{currentmarker}{}%
\end{pgfscope}%
\begin{pgfscope}%
\pgfsys@transformshift{3.576762in}{0.877661in}%
\pgfsys@useobject{currentmarker}{}%
\end{pgfscope}%
\begin{pgfscope}%
\pgfsys@transformshift{3.629213in}{0.875935in}%
\pgfsys@useobject{currentmarker}{}%
\end{pgfscope}%
\begin{pgfscope}%
\pgfsys@transformshift{3.681665in}{0.875452in}%
\pgfsys@useobject{currentmarker}{}%
\end{pgfscope}%
\begin{pgfscope}%
\pgfsys@transformshift{3.734116in}{0.875192in}%
\pgfsys@useobject{currentmarker}{}%
\end{pgfscope}%
\begin{pgfscope}%
\pgfsys@transformshift{3.786567in}{0.874649in}%
\pgfsys@useobject{currentmarker}{}%
\end{pgfscope}%
\begin{pgfscope}%
\pgfsys@transformshift{3.839019in}{0.873698in}%
\pgfsys@useobject{currentmarker}{}%
\end{pgfscope}%
\begin{pgfscope}%
\pgfsys@transformshift{3.891470in}{0.873178in}%
\pgfsys@useobject{currentmarker}{}%
\end{pgfscope}%
\begin{pgfscope}%
\pgfsys@transformshift{3.943921in}{0.873002in}%
\pgfsys@useobject{currentmarker}{}%
\end{pgfscope}%
\begin{pgfscope}%
\pgfsys@transformshift{3.996373in}{0.871232in}%
\pgfsys@useobject{currentmarker}{}%
\end{pgfscope}%
\begin{pgfscope}%
\pgfsys@transformshift{4.048824in}{0.871202in}%
\pgfsys@useobject{currentmarker}{}%
\end{pgfscope}%
\begin{pgfscope}%
\pgfsys@transformshift{4.101275in}{0.869044in}%
\pgfsys@useobject{currentmarker}{}%
\end{pgfscope}%
\begin{pgfscope}%
\pgfsys@transformshift{4.153726in}{0.869043in}%
\pgfsys@useobject{currentmarker}{}%
\end{pgfscope}%
\begin{pgfscope}%
\pgfsys@transformshift{4.206178in}{0.868629in}%
\pgfsys@useobject{currentmarker}{}%
\end{pgfscope}%
\begin{pgfscope}%
\pgfsys@transformshift{4.258629in}{0.864365in}%
\pgfsys@useobject{currentmarker}{}%
\end{pgfscope}%
\begin{pgfscope}%
\pgfsys@transformshift{4.311080in}{0.845846in}%
\pgfsys@useobject{currentmarker}{}%
\end{pgfscope}%
\begin{pgfscope}%
\pgfsys@transformshift{4.363532in}{0.781211in}%
\pgfsys@useobject{currentmarker}{}%
\end{pgfscope}%
\begin{pgfscope}%
\pgfsys@transformshift{4.415983in}{0.780353in}%
\pgfsys@useobject{currentmarker}{}%
\end{pgfscope}%
\begin{pgfscope}%
\pgfsys@transformshift{4.468434in}{0.747827in}%
\pgfsys@useobject{currentmarker}{}%
\end{pgfscope}%
\begin{pgfscope}%
\pgfsys@transformshift{4.520885in}{0.743082in}%
\pgfsys@useobject{currentmarker}{}%
\end{pgfscope}%
\begin{pgfscope}%
\pgfsys@transformshift{4.573337in}{0.734738in}%
\pgfsys@useobject{currentmarker}{}%
\end{pgfscope}%
\begin{pgfscope}%
\pgfsys@transformshift{4.625788in}{0.726470in}%
\pgfsys@useobject{currentmarker}{}%
\end{pgfscope}%
\begin{pgfscope}%
\pgfsys@transformshift{4.678239in}{0.723355in}%
\pgfsys@useobject{currentmarker}{}%
\end{pgfscope}%
\begin{pgfscope}%
\pgfsys@transformshift{4.730691in}{0.720101in}%
\pgfsys@useobject{currentmarker}{}%
\end{pgfscope}%
\begin{pgfscope}%
\pgfsys@transformshift{4.783142in}{0.717154in}%
\pgfsys@useobject{currentmarker}{}%
\end{pgfscope}%
\begin{pgfscope}%
\pgfsys@transformshift{4.835593in}{0.716636in}%
\pgfsys@useobject{currentmarker}{}%
\end{pgfscope}%
\begin{pgfscope}%
\pgfsys@transformshift{4.888045in}{0.708351in}%
\pgfsys@useobject{currentmarker}{}%
\end{pgfscope}%
\begin{pgfscope}%
\pgfsys@transformshift{4.940496in}{0.690096in}%
\pgfsys@useobject{currentmarker}{}%
\end{pgfscope}%
\begin{pgfscope}%
\pgfsys@transformshift{4.992947in}{0.589386in}%
\pgfsys@useobject{currentmarker}{}%
\end{pgfscope}%
\begin{pgfscope}%
\pgfsys@transformshift{5.045398in}{0.460676in}%
\pgfsys@useobject{currentmarker}{}%
\end{pgfscope}%
\begin{pgfscope}%
\pgfsys@transformshift{5.097850in}{0.460676in}%
\pgfsys@useobject{currentmarker}{}%
\end{pgfscope}%
\end{pgfscope}%
\begin{pgfscope}%
\pgfpathrectangle{\pgfqpoint{0.581793in}{0.460676in}}{\pgfqpoint{4.731107in}{2.584248in}}%
\pgfusepath{clip}%
\pgfsetrectcap%
\pgfsetroundjoin%
\pgfsetlinewidth{1.003750pt}%
\definecolor{currentstroke}{rgb}{0.000000,0.000000,0.000000}%
\pgfsetstrokecolor{currentstroke}%
\pgfsetdash{}{0pt}%
\pgfpathmoveto{\pgfqpoint{0.581793in}{2.775968in}}%
\pgfpathlineto{\pgfqpoint{5.312900in}{2.775968in}}%
\pgfusepath{stroke}%
\end{pgfscope}%
\begin{pgfscope}%
\pgfsetrectcap%
\pgfsetmiterjoin%
\pgfsetlinewidth{0.803000pt}%
\definecolor{currentstroke}{rgb}{0.000000,0.000000,0.000000}%
\pgfsetstrokecolor{currentstroke}%
\pgfsetdash{}{0pt}%
\pgfpathmoveto{\pgfqpoint{0.581793in}{0.460676in}}%
\pgfpathlineto{\pgfqpoint{0.581793in}{3.044924in}}%
\pgfusepath{stroke}%
\end{pgfscope}%
\begin{pgfscope}%
\pgfsetrectcap%
\pgfsetmiterjoin%
\pgfsetlinewidth{0.803000pt}%
\definecolor{currentstroke}{rgb}{0.000000,0.000000,0.000000}%
\pgfsetstrokecolor{currentstroke}%
\pgfsetdash{}{0pt}%
\pgfpathmoveto{\pgfqpoint{5.312900in}{0.460676in}}%
\pgfpathlineto{\pgfqpoint{5.312900in}{3.044924in}}%
\pgfusepath{stroke}%
\end{pgfscope}%
\begin{pgfscope}%
\pgfsetrectcap%
\pgfsetmiterjoin%
\pgfsetlinewidth{0.803000pt}%
\definecolor{currentstroke}{rgb}{0.000000,0.000000,0.000000}%
\pgfsetstrokecolor{currentstroke}%
\pgfsetdash{}{0pt}%
\pgfpathmoveto{\pgfqpoint{0.581793in}{0.460676in}}%
\pgfpathlineto{\pgfqpoint{5.312900in}{0.460676in}}%
\pgfusepath{stroke}%
\end{pgfscope}%
\begin{pgfscope}%
\pgfsetrectcap%
\pgfsetmiterjoin%
\pgfsetlinewidth{0.803000pt}%
\definecolor{currentstroke}{rgb}{0.000000,0.000000,0.000000}%
\pgfsetstrokecolor{currentstroke}%
\pgfsetdash{}{0pt}%
\pgfpathmoveto{\pgfqpoint{0.581793in}{3.044924in}}%
\pgfpathlineto{\pgfqpoint{5.312900in}{3.044924in}}%
\pgfusepath{stroke}%
\end{pgfscope}%
\begin{pgfscope}%
\pgfsetbuttcap%
\pgfsetroundjoin%
\definecolor{currentfill}{rgb}{0.000000,0.000000,0.000000}%
\pgfsetfillcolor{currentfill}%
\pgfsetlinewidth{0.803000pt}%
\definecolor{currentstroke}{rgb}{0.000000,0.000000,0.000000}%
\pgfsetstrokecolor{currentstroke}%
\pgfsetdash{}{0pt}%
\pgfsys@defobject{currentmarker}{\pgfqpoint{0.000000in}{-0.048611in}}{\pgfqpoint{0.000000in}{0.000000in}}{%
\pgfpathmoveto{\pgfqpoint{0.000000in}{0.000000in}}%
\pgfpathlineto{\pgfqpoint{0.000000in}{-0.048611in}}%
\pgfusepath{stroke,fill}%
}%
\begin{pgfscope}%
\pgfsys@transformshift{0.744392in}{0.460676in}%
\pgfsys@useobject{currentmarker}{}%
\end{pgfscope}%
\end{pgfscope}%
\begin{pgfscope}%
\definecolor{textcolor}{rgb}{0.000000,0.000000,0.000000}%
\pgfsetstrokecolor{textcolor}%
\pgfsetfillcolor{textcolor}%
\pgftext[x=0.744392in,y=0.363454in,,top]{\color{textcolor}\rmfamily\fontsize{8.330000}{9.996000}\selectfont \(\displaystyle 0\)}%
\end{pgfscope}%
\begin{pgfscope}%
\pgfsetbuttcap%
\pgfsetroundjoin%
\definecolor{currentfill}{rgb}{0.000000,0.000000,0.000000}%
\pgfsetfillcolor{currentfill}%
\pgfsetlinewidth{0.803000pt}%
\definecolor{currentstroke}{rgb}{0.000000,0.000000,0.000000}%
\pgfsetstrokecolor{currentstroke}%
\pgfsetdash{}{0pt}%
\pgfsys@defobject{currentmarker}{\pgfqpoint{0.000000in}{-0.048611in}}{\pgfqpoint{0.000000in}{0.000000in}}{%
\pgfpathmoveto{\pgfqpoint{0.000000in}{0.000000in}}%
\pgfpathlineto{\pgfqpoint{0.000000in}{-0.048611in}}%
\pgfusepath{stroke,fill}%
}%
\begin{pgfscope}%
\pgfsys@transformshift{1.793418in}{0.460676in}%
\pgfsys@useobject{currentmarker}{}%
\end{pgfscope}%
\end{pgfscope}%
\begin{pgfscope}%
\definecolor{textcolor}{rgb}{0.000000,0.000000,0.000000}%
\pgfsetstrokecolor{textcolor}%
\pgfsetfillcolor{textcolor}%
\pgftext[x=1.793418in,y=0.363454in,,top]{\color{textcolor}\rmfamily\fontsize{8.330000}{9.996000}\selectfont \(\displaystyle 20\)}%
\end{pgfscope}%
\begin{pgfscope}%
\pgfsetbuttcap%
\pgfsetroundjoin%
\definecolor{currentfill}{rgb}{0.000000,0.000000,0.000000}%
\pgfsetfillcolor{currentfill}%
\pgfsetlinewidth{0.803000pt}%
\definecolor{currentstroke}{rgb}{0.000000,0.000000,0.000000}%
\pgfsetstrokecolor{currentstroke}%
\pgfsetdash{}{0pt}%
\pgfsys@defobject{currentmarker}{\pgfqpoint{0.000000in}{-0.048611in}}{\pgfqpoint{0.000000in}{0.000000in}}{%
\pgfpathmoveto{\pgfqpoint{0.000000in}{0.000000in}}%
\pgfpathlineto{\pgfqpoint{0.000000in}{-0.048611in}}%
\pgfusepath{stroke,fill}%
}%
\begin{pgfscope}%
\pgfsys@transformshift{2.842444in}{0.460676in}%
\pgfsys@useobject{currentmarker}{}%
\end{pgfscope}%
\end{pgfscope}%
\begin{pgfscope}%
\definecolor{textcolor}{rgb}{0.000000,0.000000,0.000000}%
\pgfsetstrokecolor{textcolor}%
\pgfsetfillcolor{textcolor}%
\pgftext[x=2.842444in,y=0.363454in,,top]{\color{textcolor}\rmfamily\fontsize{8.330000}{9.996000}\selectfont \(\displaystyle 40\)}%
\end{pgfscope}%
\begin{pgfscope}%
\pgfsetbuttcap%
\pgfsetroundjoin%
\definecolor{currentfill}{rgb}{0.000000,0.000000,0.000000}%
\pgfsetfillcolor{currentfill}%
\pgfsetlinewidth{0.803000pt}%
\definecolor{currentstroke}{rgb}{0.000000,0.000000,0.000000}%
\pgfsetstrokecolor{currentstroke}%
\pgfsetdash{}{0pt}%
\pgfsys@defobject{currentmarker}{\pgfqpoint{0.000000in}{-0.048611in}}{\pgfqpoint{0.000000in}{0.000000in}}{%
\pgfpathmoveto{\pgfqpoint{0.000000in}{0.000000in}}%
\pgfpathlineto{\pgfqpoint{0.000000in}{-0.048611in}}%
\pgfusepath{stroke,fill}%
}%
\begin{pgfscope}%
\pgfsys@transformshift{3.891470in}{0.460676in}%
\pgfsys@useobject{currentmarker}{}%
\end{pgfscope}%
\end{pgfscope}%
\begin{pgfscope}%
\definecolor{textcolor}{rgb}{0.000000,0.000000,0.000000}%
\pgfsetstrokecolor{textcolor}%
\pgfsetfillcolor{textcolor}%
\pgftext[x=3.891470in,y=0.363454in,,top]{\color{textcolor}\rmfamily\fontsize{8.330000}{9.996000}\selectfont \(\displaystyle 60\)}%
\end{pgfscope}%
\begin{pgfscope}%
\pgfsetbuttcap%
\pgfsetroundjoin%
\definecolor{currentfill}{rgb}{0.000000,0.000000,0.000000}%
\pgfsetfillcolor{currentfill}%
\pgfsetlinewidth{0.803000pt}%
\definecolor{currentstroke}{rgb}{0.000000,0.000000,0.000000}%
\pgfsetstrokecolor{currentstroke}%
\pgfsetdash{}{0pt}%
\pgfsys@defobject{currentmarker}{\pgfqpoint{0.000000in}{-0.048611in}}{\pgfqpoint{0.000000in}{0.000000in}}{%
\pgfpathmoveto{\pgfqpoint{0.000000in}{0.000000in}}%
\pgfpathlineto{\pgfqpoint{0.000000in}{-0.048611in}}%
\pgfusepath{stroke,fill}%
}%
\begin{pgfscope}%
\pgfsys@transformshift{4.940496in}{0.460676in}%
\pgfsys@useobject{currentmarker}{}%
\end{pgfscope}%
\end{pgfscope}%
\begin{pgfscope}%
\definecolor{textcolor}{rgb}{0.000000,0.000000,0.000000}%
\pgfsetstrokecolor{textcolor}%
\pgfsetfillcolor{textcolor}%
\pgftext[x=4.940496in,y=0.363454in,,top]{\color{textcolor}\rmfamily\fontsize{8.330000}{9.996000}\selectfont \(\displaystyle 80\)}%
\end{pgfscope}%
\begin{pgfscope}%
\definecolor{textcolor}{rgb}{0.000000,0.000000,0.000000}%
\pgfsetstrokecolor{textcolor}%
\pgfsetfillcolor{textcolor}%
\pgftext[x=2.947347in,y=0.203949in,,top]{\color{textcolor}\rmfamily\fontsize{8.330000}{9.996000}\selectfont Instances}%
\end{pgfscope}%
\begin{pgfscope}%
\pgfsetbuttcap%
\pgfsetroundjoin%
\definecolor{currentfill}{rgb}{0.000000,0.000000,0.000000}%
\pgfsetfillcolor{currentfill}%
\pgfsetlinewidth{0.803000pt}%
\definecolor{currentstroke}{rgb}{0.000000,0.000000,0.000000}%
\pgfsetstrokecolor{currentstroke}%
\pgfsetdash{}{0pt}%
\pgfsys@defobject{currentmarker}{\pgfqpoint{-0.048611in}{0.000000in}}{\pgfqpoint{0.000000in}{0.000000in}}{%
\pgfpathmoveto{\pgfqpoint{0.000000in}{0.000000in}}%
\pgfpathlineto{\pgfqpoint{-0.048611in}{0.000000in}}%
\pgfusepath{stroke,fill}%
}%
\begin{pgfscope}%
\pgfsys@transformshift{0.581793in}{0.460676in}%
\pgfsys@useobject{currentmarker}{}%
\end{pgfscope}%
\end{pgfscope}%
\begin{pgfscope}%
\definecolor{textcolor}{rgb}{0.000000,0.000000,0.000000}%
\pgfsetstrokecolor{textcolor}%
\pgfsetfillcolor{textcolor}%
\pgftext[x=0.274692in, y=0.421225in, left, base]{\color{textcolor}\rmfamily\fontsize{8.330000}{9.996000}\selectfont \(\displaystyle 0.00\)}%
\end{pgfscope}%
\begin{pgfscope}%
\pgfsetbuttcap%
\pgfsetroundjoin%
\definecolor{currentfill}{rgb}{0.000000,0.000000,0.000000}%
\pgfsetfillcolor{currentfill}%
\pgfsetlinewidth{0.803000pt}%
\definecolor{currentstroke}{rgb}{0.000000,0.000000,0.000000}%
\pgfsetstrokecolor{currentstroke}%
\pgfsetdash{}{0pt}%
\pgfsys@defobject{currentmarker}{\pgfqpoint{-0.048611in}{0.000000in}}{\pgfqpoint{0.000000in}{0.000000in}}{%
\pgfpathmoveto{\pgfqpoint{0.000000in}{0.000000in}}%
\pgfpathlineto{\pgfqpoint{-0.048611in}{0.000000in}}%
\pgfusepath{stroke,fill}%
}%
\begin{pgfscope}%
\pgfsys@transformshift{0.581793in}{0.692205in}%
\pgfsys@useobject{currentmarker}{}%
\end{pgfscope}%
\end{pgfscope}%
\begin{pgfscope}%
\definecolor{textcolor}{rgb}{0.000000,0.000000,0.000000}%
\pgfsetstrokecolor{textcolor}%
\pgfsetfillcolor{textcolor}%
\pgftext[x=0.274692in, y=0.652754in, left, base]{\color{textcolor}\rmfamily\fontsize{8.330000}{9.996000}\selectfont \(\displaystyle 0.01\)}%
\end{pgfscope}%
\begin{pgfscope}%
\pgfsetbuttcap%
\pgfsetroundjoin%
\definecolor{currentfill}{rgb}{0.000000,0.000000,0.000000}%
\pgfsetfillcolor{currentfill}%
\pgfsetlinewidth{0.803000pt}%
\definecolor{currentstroke}{rgb}{0.000000,0.000000,0.000000}%
\pgfsetstrokecolor{currentstroke}%
\pgfsetdash{}{0pt}%
\pgfsys@defobject{currentmarker}{\pgfqpoint{-0.048611in}{0.000000in}}{\pgfqpoint{0.000000in}{0.000000in}}{%
\pgfpathmoveto{\pgfqpoint{0.000000in}{0.000000in}}%
\pgfpathlineto{\pgfqpoint{-0.048611in}{0.000000in}}%
\pgfusepath{stroke,fill}%
}%
\begin{pgfscope}%
\pgfsys@transformshift{0.581793in}{0.978391in}%
\pgfsys@useobject{currentmarker}{}%
\end{pgfscope}%
\end{pgfscope}%
\begin{pgfscope}%
\definecolor{textcolor}{rgb}{0.000000,0.000000,0.000000}%
\pgfsetstrokecolor{textcolor}%
\pgfsetfillcolor{textcolor}%
\pgftext[x=0.274692in, y=0.938940in, left, base]{\color{textcolor}\rmfamily\fontsize{8.330000}{9.996000}\selectfont \(\displaystyle 0.05\)}%
\end{pgfscope}%
\begin{pgfscope}%
\pgfsetbuttcap%
\pgfsetroundjoin%
\definecolor{currentfill}{rgb}{0.000000,0.000000,0.000000}%
\pgfsetfillcolor{currentfill}%
\pgfsetlinewidth{0.803000pt}%
\definecolor{currentstroke}{rgb}{0.000000,0.000000,0.000000}%
\pgfsetstrokecolor{currentstroke}%
\pgfsetdash{}{0pt}%
\pgfsys@defobject{currentmarker}{\pgfqpoint{-0.048611in}{0.000000in}}{\pgfqpoint{0.000000in}{0.000000in}}{%
\pgfpathmoveto{\pgfqpoint{0.000000in}{0.000000in}}%
\pgfpathlineto{\pgfqpoint{-0.048611in}{0.000000in}}%
\pgfusepath{stroke,fill}%
}%
\begin{pgfscope}%
\pgfsys@transformshift{0.581793in}{1.192835in}%
\pgfsys@useobject{currentmarker}{}%
\end{pgfscope}%
\end{pgfscope}%
\begin{pgfscope}%
\definecolor{textcolor}{rgb}{0.000000,0.000000,0.000000}%
\pgfsetstrokecolor{textcolor}%
\pgfsetfillcolor{textcolor}%
\pgftext[x=0.274692in, y=1.153384in, left, base]{\color{textcolor}\rmfamily\fontsize{8.330000}{9.996000}\selectfont \(\displaystyle 0.10\)}%
\end{pgfscope}%
\begin{pgfscope}%
\pgfsetbuttcap%
\pgfsetroundjoin%
\definecolor{currentfill}{rgb}{0.000000,0.000000,0.000000}%
\pgfsetfillcolor{currentfill}%
\pgfsetlinewidth{0.803000pt}%
\definecolor{currentstroke}{rgb}{0.000000,0.000000,0.000000}%
\pgfsetstrokecolor{currentstroke}%
\pgfsetdash{}{0pt}%
\pgfsys@defobject{currentmarker}{\pgfqpoint{-0.048611in}{0.000000in}}{\pgfqpoint{0.000000in}{0.000000in}}{%
\pgfpathmoveto{\pgfqpoint{0.000000in}{0.000000in}}%
\pgfpathlineto{\pgfqpoint{-0.048611in}{0.000000in}}%
\pgfusepath{stroke,fill}%
}%
\begin{pgfscope}%
\pgfsys@transformshift{0.581793in}{1.496106in}%
\pgfsys@useobject{currentmarker}{}%
\end{pgfscope}%
\end{pgfscope}%
\begin{pgfscope}%
\definecolor{textcolor}{rgb}{0.000000,0.000000,0.000000}%
\pgfsetstrokecolor{textcolor}%
\pgfsetfillcolor{textcolor}%
\pgftext[x=0.274692in, y=1.456655in, left, base]{\color{textcolor}\rmfamily\fontsize{8.330000}{9.996000}\selectfont \(\displaystyle 0.20\)}%
\end{pgfscope}%
\begin{pgfscope}%
\pgfsetbuttcap%
\pgfsetroundjoin%
\definecolor{currentfill}{rgb}{0.000000,0.000000,0.000000}%
\pgfsetfillcolor{currentfill}%
\pgfsetlinewidth{0.803000pt}%
\definecolor{currentstroke}{rgb}{0.000000,0.000000,0.000000}%
\pgfsetstrokecolor{currentstroke}%
\pgfsetdash{}{0pt}%
\pgfsys@defobject{currentmarker}{\pgfqpoint{-0.048611in}{0.000000in}}{\pgfqpoint{0.000000in}{0.000000in}}{%
\pgfpathmoveto{\pgfqpoint{0.000000in}{0.000000in}}%
\pgfpathlineto{\pgfqpoint{-0.048611in}{0.000000in}}%
\pgfusepath{stroke,fill}%
}%
\begin{pgfscope}%
\pgfsys@transformshift{0.581793in}{1.924995in}%
\pgfsys@useobject{currentmarker}{}%
\end{pgfscope}%
\end{pgfscope}%
\begin{pgfscope}%
\definecolor{textcolor}{rgb}{0.000000,0.000000,0.000000}%
\pgfsetstrokecolor{textcolor}%
\pgfsetfillcolor{textcolor}%
\pgftext[x=0.274692in, y=1.885544in, left, base]{\color{textcolor}\rmfamily\fontsize{8.330000}{9.996000}\selectfont \(\displaystyle 0.40\)}%
\end{pgfscope}%
\begin{pgfscope}%
\pgfsetbuttcap%
\pgfsetroundjoin%
\definecolor{currentfill}{rgb}{0.000000,0.000000,0.000000}%
\pgfsetfillcolor{currentfill}%
\pgfsetlinewidth{0.803000pt}%
\definecolor{currentstroke}{rgb}{0.000000,0.000000,0.000000}%
\pgfsetstrokecolor{currentstroke}%
\pgfsetdash{}{0pt}%
\pgfsys@defobject{currentmarker}{\pgfqpoint{-0.048611in}{0.000000in}}{\pgfqpoint{0.000000in}{0.000000in}}{%
\pgfpathmoveto{\pgfqpoint{0.000000in}{0.000000in}}%
\pgfpathlineto{\pgfqpoint{-0.048611in}{0.000000in}}%
\pgfusepath{stroke,fill}%
}%
\begin{pgfscope}%
\pgfsys@transformshift{0.581793in}{2.254093in}%
\pgfsys@useobject{currentmarker}{}%
\end{pgfscope}%
\end{pgfscope}%
\begin{pgfscope}%
\definecolor{textcolor}{rgb}{0.000000,0.000000,0.000000}%
\pgfsetstrokecolor{textcolor}%
\pgfsetfillcolor{textcolor}%
\pgftext[x=0.274692in, y=2.214642in, left, base]{\color{textcolor}\rmfamily\fontsize{8.330000}{9.996000}\selectfont \(\displaystyle 0.60\)}%
\end{pgfscope}%
\begin{pgfscope}%
\pgfsetbuttcap%
\pgfsetroundjoin%
\definecolor{currentfill}{rgb}{0.000000,0.000000,0.000000}%
\pgfsetfillcolor{currentfill}%
\pgfsetlinewidth{0.803000pt}%
\definecolor{currentstroke}{rgb}{0.000000,0.000000,0.000000}%
\pgfsetstrokecolor{currentstroke}%
\pgfsetdash{}{0pt}%
\pgfsys@defobject{currentmarker}{\pgfqpoint{-0.048611in}{0.000000in}}{\pgfqpoint{0.000000in}{0.000000in}}{%
\pgfpathmoveto{\pgfqpoint{0.000000in}{0.000000in}}%
\pgfpathlineto{\pgfqpoint{-0.048611in}{0.000000in}}%
\pgfusepath{stroke,fill}%
}%
\begin{pgfscope}%
\pgfsys@transformshift{0.581793in}{2.531536in}%
\pgfsys@useobject{currentmarker}{}%
\end{pgfscope}%
\end{pgfscope}%
\begin{pgfscope}%
\definecolor{textcolor}{rgb}{0.000000,0.000000,0.000000}%
\pgfsetstrokecolor{textcolor}%
\pgfsetfillcolor{textcolor}%
\pgftext[x=0.274692in, y=2.492085in, left, base]{\color{textcolor}\rmfamily\fontsize{8.330000}{9.996000}\selectfont \(\displaystyle 0.80\)}%
\end{pgfscope}%
\begin{pgfscope}%
\pgfsetbuttcap%
\pgfsetroundjoin%
\definecolor{currentfill}{rgb}{0.000000,0.000000,0.000000}%
\pgfsetfillcolor{currentfill}%
\pgfsetlinewidth{0.803000pt}%
\definecolor{currentstroke}{rgb}{0.000000,0.000000,0.000000}%
\pgfsetstrokecolor{currentstroke}%
\pgfsetdash{}{0pt}%
\pgfsys@defobject{currentmarker}{\pgfqpoint{-0.048611in}{0.000000in}}{\pgfqpoint{0.000000in}{0.000000in}}{%
\pgfpathmoveto{\pgfqpoint{0.000000in}{0.000000in}}%
\pgfpathlineto{\pgfqpoint{-0.048611in}{0.000000in}}%
\pgfusepath{stroke,fill}%
}%
\begin{pgfscope}%
\pgfsys@transformshift{0.581793in}{2.775968in}%
\pgfsys@useobject{currentmarker}{}%
\end{pgfscope}%
\end{pgfscope}%
\begin{pgfscope}%
\definecolor{textcolor}{rgb}{0.000000,0.000000,0.000000}%
\pgfsetstrokecolor{textcolor}%
\pgfsetfillcolor{textcolor}%
\pgftext[x=0.274692in, y=2.736517in, left, base]{\color{textcolor}\rmfamily\fontsize{8.330000}{9.996000}\selectfont \(\displaystyle 1.00\)}%
\end{pgfscope}%
\begin{pgfscope}%
\definecolor{textcolor}{rgb}{0.000000,0.000000,0.000000}%
\pgfsetstrokecolor{textcolor}%
\pgfsetfillcolor{textcolor}%
\pgftext[x=0.219136in,y=1.752800in,,bottom,rotate=90.000000]{\color{textcolor}\rmfamily\fontsize{8.330000}{9.996000}\selectfont \(\displaystyle 1-(t_{best}/t_{alg})\)}%
\end{pgfscope}%
\begin{pgfscope}%
\definecolor{textcolor}{rgb}{0.000000,0.000000,0.000000}%
\pgfsetstrokecolor{textcolor}%
\pgfsetfillcolor{textcolor}%
\pgftext[x=2.947347in,y=3.128257in,,base]{\color{textcolor}\rmfamily\fontsize{8.330000}{9.996000}\selectfont all instances - runtime}%
\end{pgfscope}%
\begin{pgfscope}%
\pgfsetbuttcap%
\pgfsetroundjoin%
\definecolor{currentfill}{rgb}{0.121569,0.466667,0.705882}%
\pgfsetfillcolor{currentfill}%
\pgfsetlinewidth{1.003750pt}%
\definecolor{currentstroke}{rgb}{0.121569,0.466667,0.705882}%
\pgfsetstrokecolor{currentstroke}%
\pgfsetdash{}{0pt}%
\pgfsys@defobject{currentmarker}{\pgfqpoint{-0.028527in}{-0.028527in}}{\pgfqpoint{0.028527in}{0.028527in}}{%
\pgfpathmoveto{\pgfqpoint{0.000000in}{-0.028527in}}%
\pgfpathcurveto{\pgfqpoint{0.007566in}{-0.028527in}}{\pgfqpoint{0.014822in}{-0.025521in}}{\pgfqpoint{0.020172in}{-0.020172in}}%
\pgfpathcurveto{\pgfqpoint{0.025521in}{-0.014822in}}{\pgfqpoint{0.028527in}{-0.007566in}}{\pgfqpoint{0.028527in}{0.000000in}}%
\pgfpathcurveto{\pgfqpoint{0.028527in}{0.007566in}}{\pgfqpoint{0.025521in}{0.014822in}}{\pgfqpoint{0.020172in}{0.020172in}}%
\pgfpathcurveto{\pgfqpoint{0.014822in}{0.025521in}}{\pgfqpoint{0.007566in}{0.028527in}}{\pgfqpoint{0.000000in}{0.028527in}}%
\pgfpathcurveto{\pgfqpoint{-0.007566in}{0.028527in}}{\pgfqpoint{-0.014822in}{0.025521in}}{\pgfqpoint{-0.020172in}{0.020172in}}%
\pgfpathcurveto{\pgfqpoint{-0.025521in}{0.014822in}}{\pgfqpoint{-0.028527in}{0.007566in}}{\pgfqpoint{-0.028527in}{0.000000in}}%
\pgfpathcurveto{\pgfqpoint{-0.028527in}{-0.007566in}}{\pgfqpoint{-0.025521in}{-0.014822in}}{\pgfqpoint{-0.020172in}{-0.020172in}}%
\pgfpathcurveto{\pgfqpoint{-0.014822in}{-0.025521in}}{\pgfqpoint{-0.007566in}{-0.028527in}}{\pgfqpoint{0.000000in}{-0.028527in}}%
\pgfpathclose%
\pgfusepath{stroke,fill}%
}%
\begin{pgfscope}%
\pgfsys@transformshift{3.637184in}{2.656165in}%
\pgfsys@useobject{currentmarker}{}%
\end{pgfscope}%
\end{pgfscope}%
\begin{pgfscope}%
\definecolor{textcolor}{rgb}{0.000000,0.000000,0.000000}%
\pgfsetstrokecolor{textcolor}%
\pgfsetfillcolor{textcolor}%
\pgftext[x=3.764447in,y=2.625796in,left,base]{\color{textcolor}\rmfamily\fontsize{8.330000}{9.996000}\selectfont maximum degree (default)}%
\end{pgfscope}%
\begin{pgfscope}%
\pgfsetbuttcap%
\pgfsetroundjoin%
\definecolor{currentfill}{rgb}{1.000000,0.498039,0.054902}%
\pgfsetfillcolor{currentfill}%
\pgfsetfillopacity{0.750000}%
\pgfsetlinewidth{1.003750pt}%
\definecolor{currentstroke}{rgb}{1.000000,0.498039,0.054902}%
\pgfsetstrokecolor{currentstroke}%
\pgfsetstrokeopacity{0.750000}%
\pgfsetdash{}{0pt}%
\pgfsys@defobject{currentmarker}{\pgfqpoint{-0.028527in}{-0.028527in}}{\pgfqpoint{0.028527in}{0.028527in}}{%
\pgfpathmoveto{\pgfqpoint{-0.000000in}{-0.028527in}}%
\pgfpathlineto{\pgfqpoint{0.028527in}{0.028527in}}%
\pgfpathlineto{\pgfqpoint{-0.028527in}{0.028527in}}%
\pgfpathclose%
\pgfusepath{stroke,fill}%
}%
\begin{pgfscope}%
\pgfsys@transformshift{3.637184in}{2.526993in}%
\pgfsys@useobject{currentmarker}{}%
\end{pgfscope}%
\end{pgfscope}%
\begin{pgfscope}%
\definecolor{textcolor}{rgb}{0.000000,0.000000,0.000000}%
\pgfsetstrokecolor{textcolor}%
\pgfsetfillcolor{textcolor}%
\pgftext[x=3.764447in,y=2.496624in,left,base]{\color{textcolor}\rmfamily\fontsize{8.330000}{9.996000}\selectfont articulation points}%
\end{pgfscope}%
\begin{pgfscope}%
\pgfsetbuttcap%
\pgfsetroundjoin%
\definecolor{currentfill}{rgb}{0.172549,0.627451,0.172549}%
\pgfsetfillcolor{currentfill}%
\pgfsetfillopacity{0.750000}%
\pgfsetlinewidth{1.003750pt}%
\definecolor{currentstroke}{rgb}{0.172549,0.627451,0.172549}%
\pgfsetstrokecolor{currentstroke}%
\pgfsetstrokeopacity{0.750000}%
\pgfsetdash{}{0pt}%
\pgfsys@defobject{currentmarker}{\pgfqpoint{-0.028527in}{-0.028527in}}{\pgfqpoint{0.028527in}{0.028527in}}{%
\pgfpathmoveto{\pgfqpoint{-0.028527in}{-0.028527in}}%
\pgfpathlineto{\pgfqpoint{0.028527in}{-0.028527in}}%
\pgfpathlineto{\pgfqpoint{0.028527in}{0.028527in}}%
\pgfpathlineto{\pgfqpoint{-0.028527in}{0.028527in}}%
\pgfpathclose%
\pgfusepath{stroke,fill}%
}%
\begin{pgfscope}%
\pgfsys@transformshift{3.637184in}{2.397821in}%
\pgfsys@useobject{currentmarker}{}%
\end{pgfscope}%
\end{pgfscope}%
\begin{pgfscope}%
\definecolor{textcolor}{rgb}{0.000000,0.000000,0.000000}%
\pgfsetstrokecolor{textcolor}%
\pgfsetfillcolor{textcolor}%
\pgftext[x=3.764447in,y=2.367452in,left,base]{\color{textcolor}\rmfamily\fontsize{8.330000}{9.996000}\selectfont edge cuts}%
\end{pgfscope}%
\begin{pgfscope}%
\pgfsetbuttcap%
\pgfsetroundjoin%
\definecolor{currentfill}{rgb}{0.839216,0.152941,0.156863}%
\pgfsetfillcolor{currentfill}%
\pgfsetfillopacity{0.750000}%
\pgfsetlinewidth{1.003750pt}%
\definecolor{currentstroke}{rgb}{0.839216,0.152941,0.156863}%
\pgfsetstrokecolor{currentstroke}%
\pgfsetstrokeopacity{0.750000}%
\pgfsetdash{}{0pt}%
\pgfsys@defobject{currentmarker}{\pgfqpoint{-0.028527in}{-0.028527in}}{\pgfqpoint{0.028527in}{0.028527in}}{%
\pgfpathmoveto{\pgfqpoint{0.000000in}{0.028527in}}%
\pgfpathlineto{\pgfqpoint{-0.028527in}{-0.028527in}}%
\pgfpathlineto{\pgfqpoint{0.028527in}{-0.028527in}}%
\pgfpathclose%
\pgfusepath{stroke,fill}%
}%
\begin{pgfscope}%
\pgfsys@transformshift{3.637184in}{2.268649in}%
\pgfsys@useobject{currentmarker}{}%
\end{pgfscope}%
\end{pgfscope}%
\begin{pgfscope}%
\definecolor{textcolor}{rgb}{0.000000,0.000000,0.000000}%
\pgfsetstrokecolor{textcolor}%
\pgfsetfillcolor{textcolor}%
\pgftext[x=3.764447in,y=2.238280in,left,base]{\color{textcolor}\rmfamily\fontsize{8.330000}{9.996000}\selectfont nested dissection}%
\end{pgfscope}%
\end{pgfpicture}%
\makeatother%
\endgroup%

	\caption{Performance plot comparing running times of decomposition-based branching strategies.}\label{fig:all_decomp_time}
\end{figure}

\begin{figure}[t]
	\centering
	%% Creator: Matplotlib, PGF backend
%%
%% To include the figure in your LaTeX document, write
%%   \input{<filename>.pgf}
%%
%% Make sure the required packages are loaded in your preamble
%%   \usepackage{pgf}
%%
%% and, on pdftex
%%   \usepackage[utf8]{inputenc}\DeclareUnicodeCharacter{2212}{-}
%%
%% or, on luatex and xetex
%%   \usepackage{unicode-math}
%%
%% Figures using additional raster images can only be included by \input if
%% they are in the same directory as the main LaTeX file. For loading figures
%% from other directories you can use the `import` package
%%   \usepackage{import}
%%
%% and then include the figures with
%%   \import{<path to file>}{<filename>.pgf}
%%
%% Matplotlib used the following preamble
%%   \renewcommand{\sfdefault}{phv}
%%   \renewcommand{\rmdefault}{ptm}
%%   \renewcommand{\ttdefault}{pcr}
%%   \normalfont\selectfont
%%
\begingroup%
\makeatletter%
\begin{pgfpicture}%
\pgfpathrectangle{\pgfpointorigin}{\pgfqpoint{5.412900in}{2.857160in}}%
\pgfusepath{use as bounding box, clip}%
\begin{pgfscope}%
\pgfsetbuttcap%
\pgfsetmiterjoin%
\definecolor{currentfill}{rgb}{1.000000,1.000000,1.000000}%
\pgfsetfillcolor{currentfill}%
\pgfsetlinewidth{0.000000pt}%
\definecolor{currentstroke}{rgb}{1.000000,1.000000,1.000000}%
\pgfsetstrokecolor{currentstroke}%
\pgfsetdash{}{0pt}%
\pgfpathmoveto{\pgfqpoint{0.000000in}{0.000000in}}%
\pgfpathlineto{\pgfqpoint{5.412900in}{0.000000in}}%
\pgfpathlineto{\pgfqpoint{5.412900in}{2.857160in}}%
\pgfpathlineto{\pgfqpoint{0.000000in}{2.857160in}}%
\pgfpathclose%
\pgfusepath{fill}%
\end{pgfscope}%
\begin{pgfscope}%
\pgfsetbuttcap%
\pgfsetmiterjoin%
\definecolor{currentfill}{rgb}{1.000000,1.000000,1.000000}%
\pgfsetfillcolor{currentfill}%
\pgfsetlinewidth{0.000000pt}%
\definecolor{currentstroke}{rgb}{0.000000,0.000000,0.000000}%
\pgfsetstrokecolor{currentstroke}%
\pgfsetstrokeopacity{0.000000}%
\pgfsetdash{}{0pt}%
\pgfpathmoveto{\pgfqpoint{0.581793in}{0.460676in}}%
\pgfpathlineto{\pgfqpoint{5.312900in}{0.460676in}}%
\pgfpathlineto{\pgfqpoint{5.312900in}{2.594924in}}%
\pgfpathlineto{\pgfqpoint{0.581793in}{2.594924in}}%
\pgfpathclose%
\pgfusepath{fill}%
\end{pgfscope}%
\begin{pgfscope}%
\pgfpathrectangle{\pgfqpoint{0.581793in}{0.460676in}}{\pgfqpoint{4.731107in}{2.134248in}}%
\pgfusepath{clip}%
\pgfsetbuttcap%
\pgfsetroundjoin%
\definecolor{currentfill}{rgb}{0.121569,0.466667,0.705882}%
\pgfsetfillcolor{currentfill}%
\pgfsetlinewidth{1.003750pt}%
\definecolor{currentstroke}{rgb}{0.121569,0.466667,0.705882}%
\pgfsetstrokecolor{currentstroke}%
\pgfsetdash{}{0pt}%
\pgfsys@defobject{currentmarker}{\pgfqpoint{-0.019018in}{-0.019018in}}{\pgfqpoint{0.019018in}{0.019018in}}{%
\pgfpathmoveto{\pgfqpoint{0.000000in}{-0.019018in}}%
\pgfpathcurveto{\pgfqpoint{0.005044in}{-0.019018in}}{\pgfqpoint{0.009881in}{-0.017014in}}{\pgfqpoint{0.013448in}{-0.013448in}}%
\pgfpathcurveto{\pgfqpoint{0.017014in}{-0.009881in}}{\pgfqpoint{0.019018in}{-0.005044in}}{\pgfqpoint{0.019018in}{0.000000in}}%
\pgfpathcurveto{\pgfqpoint{0.019018in}{0.005044in}}{\pgfqpoint{0.017014in}{0.009881in}}{\pgfqpoint{0.013448in}{0.013448in}}%
\pgfpathcurveto{\pgfqpoint{0.009881in}{0.017014in}}{\pgfqpoint{0.005044in}{0.019018in}}{\pgfqpoint{0.000000in}{0.019018in}}%
\pgfpathcurveto{\pgfqpoint{-0.005044in}{0.019018in}}{\pgfqpoint{-0.009881in}{0.017014in}}{\pgfqpoint{-0.013448in}{0.013448in}}%
\pgfpathcurveto{\pgfqpoint{-0.017014in}{0.009881in}}{\pgfqpoint{-0.019018in}{0.005044in}}{\pgfqpoint{-0.019018in}{0.000000in}}%
\pgfpathcurveto{\pgfqpoint{-0.019018in}{-0.005044in}}{\pgfqpoint{-0.017014in}{-0.009881in}}{\pgfqpoint{-0.013448in}{-0.013448in}}%
\pgfpathcurveto{\pgfqpoint{-0.009881in}{-0.017014in}}{\pgfqpoint{-0.005044in}{-0.019018in}}{\pgfqpoint{0.000000in}{-0.019018in}}%
\pgfpathclose%
\pgfusepath{stroke,fill}%
}%
\begin{pgfscope}%
\pgfsys@transformshift{0.796844in}{2.466130in}%
\pgfsys@useobject{currentmarker}{}%
\end{pgfscope}%
\begin{pgfscope}%
\pgfsys@transformshift{0.849295in}{2.466130in}%
\pgfsys@useobject{currentmarker}{}%
\end{pgfscope}%
\begin{pgfscope}%
\pgfsys@transformshift{0.901746in}{2.466130in}%
\pgfsys@useobject{currentmarker}{}%
\end{pgfscope}%
\begin{pgfscope}%
\pgfsys@transformshift{0.954198in}{2.466130in}%
\pgfsys@useobject{currentmarker}{}%
\end{pgfscope}%
\begin{pgfscope}%
\pgfsys@transformshift{1.006649in}{2.466130in}%
\pgfsys@useobject{currentmarker}{}%
\end{pgfscope}%
\begin{pgfscope}%
\pgfsys@transformshift{1.059100in}{2.466130in}%
\pgfsys@useobject{currentmarker}{}%
\end{pgfscope}%
\begin{pgfscope}%
\pgfsys@transformshift{1.111551in}{2.466130in}%
\pgfsys@useobject{currentmarker}{}%
\end{pgfscope}%
\begin{pgfscope}%
\pgfsys@transformshift{1.164003in}{2.466130in}%
\pgfsys@useobject{currentmarker}{}%
\end{pgfscope}%
\begin{pgfscope}%
\pgfsys@transformshift{1.216454in}{2.466130in}%
\pgfsys@useobject{currentmarker}{}%
\end{pgfscope}%
\begin{pgfscope}%
\pgfsys@transformshift{1.268905in}{2.126952in}%
\pgfsys@useobject{currentmarker}{}%
\end{pgfscope}%
\begin{pgfscope}%
\pgfsys@transformshift{1.321357in}{1.454743in}%
\pgfsys@useobject{currentmarker}{}%
\end{pgfscope}%
\begin{pgfscope}%
\pgfsys@transformshift{1.373808in}{1.385646in}%
\pgfsys@useobject{currentmarker}{}%
\end{pgfscope}%
\begin{pgfscope}%
\pgfsys@transformshift{1.426259in}{1.325830in}%
\pgfsys@useobject{currentmarker}{}%
\end{pgfscope}%
\begin{pgfscope}%
\pgfsys@transformshift{1.478710in}{1.145147in}%
\pgfsys@useobject{currentmarker}{}%
\end{pgfscope}%
\begin{pgfscope}%
\pgfsys@transformshift{1.531162in}{0.642774in}%
\pgfsys@useobject{currentmarker}{}%
\end{pgfscope}%
\begin{pgfscope}%
\pgfsys@transformshift{1.583613in}{0.585253in}%
\pgfsys@useobject{currentmarker}{}%
\end{pgfscope}%
\begin{pgfscope}%
\pgfsys@transformshift{1.636064in}{0.553036in}%
\pgfsys@useobject{currentmarker}{}%
\end{pgfscope}%
\begin{pgfscope}%
\pgfsys@transformshift{1.688516in}{0.461197in}%
\pgfsys@useobject{currentmarker}{}%
\end{pgfscope}%
\begin{pgfscope}%
\pgfsys@transformshift{1.740967in}{0.460676in}%
\pgfsys@useobject{currentmarker}{}%
\end{pgfscope}%
\begin{pgfscope}%
\pgfsys@transformshift{1.793418in}{0.460676in}%
\pgfsys@useobject{currentmarker}{}%
\end{pgfscope}%
\begin{pgfscope}%
\pgfsys@transformshift{1.845870in}{0.460676in}%
\pgfsys@useobject{currentmarker}{}%
\end{pgfscope}%
\begin{pgfscope}%
\pgfsys@transformshift{1.898321in}{0.460676in}%
\pgfsys@useobject{currentmarker}{}%
\end{pgfscope}%
\begin{pgfscope}%
\pgfsys@transformshift{1.950772in}{0.460676in}%
\pgfsys@useobject{currentmarker}{}%
\end{pgfscope}%
\begin{pgfscope}%
\pgfsys@transformshift{2.003223in}{0.460676in}%
\pgfsys@useobject{currentmarker}{}%
\end{pgfscope}%
\begin{pgfscope}%
\pgfsys@transformshift{2.055675in}{0.460676in}%
\pgfsys@useobject{currentmarker}{}%
\end{pgfscope}%
\begin{pgfscope}%
\pgfsys@transformshift{2.108126in}{0.460676in}%
\pgfsys@useobject{currentmarker}{}%
\end{pgfscope}%
\begin{pgfscope}%
\pgfsys@transformshift{2.160577in}{0.460676in}%
\pgfsys@useobject{currentmarker}{}%
\end{pgfscope}%
\begin{pgfscope}%
\pgfsys@transformshift{2.213029in}{0.460676in}%
\pgfsys@useobject{currentmarker}{}%
\end{pgfscope}%
\begin{pgfscope}%
\pgfsys@transformshift{2.265480in}{0.460676in}%
\pgfsys@useobject{currentmarker}{}%
\end{pgfscope}%
\begin{pgfscope}%
\pgfsys@transformshift{2.317931in}{0.460676in}%
\pgfsys@useobject{currentmarker}{}%
\end{pgfscope}%
\begin{pgfscope}%
\pgfsys@transformshift{2.370382in}{0.460676in}%
\pgfsys@useobject{currentmarker}{}%
\end{pgfscope}%
\begin{pgfscope}%
\pgfsys@transformshift{2.422834in}{0.460676in}%
\pgfsys@useobject{currentmarker}{}%
\end{pgfscope}%
\begin{pgfscope}%
\pgfsys@transformshift{2.475285in}{0.460676in}%
\pgfsys@useobject{currentmarker}{}%
\end{pgfscope}%
\begin{pgfscope}%
\pgfsys@transformshift{2.527736in}{0.460676in}%
\pgfsys@useobject{currentmarker}{}%
\end{pgfscope}%
\begin{pgfscope}%
\pgfsys@transformshift{2.580188in}{0.460676in}%
\pgfsys@useobject{currentmarker}{}%
\end{pgfscope}%
\begin{pgfscope}%
\pgfsys@transformshift{2.632639in}{0.460676in}%
\pgfsys@useobject{currentmarker}{}%
\end{pgfscope}%
\begin{pgfscope}%
\pgfsys@transformshift{2.685090in}{0.460676in}%
\pgfsys@useobject{currentmarker}{}%
\end{pgfscope}%
\begin{pgfscope}%
\pgfsys@transformshift{2.737541in}{0.460676in}%
\pgfsys@useobject{currentmarker}{}%
\end{pgfscope}%
\begin{pgfscope}%
\pgfsys@transformshift{2.789993in}{0.460676in}%
\pgfsys@useobject{currentmarker}{}%
\end{pgfscope}%
\begin{pgfscope}%
\pgfsys@transformshift{2.842444in}{0.460676in}%
\pgfsys@useobject{currentmarker}{}%
\end{pgfscope}%
\begin{pgfscope}%
\pgfsys@transformshift{2.894895in}{0.460676in}%
\pgfsys@useobject{currentmarker}{}%
\end{pgfscope}%
\begin{pgfscope}%
\pgfsys@transformshift{2.947347in}{0.460676in}%
\pgfsys@useobject{currentmarker}{}%
\end{pgfscope}%
\begin{pgfscope}%
\pgfsys@transformshift{2.999798in}{0.460676in}%
\pgfsys@useobject{currentmarker}{}%
\end{pgfscope}%
\begin{pgfscope}%
\pgfsys@transformshift{3.052249in}{0.460676in}%
\pgfsys@useobject{currentmarker}{}%
\end{pgfscope}%
\begin{pgfscope}%
\pgfsys@transformshift{3.104701in}{0.460676in}%
\pgfsys@useobject{currentmarker}{}%
\end{pgfscope}%
\begin{pgfscope}%
\pgfsys@transformshift{3.157152in}{0.460676in}%
\pgfsys@useobject{currentmarker}{}%
\end{pgfscope}%
\begin{pgfscope}%
\pgfsys@transformshift{3.209603in}{0.460676in}%
\pgfsys@useobject{currentmarker}{}%
\end{pgfscope}%
\begin{pgfscope}%
\pgfsys@transformshift{3.262054in}{0.460676in}%
\pgfsys@useobject{currentmarker}{}%
\end{pgfscope}%
\begin{pgfscope}%
\pgfsys@transformshift{3.314506in}{0.460676in}%
\pgfsys@useobject{currentmarker}{}%
\end{pgfscope}%
\begin{pgfscope}%
\pgfsys@transformshift{3.366957in}{0.460676in}%
\pgfsys@useobject{currentmarker}{}%
\end{pgfscope}%
\begin{pgfscope}%
\pgfsys@transformshift{3.419408in}{0.460676in}%
\pgfsys@useobject{currentmarker}{}%
\end{pgfscope}%
\begin{pgfscope}%
\pgfsys@transformshift{3.471860in}{0.460676in}%
\pgfsys@useobject{currentmarker}{}%
\end{pgfscope}%
\begin{pgfscope}%
\pgfsys@transformshift{3.524311in}{0.460676in}%
\pgfsys@useobject{currentmarker}{}%
\end{pgfscope}%
\begin{pgfscope}%
\pgfsys@transformshift{3.576762in}{0.460676in}%
\pgfsys@useobject{currentmarker}{}%
\end{pgfscope}%
\begin{pgfscope}%
\pgfsys@transformshift{3.629213in}{0.460676in}%
\pgfsys@useobject{currentmarker}{}%
\end{pgfscope}%
\begin{pgfscope}%
\pgfsys@transformshift{3.681665in}{0.460676in}%
\pgfsys@useobject{currentmarker}{}%
\end{pgfscope}%
\begin{pgfscope}%
\pgfsys@transformshift{3.734116in}{0.460676in}%
\pgfsys@useobject{currentmarker}{}%
\end{pgfscope}%
\begin{pgfscope}%
\pgfsys@transformshift{3.786567in}{0.460676in}%
\pgfsys@useobject{currentmarker}{}%
\end{pgfscope}%
\begin{pgfscope}%
\pgfsys@transformshift{3.839019in}{0.460676in}%
\pgfsys@useobject{currentmarker}{}%
\end{pgfscope}%
\begin{pgfscope}%
\pgfsys@transformshift{3.891470in}{0.460676in}%
\pgfsys@useobject{currentmarker}{}%
\end{pgfscope}%
\begin{pgfscope}%
\pgfsys@transformshift{3.943921in}{0.460676in}%
\pgfsys@useobject{currentmarker}{}%
\end{pgfscope}%
\begin{pgfscope}%
\pgfsys@transformshift{3.996373in}{0.460676in}%
\pgfsys@useobject{currentmarker}{}%
\end{pgfscope}%
\begin{pgfscope}%
\pgfsys@transformshift{4.048824in}{0.460676in}%
\pgfsys@useobject{currentmarker}{}%
\end{pgfscope}%
\begin{pgfscope}%
\pgfsys@transformshift{4.101275in}{0.460676in}%
\pgfsys@useobject{currentmarker}{}%
\end{pgfscope}%
\begin{pgfscope}%
\pgfsys@transformshift{4.153726in}{0.460676in}%
\pgfsys@useobject{currentmarker}{}%
\end{pgfscope}%
\begin{pgfscope}%
\pgfsys@transformshift{4.206178in}{0.460676in}%
\pgfsys@useobject{currentmarker}{}%
\end{pgfscope}%
\begin{pgfscope}%
\pgfsys@transformshift{4.258629in}{0.460676in}%
\pgfsys@useobject{currentmarker}{}%
\end{pgfscope}%
\begin{pgfscope}%
\pgfsys@transformshift{4.311080in}{0.460676in}%
\pgfsys@useobject{currentmarker}{}%
\end{pgfscope}%
\begin{pgfscope}%
\pgfsys@transformshift{4.363532in}{0.460676in}%
\pgfsys@useobject{currentmarker}{}%
\end{pgfscope}%
\begin{pgfscope}%
\pgfsys@transformshift{4.415983in}{0.460676in}%
\pgfsys@useobject{currentmarker}{}%
\end{pgfscope}%
\begin{pgfscope}%
\pgfsys@transformshift{4.468434in}{0.460676in}%
\pgfsys@useobject{currentmarker}{}%
\end{pgfscope}%
\begin{pgfscope}%
\pgfsys@transformshift{4.520885in}{0.460676in}%
\pgfsys@useobject{currentmarker}{}%
\end{pgfscope}%
\begin{pgfscope}%
\pgfsys@transformshift{4.573337in}{0.460676in}%
\pgfsys@useobject{currentmarker}{}%
\end{pgfscope}%
\begin{pgfscope}%
\pgfsys@transformshift{4.625788in}{0.460676in}%
\pgfsys@useobject{currentmarker}{}%
\end{pgfscope}%
\begin{pgfscope}%
\pgfsys@transformshift{4.678239in}{0.460676in}%
\pgfsys@useobject{currentmarker}{}%
\end{pgfscope}%
\begin{pgfscope}%
\pgfsys@transformshift{4.730691in}{0.460676in}%
\pgfsys@useobject{currentmarker}{}%
\end{pgfscope}%
\begin{pgfscope}%
\pgfsys@transformshift{4.783142in}{0.460676in}%
\pgfsys@useobject{currentmarker}{}%
\end{pgfscope}%
\begin{pgfscope}%
\pgfsys@transformshift{4.835593in}{0.460676in}%
\pgfsys@useobject{currentmarker}{}%
\end{pgfscope}%
\begin{pgfscope}%
\pgfsys@transformshift{4.888045in}{0.460676in}%
\pgfsys@useobject{currentmarker}{}%
\end{pgfscope}%
\begin{pgfscope}%
\pgfsys@transformshift{4.940496in}{0.460676in}%
\pgfsys@useobject{currentmarker}{}%
\end{pgfscope}%
\begin{pgfscope}%
\pgfsys@transformshift{4.992947in}{0.460676in}%
\pgfsys@useobject{currentmarker}{}%
\end{pgfscope}%
\begin{pgfscope}%
\pgfsys@transformshift{5.045398in}{0.460676in}%
\pgfsys@useobject{currentmarker}{}%
\end{pgfscope}%
\begin{pgfscope}%
\pgfsys@transformshift{5.097850in}{0.460676in}%
\pgfsys@useobject{currentmarker}{}%
\end{pgfscope}%
\end{pgfscope}%
\begin{pgfscope}%
\pgfpathrectangle{\pgfqpoint{0.581793in}{0.460676in}}{\pgfqpoint{4.731107in}{2.134248in}}%
\pgfusepath{clip}%
\pgfsetbuttcap%
\pgfsetroundjoin%
\definecolor{currentfill}{rgb}{1.000000,0.498039,0.054902}%
\pgfsetfillcolor{currentfill}%
\pgfsetfillopacity{0.750000}%
\pgfsetlinewidth{1.003750pt}%
\definecolor{currentstroke}{rgb}{1.000000,0.498039,0.054902}%
\pgfsetstrokecolor{currentstroke}%
\pgfsetstrokeopacity{0.750000}%
\pgfsetdash{}{0pt}%
\pgfsys@defobject{currentmarker}{\pgfqpoint{-0.019018in}{-0.019018in}}{\pgfqpoint{0.019018in}{0.019018in}}{%
\pgfpathmoveto{\pgfqpoint{-0.000000in}{-0.019018in}}%
\pgfpathlineto{\pgfqpoint{0.019018in}{0.019018in}}%
\pgfpathlineto{\pgfqpoint{-0.019018in}{0.019018in}}%
\pgfpathclose%
\pgfusepath{stroke,fill}%
}%
\begin{pgfscope}%
\pgfsys@transformshift{0.796844in}{2.475225in}%
\pgfsys@useobject{currentmarker}{}%
\end{pgfscope}%
\begin{pgfscope}%
\pgfsys@transformshift{0.849295in}{2.475225in}%
\pgfsys@useobject{currentmarker}{}%
\end{pgfscope}%
\begin{pgfscope}%
\pgfsys@transformshift{0.901746in}{2.475225in}%
\pgfsys@useobject{currentmarker}{}%
\end{pgfscope}%
\begin{pgfscope}%
\pgfsys@transformshift{0.954198in}{2.475225in}%
\pgfsys@useobject{currentmarker}{}%
\end{pgfscope}%
\begin{pgfscope}%
\pgfsys@transformshift{1.006649in}{2.475225in}%
\pgfsys@useobject{currentmarker}{}%
\end{pgfscope}%
\begin{pgfscope}%
\pgfsys@transformshift{1.059100in}{2.475225in}%
\pgfsys@useobject{currentmarker}{}%
\end{pgfscope}%
\begin{pgfscope}%
\pgfsys@transformshift{1.111551in}{2.475225in}%
\pgfsys@useobject{currentmarker}{}%
\end{pgfscope}%
\begin{pgfscope}%
\pgfsys@transformshift{1.164003in}{2.475225in}%
\pgfsys@useobject{currentmarker}{}%
\end{pgfscope}%
\begin{pgfscope}%
\pgfsys@transformshift{1.216454in}{2.107923in}%
\pgfsys@useobject{currentmarker}{}%
\end{pgfscope}%
\begin{pgfscope}%
\pgfsys@transformshift{1.268905in}{1.889785in}%
\pgfsys@useobject{currentmarker}{}%
\end{pgfscope}%
\begin{pgfscope}%
\pgfsys@transformshift{1.321357in}{1.454743in}%
\pgfsys@useobject{currentmarker}{}%
\end{pgfscope}%
\begin{pgfscope}%
\pgfsys@transformshift{1.373808in}{1.432681in}%
\pgfsys@useobject{currentmarker}{}%
\end{pgfscope}%
\begin{pgfscope}%
\pgfsys@transformshift{1.426259in}{0.861287in}%
\pgfsys@useobject{currentmarker}{}%
\end{pgfscope}%
\begin{pgfscope}%
\pgfsys@transformshift{1.478710in}{0.642774in}%
\pgfsys@useobject{currentmarker}{}%
\end{pgfscope}%
\begin{pgfscope}%
\pgfsys@transformshift{1.531162in}{0.461197in}%
\pgfsys@useobject{currentmarker}{}%
\end{pgfscope}%
\begin{pgfscope}%
\pgfsys@transformshift{1.583613in}{0.460676in}%
\pgfsys@useobject{currentmarker}{}%
\end{pgfscope}%
\begin{pgfscope}%
\pgfsys@transformshift{1.636064in}{0.460676in}%
\pgfsys@useobject{currentmarker}{}%
\end{pgfscope}%
\begin{pgfscope}%
\pgfsys@transformshift{1.688516in}{0.460676in}%
\pgfsys@useobject{currentmarker}{}%
\end{pgfscope}%
\begin{pgfscope}%
\pgfsys@transformshift{1.740967in}{0.460676in}%
\pgfsys@useobject{currentmarker}{}%
\end{pgfscope}%
\begin{pgfscope}%
\pgfsys@transformshift{1.793418in}{0.460676in}%
\pgfsys@useobject{currentmarker}{}%
\end{pgfscope}%
\begin{pgfscope}%
\pgfsys@transformshift{1.845870in}{0.460676in}%
\pgfsys@useobject{currentmarker}{}%
\end{pgfscope}%
\begin{pgfscope}%
\pgfsys@transformshift{1.898321in}{0.460676in}%
\pgfsys@useobject{currentmarker}{}%
\end{pgfscope}%
\begin{pgfscope}%
\pgfsys@transformshift{1.950772in}{0.460676in}%
\pgfsys@useobject{currentmarker}{}%
\end{pgfscope}%
\begin{pgfscope}%
\pgfsys@transformshift{2.003223in}{0.460676in}%
\pgfsys@useobject{currentmarker}{}%
\end{pgfscope}%
\begin{pgfscope}%
\pgfsys@transformshift{2.055675in}{0.460676in}%
\pgfsys@useobject{currentmarker}{}%
\end{pgfscope}%
\begin{pgfscope}%
\pgfsys@transformshift{2.108126in}{0.460676in}%
\pgfsys@useobject{currentmarker}{}%
\end{pgfscope}%
\begin{pgfscope}%
\pgfsys@transformshift{2.160577in}{0.460676in}%
\pgfsys@useobject{currentmarker}{}%
\end{pgfscope}%
\begin{pgfscope}%
\pgfsys@transformshift{2.213029in}{0.460676in}%
\pgfsys@useobject{currentmarker}{}%
\end{pgfscope}%
\begin{pgfscope}%
\pgfsys@transformshift{2.265480in}{0.460676in}%
\pgfsys@useobject{currentmarker}{}%
\end{pgfscope}%
\begin{pgfscope}%
\pgfsys@transformshift{2.317931in}{0.460676in}%
\pgfsys@useobject{currentmarker}{}%
\end{pgfscope}%
\begin{pgfscope}%
\pgfsys@transformshift{2.370382in}{0.460676in}%
\pgfsys@useobject{currentmarker}{}%
\end{pgfscope}%
\begin{pgfscope}%
\pgfsys@transformshift{2.422834in}{0.460676in}%
\pgfsys@useobject{currentmarker}{}%
\end{pgfscope}%
\begin{pgfscope}%
\pgfsys@transformshift{2.475285in}{0.460676in}%
\pgfsys@useobject{currentmarker}{}%
\end{pgfscope}%
\begin{pgfscope}%
\pgfsys@transformshift{2.527736in}{0.460676in}%
\pgfsys@useobject{currentmarker}{}%
\end{pgfscope}%
\begin{pgfscope}%
\pgfsys@transformshift{2.580188in}{0.460676in}%
\pgfsys@useobject{currentmarker}{}%
\end{pgfscope}%
\begin{pgfscope}%
\pgfsys@transformshift{2.632639in}{0.460676in}%
\pgfsys@useobject{currentmarker}{}%
\end{pgfscope}%
\begin{pgfscope}%
\pgfsys@transformshift{2.685090in}{0.460676in}%
\pgfsys@useobject{currentmarker}{}%
\end{pgfscope}%
\begin{pgfscope}%
\pgfsys@transformshift{2.737541in}{0.460676in}%
\pgfsys@useobject{currentmarker}{}%
\end{pgfscope}%
\begin{pgfscope}%
\pgfsys@transformshift{2.789993in}{0.460676in}%
\pgfsys@useobject{currentmarker}{}%
\end{pgfscope}%
\begin{pgfscope}%
\pgfsys@transformshift{2.842444in}{0.460676in}%
\pgfsys@useobject{currentmarker}{}%
\end{pgfscope}%
\begin{pgfscope}%
\pgfsys@transformshift{2.894895in}{0.460676in}%
\pgfsys@useobject{currentmarker}{}%
\end{pgfscope}%
\begin{pgfscope}%
\pgfsys@transformshift{2.947347in}{0.460676in}%
\pgfsys@useobject{currentmarker}{}%
\end{pgfscope}%
\begin{pgfscope}%
\pgfsys@transformshift{2.999798in}{0.460676in}%
\pgfsys@useobject{currentmarker}{}%
\end{pgfscope}%
\begin{pgfscope}%
\pgfsys@transformshift{3.052249in}{0.460676in}%
\pgfsys@useobject{currentmarker}{}%
\end{pgfscope}%
\begin{pgfscope}%
\pgfsys@transformshift{3.104701in}{0.460676in}%
\pgfsys@useobject{currentmarker}{}%
\end{pgfscope}%
\begin{pgfscope}%
\pgfsys@transformshift{3.157152in}{0.460676in}%
\pgfsys@useobject{currentmarker}{}%
\end{pgfscope}%
\begin{pgfscope}%
\pgfsys@transformshift{3.209603in}{0.460676in}%
\pgfsys@useobject{currentmarker}{}%
\end{pgfscope}%
\begin{pgfscope}%
\pgfsys@transformshift{3.262054in}{0.460676in}%
\pgfsys@useobject{currentmarker}{}%
\end{pgfscope}%
\begin{pgfscope}%
\pgfsys@transformshift{3.314506in}{0.460676in}%
\pgfsys@useobject{currentmarker}{}%
\end{pgfscope}%
\begin{pgfscope}%
\pgfsys@transformshift{3.366957in}{0.460676in}%
\pgfsys@useobject{currentmarker}{}%
\end{pgfscope}%
\begin{pgfscope}%
\pgfsys@transformshift{3.419408in}{0.460676in}%
\pgfsys@useobject{currentmarker}{}%
\end{pgfscope}%
\begin{pgfscope}%
\pgfsys@transformshift{3.471860in}{0.460676in}%
\pgfsys@useobject{currentmarker}{}%
\end{pgfscope}%
\begin{pgfscope}%
\pgfsys@transformshift{3.524311in}{0.460676in}%
\pgfsys@useobject{currentmarker}{}%
\end{pgfscope}%
\begin{pgfscope}%
\pgfsys@transformshift{3.576762in}{0.460676in}%
\pgfsys@useobject{currentmarker}{}%
\end{pgfscope}%
\begin{pgfscope}%
\pgfsys@transformshift{3.629213in}{0.460676in}%
\pgfsys@useobject{currentmarker}{}%
\end{pgfscope}%
\begin{pgfscope}%
\pgfsys@transformshift{3.681665in}{0.460676in}%
\pgfsys@useobject{currentmarker}{}%
\end{pgfscope}%
\begin{pgfscope}%
\pgfsys@transformshift{3.734116in}{0.460676in}%
\pgfsys@useobject{currentmarker}{}%
\end{pgfscope}%
\begin{pgfscope}%
\pgfsys@transformshift{3.786567in}{0.460676in}%
\pgfsys@useobject{currentmarker}{}%
\end{pgfscope}%
\begin{pgfscope}%
\pgfsys@transformshift{3.839019in}{0.460676in}%
\pgfsys@useobject{currentmarker}{}%
\end{pgfscope}%
\begin{pgfscope}%
\pgfsys@transformshift{3.891470in}{0.460676in}%
\pgfsys@useobject{currentmarker}{}%
\end{pgfscope}%
\begin{pgfscope}%
\pgfsys@transformshift{3.943921in}{0.460676in}%
\pgfsys@useobject{currentmarker}{}%
\end{pgfscope}%
\begin{pgfscope}%
\pgfsys@transformshift{3.996373in}{0.460676in}%
\pgfsys@useobject{currentmarker}{}%
\end{pgfscope}%
\begin{pgfscope}%
\pgfsys@transformshift{4.048824in}{0.460676in}%
\pgfsys@useobject{currentmarker}{}%
\end{pgfscope}%
\begin{pgfscope}%
\pgfsys@transformshift{4.101275in}{0.460676in}%
\pgfsys@useobject{currentmarker}{}%
\end{pgfscope}%
\begin{pgfscope}%
\pgfsys@transformshift{4.153726in}{0.460676in}%
\pgfsys@useobject{currentmarker}{}%
\end{pgfscope}%
\begin{pgfscope}%
\pgfsys@transformshift{4.206178in}{0.460676in}%
\pgfsys@useobject{currentmarker}{}%
\end{pgfscope}%
\begin{pgfscope}%
\pgfsys@transformshift{4.258629in}{0.460676in}%
\pgfsys@useobject{currentmarker}{}%
\end{pgfscope}%
\begin{pgfscope}%
\pgfsys@transformshift{4.311080in}{0.460676in}%
\pgfsys@useobject{currentmarker}{}%
\end{pgfscope}%
\begin{pgfscope}%
\pgfsys@transformshift{4.363532in}{0.460676in}%
\pgfsys@useobject{currentmarker}{}%
\end{pgfscope}%
\begin{pgfscope}%
\pgfsys@transformshift{4.415983in}{0.460676in}%
\pgfsys@useobject{currentmarker}{}%
\end{pgfscope}%
\begin{pgfscope}%
\pgfsys@transformshift{4.468434in}{0.460676in}%
\pgfsys@useobject{currentmarker}{}%
\end{pgfscope}%
\begin{pgfscope}%
\pgfsys@transformshift{4.520885in}{0.460676in}%
\pgfsys@useobject{currentmarker}{}%
\end{pgfscope}%
\begin{pgfscope}%
\pgfsys@transformshift{4.573337in}{0.460676in}%
\pgfsys@useobject{currentmarker}{}%
\end{pgfscope}%
\begin{pgfscope}%
\pgfsys@transformshift{4.625788in}{0.460676in}%
\pgfsys@useobject{currentmarker}{}%
\end{pgfscope}%
\begin{pgfscope}%
\pgfsys@transformshift{4.678239in}{0.460676in}%
\pgfsys@useobject{currentmarker}{}%
\end{pgfscope}%
\begin{pgfscope}%
\pgfsys@transformshift{4.730691in}{0.460676in}%
\pgfsys@useobject{currentmarker}{}%
\end{pgfscope}%
\begin{pgfscope}%
\pgfsys@transformshift{4.783142in}{0.460676in}%
\pgfsys@useobject{currentmarker}{}%
\end{pgfscope}%
\begin{pgfscope}%
\pgfsys@transformshift{4.835593in}{0.460676in}%
\pgfsys@useobject{currentmarker}{}%
\end{pgfscope}%
\begin{pgfscope}%
\pgfsys@transformshift{4.888045in}{0.460676in}%
\pgfsys@useobject{currentmarker}{}%
\end{pgfscope}%
\begin{pgfscope}%
\pgfsys@transformshift{4.940496in}{0.460676in}%
\pgfsys@useobject{currentmarker}{}%
\end{pgfscope}%
\begin{pgfscope}%
\pgfsys@transformshift{4.992947in}{0.460676in}%
\pgfsys@useobject{currentmarker}{}%
\end{pgfscope}%
\begin{pgfscope}%
\pgfsys@transformshift{5.045398in}{0.460676in}%
\pgfsys@useobject{currentmarker}{}%
\end{pgfscope}%
\begin{pgfscope}%
\pgfsys@transformshift{5.097850in}{0.460676in}%
\pgfsys@useobject{currentmarker}{}%
\end{pgfscope}%
\end{pgfscope}%
\begin{pgfscope}%
\pgfpathrectangle{\pgfqpoint{0.581793in}{0.460676in}}{\pgfqpoint{4.731107in}{2.134248in}}%
\pgfusepath{clip}%
\pgfsetbuttcap%
\pgfsetroundjoin%
\definecolor{currentfill}{rgb}{0.172549,0.627451,0.172549}%
\pgfsetfillcolor{currentfill}%
\pgfsetfillopacity{0.750000}%
\pgfsetlinewidth{1.003750pt}%
\definecolor{currentstroke}{rgb}{0.172549,0.627451,0.172549}%
\pgfsetstrokecolor{currentstroke}%
\pgfsetstrokeopacity{0.750000}%
\pgfsetdash{}{0pt}%
\pgfsys@defobject{currentmarker}{\pgfqpoint{-0.019018in}{-0.019018in}}{\pgfqpoint{0.019018in}{0.019018in}}{%
\pgfpathmoveto{\pgfqpoint{-0.019018in}{-0.019018in}}%
\pgfpathlineto{\pgfqpoint{0.019018in}{-0.019018in}}%
\pgfpathlineto{\pgfqpoint{0.019018in}{0.019018in}}%
\pgfpathlineto{\pgfqpoint{-0.019018in}{0.019018in}}%
\pgfpathclose%
\pgfusepath{stroke,fill}%
}%
\begin{pgfscope}%
\pgfsys@transformshift{0.796844in}{2.484279in}%
\pgfsys@useobject{currentmarker}{}%
\end{pgfscope}%
\begin{pgfscope}%
\pgfsys@transformshift{0.849295in}{2.484279in}%
\pgfsys@useobject{currentmarker}{}%
\end{pgfscope}%
\begin{pgfscope}%
\pgfsys@transformshift{0.901746in}{2.484279in}%
\pgfsys@useobject{currentmarker}{}%
\end{pgfscope}%
\begin{pgfscope}%
\pgfsys@transformshift{0.954198in}{2.484279in}%
\pgfsys@useobject{currentmarker}{}%
\end{pgfscope}%
\begin{pgfscope}%
\pgfsys@transformshift{1.006649in}{2.484279in}%
\pgfsys@useobject{currentmarker}{}%
\end{pgfscope}%
\begin{pgfscope}%
\pgfsys@transformshift{1.059100in}{2.484279in}%
\pgfsys@useobject{currentmarker}{}%
\end{pgfscope}%
\begin{pgfscope}%
\pgfsys@transformshift{1.111551in}{2.484279in}%
\pgfsys@useobject{currentmarker}{}%
\end{pgfscope}%
\begin{pgfscope}%
\pgfsys@transformshift{1.164003in}{2.484279in}%
\pgfsys@useobject{currentmarker}{}%
\end{pgfscope}%
\begin{pgfscope}%
\pgfsys@transformshift{1.216454in}{2.137805in}%
\pgfsys@useobject{currentmarker}{}%
\end{pgfscope}%
\begin{pgfscope}%
\pgfsys@transformshift{1.268905in}{1.454743in}%
\pgfsys@useobject{currentmarker}{}%
\end{pgfscope}%
\begin{pgfscope}%
\pgfsys@transformshift{1.321357in}{1.284239in}%
\pgfsys@useobject{currentmarker}{}%
\end{pgfscope}%
\begin{pgfscope}%
\pgfsys@transformshift{1.373808in}{0.996313in}%
\pgfsys@useobject{currentmarker}{}%
\end{pgfscope}%
\begin{pgfscope}%
\pgfsys@transformshift{1.426259in}{0.801651in}%
\pgfsys@useobject{currentmarker}{}%
\end{pgfscope}%
\begin{pgfscope}%
\pgfsys@transformshift{1.478710in}{0.770915in}%
\pgfsys@useobject{currentmarker}{}%
\end{pgfscope}%
\begin{pgfscope}%
\pgfsys@transformshift{1.531162in}{0.695373in}%
\pgfsys@useobject{currentmarker}{}%
\end{pgfscope}%
\begin{pgfscope}%
\pgfsys@transformshift{1.583613in}{0.525977in}%
\pgfsys@useobject{currentmarker}{}%
\end{pgfscope}%
\begin{pgfscope}%
\pgfsys@transformshift{1.636064in}{0.460676in}%
\pgfsys@useobject{currentmarker}{}%
\end{pgfscope}%
\begin{pgfscope}%
\pgfsys@transformshift{1.688516in}{0.460676in}%
\pgfsys@useobject{currentmarker}{}%
\end{pgfscope}%
\begin{pgfscope}%
\pgfsys@transformshift{1.740967in}{0.460676in}%
\pgfsys@useobject{currentmarker}{}%
\end{pgfscope}%
\begin{pgfscope}%
\pgfsys@transformshift{1.793418in}{0.460676in}%
\pgfsys@useobject{currentmarker}{}%
\end{pgfscope}%
\begin{pgfscope}%
\pgfsys@transformshift{1.845870in}{0.460676in}%
\pgfsys@useobject{currentmarker}{}%
\end{pgfscope}%
\begin{pgfscope}%
\pgfsys@transformshift{1.898321in}{0.460676in}%
\pgfsys@useobject{currentmarker}{}%
\end{pgfscope}%
\begin{pgfscope}%
\pgfsys@transformshift{1.950772in}{0.460676in}%
\pgfsys@useobject{currentmarker}{}%
\end{pgfscope}%
\begin{pgfscope}%
\pgfsys@transformshift{2.003223in}{0.460676in}%
\pgfsys@useobject{currentmarker}{}%
\end{pgfscope}%
\begin{pgfscope}%
\pgfsys@transformshift{2.055675in}{0.460676in}%
\pgfsys@useobject{currentmarker}{}%
\end{pgfscope}%
\begin{pgfscope}%
\pgfsys@transformshift{2.108126in}{0.460676in}%
\pgfsys@useobject{currentmarker}{}%
\end{pgfscope}%
\begin{pgfscope}%
\pgfsys@transformshift{2.160577in}{0.460676in}%
\pgfsys@useobject{currentmarker}{}%
\end{pgfscope}%
\begin{pgfscope}%
\pgfsys@transformshift{2.213029in}{0.460676in}%
\pgfsys@useobject{currentmarker}{}%
\end{pgfscope}%
\begin{pgfscope}%
\pgfsys@transformshift{2.265480in}{0.460676in}%
\pgfsys@useobject{currentmarker}{}%
\end{pgfscope}%
\begin{pgfscope}%
\pgfsys@transformshift{2.317931in}{0.460676in}%
\pgfsys@useobject{currentmarker}{}%
\end{pgfscope}%
\begin{pgfscope}%
\pgfsys@transformshift{2.370382in}{0.460676in}%
\pgfsys@useobject{currentmarker}{}%
\end{pgfscope}%
\begin{pgfscope}%
\pgfsys@transformshift{2.422834in}{0.460676in}%
\pgfsys@useobject{currentmarker}{}%
\end{pgfscope}%
\begin{pgfscope}%
\pgfsys@transformshift{2.475285in}{0.460676in}%
\pgfsys@useobject{currentmarker}{}%
\end{pgfscope}%
\begin{pgfscope}%
\pgfsys@transformshift{2.527736in}{0.460676in}%
\pgfsys@useobject{currentmarker}{}%
\end{pgfscope}%
\begin{pgfscope}%
\pgfsys@transformshift{2.580188in}{0.460676in}%
\pgfsys@useobject{currentmarker}{}%
\end{pgfscope}%
\begin{pgfscope}%
\pgfsys@transformshift{2.632639in}{0.460676in}%
\pgfsys@useobject{currentmarker}{}%
\end{pgfscope}%
\begin{pgfscope}%
\pgfsys@transformshift{2.685090in}{0.460676in}%
\pgfsys@useobject{currentmarker}{}%
\end{pgfscope}%
\begin{pgfscope}%
\pgfsys@transformshift{2.737541in}{0.460676in}%
\pgfsys@useobject{currentmarker}{}%
\end{pgfscope}%
\begin{pgfscope}%
\pgfsys@transformshift{2.789993in}{0.460676in}%
\pgfsys@useobject{currentmarker}{}%
\end{pgfscope}%
\begin{pgfscope}%
\pgfsys@transformshift{2.842444in}{0.460676in}%
\pgfsys@useobject{currentmarker}{}%
\end{pgfscope}%
\begin{pgfscope}%
\pgfsys@transformshift{2.894895in}{0.460676in}%
\pgfsys@useobject{currentmarker}{}%
\end{pgfscope}%
\begin{pgfscope}%
\pgfsys@transformshift{2.947347in}{0.460676in}%
\pgfsys@useobject{currentmarker}{}%
\end{pgfscope}%
\begin{pgfscope}%
\pgfsys@transformshift{2.999798in}{0.460676in}%
\pgfsys@useobject{currentmarker}{}%
\end{pgfscope}%
\begin{pgfscope}%
\pgfsys@transformshift{3.052249in}{0.460676in}%
\pgfsys@useobject{currentmarker}{}%
\end{pgfscope}%
\begin{pgfscope}%
\pgfsys@transformshift{3.104701in}{0.460676in}%
\pgfsys@useobject{currentmarker}{}%
\end{pgfscope}%
\begin{pgfscope}%
\pgfsys@transformshift{3.157152in}{0.460676in}%
\pgfsys@useobject{currentmarker}{}%
\end{pgfscope}%
\begin{pgfscope}%
\pgfsys@transformshift{3.209603in}{0.460676in}%
\pgfsys@useobject{currentmarker}{}%
\end{pgfscope}%
\begin{pgfscope}%
\pgfsys@transformshift{3.262054in}{0.460676in}%
\pgfsys@useobject{currentmarker}{}%
\end{pgfscope}%
\begin{pgfscope}%
\pgfsys@transformshift{3.314506in}{0.460676in}%
\pgfsys@useobject{currentmarker}{}%
\end{pgfscope}%
\begin{pgfscope}%
\pgfsys@transformshift{3.366957in}{0.460676in}%
\pgfsys@useobject{currentmarker}{}%
\end{pgfscope}%
\begin{pgfscope}%
\pgfsys@transformshift{3.419408in}{0.460676in}%
\pgfsys@useobject{currentmarker}{}%
\end{pgfscope}%
\begin{pgfscope}%
\pgfsys@transformshift{3.471860in}{0.460676in}%
\pgfsys@useobject{currentmarker}{}%
\end{pgfscope}%
\begin{pgfscope}%
\pgfsys@transformshift{3.524311in}{0.460676in}%
\pgfsys@useobject{currentmarker}{}%
\end{pgfscope}%
\begin{pgfscope}%
\pgfsys@transformshift{3.576762in}{0.460676in}%
\pgfsys@useobject{currentmarker}{}%
\end{pgfscope}%
\begin{pgfscope}%
\pgfsys@transformshift{3.629213in}{0.460676in}%
\pgfsys@useobject{currentmarker}{}%
\end{pgfscope}%
\begin{pgfscope}%
\pgfsys@transformshift{3.681665in}{0.460676in}%
\pgfsys@useobject{currentmarker}{}%
\end{pgfscope}%
\begin{pgfscope}%
\pgfsys@transformshift{3.734116in}{0.460676in}%
\pgfsys@useobject{currentmarker}{}%
\end{pgfscope}%
\begin{pgfscope}%
\pgfsys@transformshift{3.786567in}{0.460676in}%
\pgfsys@useobject{currentmarker}{}%
\end{pgfscope}%
\begin{pgfscope}%
\pgfsys@transformshift{3.839019in}{0.460676in}%
\pgfsys@useobject{currentmarker}{}%
\end{pgfscope}%
\begin{pgfscope}%
\pgfsys@transformshift{3.891470in}{0.460676in}%
\pgfsys@useobject{currentmarker}{}%
\end{pgfscope}%
\begin{pgfscope}%
\pgfsys@transformshift{3.943921in}{0.460676in}%
\pgfsys@useobject{currentmarker}{}%
\end{pgfscope}%
\begin{pgfscope}%
\pgfsys@transformshift{3.996373in}{0.460676in}%
\pgfsys@useobject{currentmarker}{}%
\end{pgfscope}%
\begin{pgfscope}%
\pgfsys@transformshift{4.048824in}{0.460676in}%
\pgfsys@useobject{currentmarker}{}%
\end{pgfscope}%
\begin{pgfscope}%
\pgfsys@transformshift{4.101275in}{0.460676in}%
\pgfsys@useobject{currentmarker}{}%
\end{pgfscope}%
\begin{pgfscope}%
\pgfsys@transformshift{4.153726in}{0.460676in}%
\pgfsys@useobject{currentmarker}{}%
\end{pgfscope}%
\begin{pgfscope}%
\pgfsys@transformshift{4.206178in}{0.460676in}%
\pgfsys@useobject{currentmarker}{}%
\end{pgfscope}%
\begin{pgfscope}%
\pgfsys@transformshift{4.258629in}{0.460676in}%
\pgfsys@useobject{currentmarker}{}%
\end{pgfscope}%
\begin{pgfscope}%
\pgfsys@transformshift{4.311080in}{0.460676in}%
\pgfsys@useobject{currentmarker}{}%
\end{pgfscope}%
\begin{pgfscope}%
\pgfsys@transformshift{4.363532in}{0.460676in}%
\pgfsys@useobject{currentmarker}{}%
\end{pgfscope}%
\begin{pgfscope}%
\pgfsys@transformshift{4.415983in}{0.460676in}%
\pgfsys@useobject{currentmarker}{}%
\end{pgfscope}%
\begin{pgfscope}%
\pgfsys@transformshift{4.468434in}{0.460676in}%
\pgfsys@useobject{currentmarker}{}%
\end{pgfscope}%
\begin{pgfscope}%
\pgfsys@transformshift{4.520885in}{0.460676in}%
\pgfsys@useobject{currentmarker}{}%
\end{pgfscope}%
\begin{pgfscope}%
\pgfsys@transformshift{4.573337in}{0.460676in}%
\pgfsys@useobject{currentmarker}{}%
\end{pgfscope}%
\begin{pgfscope}%
\pgfsys@transformshift{4.625788in}{0.460676in}%
\pgfsys@useobject{currentmarker}{}%
\end{pgfscope}%
\begin{pgfscope}%
\pgfsys@transformshift{4.678239in}{0.460676in}%
\pgfsys@useobject{currentmarker}{}%
\end{pgfscope}%
\begin{pgfscope}%
\pgfsys@transformshift{4.730691in}{0.460676in}%
\pgfsys@useobject{currentmarker}{}%
\end{pgfscope}%
\begin{pgfscope}%
\pgfsys@transformshift{4.783142in}{0.460676in}%
\pgfsys@useobject{currentmarker}{}%
\end{pgfscope}%
\begin{pgfscope}%
\pgfsys@transformshift{4.835593in}{0.460676in}%
\pgfsys@useobject{currentmarker}{}%
\end{pgfscope}%
\begin{pgfscope}%
\pgfsys@transformshift{4.888045in}{0.460676in}%
\pgfsys@useobject{currentmarker}{}%
\end{pgfscope}%
\begin{pgfscope}%
\pgfsys@transformshift{4.940496in}{0.460676in}%
\pgfsys@useobject{currentmarker}{}%
\end{pgfscope}%
\begin{pgfscope}%
\pgfsys@transformshift{4.992947in}{0.460676in}%
\pgfsys@useobject{currentmarker}{}%
\end{pgfscope}%
\begin{pgfscope}%
\pgfsys@transformshift{5.045398in}{0.460676in}%
\pgfsys@useobject{currentmarker}{}%
\end{pgfscope}%
\begin{pgfscope}%
\pgfsys@transformshift{5.097850in}{0.460676in}%
\pgfsys@useobject{currentmarker}{}%
\end{pgfscope}%
\end{pgfscope}%
\begin{pgfscope}%
\pgfpathrectangle{\pgfqpoint{0.581793in}{0.460676in}}{\pgfqpoint{4.731107in}{2.134248in}}%
\pgfusepath{clip}%
\pgfsetbuttcap%
\pgfsetroundjoin%
\definecolor{currentfill}{rgb}{0.839216,0.152941,0.156863}%
\pgfsetfillcolor{currentfill}%
\pgfsetfillopacity{0.750000}%
\pgfsetlinewidth{1.003750pt}%
\definecolor{currentstroke}{rgb}{0.839216,0.152941,0.156863}%
\pgfsetstrokecolor{currentstroke}%
\pgfsetstrokeopacity{0.750000}%
\pgfsetdash{}{0pt}%
\pgfsys@defobject{currentmarker}{\pgfqpoint{-0.019018in}{-0.019018in}}{\pgfqpoint{0.019018in}{0.019018in}}{%
\pgfpathmoveto{\pgfqpoint{0.000000in}{0.019018in}}%
\pgfpathlineto{\pgfqpoint{-0.019018in}{-0.019018in}}%
\pgfpathlineto{\pgfqpoint{0.019018in}{-0.019018in}}%
\pgfpathclose%
\pgfusepath{stroke,fill}%
}%
\begin{pgfscope}%
\pgfsys@transformshift{0.796844in}{2.493293in}%
\pgfsys@useobject{currentmarker}{}%
\end{pgfscope}%
\begin{pgfscope}%
\pgfsys@transformshift{0.849295in}{2.493293in}%
\pgfsys@useobject{currentmarker}{}%
\end{pgfscope}%
\begin{pgfscope}%
\pgfsys@transformshift{0.901746in}{2.493293in}%
\pgfsys@useobject{currentmarker}{}%
\end{pgfscope}%
\begin{pgfscope}%
\pgfsys@transformshift{0.954198in}{2.493293in}%
\pgfsys@useobject{currentmarker}{}%
\end{pgfscope}%
\begin{pgfscope}%
\pgfsys@transformshift{1.006649in}{2.493293in}%
\pgfsys@useobject{currentmarker}{}%
\end{pgfscope}%
\begin{pgfscope}%
\pgfsys@transformshift{1.059100in}{2.493293in}%
\pgfsys@useobject{currentmarker}{}%
\end{pgfscope}%
\begin{pgfscope}%
\pgfsys@transformshift{1.111551in}{2.493293in}%
\pgfsys@useobject{currentmarker}{}%
\end{pgfscope}%
\begin{pgfscope}%
\pgfsys@transformshift{1.164003in}{2.493293in}%
\pgfsys@useobject{currentmarker}{}%
\end{pgfscope}%
\begin{pgfscope}%
\pgfsys@transformshift{1.216454in}{2.493293in}%
\pgfsys@useobject{currentmarker}{}%
\end{pgfscope}%
\begin{pgfscope}%
\pgfsys@transformshift{1.268905in}{2.137075in}%
\pgfsys@useobject{currentmarker}{}%
\end{pgfscope}%
\begin{pgfscope}%
\pgfsys@transformshift{1.321357in}{2.083919in}%
\pgfsys@useobject{currentmarker}{}%
\end{pgfscope}%
\begin{pgfscope}%
\pgfsys@transformshift{1.373808in}{2.035925in}%
\pgfsys@useobject{currentmarker}{}%
\end{pgfscope}%
\begin{pgfscope}%
\pgfsys@transformshift{1.426259in}{1.824829in}%
\pgfsys@useobject{currentmarker}{}%
\end{pgfscope}%
\begin{pgfscope}%
\pgfsys@transformshift{1.478710in}{1.325830in}%
\pgfsys@useobject{currentmarker}{}%
\end{pgfscope}%
\begin{pgfscope}%
\pgfsys@transformshift{1.531162in}{1.276737in}%
\pgfsys@useobject{currentmarker}{}%
\end{pgfscope}%
\begin{pgfscope}%
\pgfsys@transformshift{1.583613in}{0.876017in}%
\pgfsys@useobject{currentmarker}{}%
\end{pgfscope}%
\begin{pgfscope}%
\pgfsys@transformshift{1.636064in}{0.858250in}%
\pgfsys@useobject{currentmarker}{}%
\end{pgfscope}%
\begin{pgfscope}%
\pgfsys@transformshift{1.688516in}{0.835675in}%
\pgfsys@useobject{currentmarker}{}%
\end{pgfscope}%
\begin{pgfscope}%
\pgfsys@transformshift{1.740967in}{0.835675in}%
\pgfsys@useobject{currentmarker}{}%
\end{pgfscope}%
\begin{pgfscope}%
\pgfsys@transformshift{1.793418in}{0.835675in}%
\pgfsys@useobject{currentmarker}{}%
\end{pgfscope}%
\begin{pgfscope}%
\pgfsys@transformshift{1.845870in}{0.835675in}%
\pgfsys@useobject{currentmarker}{}%
\end{pgfscope}%
\begin{pgfscope}%
\pgfsys@transformshift{1.898321in}{0.835675in}%
\pgfsys@useobject{currentmarker}{}%
\end{pgfscope}%
\begin{pgfscope}%
\pgfsys@transformshift{1.950772in}{0.835675in}%
\pgfsys@useobject{currentmarker}{}%
\end{pgfscope}%
\begin{pgfscope}%
\pgfsys@transformshift{2.003223in}{0.835675in}%
\pgfsys@useobject{currentmarker}{}%
\end{pgfscope}%
\begin{pgfscope}%
\pgfsys@transformshift{2.055675in}{0.835675in}%
\pgfsys@useobject{currentmarker}{}%
\end{pgfscope}%
\begin{pgfscope}%
\pgfsys@transformshift{2.108126in}{0.835675in}%
\pgfsys@useobject{currentmarker}{}%
\end{pgfscope}%
\begin{pgfscope}%
\pgfsys@transformshift{2.160577in}{0.835675in}%
\pgfsys@useobject{currentmarker}{}%
\end{pgfscope}%
\begin{pgfscope}%
\pgfsys@transformshift{2.213029in}{0.835675in}%
\pgfsys@useobject{currentmarker}{}%
\end{pgfscope}%
\begin{pgfscope}%
\pgfsys@transformshift{2.265480in}{0.835675in}%
\pgfsys@useobject{currentmarker}{}%
\end{pgfscope}%
\begin{pgfscope}%
\pgfsys@transformshift{2.317931in}{0.835675in}%
\pgfsys@useobject{currentmarker}{}%
\end{pgfscope}%
\begin{pgfscope}%
\pgfsys@transformshift{2.370382in}{0.835675in}%
\pgfsys@useobject{currentmarker}{}%
\end{pgfscope}%
\begin{pgfscope}%
\pgfsys@transformshift{2.422834in}{0.835675in}%
\pgfsys@useobject{currentmarker}{}%
\end{pgfscope}%
\begin{pgfscope}%
\pgfsys@transformshift{2.475285in}{0.835675in}%
\pgfsys@useobject{currentmarker}{}%
\end{pgfscope}%
\begin{pgfscope}%
\pgfsys@transformshift{2.527736in}{0.835675in}%
\pgfsys@useobject{currentmarker}{}%
\end{pgfscope}%
\begin{pgfscope}%
\pgfsys@transformshift{2.580188in}{0.835675in}%
\pgfsys@useobject{currentmarker}{}%
\end{pgfscope}%
\begin{pgfscope}%
\pgfsys@transformshift{2.632639in}{0.835675in}%
\pgfsys@useobject{currentmarker}{}%
\end{pgfscope}%
\begin{pgfscope}%
\pgfsys@transformshift{2.685090in}{0.835675in}%
\pgfsys@useobject{currentmarker}{}%
\end{pgfscope}%
\begin{pgfscope}%
\pgfsys@transformshift{2.737541in}{0.835675in}%
\pgfsys@useobject{currentmarker}{}%
\end{pgfscope}%
\begin{pgfscope}%
\pgfsys@transformshift{2.789993in}{0.835675in}%
\pgfsys@useobject{currentmarker}{}%
\end{pgfscope}%
\begin{pgfscope}%
\pgfsys@transformshift{2.842444in}{0.835675in}%
\pgfsys@useobject{currentmarker}{}%
\end{pgfscope}%
\begin{pgfscope}%
\pgfsys@transformshift{2.894895in}{0.835675in}%
\pgfsys@useobject{currentmarker}{}%
\end{pgfscope}%
\begin{pgfscope}%
\pgfsys@transformshift{2.947347in}{0.835675in}%
\pgfsys@useobject{currentmarker}{}%
\end{pgfscope}%
\begin{pgfscope}%
\pgfsys@transformshift{2.999798in}{0.835675in}%
\pgfsys@useobject{currentmarker}{}%
\end{pgfscope}%
\begin{pgfscope}%
\pgfsys@transformshift{3.052249in}{0.835675in}%
\pgfsys@useobject{currentmarker}{}%
\end{pgfscope}%
\begin{pgfscope}%
\pgfsys@transformshift{3.104701in}{0.835675in}%
\pgfsys@useobject{currentmarker}{}%
\end{pgfscope}%
\begin{pgfscope}%
\pgfsys@transformshift{3.157152in}{0.835675in}%
\pgfsys@useobject{currentmarker}{}%
\end{pgfscope}%
\begin{pgfscope}%
\pgfsys@transformshift{3.209603in}{0.835675in}%
\pgfsys@useobject{currentmarker}{}%
\end{pgfscope}%
\begin{pgfscope}%
\pgfsys@transformshift{3.262054in}{0.835675in}%
\pgfsys@useobject{currentmarker}{}%
\end{pgfscope}%
\begin{pgfscope}%
\pgfsys@transformshift{3.314506in}{0.835675in}%
\pgfsys@useobject{currentmarker}{}%
\end{pgfscope}%
\begin{pgfscope}%
\pgfsys@transformshift{3.366957in}{0.835675in}%
\pgfsys@useobject{currentmarker}{}%
\end{pgfscope}%
\begin{pgfscope}%
\pgfsys@transformshift{3.419408in}{0.835675in}%
\pgfsys@useobject{currentmarker}{}%
\end{pgfscope}%
\begin{pgfscope}%
\pgfsys@transformshift{3.471860in}{0.835675in}%
\pgfsys@useobject{currentmarker}{}%
\end{pgfscope}%
\begin{pgfscope}%
\pgfsys@transformshift{3.524311in}{0.835675in}%
\pgfsys@useobject{currentmarker}{}%
\end{pgfscope}%
\begin{pgfscope}%
\pgfsys@transformshift{3.576762in}{0.835675in}%
\pgfsys@useobject{currentmarker}{}%
\end{pgfscope}%
\begin{pgfscope}%
\pgfsys@transformshift{3.629213in}{0.585253in}%
\pgfsys@useobject{currentmarker}{}%
\end{pgfscope}%
\begin{pgfscope}%
\pgfsys@transformshift{3.681665in}{0.553036in}%
\pgfsys@useobject{currentmarker}{}%
\end{pgfscope}%
\begin{pgfscope}%
\pgfsys@transformshift{3.734116in}{0.467134in}%
\pgfsys@useobject{currentmarker}{}%
\end{pgfscope}%
\begin{pgfscope}%
\pgfsys@transformshift{3.786567in}{0.460676in}%
\pgfsys@useobject{currentmarker}{}%
\end{pgfscope}%
\begin{pgfscope}%
\pgfsys@transformshift{3.839019in}{0.460676in}%
\pgfsys@useobject{currentmarker}{}%
\end{pgfscope}%
\begin{pgfscope}%
\pgfsys@transformshift{3.891470in}{0.460676in}%
\pgfsys@useobject{currentmarker}{}%
\end{pgfscope}%
\begin{pgfscope}%
\pgfsys@transformshift{3.943921in}{0.460676in}%
\pgfsys@useobject{currentmarker}{}%
\end{pgfscope}%
\begin{pgfscope}%
\pgfsys@transformshift{3.996373in}{0.460676in}%
\pgfsys@useobject{currentmarker}{}%
\end{pgfscope}%
\begin{pgfscope}%
\pgfsys@transformshift{4.048824in}{0.460676in}%
\pgfsys@useobject{currentmarker}{}%
\end{pgfscope}%
\begin{pgfscope}%
\pgfsys@transformshift{4.101275in}{0.460676in}%
\pgfsys@useobject{currentmarker}{}%
\end{pgfscope}%
\begin{pgfscope}%
\pgfsys@transformshift{4.153726in}{0.460676in}%
\pgfsys@useobject{currentmarker}{}%
\end{pgfscope}%
\begin{pgfscope}%
\pgfsys@transformshift{4.206178in}{0.460676in}%
\pgfsys@useobject{currentmarker}{}%
\end{pgfscope}%
\begin{pgfscope}%
\pgfsys@transformshift{4.258629in}{0.460676in}%
\pgfsys@useobject{currentmarker}{}%
\end{pgfscope}%
\begin{pgfscope}%
\pgfsys@transformshift{4.311080in}{0.460676in}%
\pgfsys@useobject{currentmarker}{}%
\end{pgfscope}%
\begin{pgfscope}%
\pgfsys@transformshift{4.363532in}{0.460676in}%
\pgfsys@useobject{currentmarker}{}%
\end{pgfscope}%
\begin{pgfscope}%
\pgfsys@transformshift{4.415983in}{0.460676in}%
\pgfsys@useobject{currentmarker}{}%
\end{pgfscope}%
\begin{pgfscope}%
\pgfsys@transformshift{4.468434in}{0.460676in}%
\pgfsys@useobject{currentmarker}{}%
\end{pgfscope}%
\begin{pgfscope}%
\pgfsys@transformshift{4.520885in}{0.460676in}%
\pgfsys@useobject{currentmarker}{}%
\end{pgfscope}%
\begin{pgfscope}%
\pgfsys@transformshift{4.573337in}{0.460676in}%
\pgfsys@useobject{currentmarker}{}%
\end{pgfscope}%
\begin{pgfscope}%
\pgfsys@transformshift{4.625788in}{0.460676in}%
\pgfsys@useobject{currentmarker}{}%
\end{pgfscope}%
\begin{pgfscope}%
\pgfsys@transformshift{4.678239in}{0.460676in}%
\pgfsys@useobject{currentmarker}{}%
\end{pgfscope}%
\begin{pgfscope}%
\pgfsys@transformshift{4.730691in}{0.460676in}%
\pgfsys@useobject{currentmarker}{}%
\end{pgfscope}%
\begin{pgfscope}%
\pgfsys@transformshift{4.783142in}{0.460676in}%
\pgfsys@useobject{currentmarker}{}%
\end{pgfscope}%
\begin{pgfscope}%
\pgfsys@transformshift{4.835593in}{0.460676in}%
\pgfsys@useobject{currentmarker}{}%
\end{pgfscope}%
\begin{pgfscope}%
\pgfsys@transformshift{4.888045in}{0.460676in}%
\pgfsys@useobject{currentmarker}{}%
\end{pgfscope}%
\begin{pgfscope}%
\pgfsys@transformshift{4.940496in}{0.460676in}%
\pgfsys@useobject{currentmarker}{}%
\end{pgfscope}%
\begin{pgfscope}%
\pgfsys@transformshift{4.992947in}{0.460676in}%
\pgfsys@useobject{currentmarker}{}%
\end{pgfscope}%
\begin{pgfscope}%
\pgfsys@transformshift{5.045398in}{0.460676in}%
\pgfsys@useobject{currentmarker}{}%
\end{pgfscope}%
\begin{pgfscope}%
\pgfsys@transformshift{5.097850in}{0.460676in}%
\pgfsys@useobject{currentmarker}{}%
\end{pgfscope}%
\end{pgfscope}%
\begin{pgfscope}%
\pgfpathrectangle{\pgfqpoint{0.581793in}{0.460676in}}{\pgfqpoint{4.731107in}{2.134248in}}%
\pgfusepath{clip}%
\pgfsetrectcap%
\pgfsetroundjoin%
\pgfsetlinewidth{1.003750pt}%
\definecolor{currentstroke}{rgb}{0.000000,0.000000,0.000000}%
\pgfsetstrokecolor{currentstroke}%
\pgfsetdash{}{0pt}%
\pgfpathmoveto{\pgfqpoint{0.581793in}{2.372801in}}%
\pgfpathlineto{\pgfqpoint{5.312900in}{2.372801in}}%
\pgfusepath{stroke}%
\end{pgfscope}%
\begin{pgfscope}%
\pgfsetrectcap%
\pgfsetmiterjoin%
\pgfsetlinewidth{0.803000pt}%
\definecolor{currentstroke}{rgb}{0.000000,0.000000,0.000000}%
\pgfsetstrokecolor{currentstroke}%
\pgfsetdash{}{0pt}%
\pgfpathmoveto{\pgfqpoint{0.581793in}{0.460676in}}%
\pgfpathlineto{\pgfqpoint{0.581793in}{2.594924in}}%
\pgfusepath{stroke}%
\end{pgfscope}%
\begin{pgfscope}%
\pgfsetrectcap%
\pgfsetmiterjoin%
\pgfsetlinewidth{0.803000pt}%
\definecolor{currentstroke}{rgb}{0.000000,0.000000,0.000000}%
\pgfsetstrokecolor{currentstroke}%
\pgfsetdash{}{0pt}%
\pgfpathmoveto{\pgfqpoint{5.312900in}{0.460676in}}%
\pgfpathlineto{\pgfqpoint{5.312900in}{2.594924in}}%
\pgfusepath{stroke}%
\end{pgfscope}%
\begin{pgfscope}%
\pgfsetrectcap%
\pgfsetmiterjoin%
\pgfsetlinewidth{0.803000pt}%
\definecolor{currentstroke}{rgb}{0.000000,0.000000,0.000000}%
\pgfsetstrokecolor{currentstroke}%
\pgfsetdash{}{0pt}%
\pgfpathmoveto{\pgfqpoint{0.581793in}{0.460676in}}%
\pgfpathlineto{\pgfqpoint{5.312900in}{0.460676in}}%
\pgfusepath{stroke}%
\end{pgfscope}%
\begin{pgfscope}%
\pgfsetrectcap%
\pgfsetmiterjoin%
\pgfsetlinewidth{0.803000pt}%
\definecolor{currentstroke}{rgb}{0.000000,0.000000,0.000000}%
\pgfsetstrokecolor{currentstroke}%
\pgfsetdash{}{0pt}%
\pgfpathmoveto{\pgfqpoint{0.581793in}{2.594924in}}%
\pgfpathlineto{\pgfqpoint{5.312900in}{2.594924in}}%
\pgfusepath{stroke}%
\end{pgfscope}%
\begin{pgfscope}%
\pgfsetbuttcap%
\pgfsetroundjoin%
\definecolor{currentfill}{rgb}{0.000000,0.000000,0.000000}%
\pgfsetfillcolor{currentfill}%
\pgfsetlinewidth{0.803000pt}%
\definecolor{currentstroke}{rgb}{0.000000,0.000000,0.000000}%
\pgfsetstrokecolor{currentstroke}%
\pgfsetdash{}{0pt}%
\pgfsys@defobject{currentmarker}{\pgfqpoint{0.000000in}{-0.048611in}}{\pgfqpoint{0.000000in}{0.000000in}}{%
\pgfpathmoveto{\pgfqpoint{0.000000in}{0.000000in}}%
\pgfpathlineto{\pgfqpoint{0.000000in}{-0.048611in}}%
\pgfusepath{stroke,fill}%
}%
\begin{pgfscope}%
\pgfsys@transformshift{0.744392in}{0.460676in}%
\pgfsys@useobject{currentmarker}{}%
\end{pgfscope}%
\end{pgfscope}%
\begin{pgfscope}%
\definecolor{textcolor}{rgb}{0.000000,0.000000,0.000000}%
\pgfsetstrokecolor{textcolor}%
\pgfsetfillcolor{textcolor}%
\pgftext[x=0.744392in,y=0.363454in,,top]{\color{textcolor}\rmfamily\fontsize{8.330000}{9.996000}\selectfont \(\displaystyle 0\)}%
\end{pgfscope}%
\begin{pgfscope}%
\pgfsetbuttcap%
\pgfsetroundjoin%
\definecolor{currentfill}{rgb}{0.000000,0.000000,0.000000}%
\pgfsetfillcolor{currentfill}%
\pgfsetlinewidth{0.803000pt}%
\definecolor{currentstroke}{rgb}{0.000000,0.000000,0.000000}%
\pgfsetstrokecolor{currentstroke}%
\pgfsetdash{}{0pt}%
\pgfsys@defobject{currentmarker}{\pgfqpoint{0.000000in}{-0.048611in}}{\pgfqpoint{0.000000in}{0.000000in}}{%
\pgfpathmoveto{\pgfqpoint{0.000000in}{0.000000in}}%
\pgfpathlineto{\pgfqpoint{0.000000in}{-0.048611in}}%
\pgfusepath{stroke,fill}%
}%
\begin{pgfscope}%
\pgfsys@transformshift{1.793418in}{0.460676in}%
\pgfsys@useobject{currentmarker}{}%
\end{pgfscope}%
\end{pgfscope}%
\begin{pgfscope}%
\definecolor{textcolor}{rgb}{0.000000,0.000000,0.000000}%
\pgfsetstrokecolor{textcolor}%
\pgfsetfillcolor{textcolor}%
\pgftext[x=1.793418in,y=0.363454in,,top]{\color{textcolor}\rmfamily\fontsize{8.330000}{9.996000}\selectfont \(\displaystyle 20\)}%
\end{pgfscope}%
\begin{pgfscope}%
\pgfsetbuttcap%
\pgfsetroundjoin%
\definecolor{currentfill}{rgb}{0.000000,0.000000,0.000000}%
\pgfsetfillcolor{currentfill}%
\pgfsetlinewidth{0.803000pt}%
\definecolor{currentstroke}{rgb}{0.000000,0.000000,0.000000}%
\pgfsetstrokecolor{currentstroke}%
\pgfsetdash{}{0pt}%
\pgfsys@defobject{currentmarker}{\pgfqpoint{0.000000in}{-0.048611in}}{\pgfqpoint{0.000000in}{0.000000in}}{%
\pgfpathmoveto{\pgfqpoint{0.000000in}{0.000000in}}%
\pgfpathlineto{\pgfqpoint{0.000000in}{-0.048611in}}%
\pgfusepath{stroke,fill}%
}%
\begin{pgfscope}%
\pgfsys@transformshift{2.842444in}{0.460676in}%
\pgfsys@useobject{currentmarker}{}%
\end{pgfscope}%
\end{pgfscope}%
\begin{pgfscope}%
\definecolor{textcolor}{rgb}{0.000000,0.000000,0.000000}%
\pgfsetstrokecolor{textcolor}%
\pgfsetfillcolor{textcolor}%
\pgftext[x=2.842444in,y=0.363454in,,top]{\color{textcolor}\rmfamily\fontsize{8.330000}{9.996000}\selectfont \(\displaystyle 40\)}%
\end{pgfscope}%
\begin{pgfscope}%
\pgfsetbuttcap%
\pgfsetroundjoin%
\definecolor{currentfill}{rgb}{0.000000,0.000000,0.000000}%
\pgfsetfillcolor{currentfill}%
\pgfsetlinewidth{0.803000pt}%
\definecolor{currentstroke}{rgb}{0.000000,0.000000,0.000000}%
\pgfsetstrokecolor{currentstroke}%
\pgfsetdash{}{0pt}%
\pgfsys@defobject{currentmarker}{\pgfqpoint{0.000000in}{-0.048611in}}{\pgfqpoint{0.000000in}{0.000000in}}{%
\pgfpathmoveto{\pgfqpoint{0.000000in}{0.000000in}}%
\pgfpathlineto{\pgfqpoint{0.000000in}{-0.048611in}}%
\pgfusepath{stroke,fill}%
}%
\begin{pgfscope}%
\pgfsys@transformshift{3.891470in}{0.460676in}%
\pgfsys@useobject{currentmarker}{}%
\end{pgfscope}%
\end{pgfscope}%
\begin{pgfscope}%
\definecolor{textcolor}{rgb}{0.000000,0.000000,0.000000}%
\pgfsetstrokecolor{textcolor}%
\pgfsetfillcolor{textcolor}%
\pgftext[x=3.891470in,y=0.363454in,,top]{\color{textcolor}\rmfamily\fontsize{8.330000}{9.996000}\selectfont \(\displaystyle 60\)}%
\end{pgfscope}%
\begin{pgfscope}%
\pgfsetbuttcap%
\pgfsetroundjoin%
\definecolor{currentfill}{rgb}{0.000000,0.000000,0.000000}%
\pgfsetfillcolor{currentfill}%
\pgfsetlinewidth{0.803000pt}%
\definecolor{currentstroke}{rgb}{0.000000,0.000000,0.000000}%
\pgfsetstrokecolor{currentstroke}%
\pgfsetdash{}{0pt}%
\pgfsys@defobject{currentmarker}{\pgfqpoint{0.000000in}{-0.048611in}}{\pgfqpoint{0.000000in}{0.000000in}}{%
\pgfpathmoveto{\pgfqpoint{0.000000in}{0.000000in}}%
\pgfpathlineto{\pgfqpoint{0.000000in}{-0.048611in}}%
\pgfusepath{stroke,fill}%
}%
\begin{pgfscope}%
\pgfsys@transformshift{4.940496in}{0.460676in}%
\pgfsys@useobject{currentmarker}{}%
\end{pgfscope}%
\end{pgfscope}%
\begin{pgfscope}%
\definecolor{textcolor}{rgb}{0.000000,0.000000,0.000000}%
\pgfsetstrokecolor{textcolor}%
\pgfsetfillcolor{textcolor}%
\pgftext[x=4.940496in,y=0.363454in,,top]{\color{textcolor}\rmfamily\fontsize{8.330000}{9.996000}\selectfont \(\displaystyle 80\)}%
\end{pgfscope}%
\begin{pgfscope}%
\definecolor{textcolor}{rgb}{0.000000,0.000000,0.000000}%
\pgfsetstrokecolor{textcolor}%
\pgfsetfillcolor{textcolor}%
\pgftext[x=2.947347in,y=0.203949in,,top]{\color{textcolor}\rmfamily\fontsize{8.330000}{9.996000}\selectfont Instances}%
\end{pgfscope}%
\begin{pgfscope}%
\pgfsetbuttcap%
\pgfsetroundjoin%
\definecolor{currentfill}{rgb}{0.000000,0.000000,0.000000}%
\pgfsetfillcolor{currentfill}%
\pgfsetlinewidth{0.803000pt}%
\definecolor{currentstroke}{rgb}{0.000000,0.000000,0.000000}%
\pgfsetstrokecolor{currentstroke}%
\pgfsetdash{}{0pt}%
\pgfsys@defobject{currentmarker}{\pgfqpoint{-0.048611in}{0.000000in}}{\pgfqpoint{0.000000in}{0.000000in}}{%
\pgfpathmoveto{\pgfqpoint{0.000000in}{0.000000in}}%
\pgfpathlineto{\pgfqpoint{-0.048611in}{0.000000in}}%
\pgfusepath{stroke,fill}%
}%
\begin{pgfscope}%
\pgfsys@transformshift{0.581793in}{0.460676in}%
\pgfsys@useobject{currentmarker}{}%
\end{pgfscope}%
\end{pgfscope}%
\begin{pgfscope}%
\definecolor{textcolor}{rgb}{0.000000,0.000000,0.000000}%
\pgfsetstrokecolor{textcolor}%
\pgfsetfillcolor{textcolor}%
\pgftext[x=0.274692in, y=0.421225in, left, base]{\color{textcolor}\rmfamily\fontsize{8.330000}{9.996000}\selectfont \(\displaystyle 0.00\)}%
\end{pgfscope}%
\begin{pgfscope}%
\pgfsetbuttcap%
\pgfsetroundjoin%
\definecolor{currentfill}{rgb}{0.000000,0.000000,0.000000}%
\pgfsetfillcolor{currentfill}%
\pgfsetlinewidth{0.803000pt}%
\definecolor{currentstroke}{rgb}{0.000000,0.000000,0.000000}%
\pgfsetstrokecolor{currentstroke}%
\pgfsetdash{}{0pt}%
\pgfsys@defobject{currentmarker}{\pgfqpoint{-0.048611in}{0.000000in}}{\pgfqpoint{0.000000in}{0.000000in}}{%
\pgfpathmoveto{\pgfqpoint{0.000000in}{0.000000in}}%
\pgfpathlineto{\pgfqpoint{-0.048611in}{0.000000in}}%
\pgfusepath{stroke,fill}%
}%
\begin{pgfscope}%
\pgfsys@transformshift{0.581793in}{0.651888in}%
\pgfsys@useobject{currentmarker}{}%
\end{pgfscope}%
\end{pgfscope}%
\begin{pgfscope}%
\definecolor{textcolor}{rgb}{0.000000,0.000000,0.000000}%
\pgfsetstrokecolor{textcolor}%
\pgfsetfillcolor{textcolor}%
\pgftext[x=0.274692in, y=0.612437in, left, base]{\color{textcolor}\rmfamily\fontsize{8.330000}{9.996000}\selectfont \(\displaystyle 0.01\)}%
\end{pgfscope}%
\begin{pgfscope}%
\pgfsetbuttcap%
\pgfsetroundjoin%
\definecolor{currentfill}{rgb}{0.000000,0.000000,0.000000}%
\pgfsetfillcolor{currentfill}%
\pgfsetlinewidth{0.803000pt}%
\definecolor{currentstroke}{rgb}{0.000000,0.000000,0.000000}%
\pgfsetstrokecolor{currentstroke}%
\pgfsetdash{}{0pt}%
\pgfsys@defobject{currentmarker}{\pgfqpoint{-0.048611in}{0.000000in}}{\pgfqpoint{0.000000in}{0.000000in}}{%
\pgfpathmoveto{\pgfqpoint{0.000000in}{0.000000in}}%
\pgfpathlineto{\pgfqpoint{-0.048611in}{0.000000in}}%
\pgfusepath{stroke,fill}%
}%
\begin{pgfscope}%
\pgfsys@transformshift{0.581793in}{0.888240in}%
\pgfsys@useobject{currentmarker}{}%
\end{pgfscope}%
\end{pgfscope}%
\begin{pgfscope}%
\definecolor{textcolor}{rgb}{0.000000,0.000000,0.000000}%
\pgfsetstrokecolor{textcolor}%
\pgfsetfillcolor{textcolor}%
\pgftext[x=0.274692in, y=0.848789in, left, base]{\color{textcolor}\rmfamily\fontsize{8.330000}{9.996000}\selectfont \(\displaystyle 0.05\)}%
\end{pgfscope}%
\begin{pgfscope}%
\pgfsetbuttcap%
\pgfsetroundjoin%
\definecolor{currentfill}{rgb}{0.000000,0.000000,0.000000}%
\pgfsetfillcolor{currentfill}%
\pgfsetlinewidth{0.803000pt}%
\definecolor{currentstroke}{rgb}{0.000000,0.000000,0.000000}%
\pgfsetstrokecolor{currentstroke}%
\pgfsetdash{}{0pt}%
\pgfsys@defobject{currentmarker}{\pgfqpoint{-0.048611in}{0.000000in}}{\pgfqpoint{0.000000in}{0.000000in}}{%
\pgfpathmoveto{\pgfqpoint{0.000000in}{0.000000in}}%
\pgfpathlineto{\pgfqpoint{-0.048611in}{0.000000in}}%
\pgfusepath{stroke,fill}%
}%
\begin{pgfscope}%
\pgfsys@transformshift{0.581793in}{1.065343in}%
\pgfsys@useobject{currentmarker}{}%
\end{pgfscope}%
\end{pgfscope}%
\begin{pgfscope}%
\definecolor{textcolor}{rgb}{0.000000,0.000000,0.000000}%
\pgfsetstrokecolor{textcolor}%
\pgfsetfillcolor{textcolor}%
\pgftext[x=0.274692in, y=1.025892in, left, base]{\color{textcolor}\rmfamily\fontsize{8.330000}{9.996000}\selectfont \(\displaystyle 0.10\)}%
\end{pgfscope}%
\begin{pgfscope}%
\pgfsetbuttcap%
\pgfsetroundjoin%
\definecolor{currentfill}{rgb}{0.000000,0.000000,0.000000}%
\pgfsetfillcolor{currentfill}%
\pgfsetlinewidth{0.803000pt}%
\definecolor{currentstroke}{rgb}{0.000000,0.000000,0.000000}%
\pgfsetstrokecolor{currentstroke}%
\pgfsetdash{}{0pt}%
\pgfsys@defobject{currentmarker}{\pgfqpoint{-0.048611in}{0.000000in}}{\pgfqpoint{0.000000in}{0.000000in}}{%
\pgfpathmoveto{\pgfqpoint{0.000000in}{0.000000in}}%
\pgfpathlineto{\pgfqpoint{-0.048611in}{0.000000in}}%
\pgfusepath{stroke,fill}%
}%
\begin{pgfscope}%
\pgfsys@transformshift{0.581793in}{1.315804in}%
\pgfsys@useobject{currentmarker}{}%
\end{pgfscope}%
\end{pgfscope}%
\begin{pgfscope}%
\definecolor{textcolor}{rgb}{0.000000,0.000000,0.000000}%
\pgfsetstrokecolor{textcolor}%
\pgfsetfillcolor{textcolor}%
\pgftext[x=0.274692in, y=1.276353in, left, base]{\color{textcolor}\rmfamily\fontsize{8.330000}{9.996000}\selectfont \(\displaystyle 0.20\)}%
\end{pgfscope}%
\begin{pgfscope}%
\pgfsetbuttcap%
\pgfsetroundjoin%
\definecolor{currentfill}{rgb}{0.000000,0.000000,0.000000}%
\pgfsetfillcolor{currentfill}%
\pgfsetlinewidth{0.803000pt}%
\definecolor{currentstroke}{rgb}{0.000000,0.000000,0.000000}%
\pgfsetstrokecolor{currentstroke}%
\pgfsetdash{}{0pt}%
\pgfsys@defobject{currentmarker}{\pgfqpoint{-0.048611in}{0.000000in}}{\pgfqpoint{0.000000in}{0.000000in}}{%
\pgfpathmoveto{\pgfqpoint{0.000000in}{0.000000in}}%
\pgfpathlineto{\pgfqpoint{-0.048611in}{0.000000in}}%
\pgfusepath{stroke,fill}%
}%
\begin{pgfscope}%
\pgfsys@transformshift{0.581793in}{1.670010in}%
\pgfsys@useobject{currentmarker}{}%
\end{pgfscope}%
\end{pgfscope}%
\begin{pgfscope}%
\definecolor{textcolor}{rgb}{0.000000,0.000000,0.000000}%
\pgfsetstrokecolor{textcolor}%
\pgfsetfillcolor{textcolor}%
\pgftext[x=0.274692in, y=1.630559in, left, base]{\color{textcolor}\rmfamily\fontsize{8.330000}{9.996000}\selectfont \(\displaystyle 0.40\)}%
\end{pgfscope}%
\begin{pgfscope}%
\pgfsetbuttcap%
\pgfsetroundjoin%
\definecolor{currentfill}{rgb}{0.000000,0.000000,0.000000}%
\pgfsetfillcolor{currentfill}%
\pgfsetlinewidth{0.803000pt}%
\definecolor{currentstroke}{rgb}{0.000000,0.000000,0.000000}%
\pgfsetstrokecolor{currentstroke}%
\pgfsetdash{}{0pt}%
\pgfsys@defobject{currentmarker}{\pgfqpoint{-0.048611in}{0.000000in}}{\pgfqpoint{0.000000in}{0.000000in}}{%
\pgfpathmoveto{\pgfqpoint{0.000000in}{0.000000in}}%
\pgfpathlineto{\pgfqpoint{-0.048611in}{0.000000in}}%
\pgfusepath{stroke,fill}%
}%
\begin{pgfscope}%
\pgfsys@transformshift{0.581793in}{1.941802in}%
\pgfsys@useobject{currentmarker}{}%
\end{pgfscope}%
\end{pgfscope}%
\begin{pgfscope}%
\definecolor{textcolor}{rgb}{0.000000,0.000000,0.000000}%
\pgfsetstrokecolor{textcolor}%
\pgfsetfillcolor{textcolor}%
\pgftext[x=0.274692in, y=1.902351in, left, base]{\color{textcolor}\rmfamily\fontsize{8.330000}{9.996000}\selectfont \(\displaystyle 0.60\)}%
\end{pgfscope}%
\begin{pgfscope}%
\pgfsetbuttcap%
\pgfsetroundjoin%
\definecolor{currentfill}{rgb}{0.000000,0.000000,0.000000}%
\pgfsetfillcolor{currentfill}%
\pgfsetlinewidth{0.803000pt}%
\definecolor{currentstroke}{rgb}{0.000000,0.000000,0.000000}%
\pgfsetstrokecolor{currentstroke}%
\pgfsetdash{}{0pt}%
\pgfsys@defobject{currentmarker}{\pgfqpoint{-0.048611in}{0.000000in}}{\pgfqpoint{0.000000in}{0.000000in}}{%
\pgfpathmoveto{\pgfqpoint{0.000000in}{0.000000in}}%
\pgfpathlineto{\pgfqpoint{-0.048611in}{0.000000in}}%
\pgfusepath{stroke,fill}%
}%
\begin{pgfscope}%
\pgfsys@transformshift{0.581793in}{2.170933in}%
\pgfsys@useobject{currentmarker}{}%
\end{pgfscope}%
\end{pgfscope}%
\begin{pgfscope}%
\definecolor{textcolor}{rgb}{0.000000,0.000000,0.000000}%
\pgfsetstrokecolor{textcolor}%
\pgfsetfillcolor{textcolor}%
\pgftext[x=0.274692in, y=2.131482in, left, base]{\color{textcolor}\rmfamily\fontsize{8.330000}{9.996000}\selectfont \(\displaystyle 0.80\)}%
\end{pgfscope}%
\begin{pgfscope}%
\pgfsetbuttcap%
\pgfsetroundjoin%
\definecolor{currentfill}{rgb}{0.000000,0.000000,0.000000}%
\pgfsetfillcolor{currentfill}%
\pgfsetlinewidth{0.803000pt}%
\definecolor{currentstroke}{rgb}{0.000000,0.000000,0.000000}%
\pgfsetstrokecolor{currentstroke}%
\pgfsetdash{}{0pt}%
\pgfsys@defobject{currentmarker}{\pgfqpoint{-0.048611in}{0.000000in}}{\pgfqpoint{0.000000in}{0.000000in}}{%
\pgfpathmoveto{\pgfqpoint{0.000000in}{0.000000in}}%
\pgfpathlineto{\pgfqpoint{-0.048611in}{0.000000in}}%
\pgfusepath{stroke,fill}%
}%
\begin{pgfscope}%
\pgfsys@transformshift{0.581793in}{2.372801in}%
\pgfsys@useobject{currentmarker}{}%
\end{pgfscope}%
\end{pgfscope}%
\begin{pgfscope}%
\definecolor{textcolor}{rgb}{0.000000,0.000000,0.000000}%
\pgfsetstrokecolor{textcolor}%
\pgfsetfillcolor{textcolor}%
\pgftext[x=0.274692in, y=2.333350in, left, base]{\color{textcolor}\rmfamily\fontsize{8.330000}{9.996000}\selectfont \(\displaystyle 1.00\)}%
\end{pgfscope}%
\begin{pgfscope}%
\definecolor{textcolor}{rgb}{0.000000,0.000000,0.000000}%
\pgfsetstrokecolor{textcolor}%
\pgfsetfillcolor{textcolor}%
\pgftext[x=0.219136in,y=1.527800in,,bottom,rotate=90.000000]{\color{textcolor}\rmfamily\fontsize{8.330000}{9.996000}\selectfont \(\displaystyle 1-(b_{best}/b_{alg})\)}%
\end{pgfscope}%
\begin{pgfscope}%
\definecolor{textcolor}{rgb}{0.000000,0.000000,0.000000}%
\pgfsetstrokecolor{textcolor}%
\pgfsetfillcolor{textcolor}%
\pgftext[x=2.947347in,y=2.678257in,,base]{\color{textcolor}\rmfamily\fontsize{8.330000}{9.996000}\selectfont all instances - number of branches}%
\end{pgfscope}%
\begin{pgfscope}%
\pgfsetbuttcap%
\pgfsetroundjoin%
\definecolor{currentfill}{rgb}{0.121569,0.466667,0.705882}%
\pgfsetfillcolor{currentfill}%
\pgfsetlinewidth{1.003750pt}%
\definecolor{currentstroke}{rgb}{0.121569,0.466667,0.705882}%
\pgfsetstrokecolor{currentstroke}%
\pgfsetdash{}{0pt}%
\pgfsys@defobject{currentmarker}{\pgfqpoint{-0.028527in}{-0.028527in}}{\pgfqpoint{0.028527in}{0.028527in}}{%
\pgfpathmoveto{\pgfqpoint{0.000000in}{-0.028527in}}%
\pgfpathcurveto{\pgfqpoint{0.007566in}{-0.028527in}}{\pgfqpoint{0.014822in}{-0.025521in}}{\pgfqpoint{0.020172in}{-0.020172in}}%
\pgfpathcurveto{\pgfqpoint{0.025521in}{-0.014822in}}{\pgfqpoint{0.028527in}{-0.007566in}}{\pgfqpoint{0.028527in}{0.000000in}}%
\pgfpathcurveto{\pgfqpoint{0.028527in}{0.007566in}}{\pgfqpoint{0.025521in}{0.014822in}}{\pgfqpoint{0.020172in}{0.020172in}}%
\pgfpathcurveto{\pgfqpoint{0.014822in}{0.025521in}}{\pgfqpoint{0.007566in}{0.028527in}}{\pgfqpoint{0.000000in}{0.028527in}}%
\pgfpathcurveto{\pgfqpoint{-0.007566in}{0.028527in}}{\pgfqpoint{-0.014822in}{0.025521in}}{\pgfqpoint{-0.020172in}{0.020172in}}%
\pgfpathcurveto{\pgfqpoint{-0.025521in}{0.014822in}}{\pgfqpoint{-0.028527in}{0.007566in}}{\pgfqpoint{-0.028527in}{0.000000in}}%
\pgfpathcurveto{\pgfqpoint{-0.028527in}{-0.007566in}}{\pgfqpoint{-0.025521in}{-0.014822in}}{\pgfqpoint{-0.020172in}{-0.020172in}}%
\pgfpathcurveto{\pgfqpoint{-0.014822in}{-0.025521in}}{\pgfqpoint{-0.007566in}{-0.028527in}}{\pgfqpoint{0.000000in}{-0.028527in}}%
\pgfpathclose%
\pgfusepath{stroke,fill}%
}%
\begin{pgfscope}%
\pgfsys@transformshift{3.637184in}{2.284915in}%
\pgfsys@useobject{currentmarker}{}%
\end{pgfscope}%
\end{pgfscope}%
\begin{pgfscope}%
\definecolor{textcolor}{rgb}{0.000000,0.000000,0.000000}%
\pgfsetstrokecolor{textcolor}%
\pgfsetfillcolor{textcolor}%
\pgftext[x=3.764447in,y=2.254546in,left,base]{\color{textcolor}\rmfamily\fontsize{8.330000}{9.996000}\selectfont maximum degree (default)}%
\end{pgfscope}%
\begin{pgfscope}%
\pgfsetbuttcap%
\pgfsetroundjoin%
\definecolor{currentfill}{rgb}{1.000000,0.498039,0.054902}%
\pgfsetfillcolor{currentfill}%
\pgfsetfillopacity{0.750000}%
\pgfsetlinewidth{1.003750pt}%
\definecolor{currentstroke}{rgb}{1.000000,0.498039,0.054902}%
\pgfsetstrokecolor{currentstroke}%
\pgfsetstrokeopacity{0.750000}%
\pgfsetdash{}{0pt}%
\pgfsys@defobject{currentmarker}{\pgfqpoint{-0.028527in}{-0.028527in}}{\pgfqpoint{0.028527in}{0.028527in}}{%
\pgfpathmoveto{\pgfqpoint{-0.000000in}{-0.028527in}}%
\pgfpathlineto{\pgfqpoint{0.028527in}{0.028527in}}%
\pgfpathlineto{\pgfqpoint{-0.028527in}{0.028527in}}%
\pgfpathclose%
\pgfusepath{stroke,fill}%
}%
\begin{pgfscope}%
\pgfsys@transformshift{3.637184in}{2.155743in}%
\pgfsys@useobject{currentmarker}{}%
\end{pgfscope}%
\end{pgfscope}%
\begin{pgfscope}%
\definecolor{textcolor}{rgb}{0.000000,0.000000,0.000000}%
\pgfsetstrokecolor{textcolor}%
\pgfsetfillcolor{textcolor}%
\pgftext[x=3.764447in,y=2.125374in,left,base]{\color{textcolor}\rmfamily\fontsize{8.330000}{9.996000}\selectfont articulation points}%
\end{pgfscope}%
\begin{pgfscope}%
\pgfsetbuttcap%
\pgfsetroundjoin%
\definecolor{currentfill}{rgb}{0.172549,0.627451,0.172549}%
\pgfsetfillcolor{currentfill}%
\pgfsetfillopacity{0.750000}%
\pgfsetlinewidth{1.003750pt}%
\definecolor{currentstroke}{rgb}{0.172549,0.627451,0.172549}%
\pgfsetstrokecolor{currentstroke}%
\pgfsetstrokeopacity{0.750000}%
\pgfsetdash{}{0pt}%
\pgfsys@defobject{currentmarker}{\pgfqpoint{-0.028527in}{-0.028527in}}{\pgfqpoint{0.028527in}{0.028527in}}{%
\pgfpathmoveto{\pgfqpoint{-0.028527in}{-0.028527in}}%
\pgfpathlineto{\pgfqpoint{0.028527in}{-0.028527in}}%
\pgfpathlineto{\pgfqpoint{0.028527in}{0.028527in}}%
\pgfpathlineto{\pgfqpoint{-0.028527in}{0.028527in}}%
\pgfpathclose%
\pgfusepath{stroke,fill}%
}%
\begin{pgfscope}%
\pgfsys@transformshift{3.637184in}{2.026571in}%
\pgfsys@useobject{currentmarker}{}%
\end{pgfscope}%
\end{pgfscope}%
\begin{pgfscope}%
\definecolor{textcolor}{rgb}{0.000000,0.000000,0.000000}%
\pgfsetstrokecolor{textcolor}%
\pgfsetfillcolor{textcolor}%
\pgftext[x=3.764447in,y=1.996202in,left,base]{\color{textcolor}\rmfamily\fontsize{8.330000}{9.996000}\selectfont edge cuts}%
\end{pgfscope}%
\begin{pgfscope}%
\pgfsetbuttcap%
\pgfsetroundjoin%
\definecolor{currentfill}{rgb}{0.839216,0.152941,0.156863}%
\pgfsetfillcolor{currentfill}%
\pgfsetfillopacity{0.750000}%
\pgfsetlinewidth{1.003750pt}%
\definecolor{currentstroke}{rgb}{0.839216,0.152941,0.156863}%
\pgfsetstrokecolor{currentstroke}%
\pgfsetstrokeopacity{0.750000}%
\pgfsetdash{}{0pt}%
\pgfsys@defobject{currentmarker}{\pgfqpoint{-0.028527in}{-0.028527in}}{\pgfqpoint{0.028527in}{0.028527in}}{%
\pgfpathmoveto{\pgfqpoint{0.000000in}{0.028527in}}%
\pgfpathlineto{\pgfqpoint{-0.028527in}{-0.028527in}}%
\pgfpathlineto{\pgfqpoint{0.028527in}{-0.028527in}}%
\pgfpathclose%
\pgfusepath{stroke,fill}%
}%
\begin{pgfscope}%
\pgfsys@transformshift{3.637184in}{1.897399in}%
\pgfsys@useobject{currentmarker}{}%
\end{pgfscope}%
\end{pgfscope}%
\begin{pgfscope}%
\definecolor{textcolor}{rgb}{0.000000,0.000000,0.000000}%
\pgfsetstrokecolor{textcolor}%
\pgfsetfillcolor{textcolor}%
\pgftext[x=3.764447in,y=1.867030in,left,base]{\color{textcolor}\rmfamily\fontsize{8.330000}{9.996000}\selectfont nested dissection}%
\end{pgfscope}%
\end{pgfpicture}%
\makeatother%
\endgroup%

	\caption{Performance plot comparing branches of decomposition-based branching strategies.}\label{fig:all_decomp_branch}
\end{figure}


% \FloatBarrier
\subsection{Reduction Branching}
Figure~\ref{fig:all_reduction_time} shows the performance plot (see
Section~\ref{sec:experiments_decomp}) for our reduction-based branching strategies. Here we see that targeting the funnel reduction
results in the fastest time for the most number of instances, followed by almost
twins. In fact, the
funnel-based technique is the fastest for all but 5 of the PACE instances (excluding
one which was not solved by any algorithm). Furthermore, branching on almost
twins is faster than maximum degree branching on all DIMACS clique instances that
were solved by at least one of the techniques. We also show results for a strategy that targets all reduction rules
described in Section~\ref{sec:reduction_branching} (called \emph{combined}). Even though this approach
leads to the lowest number of branches for most instances (see Figure~\ref{fig:all_reduction_branch}), the time required to
identify candidate vertices for all reduction rules causes too big of an
overhead to be competitive. Optimizing the algorithms to identify candidate
vertices could lead to making this combined strategy
dominant. Table~\ref{tab:summary_reduction} shows that while all reduction-based
techniques combined solve $88\%$ of solved instances faster that the default
strategy maximum degree (see Appendix~\ref{app:detailed_results}), only the twin
based technique results in a small speedup of $1\%$ over all instances.
Thus, for $84\%$ of all ($93\%$ of solved) instances tested, one of our strategies (decomposition or reduction-based) is faster than the default strategy.

\begin{table}[t]
  \caption{Speedup of our reduction-based techniques over the default
    strategy (maximum degree). ``all'' are geometric means of the speedup over all
  instances of that type (timeouts are assigned a running time of ten hours), ``solved'' are geometric means of the
  speedup over all instances that have been solved by all techniques.}\label{tab:summary_reduction}

  \centering
  \begin{tabular}{|l|rr|rr|rr|rr|}
    \hline
    Instance type $\rightarrow$ & \multicolumn{2}{c|}{PACE} & \multicolumn{2}{c|}{DIMACS} & \multicolumn{2}{c|}{sparse nets.} & \multicolumn{2}{c|}{All Instances}                                                                                                          \\
    reduction rule $\downarrow$ & \multicolumn{1}{c}{all}   & \multicolumn{1}{c|}{solved} & \multicolumn{1}{c}{all}   & \multicolumn{1}{c|}{solved} & \multicolumn{1}{c}{all} & \multicolumn{1}{c|}{solved} & \multicolumn{1}{c}{all} & \multicolumn{1}{c|}{solved} \\
    \hline
    Twin                        & \numprint{1.00}     & \numprint{1.00}      & \textbf{\numprint{1.02}}     & \textbf{\numprint{1.02}}     & \numprint{0.99}  & \numprint{0.99}      & \textbf{\numprint{1.01}}   & \textbf{\numprint{1.01}}       \\
    Funnel                      & \textbf{\numprint{1.02}}     & \textbf{\numprint{1.02}}      & \numprint{0.96}     & \numprint{0.96}       & \numprint{0.71}   & \numprint{0.71}       & \numprint{0.96}   & \numprint{0.96}       \\
    Unconfined                  & \numprint{0.85}    & \numprint{0.85}       & \numprint{1.01}     & \numprint{1.01}       & \numprint{0.71}   & \numprint{0.71}       & \numprint{0.89}   & \numprint{0.89}      \\
    Combined                    & \numprint{0.96}    & \numprint{0.96}     & \numprint{0.94}     & \numprint{0.94}      & \numprint{0.68}   & \numprint{0.68}      & \numprint{0.92}   & \numprint{0.92}       \\
    \hline
    \end{tabular}
\end{table}

%\begin{figure}[t]
%  \includegraphics[width=\textwidth]{plots/all_reduction_based_time}
%  \caption{Performance plot comparing running time of reduction-based branching strategies.}
%  \label{fig:all_reduction_time}
%\end{figure}

%\begin{figure}[t]
%  \includegraphics[width=\textwidth]{plots/all_reduction_based_branch}
%  \caption{Performance plot comparing branches of reduction-based branching strategies.}
%  \label{fig:all_reduction_branch}
%\end{figure}

\begin{figure}[t]
	\centering
	%% Creator: Matplotlib, PGF backend
%%
%% To include the figure in your LaTeX document, write
%%   \input{<filename>.pgf}
%%
%% Make sure the required packages are loaded in your preamble
%%   \usepackage{pgf}
%%
%% and, on pdftex
%%   \usepackage[utf8]{inputenc}\DeclareUnicodeCharacter{2212}{-}
%%
%% or, on luatex and xetex
%%   \usepackage{unicode-math}
%%
%% Figures using additional raster images can only be included by \input if
%% they are in the same directory as the main LaTeX file. For loading figures
%% from other directories you can use the `import` package
%%   \usepackage{import}
%%
%% and then include the figures with
%%   \import{<path to file>}{<filename>.pgf}
%%
%% Matplotlib used the following preamble
%%   \renewcommand{\sfdefault}{phv}
%%   \renewcommand{\rmdefault}{ptm}
%%   \renewcommand{\ttdefault}{pcr}
%%   \normalfont\selectfont
%%
\begingroup%
\makeatletter%
\begin{pgfpicture}%
\pgfpathrectangle{\pgfpointorigin}{\pgfqpoint{5.412900in}{2.857160in}}%
\pgfusepath{use as bounding box, clip}%
\begin{pgfscope}%
\pgfsetbuttcap%
\pgfsetmiterjoin%
\definecolor{currentfill}{rgb}{1.000000,1.000000,1.000000}%
\pgfsetfillcolor{currentfill}%
\pgfsetlinewidth{0.000000pt}%
\definecolor{currentstroke}{rgb}{1.000000,1.000000,1.000000}%
\pgfsetstrokecolor{currentstroke}%
\pgfsetdash{}{0pt}%
\pgfpathmoveto{\pgfqpoint{0.000000in}{0.000000in}}%
\pgfpathlineto{\pgfqpoint{5.412900in}{0.000000in}}%
\pgfpathlineto{\pgfqpoint{5.412900in}{2.857160in}}%
\pgfpathlineto{\pgfqpoint{0.000000in}{2.857160in}}%
\pgfpathclose%
\pgfusepath{fill}%
\end{pgfscope}%
\begin{pgfscope}%
\pgfsetbuttcap%
\pgfsetmiterjoin%
\definecolor{currentfill}{rgb}{1.000000,1.000000,1.000000}%
\pgfsetfillcolor{currentfill}%
\pgfsetlinewidth{0.000000pt}%
\definecolor{currentstroke}{rgb}{0.000000,0.000000,0.000000}%
\pgfsetstrokecolor{currentstroke}%
\pgfsetstrokeopacity{0.000000}%
\pgfsetdash{}{0pt}%
\pgfpathmoveto{\pgfqpoint{0.581793in}{0.460676in}}%
\pgfpathlineto{\pgfqpoint{5.312900in}{0.460676in}}%
\pgfpathlineto{\pgfqpoint{5.312900in}{2.594924in}}%
\pgfpathlineto{\pgfqpoint{0.581793in}{2.594924in}}%
\pgfpathclose%
\pgfusepath{fill}%
\end{pgfscope}%
\begin{pgfscope}%
\pgfpathrectangle{\pgfqpoint{0.581793in}{0.460676in}}{\pgfqpoint{4.731107in}{2.134248in}}%
\pgfusepath{clip}%
\pgfsetbuttcap%
\pgfsetroundjoin%
\definecolor{currentfill}{rgb}{0.121569,0.466667,0.705882}%
\pgfsetfillcolor{currentfill}%
\pgfsetlinewidth{1.003750pt}%
\definecolor{currentstroke}{rgb}{0.121569,0.466667,0.705882}%
\pgfsetstrokecolor{currentstroke}%
\pgfsetdash{}{0pt}%
\pgfsys@defobject{currentmarker}{\pgfqpoint{-0.019018in}{-0.019018in}}{\pgfqpoint{0.019018in}{0.019018in}}{%
\pgfpathmoveto{\pgfqpoint{0.000000in}{-0.019018in}}%
\pgfpathcurveto{\pgfqpoint{0.005044in}{-0.019018in}}{\pgfqpoint{0.009881in}{-0.017014in}}{\pgfqpoint{0.013448in}{-0.013448in}}%
\pgfpathcurveto{\pgfqpoint{0.017014in}{-0.009881in}}{\pgfqpoint{0.019018in}{-0.005044in}}{\pgfqpoint{0.019018in}{0.000000in}}%
\pgfpathcurveto{\pgfqpoint{0.019018in}{0.005044in}}{\pgfqpoint{0.017014in}{0.009881in}}{\pgfqpoint{0.013448in}{0.013448in}}%
\pgfpathcurveto{\pgfqpoint{0.009881in}{0.017014in}}{\pgfqpoint{0.005044in}{0.019018in}}{\pgfqpoint{0.000000in}{0.019018in}}%
\pgfpathcurveto{\pgfqpoint{-0.005044in}{0.019018in}}{\pgfqpoint{-0.009881in}{0.017014in}}{\pgfqpoint{-0.013448in}{0.013448in}}%
\pgfpathcurveto{\pgfqpoint{-0.017014in}{0.009881in}}{\pgfqpoint{-0.019018in}{0.005044in}}{\pgfqpoint{-0.019018in}{0.000000in}}%
\pgfpathcurveto{\pgfqpoint{-0.019018in}{-0.005044in}}{\pgfqpoint{-0.017014in}{-0.009881in}}{\pgfqpoint{-0.013448in}{-0.013448in}}%
\pgfpathcurveto{\pgfqpoint{-0.009881in}{-0.017014in}}{\pgfqpoint{-0.005044in}{-0.019018in}}{\pgfqpoint{0.000000in}{-0.019018in}}%
\pgfpathclose%
\pgfusepath{stroke,fill}%
}%
\begin{pgfscope}%
\pgfsys@transformshift{0.796844in}{2.457315in}%
\pgfsys@useobject{currentmarker}{}%
\end{pgfscope}%
\begin{pgfscope}%
\pgfsys@transformshift{0.846280in}{1.984815in}%
\pgfsys@useobject{currentmarker}{}%
\end{pgfscope}%
\begin{pgfscope}%
\pgfsys@transformshift{0.895717in}{1.654145in}%
\pgfsys@useobject{currentmarker}{}%
\end{pgfscope}%
\begin{pgfscope}%
\pgfsys@transformshift{0.945154in}{1.292018in}%
\pgfsys@useobject{currentmarker}{}%
\end{pgfscope}%
\begin{pgfscope}%
\pgfsys@transformshift{0.994591in}{1.111342in}%
\pgfsys@useobject{currentmarker}{}%
\end{pgfscope}%
\begin{pgfscope}%
\pgfsys@transformshift{1.044028in}{1.075045in}%
\pgfsys@useobject{currentmarker}{}%
\end{pgfscope}%
\begin{pgfscope}%
\pgfsys@transformshift{1.093465in}{1.070503in}%
\pgfsys@useobject{currentmarker}{}%
\end{pgfscope}%
\begin{pgfscope}%
\pgfsys@transformshift{1.142902in}{1.058965in}%
\pgfsys@useobject{currentmarker}{}%
\end{pgfscope}%
\begin{pgfscope}%
\pgfsys@transformshift{1.192338in}{1.058427in}%
\pgfsys@useobject{currentmarker}{}%
\end{pgfscope}%
\begin{pgfscope}%
\pgfsys@transformshift{1.241775in}{1.055950in}%
\pgfsys@useobject{currentmarker}{}%
\end{pgfscope}%
\begin{pgfscope}%
\pgfsys@transformshift{1.291212in}{1.029940in}%
\pgfsys@useobject{currentmarker}{}%
\end{pgfscope}%
\begin{pgfscope}%
\pgfsys@transformshift{1.340649in}{0.981313in}%
\pgfsys@useobject{currentmarker}{}%
\end{pgfscope}%
\begin{pgfscope}%
\pgfsys@transformshift{1.390086in}{0.978378in}%
\pgfsys@useobject{currentmarker}{}%
\end{pgfscope}%
\begin{pgfscope}%
\pgfsys@transformshift{1.439523in}{0.977316in}%
\pgfsys@useobject{currentmarker}{}%
\end{pgfscope}%
\begin{pgfscope}%
\pgfsys@transformshift{1.488960in}{0.971735in}%
\pgfsys@useobject{currentmarker}{}%
\end{pgfscope}%
\begin{pgfscope}%
\pgfsys@transformshift{1.538396in}{0.967645in}%
\pgfsys@useobject{currentmarker}{}%
\end{pgfscope}%
\begin{pgfscope}%
\pgfsys@transformshift{1.587833in}{0.965376in}%
\pgfsys@useobject{currentmarker}{}%
\end{pgfscope}%
\begin{pgfscope}%
\pgfsys@transformshift{1.637270in}{0.964104in}%
\pgfsys@useobject{currentmarker}{}%
\end{pgfscope}%
\begin{pgfscope}%
\pgfsys@transformshift{1.686707in}{0.946480in}%
\pgfsys@useobject{currentmarker}{}%
\end{pgfscope}%
\begin{pgfscope}%
\pgfsys@transformshift{1.736144in}{0.946111in}%
\pgfsys@useobject{currentmarker}{}%
\end{pgfscope}%
\begin{pgfscope}%
\pgfsys@transformshift{1.785581in}{0.924254in}%
\pgfsys@useobject{currentmarker}{}%
\end{pgfscope}%
\begin{pgfscope}%
\pgfsys@transformshift{1.835018in}{0.922742in}%
\pgfsys@useobject{currentmarker}{}%
\end{pgfscope}%
\begin{pgfscope}%
\pgfsys@transformshift{1.884454in}{0.919853in}%
\pgfsys@useobject{currentmarker}{}%
\end{pgfscope}%
\begin{pgfscope}%
\pgfsys@transformshift{1.933891in}{0.914150in}%
\pgfsys@useobject{currentmarker}{}%
\end{pgfscope}%
\begin{pgfscope}%
\pgfsys@transformshift{1.983328in}{0.911348in}%
\pgfsys@useobject{currentmarker}{}%
\end{pgfscope}%
\begin{pgfscope}%
\pgfsys@transformshift{2.032765in}{0.909348in}%
\pgfsys@useobject{currentmarker}{}%
\end{pgfscope}%
\begin{pgfscope}%
\pgfsys@transformshift{2.082202in}{0.905104in}%
\pgfsys@useobject{currentmarker}{}%
\end{pgfscope}%
\begin{pgfscope}%
\pgfsys@transformshift{2.131639in}{0.899419in}%
\pgfsys@useobject{currentmarker}{}%
\end{pgfscope}%
\begin{pgfscope}%
\pgfsys@transformshift{2.181075in}{0.898448in}%
\pgfsys@useobject{currentmarker}{}%
\end{pgfscope}%
\begin{pgfscope}%
\pgfsys@transformshift{2.230512in}{0.896081in}%
\pgfsys@useobject{currentmarker}{}%
\end{pgfscope}%
\begin{pgfscope}%
\pgfsys@transformshift{2.279949in}{0.883326in}%
\pgfsys@useobject{currentmarker}{}%
\end{pgfscope}%
\begin{pgfscope}%
\pgfsys@transformshift{2.329386in}{0.868688in}%
\pgfsys@useobject{currentmarker}{}%
\end{pgfscope}%
\begin{pgfscope}%
\pgfsys@transformshift{2.378823in}{0.862950in}%
\pgfsys@useobject{currentmarker}{}%
\end{pgfscope}%
\begin{pgfscope}%
\pgfsys@transformshift{2.428260in}{0.862609in}%
\pgfsys@useobject{currentmarker}{}%
\end{pgfscope}%
\begin{pgfscope}%
\pgfsys@transformshift{2.477697in}{0.861884in}%
\pgfsys@useobject{currentmarker}{}%
\end{pgfscope}%
\begin{pgfscope}%
\pgfsys@transformshift{2.527133in}{0.860937in}%
\pgfsys@useobject{currentmarker}{}%
\end{pgfscope}%
\begin{pgfscope}%
\pgfsys@transformshift{2.576570in}{0.832749in}%
\pgfsys@useobject{currentmarker}{}%
\end{pgfscope}%
\begin{pgfscope}%
\pgfsys@transformshift{2.626007in}{0.828800in}%
\pgfsys@useobject{currentmarker}{}%
\end{pgfscope}%
\begin{pgfscope}%
\pgfsys@transformshift{2.675444in}{0.816429in}%
\pgfsys@useobject{currentmarker}{}%
\end{pgfscope}%
\begin{pgfscope}%
\pgfsys@transformshift{2.724881in}{0.779028in}%
\pgfsys@useobject{currentmarker}{}%
\end{pgfscope}%
\begin{pgfscope}%
\pgfsys@transformshift{2.774318in}{0.760767in}%
\pgfsys@useobject{currentmarker}{}%
\end{pgfscope}%
\begin{pgfscope}%
\pgfsys@transformshift{2.823755in}{0.760509in}%
\pgfsys@useobject{currentmarker}{}%
\end{pgfscope}%
\begin{pgfscope}%
\pgfsys@transformshift{2.873191in}{0.760302in}%
\pgfsys@useobject{currentmarker}{}%
\end{pgfscope}%
\begin{pgfscope}%
\pgfsys@transformshift{2.922628in}{0.760173in}%
\pgfsys@useobject{currentmarker}{}%
\end{pgfscope}%
\begin{pgfscope}%
\pgfsys@transformshift{2.972065in}{0.760051in}%
\pgfsys@useobject{currentmarker}{}%
\end{pgfscope}%
\begin{pgfscope}%
\pgfsys@transformshift{3.021502in}{0.757581in}%
\pgfsys@useobject{currentmarker}{}%
\end{pgfscope}%
\begin{pgfscope}%
\pgfsys@transformshift{3.070939in}{0.757084in}%
\pgfsys@useobject{currentmarker}{}%
\end{pgfscope}%
\begin{pgfscope}%
\pgfsys@transformshift{3.120376in}{0.756645in}%
\pgfsys@useobject{currentmarker}{}%
\end{pgfscope}%
\begin{pgfscope}%
\pgfsys@transformshift{3.169812in}{0.756562in}%
\pgfsys@useobject{currentmarker}{}%
\end{pgfscope}%
\begin{pgfscope}%
\pgfsys@transformshift{3.219249in}{0.756097in}%
\pgfsys@useobject{currentmarker}{}%
\end{pgfscope}%
\begin{pgfscope}%
\pgfsys@transformshift{3.268686in}{0.755786in}%
\pgfsys@useobject{currentmarker}{}%
\end{pgfscope}%
\begin{pgfscope}%
\pgfsys@transformshift{3.318123in}{0.755320in}%
\pgfsys@useobject{currentmarker}{}%
\end{pgfscope}%
\begin{pgfscope}%
\pgfsys@transformshift{3.367560in}{0.754061in}%
\pgfsys@useobject{currentmarker}{}%
\end{pgfscope}%
\begin{pgfscope}%
\pgfsys@transformshift{3.416997in}{0.753877in}%
\pgfsys@useobject{currentmarker}{}%
\end{pgfscope}%
\begin{pgfscope}%
\pgfsys@transformshift{3.466434in}{0.747475in}%
\pgfsys@useobject{currentmarker}{}%
\end{pgfscope}%
\begin{pgfscope}%
\pgfsys@transformshift{3.515870in}{0.745376in}%
\pgfsys@useobject{currentmarker}{}%
\end{pgfscope}%
\begin{pgfscope}%
\pgfsys@transformshift{3.565307in}{0.745123in}%
\pgfsys@useobject{currentmarker}{}%
\end{pgfscope}%
\begin{pgfscope}%
\pgfsys@transformshift{3.614744in}{0.733659in}%
\pgfsys@useobject{currentmarker}{}%
\end{pgfscope}%
\begin{pgfscope}%
\pgfsys@transformshift{3.664181in}{0.681354in}%
\pgfsys@useobject{currentmarker}{}%
\end{pgfscope}%
\begin{pgfscope}%
\pgfsys@transformshift{3.713618in}{0.679430in}%
\pgfsys@useobject{currentmarker}{}%
\end{pgfscope}%
\begin{pgfscope}%
\pgfsys@transformshift{3.763055in}{0.678854in}%
\pgfsys@useobject{currentmarker}{}%
\end{pgfscope}%
\begin{pgfscope}%
\pgfsys@transformshift{3.812492in}{0.676891in}%
\pgfsys@useobject{currentmarker}{}%
\end{pgfscope}%
\begin{pgfscope}%
\pgfsys@transformshift{3.861928in}{0.674050in}%
\pgfsys@useobject{currentmarker}{}%
\end{pgfscope}%
\begin{pgfscope}%
\pgfsys@transformshift{3.911365in}{0.665602in}%
\pgfsys@useobject{currentmarker}{}%
\end{pgfscope}%
\begin{pgfscope}%
\pgfsys@transformshift{3.960802in}{0.644563in}%
\pgfsys@useobject{currentmarker}{}%
\end{pgfscope}%
\begin{pgfscope}%
\pgfsys@transformshift{4.010239in}{0.636234in}%
\pgfsys@useobject{currentmarker}{}%
\end{pgfscope}%
\begin{pgfscope}%
\pgfsys@transformshift{4.059676in}{0.624346in}%
\pgfsys@useobject{currentmarker}{}%
\end{pgfscope}%
\begin{pgfscope}%
\pgfsys@transformshift{4.109113in}{0.611331in}%
\pgfsys@useobject{currentmarker}{}%
\end{pgfscope}%
\begin{pgfscope}%
\pgfsys@transformshift{4.158550in}{0.593144in}%
\pgfsys@useobject{currentmarker}{}%
\end{pgfscope}%
\begin{pgfscope}%
\pgfsys@transformshift{4.207986in}{0.586974in}%
\pgfsys@useobject{currentmarker}{}%
\end{pgfscope}%
\begin{pgfscope}%
\pgfsys@transformshift{4.257423in}{0.557410in}%
\pgfsys@useobject{currentmarker}{}%
\end{pgfscope}%
\begin{pgfscope}%
\pgfsys@transformshift{4.306860in}{0.514774in}%
\pgfsys@useobject{currentmarker}{}%
\end{pgfscope}%
\begin{pgfscope}%
\pgfsys@transformshift{4.356297in}{0.507424in}%
\pgfsys@useobject{currentmarker}{}%
\end{pgfscope}%
\begin{pgfscope}%
\pgfsys@transformshift{4.405734in}{0.507297in}%
\pgfsys@useobject{currentmarker}{}%
\end{pgfscope}%
\begin{pgfscope}%
\pgfsys@transformshift{4.455171in}{0.460676in}%
\pgfsys@useobject{currentmarker}{}%
\end{pgfscope}%
\begin{pgfscope}%
\pgfsys@transformshift{4.504607in}{0.460676in}%
\pgfsys@useobject{currentmarker}{}%
\end{pgfscope}%
\begin{pgfscope}%
\pgfsys@transformshift{4.554044in}{0.460676in}%
\pgfsys@useobject{currentmarker}{}%
\end{pgfscope}%
\begin{pgfscope}%
\pgfsys@transformshift{4.603481in}{0.460676in}%
\pgfsys@useobject{currentmarker}{}%
\end{pgfscope}%
\begin{pgfscope}%
\pgfsys@transformshift{4.652918in}{0.460676in}%
\pgfsys@useobject{currentmarker}{}%
\end{pgfscope}%
\begin{pgfscope}%
\pgfsys@transformshift{4.702355in}{0.460676in}%
\pgfsys@useobject{currentmarker}{}%
\end{pgfscope}%
\begin{pgfscope}%
\pgfsys@transformshift{4.751792in}{0.460676in}%
\pgfsys@useobject{currentmarker}{}%
\end{pgfscope}%
\begin{pgfscope}%
\pgfsys@transformshift{4.801229in}{0.460676in}%
\pgfsys@useobject{currentmarker}{}%
\end{pgfscope}%
\begin{pgfscope}%
\pgfsys@transformshift{4.850665in}{0.460676in}%
\pgfsys@useobject{currentmarker}{}%
\end{pgfscope}%
\begin{pgfscope}%
\pgfsys@transformshift{4.900102in}{0.460676in}%
\pgfsys@useobject{currentmarker}{}%
\end{pgfscope}%
\begin{pgfscope}%
\pgfsys@transformshift{4.949539in}{0.460676in}%
\pgfsys@useobject{currentmarker}{}%
\end{pgfscope}%
\begin{pgfscope}%
\pgfsys@transformshift{4.998976in}{0.460676in}%
\pgfsys@useobject{currentmarker}{}%
\end{pgfscope}%
\begin{pgfscope}%
\pgfsys@transformshift{5.048413in}{0.460676in}%
\pgfsys@useobject{currentmarker}{}%
\end{pgfscope}%
\begin{pgfscope}%
\pgfsys@transformshift{5.097850in}{0.460676in}%
\pgfsys@useobject{currentmarker}{}%
\end{pgfscope}%
\end{pgfscope}%
\begin{pgfscope}%
\pgfpathrectangle{\pgfqpoint{0.581793in}{0.460676in}}{\pgfqpoint{4.731107in}{2.134248in}}%
\pgfusepath{clip}%
\pgfsetbuttcap%
\pgfsetroundjoin%
\definecolor{currentfill}{rgb}{1.000000,0.498039,0.054902}%
\pgfsetfillcolor{currentfill}%
\pgfsetfillopacity{0.750000}%
\pgfsetlinewidth{1.003750pt}%
\definecolor{currentstroke}{rgb}{1.000000,0.498039,0.054902}%
\pgfsetstrokecolor{currentstroke}%
\pgfsetstrokeopacity{0.750000}%
\pgfsetdash{}{0pt}%
\pgfsys@defobject{currentmarker}{\pgfqpoint{-0.019018in}{-0.019018in}}{\pgfqpoint{0.019018in}{0.019018in}}{%
\pgfpathmoveto{\pgfqpoint{-0.000000in}{-0.019018in}}%
\pgfpathlineto{\pgfqpoint{0.019018in}{0.019018in}}%
\pgfpathlineto{\pgfqpoint{-0.019018in}{0.019018in}}%
\pgfpathclose%
\pgfusepath{stroke,fill}%
}%
\begin{pgfscope}%
\pgfsys@transformshift{0.796844in}{2.493293in}%
\pgfsys@useobject{currentmarker}{}%
\end{pgfscope}%
\begin{pgfscope}%
\pgfsys@transformshift{0.846280in}{2.493293in}%
\pgfsys@useobject{currentmarker}{}%
\end{pgfscope}%
\begin{pgfscope}%
\pgfsys@transformshift{0.895717in}{2.223287in}%
\pgfsys@useobject{currentmarker}{}%
\end{pgfscope}%
\begin{pgfscope}%
\pgfsys@transformshift{0.945154in}{2.178359in}%
\pgfsys@useobject{currentmarker}{}%
\end{pgfscope}%
\begin{pgfscope}%
\pgfsys@transformshift{0.994591in}{1.988047in}%
\pgfsys@useobject{currentmarker}{}%
\end{pgfscope}%
\begin{pgfscope}%
\pgfsys@transformshift{1.044028in}{1.982807in}%
\pgfsys@useobject{currentmarker}{}%
\end{pgfscope}%
\begin{pgfscope}%
\pgfsys@transformshift{1.093465in}{1.753259in}%
\pgfsys@useobject{currentmarker}{}%
\end{pgfscope}%
\begin{pgfscope}%
\pgfsys@transformshift{1.142902in}{1.608033in}%
\pgfsys@useobject{currentmarker}{}%
\end{pgfscope}%
\begin{pgfscope}%
\pgfsys@transformshift{1.192338in}{1.512845in}%
\pgfsys@useobject{currentmarker}{}%
\end{pgfscope}%
\begin{pgfscope}%
\pgfsys@transformshift{1.241775in}{1.425035in}%
\pgfsys@useobject{currentmarker}{}%
\end{pgfscope}%
\begin{pgfscope}%
\pgfsys@transformshift{1.291212in}{1.418652in}%
\pgfsys@useobject{currentmarker}{}%
\end{pgfscope}%
\begin{pgfscope}%
\pgfsys@transformshift{1.340649in}{1.386250in}%
\pgfsys@useobject{currentmarker}{}%
\end{pgfscope}%
\begin{pgfscope}%
\pgfsys@transformshift{1.390086in}{1.374948in}%
\pgfsys@useobject{currentmarker}{}%
\end{pgfscope}%
\begin{pgfscope}%
\pgfsys@transformshift{1.439523in}{1.339541in}%
\pgfsys@useobject{currentmarker}{}%
\end{pgfscope}%
\begin{pgfscope}%
\pgfsys@transformshift{1.488960in}{1.335616in}%
\pgfsys@useobject{currentmarker}{}%
\end{pgfscope}%
\begin{pgfscope}%
\pgfsys@transformshift{1.538396in}{1.180053in}%
\pgfsys@useobject{currentmarker}{}%
\end{pgfscope}%
\begin{pgfscope}%
\pgfsys@transformshift{1.587833in}{1.155205in}%
\pgfsys@useobject{currentmarker}{}%
\end{pgfscope}%
\begin{pgfscope}%
\pgfsys@transformshift{1.637270in}{1.137303in}%
\pgfsys@useobject{currentmarker}{}%
\end{pgfscope}%
\begin{pgfscope}%
\pgfsys@transformshift{1.686707in}{1.113526in}%
\pgfsys@useobject{currentmarker}{}%
\end{pgfscope}%
\begin{pgfscope}%
\pgfsys@transformshift{1.736144in}{1.083629in}%
\pgfsys@useobject{currentmarker}{}%
\end{pgfscope}%
\begin{pgfscope}%
\pgfsys@transformshift{1.785581in}{1.077867in}%
\pgfsys@useobject{currentmarker}{}%
\end{pgfscope}%
\begin{pgfscope}%
\pgfsys@transformshift{1.835018in}{1.062098in}%
\pgfsys@useobject{currentmarker}{}%
\end{pgfscope}%
\begin{pgfscope}%
\pgfsys@transformshift{1.884454in}{1.038675in}%
\pgfsys@useobject{currentmarker}{}%
\end{pgfscope}%
\begin{pgfscope}%
\pgfsys@transformshift{1.933891in}{1.033367in}%
\pgfsys@useobject{currentmarker}{}%
\end{pgfscope}%
\begin{pgfscope}%
\pgfsys@transformshift{1.983328in}{1.028215in}%
\pgfsys@useobject{currentmarker}{}%
\end{pgfscope}%
\begin{pgfscope}%
\pgfsys@transformshift{2.032765in}{1.025980in}%
\pgfsys@useobject{currentmarker}{}%
\end{pgfscope}%
\begin{pgfscope}%
\pgfsys@transformshift{2.082202in}{1.012764in}%
\pgfsys@useobject{currentmarker}{}%
\end{pgfscope}%
\begin{pgfscope}%
\pgfsys@transformshift{2.131639in}{1.012285in}%
\pgfsys@useobject{currentmarker}{}%
\end{pgfscope}%
\begin{pgfscope}%
\pgfsys@transformshift{2.181075in}{1.007598in}%
\pgfsys@useobject{currentmarker}{}%
\end{pgfscope}%
\begin{pgfscope}%
\pgfsys@transformshift{2.230512in}{1.001846in}%
\pgfsys@useobject{currentmarker}{}%
\end{pgfscope}%
\begin{pgfscope}%
\pgfsys@transformshift{2.279949in}{1.000623in}%
\pgfsys@useobject{currentmarker}{}%
\end{pgfscope}%
\begin{pgfscope}%
\pgfsys@transformshift{2.329386in}{0.998817in}%
\pgfsys@useobject{currentmarker}{}%
\end{pgfscope}%
\begin{pgfscope}%
\pgfsys@transformshift{2.378823in}{0.994060in}%
\pgfsys@useobject{currentmarker}{}%
\end{pgfscope}%
\begin{pgfscope}%
\pgfsys@transformshift{2.428260in}{0.981879in}%
\pgfsys@useobject{currentmarker}{}%
\end{pgfscope}%
\begin{pgfscope}%
\pgfsys@transformshift{2.477697in}{0.975175in}%
\pgfsys@useobject{currentmarker}{}%
\end{pgfscope}%
\begin{pgfscope}%
\pgfsys@transformshift{2.527133in}{0.974138in}%
\pgfsys@useobject{currentmarker}{}%
\end{pgfscope}%
\begin{pgfscope}%
\pgfsys@transformshift{2.576570in}{0.965676in}%
\pgfsys@useobject{currentmarker}{}%
\end{pgfscope}%
\begin{pgfscope}%
\pgfsys@transformshift{2.626007in}{0.964422in}%
\pgfsys@useobject{currentmarker}{}%
\end{pgfscope}%
\begin{pgfscope}%
\pgfsys@transformshift{2.675444in}{0.957256in}%
\pgfsys@useobject{currentmarker}{}%
\end{pgfscope}%
\begin{pgfscope}%
\pgfsys@transformshift{2.724881in}{0.948979in}%
\pgfsys@useobject{currentmarker}{}%
\end{pgfscope}%
\begin{pgfscope}%
\pgfsys@transformshift{2.774318in}{0.941419in}%
\pgfsys@useobject{currentmarker}{}%
\end{pgfscope}%
\begin{pgfscope}%
\pgfsys@transformshift{2.823755in}{0.939019in}%
\pgfsys@useobject{currentmarker}{}%
\end{pgfscope}%
\begin{pgfscope}%
\pgfsys@transformshift{2.873191in}{0.933457in}%
\pgfsys@useobject{currentmarker}{}%
\end{pgfscope}%
\begin{pgfscope}%
\pgfsys@transformshift{2.922628in}{0.926175in}%
\pgfsys@useobject{currentmarker}{}%
\end{pgfscope}%
\begin{pgfscope}%
\pgfsys@transformshift{2.972065in}{0.922608in}%
\pgfsys@useobject{currentmarker}{}%
\end{pgfscope}%
\begin{pgfscope}%
\pgfsys@transformshift{3.021502in}{0.913468in}%
\pgfsys@useobject{currentmarker}{}%
\end{pgfscope}%
\begin{pgfscope}%
\pgfsys@transformshift{3.070939in}{0.911181in}%
\pgfsys@useobject{currentmarker}{}%
\end{pgfscope}%
\begin{pgfscope}%
\pgfsys@transformshift{3.120376in}{0.905485in}%
\pgfsys@useobject{currentmarker}{}%
\end{pgfscope}%
\begin{pgfscope}%
\pgfsys@transformshift{3.169812in}{0.904239in}%
\pgfsys@useobject{currentmarker}{}%
\end{pgfscope}%
\begin{pgfscope}%
\pgfsys@transformshift{3.219249in}{0.904105in}%
\pgfsys@useobject{currentmarker}{}%
\end{pgfscope}%
\begin{pgfscope}%
\pgfsys@transformshift{3.268686in}{0.887771in}%
\pgfsys@useobject{currentmarker}{}%
\end{pgfscope}%
\begin{pgfscope}%
\pgfsys@transformshift{3.318123in}{0.882661in}%
\pgfsys@useobject{currentmarker}{}%
\end{pgfscope}%
\begin{pgfscope}%
\pgfsys@transformshift{3.367560in}{0.876614in}%
\pgfsys@useobject{currentmarker}{}%
\end{pgfscope}%
\begin{pgfscope}%
\pgfsys@transformshift{3.416997in}{0.875693in}%
\pgfsys@useobject{currentmarker}{}%
\end{pgfscope}%
\begin{pgfscope}%
\pgfsys@transformshift{3.466434in}{0.872685in}%
\pgfsys@useobject{currentmarker}{}%
\end{pgfscope}%
\begin{pgfscope}%
\pgfsys@transformshift{3.515870in}{0.871745in}%
\pgfsys@useobject{currentmarker}{}%
\end{pgfscope}%
\begin{pgfscope}%
\pgfsys@transformshift{3.565307in}{0.864204in}%
\pgfsys@useobject{currentmarker}{}%
\end{pgfscope}%
\begin{pgfscope}%
\pgfsys@transformshift{3.614744in}{0.859200in}%
\pgfsys@useobject{currentmarker}{}%
\end{pgfscope}%
\begin{pgfscope}%
\pgfsys@transformshift{3.664181in}{0.857718in}%
\pgfsys@useobject{currentmarker}{}%
\end{pgfscope}%
\begin{pgfscope}%
\pgfsys@transformshift{3.713618in}{0.856053in}%
\pgfsys@useobject{currentmarker}{}%
\end{pgfscope}%
\begin{pgfscope}%
\pgfsys@transformshift{3.763055in}{0.852230in}%
\pgfsys@useobject{currentmarker}{}%
\end{pgfscope}%
\begin{pgfscope}%
\pgfsys@transformshift{3.812492in}{0.846052in}%
\pgfsys@useobject{currentmarker}{}%
\end{pgfscope}%
\begin{pgfscope}%
\pgfsys@transformshift{3.861928in}{0.842103in}%
\pgfsys@useobject{currentmarker}{}%
\end{pgfscope}%
\begin{pgfscope}%
\pgfsys@transformshift{3.911365in}{0.840484in}%
\pgfsys@useobject{currentmarker}{}%
\end{pgfscope}%
\begin{pgfscope}%
\pgfsys@transformshift{3.960802in}{0.835420in}%
\pgfsys@useobject{currentmarker}{}%
\end{pgfscope}%
\begin{pgfscope}%
\pgfsys@transformshift{4.010239in}{0.831448in}%
\pgfsys@useobject{currentmarker}{}%
\end{pgfscope}%
\begin{pgfscope}%
\pgfsys@transformshift{4.059676in}{0.830156in}%
\pgfsys@useobject{currentmarker}{}%
\end{pgfscope}%
\begin{pgfscope}%
\pgfsys@transformshift{4.109113in}{0.828107in}%
\pgfsys@useobject{currentmarker}{}%
\end{pgfscope}%
\begin{pgfscope}%
\pgfsys@transformshift{4.158550in}{0.826587in}%
\pgfsys@useobject{currentmarker}{}%
\end{pgfscope}%
\begin{pgfscope}%
\pgfsys@transformshift{4.207986in}{0.825745in}%
\pgfsys@useobject{currentmarker}{}%
\end{pgfscope}%
\begin{pgfscope}%
\pgfsys@transformshift{4.257423in}{0.817927in}%
\pgfsys@useobject{currentmarker}{}%
\end{pgfscope}%
\begin{pgfscope}%
\pgfsys@transformshift{4.306860in}{0.815199in}%
\pgfsys@useobject{currentmarker}{}%
\end{pgfscope}%
\begin{pgfscope}%
\pgfsys@transformshift{4.356297in}{0.806980in}%
\pgfsys@useobject{currentmarker}{}%
\end{pgfscope}%
\begin{pgfscope}%
\pgfsys@transformshift{4.405734in}{0.802218in}%
\pgfsys@useobject{currentmarker}{}%
\end{pgfscope}%
\begin{pgfscope}%
\pgfsys@transformshift{4.455171in}{0.801932in}%
\pgfsys@useobject{currentmarker}{}%
\end{pgfscope}%
\begin{pgfscope}%
\pgfsys@transformshift{4.504607in}{0.798273in}%
\pgfsys@useobject{currentmarker}{}%
\end{pgfscope}%
\begin{pgfscope}%
\pgfsys@transformshift{4.554044in}{0.788876in}%
\pgfsys@useobject{currentmarker}{}%
\end{pgfscope}%
\begin{pgfscope}%
\pgfsys@transformshift{4.603481in}{0.777263in}%
\pgfsys@useobject{currentmarker}{}%
\end{pgfscope}%
\begin{pgfscope}%
\pgfsys@transformshift{4.652918in}{0.771555in}%
\pgfsys@useobject{currentmarker}{}%
\end{pgfscope}%
\begin{pgfscope}%
\pgfsys@transformshift{4.702355in}{0.764889in}%
\pgfsys@useobject{currentmarker}{}%
\end{pgfscope}%
\begin{pgfscope}%
\pgfsys@transformshift{4.751792in}{0.758917in}%
\pgfsys@useobject{currentmarker}{}%
\end{pgfscope}%
\begin{pgfscope}%
\pgfsys@transformshift{4.801229in}{0.756401in}%
\pgfsys@useobject{currentmarker}{}%
\end{pgfscope}%
\begin{pgfscope}%
\pgfsys@transformshift{4.850665in}{0.755406in}%
\pgfsys@useobject{currentmarker}{}%
\end{pgfscope}%
\begin{pgfscope}%
\pgfsys@transformshift{4.900102in}{0.734903in}%
\pgfsys@useobject{currentmarker}{}%
\end{pgfscope}%
\begin{pgfscope}%
\pgfsys@transformshift{4.949539in}{0.728066in}%
\pgfsys@useobject{currentmarker}{}%
\end{pgfscope}%
\begin{pgfscope}%
\pgfsys@transformshift{4.998976in}{0.710746in}%
\pgfsys@useobject{currentmarker}{}%
\end{pgfscope}%
\begin{pgfscope}%
\pgfsys@transformshift{5.048413in}{0.697983in}%
\pgfsys@useobject{currentmarker}{}%
\end{pgfscope}%
\begin{pgfscope}%
\pgfsys@transformshift{5.097850in}{0.460676in}%
\pgfsys@useobject{currentmarker}{}%
\end{pgfscope}%
\end{pgfscope}%
\begin{pgfscope}%
\pgfpathrectangle{\pgfqpoint{0.581793in}{0.460676in}}{\pgfqpoint{4.731107in}{2.134248in}}%
\pgfusepath{clip}%
\pgfsetbuttcap%
\pgfsetroundjoin%
\definecolor{currentfill}{rgb}{0.172549,0.627451,0.172549}%
\pgfsetfillcolor{currentfill}%
\pgfsetfillopacity{0.750000}%
\pgfsetlinewidth{1.003750pt}%
\definecolor{currentstroke}{rgb}{0.172549,0.627451,0.172549}%
\pgfsetstrokecolor{currentstroke}%
\pgfsetstrokeopacity{0.750000}%
\pgfsetdash{}{0pt}%
\pgfsys@defobject{currentmarker}{\pgfqpoint{-0.019018in}{-0.019018in}}{\pgfqpoint{0.019018in}{0.019018in}}{%
\pgfpathmoveto{\pgfqpoint{-0.019018in}{-0.019018in}}%
\pgfpathlineto{\pgfqpoint{0.019018in}{-0.019018in}}%
\pgfpathlineto{\pgfqpoint{0.019018in}{0.019018in}}%
\pgfpathlineto{\pgfqpoint{-0.019018in}{0.019018in}}%
\pgfpathclose%
\pgfusepath{stroke,fill}%
}%
\begin{pgfscope}%
\pgfsys@transformshift{0.796844in}{2.466370in}%
\pgfsys@useobject{currentmarker}{}%
\end{pgfscope}%
\begin{pgfscope}%
\pgfsys@transformshift{0.846280in}{2.466370in}%
\pgfsys@useobject{currentmarker}{}%
\end{pgfscope}%
\begin{pgfscope}%
\pgfsys@transformshift{0.895717in}{1.985356in}%
\pgfsys@useobject{currentmarker}{}%
\end{pgfscope}%
\begin{pgfscope}%
\pgfsys@transformshift{0.945154in}{1.630456in}%
\pgfsys@useobject{currentmarker}{}%
\end{pgfscope}%
\begin{pgfscope}%
\pgfsys@transformshift{0.994591in}{1.292006in}%
\pgfsys@useobject{currentmarker}{}%
\end{pgfscope}%
\begin{pgfscope}%
\pgfsys@transformshift{1.044028in}{1.103121in}%
\pgfsys@useobject{currentmarker}{}%
\end{pgfscope}%
\begin{pgfscope}%
\pgfsys@transformshift{1.093465in}{1.058927in}%
\pgfsys@useobject{currentmarker}{}%
\end{pgfscope}%
\begin{pgfscope}%
\pgfsys@transformshift{1.142902in}{1.058750in}%
\pgfsys@useobject{currentmarker}{}%
\end{pgfscope}%
\begin{pgfscope}%
\pgfsys@transformshift{1.192338in}{1.053068in}%
\pgfsys@useobject{currentmarker}{}%
\end{pgfscope}%
\begin{pgfscope}%
\pgfsys@transformshift{1.241775in}{1.053055in}%
\pgfsys@useobject{currentmarker}{}%
\end{pgfscope}%
\begin{pgfscope}%
\pgfsys@transformshift{1.291212in}{1.031466in}%
\pgfsys@useobject{currentmarker}{}%
\end{pgfscope}%
\begin{pgfscope}%
\pgfsys@transformshift{1.340649in}{0.979651in}%
\pgfsys@useobject{currentmarker}{}%
\end{pgfscope}%
\begin{pgfscope}%
\pgfsys@transformshift{1.390086in}{0.977099in}%
\pgfsys@useobject{currentmarker}{}%
\end{pgfscope}%
\begin{pgfscope}%
\pgfsys@transformshift{1.439523in}{0.968369in}%
\pgfsys@useobject{currentmarker}{}%
\end{pgfscope}%
\begin{pgfscope}%
\pgfsys@transformshift{1.488960in}{0.961691in}%
\pgfsys@useobject{currentmarker}{}%
\end{pgfscope}%
\begin{pgfscope}%
\pgfsys@transformshift{1.538396in}{0.956234in}%
\pgfsys@useobject{currentmarker}{}%
\end{pgfscope}%
\begin{pgfscope}%
\pgfsys@transformshift{1.587833in}{0.956178in}%
\pgfsys@useobject{currentmarker}{}%
\end{pgfscope}%
\begin{pgfscope}%
\pgfsys@transformshift{1.637270in}{0.955437in}%
\pgfsys@useobject{currentmarker}{}%
\end{pgfscope}%
\begin{pgfscope}%
\pgfsys@transformshift{1.686707in}{0.938299in}%
\pgfsys@useobject{currentmarker}{}%
\end{pgfscope}%
\begin{pgfscope}%
\pgfsys@transformshift{1.736144in}{0.921192in}%
\pgfsys@useobject{currentmarker}{}%
\end{pgfscope}%
\begin{pgfscope}%
\pgfsys@transformshift{1.785581in}{0.911558in}%
\pgfsys@useobject{currentmarker}{}%
\end{pgfscope}%
\begin{pgfscope}%
\pgfsys@transformshift{1.835018in}{0.911229in}%
\pgfsys@useobject{currentmarker}{}%
\end{pgfscope}%
\begin{pgfscope}%
\pgfsys@transformshift{1.884454in}{0.910037in}%
\pgfsys@useobject{currentmarker}{}%
\end{pgfscope}%
\begin{pgfscope}%
\pgfsys@transformshift{1.933891in}{0.908771in}%
\pgfsys@useobject{currentmarker}{}%
\end{pgfscope}%
\begin{pgfscope}%
\pgfsys@transformshift{1.983328in}{0.908216in}%
\pgfsys@useobject{currentmarker}{}%
\end{pgfscope}%
\begin{pgfscope}%
\pgfsys@transformshift{2.032765in}{0.908044in}%
\pgfsys@useobject{currentmarker}{}%
\end{pgfscope}%
\begin{pgfscope}%
\pgfsys@transformshift{2.082202in}{0.889259in}%
\pgfsys@useobject{currentmarker}{}%
\end{pgfscope}%
\begin{pgfscope}%
\pgfsys@transformshift{2.131639in}{0.886647in}%
\pgfsys@useobject{currentmarker}{}%
\end{pgfscope}%
\begin{pgfscope}%
\pgfsys@transformshift{2.181075in}{0.878469in}%
\pgfsys@useobject{currentmarker}{}%
\end{pgfscope}%
\begin{pgfscope}%
\pgfsys@transformshift{2.230512in}{0.855041in}%
\pgfsys@useobject{currentmarker}{}%
\end{pgfscope}%
\begin{pgfscope}%
\pgfsys@transformshift{2.279949in}{0.854547in}%
\pgfsys@useobject{currentmarker}{}%
\end{pgfscope}%
\begin{pgfscope}%
\pgfsys@transformshift{2.329386in}{0.853961in}%
\pgfsys@useobject{currentmarker}{}%
\end{pgfscope}%
\begin{pgfscope}%
\pgfsys@transformshift{2.378823in}{0.848267in}%
\pgfsys@useobject{currentmarker}{}%
\end{pgfscope}%
\begin{pgfscope}%
\pgfsys@transformshift{2.428260in}{0.821870in}%
\pgfsys@useobject{currentmarker}{}%
\end{pgfscope}%
\begin{pgfscope}%
\pgfsys@transformshift{2.477697in}{0.819683in}%
\pgfsys@useobject{currentmarker}{}%
\end{pgfscope}%
\begin{pgfscope}%
\pgfsys@transformshift{2.527133in}{0.819585in}%
\pgfsys@useobject{currentmarker}{}%
\end{pgfscope}%
\begin{pgfscope}%
\pgfsys@transformshift{2.576570in}{0.800932in}%
\pgfsys@useobject{currentmarker}{}%
\end{pgfscope}%
\begin{pgfscope}%
\pgfsys@transformshift{2.626007in}{0.796776in}%
\pgfsys@useobject{currentmarker}{}%
\end{pgfscope}%
\begin{pgfscope}%
\pgfsys@transformshift{2.675444in}{0.795942in}%
\pgfsys@useobject{currentmarker}{}%
\end{pgfscope}%
\begin{pgfscope}%
\pgfsys@transformshift{2.724881in}{0.781286in}%
\pgfsys@useobject{currentmarker}{}%
\end{pgfscope}%
\begin{pgfscope}%
\pgfsys@transformshift{2.774318in}{0.733119in}%
\pgfsys@useobject{currentmarker}{}%
\end{pgfscope}%
\begin{pgfscope}%
\pgfsys@transformshift{2.823755in}{0.720808in}%
\pgfsys@useobject{currentmarker}{}%
\end{pgfscope}%
\begin{pgfscope}%
\pgfsys@transformshift{2.873191in}{0.717181in}%
\pgfsys@useobject{currentmarker}{}%
\end{pgfscope}%
\begin{pgfscope}%
\pgfsys@transformshift{2.922628in}{0.714914in}%
\pgfsys@useobject{currentmarker}{}%
\end{pgfscope}%
\begin{pgfscope}%
\pgfsys@transformshift{2.972065in}{0.698497in}%
\pgfsys@useobject{currentmarker}{}%
\end{pgfscope}%
\begin{pgfscope}%
\pgfsys@transformshift{3.021502in}{0.688884in}%
\pgfsys@useobject{currentmarker}{}%
\end{pgfscope}%
\begin{pgfscope}%
\pgfsys@transformshift{3.070939in}{0.667332in}%
\pgfsys@useobject{currentmarker}{}%
\end{pgfscope}%
\begin{pgfscope}%
\pgfsys@transformshift{3.120376in}{0.659199in}%
\pgfsys@useobject{currentmarker}{}%
\end{pgfscope}%
\begin{pgfscope}%
\pgfsys@transformshift{3.169812in}{0.651288in}%
\pgfsys@useobject{currentmarker}{}%
\end{pgfscope}%
\begin{pgfscope}%
\pgfsys@transformshift{3.219249in}{0.640028in}%
\pgfsys@useobject{currentmarker}{}%
\end{pgfscope}%
\begin{pgfscope}%
\pgfsys@transformshift{3.268686in}{0.633362in}%
\pgfsys@useobject{currentmarker}{}%
\end{pgfscope}%
\begin{pgfscope}%
\pgfsys@transformshift{3.318123in}{0.606314in}%
\pgfsys@useobject{currentmarker}{}%
\end{pgfscope}%
\begin{pgfscope}%
\pgfsys@transformshift{3.367560in}{0.591367in}%
\pgfsys@useobject{currentmarker}{}%
\end{pgfscope}%
\begin{pgfscope}%
\pgfsys@transformshift{3.416997in}{0.574463in}%
\pgfsys@useobject{currentmarker}{}%
\end{pgfscope}%
\begin{pgfscope}%
\pgfsys@transformshift{3.466434in}{0.543905in}%
\pgfsys@useobject{currentmarker}{}%
\end{pgfscope}%
\begin{pgfscope}%
\pgfsys@transformshift{3.515870in}{0.539928in}%
\pgfsys@useobject{currentmarker}{}%
\end{pgfscope}%
\begin{pgfscope}%
\pgfsys@transformshift{3.565307in}{0.537985in}%
\pgfsys@useobject{currentmarker}{}%
\end{pgfscope}%
\begin{pgfscope}%
\pgfsys@transformshift{3.614744in}{0.532642in}%
\pgfsys@useobject{currentmarker}{}%
\end{pgfscope}%
\begin{pgfscope}%
\pgfsys@transformshift{3.664181in}{0.484234in}%
\pgfsys@useobject{currentmarker}{}%
\end{pgfscope}%
\begin{pgfscope}%
\pgfsys@transformshift{3.713618in}{0.460676in}%
\pgfsys@useobject{currentmarker}{}%
\end{pgfscope}%
\begin{pgfscope}%
\pgfsys@transformshift{3.763055in}{0.460676in}%
\pgfsys@useobject{currentmarker}{}%
\end{pgfscope}%
\begin{pgfscope}%
\pgfsys@transformshift{3.812492in}{0.460676in}%
\pgfsys@useobject{currentmarker}{}%
\end{pgfscope}%
\begin{pgfscope}%
\pgfsys@transformshift{3.861928in}{0.460676in}%
\pgfsys@useobject{currentmarker}{}%
\end{pgfscope}%
\begin{pgfscope}%
\pgfsys@transformshift{3.911365in}{0.460676in}%
\pgfsys@useobject{currentmarker}{}%
\end{pgfscope}%
\begin{pgfscope}%
\pgfsys@transformshift{3.960802in}{0.460676in}%
\pgfsys@useobject{currentmarker}{}%
\end{pgfscope}%
\begin{pgfscope}%
\pgfsys@transformshift{4.010239in}{0.460676in}%
\pgfsys@useobject{currentmarker}{}%
\end{pgfscope}%
\begin{pgfscope}%
\pgfsys@transformshift{4.059676in}{0.460676in}%
\pgfsys@useobject{currentmarker}{}%
\end{pgfscope}%
\begin{pgfscope}%
\pgfsys@transformshift{4.109113in}{0.460676in}%
\pgfsys@useobject{currentmarker}{}%
\end{pgfscope}%
\begin{pgfscope}%
\pgfsys@transformshift{4.158550in}{0.460676in}%
\pgfsys@useobject{currentmarker}{}%
\end{pgfscope}%
\begin{pgfscope}%
\pgfsys@transformshift{4.207986in}{0.460676in}%
\pgfsys@useobject{currentmarker}{}%
\end{pgfscope}%
\begin{pgfscope}%
\pgfsys@transformshift{4.257423in}{0.460676in}%
\pgfsys@useobject{currentmarker}{}%
\end{pgfscope}%
\begin{pgfscope}%
\pgfsys@transformshift{4.306860in}{0.460676in}%
\pgfsys@useobject{currentmarker}{}%
\end{pgfscope}%
\begin{pgfscope}%
\pgfsys@transformshift{4.356297in}{0.460676in}%
\pgfsys@useobject{currentmarker}{}%
\end{pgfscope}%
\begin{pgfscope}%
\pgfsys@transformshift{4.405734in}{0.460676in}%
\pgfsys@useobject{currentmarker}{}%
\end{pgfscope}%
\begin{pgfscope}%
\pgfsys@transformshift{4.455171in}{0.460676in}%
\pgfsys@useobject{currentmarker}{}%
\end{pgfscope}%
\begin{pgfscope}%
\pgfsys@transformshift{4.504607in}{0.460676in}%
\pgfsys@useobject{currentmarker}{}%
\end{pgfscope}%
\begin{pgfscope}%
\pgfsys@transformshift{4.554044in}{0.460676in}%
\pgfsys@useobject{currentmarker}{}%
\end{pgfscope}%
\begin{pgfscope}%
\pgfsys@transformshift{4.603481in}{0.460676in}%
\pgfsys@useobject{currentmarker}{}%
\end{pgfscope}%
\begin{pgfscope}%
\pgfsys@transformshift{4.652918in}{0.460676in}%
\pgfsys@useobject{currentmarker}{}%
\end{pgfscope}%
\begin{pgfscope}%
\pgfsys@transformshift{4.702355in}{0.460676in}%
\pgfsys@useobject{currentmarker}{}%
\end{pgfscope}%
\begin{pgfscope}%
\pgfsys@transformshift{4.751792in}{0.460676in}%
\pgfsys@useobject{currentmarker}{}%
\end{pgfscope}%
\begin{pgfscope}%
\pgfsys@transformshift{4.801229in}{0.460676in}%
\pgfsys@useobject{currentmarker}{}%
\end{pgfscope}%
\begin{pgfscope}%
\pgfsys@transformshift{4.850665in}{0.460676in}%
\pgfsys@useobject{currentmarker}{}%
\end{pgfscope}%
\begin{pgfscope}%
\pgfsys@transformshift{4.900102in}{0.460676in}%
\pgfsys@useobject{currentmarker}{}%
\end{pgfscope}%
\begin{pgfscope}%
\pgfsys@transformshift{4.949539in}{0.460676in}%
\pgfsys@useobject{currentmarker}{}%
\end{pgfscope}%
\begin{pgfscope}%
\pgfsys@transformshift{4.998976in}{0.460676in}%
\pgfsys@useobject{currentmarker}{}%
\end{pgfscope}%
\begin{pgfscope}%
\pgfsys@transformshift{5.048413in}{0.460676in}%
\pgfsys@useobject{currentmarker}{}%
\end{pgfscope}%
\begin{pgfscope}%
\pgfsys@transformshift{5.097850in}{0.460676in}%
\pgfsys@useobject{currentmarker}{}%
\end{pgfscope}%
\end{pgfscope}%
\begin{pgfscope}%
\pgfpathrectangle{\pgfqpoint{0.581793in}{0.460676in}}{\pgfqpoint{4.731107in}{2.134248in}}%
\pgfusepath{clip}%
\pgfsetbuttcap%
\pgfsetroundjoin%
\definecolor{currentfill}{rgb}{0.839216,0.152941,0.156863}%
\pgfsetfillcolor{currentfill}%
\pgfsetfillopacity{0.750000}%
\pgfsetlinewidth{1.003750pt}%
\definecolor{currentstroke}{rgb}{0.839216,0.152941,0.156863}%
\pgfsetstrokecolor{currentstroke}%
\pgfsetstrokeopacity{0.750000}%
\pgfsetdash{}{0pt}%
\pgfsys@defobject{currentmarker}{\pgfqpoint{-0.019018in}{-0.019018in}}{\pgfqpoint{0.019018in}{0.019018in}}{%
\pgfpathmoveto{\pgfqpoint{0.000000in}{0.019018in}}%
\pgfpathlineto{\pgfqpoint{-0.019018in}{-0.019018in}}%
\pgfpathlineto{\pgfqpoint{0.019018in}{-0.019018in}}%
\pgfpathclose%
\pgfusepath{stroke,fill}%
}%
\begin{pgfscope}%
\pgfsys@transformshift{0.796844in}{2.475384in}%
\pgfsys@useobject{currentmarker}{}%
\end{pgfscope}%
\begin{pgfscope}%
\pgfsys@transformshift{0.846280in}{2.475384in}%
\pgfsys@useobject{currentmarker}{}%
\end{pgfscope}%
\begin{pgfscope}%
\pgfsys@transformshift{0.895717in}{2.208501in}%
\pgfsys@useobject{currentmarker}{}%
\end{pgfscope}%
\begin{pgfscope}%
\pgfsys@transformshift{0.945154in}{2.187739in}%
\pgfsys@useobject{currentmarker}{}%
\end{pgfscope}%
\begin{pgfscope}%
\pgfsys@transformshift{0.994591in}{1.780588in}%
\pgfsys@useobject{currentmarker}{}%
\end{pgfscope}%
\begin{pgfscope}%
\pgfsys@transformshift{1.044028in}{1.605000in}%
\pgfsys@useobject{currentmarker}{}%
\end{pgfscope}%
\begin{pgfscope}%
\pgfsys@transformshift{1.093465in}{1.451330in}%
\pgfsys@useobject{currentmarker}{}%
\end{pgfscope}%
\begin{pgfscope}%
\pgfsys@transformshift{1.142902in}{1.382239in}%
\pgfsys@useobject{currentmarker}{}%
\end{pgfscope}%
\begin{pgfscope}%
\pgfsys@transformshift{1.192338in}{1.354410in}%
\pgfsys@useobject{currentmarker}{}%
\end{pgfscope}%
\begin{pgfscope}%
\pgfsys@transformshift{1.241775in}{1.304412in}%
\pgfsys@useobject{currentmarker}{}%
\end{pgfscope}%
\begin{pgfscope}%
\pgfsys@transformshift{1.291212in}{1.252705in}%
\pgfsys@useobject{currentmarker}{}%
\end{pgfscope}%
\begin{pgfscope}%
\pgfsys@transformshift{1.340649in}{1.130658in}%
\pgfsys@useobject{currentmarker}{}%
\end{pgfscope}%
\begin{pgfscope}%
\pgfsys@transformshift{1.390086in}{1.123425in}%
\pgfsys@useobject{currentmarker}{}%
\end{pgfscope}%
\begin{pgfscope}%
\pgfsys@transformshift{1.439523in}{1.070970in}%
\pgfsys@useobject{currentmarker}{}%
\end{pgfscope}%
\begin{pgfscope}%
\pgfsys@transformshift{1.488960in}{1.031277in}%
\pgfsys@useobject{currentmarker}{}%
\end{pgfscope}%
\begin{pgfscope}%
\pgfsys@transformshift{1.538396in}{0.977902in}%
\pgfsys@useobject{currentmarker}{}%
\end{pgfscope}%
\begin{pgfscope}%
\pgfsys@transformshift{1.587833in}{0.971804in}%
\pgfsys@useobject{currentmarker}{}%
\end{pgfscope}%
\begin{pgfscope}%
\pgfsys@transformshift{1.637270in}{0.963137in}%
\pgfsys@useobject{currentmarker}{}%
\end{pgfscope}%
\begin{pgfscope}%
\pgfsys@transformshift{1.686707in}{0.931584in}%
\pgfsys@useobject{currentmarker}{}%
\end{pgfscope}%
\begin{pgfscope}%
\pgfsys@transformshift{1.736144in}{0.925581in}%
\pgfsys@useobject{currentmarker}{}%
\end{pgfscope}%
\begin{pgfscope}%
\pgfsys@transformshift{1.785581in}{0.920040in}%
\pgfsys@useobject{currentmarker}{}%
\end{pgfscope}%
\begin{pgfscope}%
\pgfsys@transformshift{1.835018in}{0.915091in}%
\pgfsys@useobject{currentmarker}{}%
\end{pgfscope}%
\begin{pgfscope}%
\pgfsys@transformshift{1.884454in}{0.900755in}%
\pgfsys@useobject{currentmarker}{}%
\end{pgfscope}%
\begin{pgfscope}%
\pgfsys@transformshift{1.933891in}{0.891716in}%
\pgfsys@useobject{currentmarker}{}%
\end{pgfscope}%
\begin{pgfscope}%
\pgfsys@transformshift{1.983328in}{0.886414in}%
\pgfsys@useobject{currentmarker}{}%
\end{pgfscope}%
\begin{pgfscope}%
\pgfsys@transformshift{2.032765in}{0.886203in}%
\pgfsys@useobject{currentmarker}{}%
\end{pgfscope}%
\begin{pgfscope}%
\pgfsys@transformshift{2.082202in}{0.882935in}%
\pgfsys@useobject{currentmarker}{}%
\end{pgfscope}%
\begin{pgfscope}%
\pgfsys@transformshift{2.131639in}{0.867233in}%
\pgfsys@useobject{currentmarker}{}%
\end{pgfscope}%
\begin{pgfscope}%
\pgfsys@transformshift{2.181075in}{0.864619in}%
\pgfsys@useobject{currentmarker}{}%
\end{pgfscope}%
\begin{pgfscope}%
\pgfsys@transformshift{2.230512in}{0.863109in}%
\pgfsys@useobject{currentmarker}{}%
\end{pgfscope}%
\begin{pgfscope}%
\pgfsys@transformshift{2.279949in}{0.863021in}%
\pgfsys@useobject{currentmarker}{}%
\end{pgfscope}%
\begin{pgfscope}%
\pgfsys@transformshift{2.329386in}{0.860917in}%
\pgfsys@useobject{currentmarker}{}%
\end{pgfscope}%
\begin{pgfscope}%
\pgfsys@transformshift{2.378823in}{0.845165in}%
\pgfsys@useobject{currentmarker}{}%
\end{pgfscope}%
\begin{pgfscope}%
\pgfsys@transformshift{2.428260in}{0.841568in}%
\pgfsys@useobject{currentmarker}{}%
\end{pgfscope}%
\begin{pgfscope}%
\pgfsys@transformshift{2.477697in}{0.841541in}%
\pgfsys@useobject{currentmarker}{}%
\end{pgfscope}%
\begin{pgfscope}%
\pgfsys@transformshift{2.527133in}{0.833384in}%
\pgfsys@useobject{currentmarker}{}%
\end{pgfscope}%
\begin{pgfscope}%
\pgfsys@transformshift{2.576570in}{0.819957in}%
\pgfsys@useobject{currentmarker}{}%
\end{pgfscope}%
\begin{pgfscope}%
\pgfsys@transformshift{2.626007in}{0.815331in}%
\pgfsys@useobject{currentmarker}{}%
\end{pgfscope}%
\begin{pgfscope}%
\pgfsys@transformshift{2.675444in}{0.808425in}%
\pgfsys@useobject{currentmarker}{}%
\end{pgfscope}%
\begin{pgfscope}%
\pgfsys@transformshift{2.724881in}{0.796893in}%
\pgfsys@useobject{currentmarker}{}%
\end{pgfscope}%
\begin{pgfscope}%
\pgfsys@transformshift{2.774318in}{0.783760in}%
\pgfsys@useobject{currentmarker}{}%
\end{pgfscope}%
\begin{pgfscope}%
\pgfsys@transformshift{2.823755in}{0.713184in}%
\pgfsys@useobject{currentmarker}{}%
\end{pgfscope}%
\begin{pgfscope}%
\pgfsys@transformshift{2.873191in}{0.687182in}%
\pgfsys@useobject{currentmarker}{}%
\end{pgfscope}%
\begin{pgfscope}%
\pgfsys@transformshift{2.922628in}{0.642671in}%
\pgfsys@useobject{currentmarker}{}%
\end{pgfscope}%
\begin{pgfscope}%
\pgfsys@transformshift{2.972065in}{0.541006in}%
\pgfsys@useobject{currentmarker}{}%
\end{pgfscope}%
\begin{pgfscope}%
\pgfsys@transformshift{3.021502in}{0.532295in}%
\pgfsys@useobject{currentmarker}{}%
\end{pgfscope}%
\begin{pgfscope}%
\pgfsys@transformshift{3.070939in}{0.460676in}%
\pgfsys@useobject{currentmarker}{}%
\end{pgfscope}%
\begin{pgfscope}%
\pgfsys@transformshift{3.120376in}{0.460676in}%
\pgfsys@useobject{currentmarker}{}%
\end{pgfscope}%
\begin{pgfscope}%
\pgfsys@transformshift{3.169812in}{0.460676in}%
\pgfsys@useobject{currentmarker}{}%
\end{pgfscope}%
\begin{pgfscope}%
\pgfsys@transformshift{3.219249in}{0.460676in}%
\pgfsys@useobject{currentmarker}{}%
\end{pgfscope}%
\begin{pgfscope}%
\pgfsys@transformshift{3.268686in}{0.460676in}%
\pgfsys@useobject{currentmarker}{}%
\end{pgfscope}%
\begin{pgfscope}%
\pgfsys@transformshift{3.318123in}{0.460676in}%
\pgfsys@useobject{currentmarker}{}%
\end{pgfscope}%
\begin{pgfscope}%
\pgfsys@transformshift{3.367560in}{0.460676in}%
\pgfsys@useobject{currentmarker}{}%
\end{pgfscope}%
\begin{pgfscope}%
\pgfsys@transformshift{3.416997in}{0.460676in}%
\pgfsys@useobject{currentmarker}{}%
\end{pgfscope}%
\begin{pgfscope}%
\pgfsys@transformshift{3.466434in}{0.460676in}%
\pgfsys@useobject{currentmarker}{}%
\end{pgfscope}%
\begin{pgfscope}%
\pgfsys@transformshift{3.515870in}{0.460676in}%
\pgfsys@useobject{currentmarker}{}%
\end{pgfscope}%
\begin{pgfscope}%
\pgfsys@transformshift{3.565307in}{0.460676in}%
\pgfsys@useobject{currentmarker}{}%
\end{pgfscope}%
\begin{pgfscope}%
\pgfsys@transformshift{3.614744in}{0.460676in}%
\pgfsys@useobject{currentmarker}{}%
\end{pgfscope}%
\begin{pgfscope}%
\pgfsys@transformshift{3.664181in}{0.460676in}%
\pgfsys@useobject{currentmarker}{}%
\end{pgfscope}%
\begin{pgfscope}%
\pgfsys@transformshift{3.713618in}{0.460676in}%
\pgfsys@useobject{currentmarker}{}%
\end{pgfscope}%
\begin{pgfscope}%
\pgfsys@transformshift{3.763055in}{0.460676in}%
\pgfsys@useobject{currentmarker}{}%
\end{pgfscope}%
\begin{pgfscope}%
\pgfsys@transformshift{3.812492in}{0.460676in}%
\pgfsys@useobject{currentmarker}{}%
\end{pgfscope}%
\begin{pgfscope}%
\pgfsys@transformshift{3.861928in}{0.460676in}%
\pgfsys@useobject{currentmarker}{}%
\end{pgfscope}%
\begin{pgfscope}%
\pgfsys@transformshift{3.911365in}{0.460676in}%
\pgfsys@useobject{currentmarker}{}%
\end{pgfscope}%
\begin{pgfscope}%
\pgfsys@transformshift{3.960802in}{0.460676in}%
\pgfsys@useobject{currentmarker}{}%
\end{pgfscope}%
\begin{pgfscope}%
\pgfsys@transformshift{4.010239in}{0.460676in}%
\pgfsys@useobject{currentmarker}{}%
\end{pgfscope}%
\begin{pgfscope}%
\pgfsys@transformshift{4.059676in}{0.460676in}%
\pgfsys@useobject{currentmarker}{}%
\end{pgfscope}%
\begin{pgfscope}%
\pgfsys@transformshift{4.109113in}{0.460676in}%
\pgfsys@useobject{currentmarker}{}%
\end{pgfscope}%
\begin{pgfscope}%
\pgfsys@transformshift{4.158550in}{0.460676in}%
\pgfsys@useobject{currentmarker}{}%
\end{pgfscope}%
\begin{pgfscope}%
\pgfsys@transformshift{4.207986in}{0.460676in}%
\pgfsys@useobject{currentmarker}{}%
\end{pgfscope}%
\begin{pgfscope}%
\pgfsys@transformshift{4.257423in}{0.460676in}%
\pgfsys@useobject{currentmarker}{}%
\end{pgfscope}%
\begin{pgfscope}%
\pgfsys@transformshift{4.306860in}{0.460676in}%
\pgfsys@useobject{currentmarker}{}%
\end{pgfscope}%
\begin{pgfscope}%
\pgfsys@transformshift{4.356297in}{0.460676in}%
\pgfsys@useobject{currentmarker}{}%
\end{pgfscope}%
\begin{pgfscope}%
\pgfsys@transformshift{4.405734in}{0.460676in}%
\pgfsys@useobject{currentmarker}{}%
\end{pgfscope}%
\begin{pgfscope}%
\pgfsys@transformshift{4.455171in}{0.460676in}%
\pgfsys@useobject{currentmarker}{}%
\end{pgfscope}%
\begin{pgfscope}%
\pgfsys@transformshift{4.504607in}{0.460676in}%
\pgfsys@useobject{currentmarker}{}%
\end{pgfscope}%
\begin{pgfscope}%
\pgfsys@transformshift{4.554044in}{0.460676in}%
\pgfsys@useobject{currentmarker}{}%
\end{pgfscope}%
\begin{pgfscope}%
\pgfsys@transformshift{4.603481in}{0.460676in}%
\pgfsys@useobject{currentmarker}{}%
\end{pgfscope}%
\begin{pgfscope}%
\pgfsys@transformshift{4.652918in}{0.460676in}%
\pgfsys@useobject{currentmarker}{}%
\end{pgfscope}%
\begin{pgfscope}%
\pgfsys@transformshift{4.702355in}{0.460676in}%
\pgfsys@useobject{currentmarker}{}%
\end{pgfscope}%
\begin{pgfscope}%
\pgfsys@transformshift{4.751792in}{0.460676in}%
\pgfsys@useobject{currentmarker}{}%
\end{pgfscope}%
\begin{pgfscope}%
\pgfsys@transformshift{4.801229in}{0.460676in}%
\pgfsys@useobject{currentmarker}{}%
\end{pgfscope}%
\begin{pgfscope}%
\pgfsys@transformshift{4.850665in}{0.460676in}%
\pgfsys@useobject{currentmarker}{}%
\end{pgfscope}%
\begin{pgfscope}%
\pgfsys@transformshift{4.900102in}{0.460676in}%
\pgfsys@useobject{currentmarker}{}%
\end{pgfscope}%
\begin{pgfscope}%
\pgfsys@transformshift{4.949539in}{0.460676in}%
\pgfsys@useobject{currentmarker}{}%
\end{pgfscope}%
\begin{pgfscope}%
\pgfsys@transformshift{4.998976in}{0.460676in}%
\pgfsys@useobject{currentmarker}{}%
\end{pgfscope}%
\begin{pgfscope}%
\pgfsys@transformshift{5.048413in}{0.460676in}%
\pgfsys@useobject{currentmarker}{}%
\end{pgfscope}%
\begin{pgfscope}%
\pgfsys@transformshift{5.097850in}{0.460676in}%
\pgfsys@useobject{currentmarker}{}%
\end{pgfscope}%
\end{pgfscope}%
\begin{pgfscope}%
\pgfpathrectangle{\pgfqpoint{0.581793in}{0.460676in}}{\pgfqpoint{4.731107in}{2.134248in}}%
\pgfusepath{clip}%
\pgfsetbuttcap%
\pgfsetroundjoin%
\definecolor{currentfill}{rgb}{0.580392,0.403922,0.741176}%
\pgfsetfillcolor{currentfill}%
\pgfsetfillopacity{0.750000}%
\pgfsetlinewidth{1.003750pt}%
\definecolor{currentstroke}{rgb}{0.580392,0.403922,0.741176}%
\pgfsetstrokecolor{currentstroke}%
\pgfsetstrokeopacity{0.750000}%
\pgfsetdash{}{0pt}%
\pgfsys@defobject{currentmarker}{\pgfqpoint{-0.026896in}{-0.026896in}}{\pgfqpoint{0.026896in}{0.026896in}}{%
\pgfpathmoveto{\pgfqpoint{-0.000000in}{-0.026896in}}%
\pgfpathlineto{\pgfqpoint{0.026896in}{0.000000in}}%
\pgfpathlineto{\pgfqpoint{0.000000in}{0.026896in}}%
\pgfpathlineto{\pgfqpoint{-0.026896in}{0.000000in}}%
\pgfpathclose%
\pgfusepath{stroke,fill}%
}%
\begin{pgfscope}%
\pgfsys@transformshift{0.796844in}{2.484359in}%
\pgfsys@useobject{currentmarker}{}%
\end{pgfscope}%
\begin{pgfscope}%
\pgfsys@transformshift{0.846280in}{2.484359in}%
\pgfsys@useobject{currentmarker}{}%
\end{pgfscope}%
\begin{pgfscope}%
\pgfsys@transformshift{0.895717in}{2.353957in}%
\pgfsys@useobject{currentmarker}{}%
\end{pgfscope}%
\begin{pgfscope}%
\pgfsys@transformshift{0.945154in}{2.101132in}%
\pgfsys@useobject{currentmarker}{}%
\end{pgfscope}%
\begin{pgfscope}%
\pgfsys@transformshift{0.994591in}{2.044186in}%
\pgfsys@useobject{currentmarker}{}%
\end{pgfscope}%
\begin{pgfscope}%
\pgfsys@transformshift{1.044028in}{2.023726in}%
\pgfsys@useobject{currentmarker}{}%
\end{pgfscope}%
\begin{pgfscope}%
\pgfsys@transformshift{1.093465in}{1.981579in}%
\pgfsys@useobject{currentmarker}{}%
\end{pgfscope}%
\begin{pgfscope}%
\pgfsys@transformshift{1.142902in}{1.467767in}%
\pgfsys@useobject{currentmarker}{}%
\end{pgfscope}%
\begin{pgfscope}%
\pgfsys@transformshift{1.192338in}{1.335348in}%
\pgfsys@useobject{currentmarker}{}%
\end{pgfscope}%
\begin{pgfscope}%
\pgfsys@transformshift{1.241775in}{1.315214in}%
\pgfsys@useobject{currentmarker}{}%
\end{pgfscope}%
\begin{pgfscope}%
\pgfsys@transformshift{1.291212in}{1.284207in}%
\pgfsys@useobject{currentmarker}{}%
\end{pgfscope}%
\begin{pgfscope}%
\pgfsys@transformshift{1.340649in}{1.215122in}%
\pgfsys@useobject{currentmarker}{}%
\end{pgfscope}%
\begin{pgfscope}%
\pgfsys@transformshift{1.390086in}{1.195685in}%
\pgfsys@useobject{currentmarker}{}%
\end{pgfscope}%
\begin{pgfscope}%
\pgfsys@transformshift{1.439523in}{1.182468in}%
\pgfsys@useobject{currentmarker}{}%
\end{pgfscope}%
\begin{pgfscope}%
\pgfsys@transformshift{1.488960in}{1.168404in}%
\pgfsys@useobject{currentmarker}{}%
\end{pgfscope}%
\begin{pgfscope}%
\pgfsys@transformshift{1.538396in}{1.151700in}%
\pgfsys@useobject{currentmarker}{}%
\end{pgfscope}%
\begin{pgfscope}%
\pgfsys@transformshift{1.587833in}{1.138336in}%
\pgfsys@useobject{currentmarker}{}%
\end{pgfscope}%
\begin{pgfscope}%
\pgfsys@transformshift{1.637270in}{1.125851in}%
\pgfsys@useobject{currentmarker}{}%
\end{pgfscope}%
\begin{pgfscope}%
\pgfsys@transformshift{1.686707in}{1.107357in}%
\pgfsys@useobject{currentmarker}{}%
\end{pgfscope}%
\begin{pgfscope}%
\pgfsys@transformshift{1.736144in}{1.099358in}%
\pgfsys@useobject{currentmarker}{}%
\end{pgfscope}%
\begin{pgfscope}%
\pgfsys@transformshift{1.785581in}{1.095565in}%
\pgfsys@useobject{currentmarker}{}%
\end{pgfscope}%
\begin{pgfscope}%
\pgfsys@transformshift{1.835018in}{1.089303in}%
\pgfsys@useobject{currentmarker}{}%
\end{pgfscope}%
\begin{pgfscope}%
\pgfsys@transformshift{1.884454in}{1.081653in}%
\pgfsys@useobject{currentmarker}{}%
\end{pgfscope}%
\begin{pgfscope}%
\pgfsys@transformshift{1.933891in}{1.080482in}%
\pgfsys@useobject{currentmarker}{}%
\end{pgfscope}%
\begin{pgfscope}%
\pgfsys@transformshift{1.983328in}{1.071007in}%
\pgfsys@useobject{currentmarker}{}%
\end{pgfscope}%
\begin{pgfscope}%
\pgfsys@transformshift{2.032765in}{1.068744in}%
\pgfsys@useobject{currentmarker}{}%
\end{pgfscope}%
\begin{pgfscope}%
\pgfsys@transformshift{2.082202in}{1.054304in}%
\pgfsys@useobject{currentmarker}{}%
\end{pgfscope}%
\begin{pgfscope}%
\pgfsys@transformshift{2.131639in}{1.052123in}%
\pgfsys@useobject{currentmarker}{}%
\end{pgfscope}%
\begin{pgfscope}%
\pgfsys@transformshift{2.181075in}{1.043487in}%
\pgfsys@useobject{currentmarker}{}%
\end{pgfscope}%
\begin{pgfscope}%
\pgfsys@transformshift{2.230512in}{1.034325in}%
\pgfsys@useobject{currentmarker}{}%
\end{pgfscope}%
\begin{pgfscope}%
\pgfsys@transformshift{2.279949in}{1.031721in}%
\pgfsys@useobject{currentmarker}{}%
\end{pgfscope}%
\begin{pgfscope}%
\pgfsys@transformshift{2.329386in}{1.031580in}%
\pgfsys@useobject{currentmarker}{}%
\end{pgfscope}%
\begin{pgfscope}%
\pgfsys@transformshift{2.378823in}{1.031576in}%
\pgfsys@useobject{currentmarker}{}%
\end{pgfscope}%
\begin{pgfscope}%
\pgfsys@transformshift{2.428260in}{1.024986in}%
\pgfsys@useobject{currentmarker}{}%
\end{pgfscope}%
\begin{pgfscope}%
\pgfsys@transformshift{2.477697in}{1.001236in}%
\pgfsys@useobject{currentmarker}{}%
\end{pgfscope}%
\begin{pgfscope}%
\pgfsys@transformshift{2.527133in}{0.993533in}%
\pgfsys@useobject{currentmarker}{}%
\end{pgfscope}%
\begin{pgfscope}%
\pgfsys@transformshift{2.576570in}{0.991108in}%
\pgfsys@useobject{currentmarker}{}%
\end{pgfscope}%
\begin{pgfscope}%
\pgfsys@transformshift{2.626007in}{0.983325in}%
\pgfsys@useobject{currentmarker}{}%
\end{pgfscope}%
\begin{pgfscope}%
\pgfsys@transformshift{2.675444in}{0.978150in}%
\pgfsys@useobject{currentmarker}{}%
\end{pgfscope}%
\begin{pgfscope}%
\pgfsys@transformshift{2.724881in}{0.973801in}%
\pgfsys@useobject{currentmarker}{}%
\end{pgfscope}%
\begin{pgfscope}%
\pgfsys@transformshift{2.774318in}{0.972499in}%
\pgfsys@useobject{currentmarker}{}%
\end{pgfscope}%
\begin{pgfscope}%
\pgfsys@transformshift{2.823755in}{0.968195in}%
\pgfsys@useobject{currentmarker}{}%
\end{pgfscope}%
\begin{pgfscope}%
\pgfsys@transformshift{2.873191in}{0.957327in}%
\pgfsys@useobject{currentmarker}{}%
\end{pgfscope}%
\begin{pgfscope}%
\pgfsys@transformshift{2.922628in}{0.953215in}%
\pgfsys@useobject{currentmarker}{}%
\end{pgfscope}%
\begin{pgfscope}%
\pgfsys@transformshift{2.972065in}{0.946312in}%
\pgfsys@useobject{currentmarker}{}%
\end{pgfscope}%
\begin{pgfscope}%
\pgfsys@transformshift{3.021502in}{0.944749in}%
\pgfsys@useobject{currentmarker}{}%
\end{pgfscope}%
\begin{pgfscope}%
\pgfsys@transformshift{3.070939in}{0.936007in}%
\pgfsys@useobject{currentmarker}{}%
\end{pgfscope}%
\begin{pgfscope}%
\pgfsys@transformshift{3.120376in}{0.926855in}%
\pgfsys@useobject{currentmarker}{}%
\end{pgfscope}%
\begin{pgfscope}%
\pgfsys@transformshift{3.169812in}{0.926733in}%
\pgfsys@useobject{currentmarker}{}%
\end{pgfscope}%
\begin{pgfscope}%
\pgfsys@transformshift{3.219249in}{0.924949in}%
\pgfsys@useobject{currentmarker}{}%
\end{pgfscope}%
\begin{pgfscope}%
\pgfsys@transformshift{3.268686in}{0.923522in}%
\pgfsys@useobject{currentmarker}{}%
\end{pgfscope}%
\begin{pgfscope}%
\pgfsys@transformshift{3.318123in}{0.921467in}%
\pgfsys@useobject{currentmarker}{}%
\end{pgfscope}%
\begin{pgfscope}%
\pgfsys@transformshift{3.367560in}{0.914114in}%
\pgfsys@useobject{currentmarker}{}%
\end{pgfscope}%
\begin{pgfscope}%
\pgfsys@transformshift{3.416997in}{0.913660in}%
\pgfsys@useobject{currentmarker}{}%
\end{pgfscope}%
\begin{pgfscope}%
\pgfsys@transformshift{3.466434in}{0.902896in}%
\pgfsys@useobject{currentmarker}{}%
\end{pgfscope}%
\begin{pgfscope}%
\pgfsys@transformshift{3.515870in}{0.898457in}%
\pgfsys@useobject{currentmarker}{}%
\end{pgfscope}%
\begin{pgfscope}%
\pgfsys@transformshift{3.565307in}{0.895343in}%
\pgfsys@useobject{currentmarker}{}%
\end{pgfscope}%
\begin{pgfscope}%
\pgfsys@transformshift{3.614744in}{0.888012in}%
\pgfsys@useobject{currentmarker}{}%
\end{pgfscope}%
\begin{pgfscope}%
\pgfsys@transformshift{3.664181in}{0.878920in}%
\pgfsys@useobject{currentmarker}{}%
\end{pgfscope}%
\begin{pgfscope}%
\pgfsys@transformshift{3.713618in}{0.863901in}%
\pgfsys@useobject{currentmarker}{}%
\end{pgfscope}%
\begin{pgfscope}%
\pgfsys@transformshift{3.763055in}{0.858387in}%
\pgfsys@useobject{currentmarker}{}%
\end{pgfscope}%
\begin{pgfscope}%
\pgfsys@transformshift{3.812492in}{0.855790in}%
\pgfsys@useobject{currentmarker}{}%
\end{pgfscope}%
\begin{pgfscope}%
\pgfsys@transformshift{3.861928in}{0.834443in}%
\pgfsys@useobject{currentmarker}{}%
\end{pgfscope}%
\begin{pgfscope}%
\pgfsys@transformshift{3.911365in}{0.825708in}%
\pgfsys@useobject{currentmarker}{}%
\end{pgfscope}%
\begin{pgfscope}%
\pgfsys@transformshift{3.960802in}{0.825595in}%
\pgfsys@useobject{currentmarker}{}%
\end{pgfscope}%
\begin{pgfscope}%
\pgfsys@transformshift{4.010239in}{0.824912in}%
\pgfsys@useobject{currentmarker}{}%
\end{pgfscope}%
\begin{pgfscope}%
\pgfsys@transformshift{4.059676in}{0.798637in}%
\pgfsys@useobject{currentmarker}{}%
\end{pgfscope}%
\begin{pgfscope}%
\pgfsys@transformshift{4.109113in}{0.782192in}%
\pgfsys@useobject{currentmarker}{}%
\end{pgfscope}%
\begin{pgfscope}%
\pgfsys@transformshift{4.158550in}{0.777206in}%
\pgfsys@useobject{currentmarker}{}%
\end{pgfscope}%
\begin{pgfscope}%
\pgfsys@transformshift{4.207986in}{0.737149in}%
\pgfsys@useobject{currentmarker}{}%
\end{pgfscope}%
\begin{pgfscope}%
\pgfsys@transformshift{4.257423in}{0.724026in}%
\pgfsys@useobject{currentmarker}{}%
\end{pgfscope}%
\begin{pgfscope}%
\pgfsys@transformshift{4.306860in}{0.718025in}%
\pgfsys@useobject{currentmarker}{}%
\end{pgfscope}%
\begin{pgfscope}%
\pgfsys@transformshift{4.356297in}{0.696802in}%
\pgfsys@useobject{currentmarker}{}%
\end{pgfscope}%
\begin{pgfscope}%
\pgfsys@transformshift{4.405734in}{0.695623in}%
\pgfsys@useobject{currentmarker}{}%
\end{pgfscope}%
\begin{pgfscope}%
\pgfsys@transformshift{4.455171in}{0.694850in}%
\pgfsys@useobject{currentmarker}{}%
\end{pgfscope}%
\begin{pgfscope}%
\pgfsys@transformshift{4.504607in}{0.684939in}%
\pgfsys@useobject{currentmarker}{}%
\end{pgfscope}%
\begin{pgfscope}%
\pgfsys@transformshift{4.554044in}{0.684327in}%
\pgfsys@useobject{currentmarker}{}%
\end{pgfscope}%
\begin{pgfscope}%
\pgfsys@transformshift{4.603481in}{0.683605in}%
\pgfsys@useobject{currentmarker}{}%
\end{pgfscope}%
\begin{pgfscope}%
\pgfsys@transformshift{4.652918in}{0.683099in}%
\pgfsys@useobject{currentmarker}{}%
\end{pgfscope}%
\begin{pgfscope}%
\pgfsys@transformshift{4.702355in}{0.670942in}%
\pgfsys@useobject{currentmarker}{}%
\end{pgfscope}%
\begin{pgfscope}%
\pgfsys@transformshift{4.751792in}{0.667203in}%
\pgfsys@useobject{currentmarker}{}%
\end{pgfscope}%
\begin{pgfscope}%
\pgfsys@transformshift{4.801229in}{0.660774in}%
\pgfsys@useobject{currentmarker}{}%
\end{pgfscope}%
\begin{pgfscope}%
\pgfsys@transformshift{4.850665in}{0.646217in}%
\pgfsys@useobject{currentmarker}{}%
\end{pgfscope}%
\begin{pgfscope}%
\pgfsys@transformshift{4.900102in}{0.610134in}%
\pgfsys@useobject{currentmarker}{}%
\end{pgfscope}%
\begin{pgfscope}%
\pgfsys@transformshift{4.949539in}{0.588595in}%
\pgfsys@useobject{currentmarker}{}%
\end{pgfscope}%
\begin{pgfscope}%
\pgfsys@transformshift{4.998976in}{0.581502in}%
\pgfsys@useobject{currentmarker}{}%
\end{pgfscope}%
\begin{pgfscope}%
\pgfsys@transformshift{5.048413in}{0.498490in}%
\pgfsys@useobject{currentmarker}{}%
\end{pgfscope}%
\begin{pgfscope}%
\pgfsys@transformshift{5.097850in}{0.460676in}%
\pgfsys@useobject{currentmarker}{}%
\end{pgfscope}%
\end{pgfscope}%
\begin{pgfscope}%
\pgfpathrectangle{\pgfqpoint{0.581793in}{0.460676in}}{\pgfqpoint{4.731107in}{2.134248in}}%
\pgfusepath{clip}%
\pgfsetrectcap%
\pgfsetroundjoin%
\pgfsetlinewidth{1.003750pt}%
\definecolor{currentstroke}{rgb}{0.000000,0.000000,0.000000}%
\pgfsetstrokecolor{currentstroke}%
\pgfsetdash{}{0pt}%
\pgfpathmoveto{\pgfqpoint{0.581793in}{2.364397in}}%
\pgfpathlineto{\pgfqpoint{5.312900in}{2.364397in}}%
\pgfusepath{stroke}%
\end{pgfscope}%
\begin{pgfscope}%
\pgfsetrectcap%
\pgfsetmiterjoin%
\pgfsetlinewidth{0.803000pt}%
\definecolor{currentstroke}{rgb}{0.000000,0.000000,0.000000}%
\pgfsetstrokecolor{currentstroke}%
\pgfsetdash{}{0pt}%
\pgfpathmoveto{\pgfqpoint{0.581793in}{0.460676in}}%
\pgfpathlineto{\pgfqpoint{0.581793in}{2.594924in}}%
\pgfusepath{stroke}%
\end{pgfscope}%
\begin{pgfscope}%
\pgfsetrectcap%
\pgfsetmiterjoin%
\pgfsetlinewidth{0.803000pt}%
\definecolor{currentstroke}{rgb}{0.000000,0.000000,0.000000}%
\pgfsetstrokecolor{currentstroke}%
\pgfsetdash{}{0pt}%
\pgfpathmoveto{\pgfqpoint{5.312900in}{0.460676in}}%
\pgfpathlineto{\pgfqpoint{5.312900in}{2.594924in}}%
\pgfusepath{stroke}%
\end{pgfscope}%
\begin{pgfscope}%
\pgfsetrectcap%
\pgfsetmiterjoin%
\pgfsetlinewidth{0.803000pt}%
\definecolor{currentstroke}{rgb}{0.000000,0.000000,0.000000}%
\pgfsetstrokecolor{currentstroke}%
\pgfsetdash{}{0pt}%
\pgfpathmoveto{\pgfqpoint{0.581793in}{0.460676in}}%
\pgfpathlineto{\pgfqpoint{5.312900in}{0.460676in}}%
\pgfusepath{stroke}%
\end{pgfscope}%
\begin{pgfscope}%
\pgfsetrectcap%
\pgfsetmiterjoin%
\pgfsetlinewidth{0.803000pt}%
\definecolor{currentstroke}{rgb}{0.000000,0.000000,0.000000}%
\pgfsetstrokecolor{currentstroke}%
\pgfsetdash{}{0pt}%
\pgfpathmoveto{\pgfqpoint{0.581793in}{2.594924in}}%
\pgfpathlineto{\pgfqpoint{5.312900in}{2.594924in}}%
\pgfusepath{stroke}%
\end{pgfscope}%
\begin{pgfscope}%
\pgfsetbuttcap%
\pgfsetroundjoin%
\definecolor{currentfill}{rgb}{0.000000,0.000000,0.000000}%
\pgfsetfillcolor{currentfill}%
\pgfsetlinewidth{0.803000pt}%
\definecolor{currentstroke}{rgb}{0.000000,0.000000,0.000000}%
\pgfsetstrokecolor{currentstroke}%
\pgfsetdash{}{0pt}%
\pgfsys@defobject{currentmarker}{\pgfqpoint{0.000000in}{-0.048611in}}{\pgfqpoint{0.000000in}{0.000000in}}{%
\pgfpathmoveto{\pgfqpoint{0.000000in}{0.000000in}}%
\pgfpathlineto{\pgfqpoint{0.000000in}{-0.048611in}}%
\pgfusepath{stroke,fill}%
}%
\begin{pgfscope}%
\pgfsys@transformshift{0.747407in}{0.460676in}%
\pgfsys@useobject{currentmarker}{}%
\end{pgfscope}%
\end{pgfscope}%
\begin{pgfscope}%
\definecolor{textcolor}{rgb}{0.000000,0.000000,0.000000}%
\pgfsetstrokecolor{textcolor}%
\pgfsetfillcolor{textcolor}%
\pgftext[x=0.747407in,y=0.363454in,,top]{\color{textcolor}\rmfamily\fontsize{8.330000}{9.996000}\selectfont \(\displaystyle 0\)}%
\end{pgfscope}%
\begin{pgfscope}%
\pgfsetbuttcap%
\pgfsetroundjoin%
\definecolor{currentfill}{rgb}{0.000000,0.000000,0.000000}%
\pgfsetfillcolor{currentfill}%
\pgfsetlinewidth{0.803000pt}%
\definecolor{currentstroke}{rgb}{0.000000,0.000000,0.000000}%
\pgfsetstrokecolor{currentstroke}%
\pgfsetdash{}{0pt}%
\pgfsys@defobject{currentmarker}{\pgfqpoint{0.000000in}{-0.048611in}}{\pgfqpoint{0.000000in}{0.000000in}}{%
\pgfpathmoveto{\pgfqpoint{0.000000in}{0.000000in}}%
\pgfpathlineto{\pgfqpoint{0.000000in}{-0.048611in}}%
\pgfusepath{stroke,fill}%
}%
\begin{pgfscope}%
\pgfsys@transformshift{1.736144in}{0.460676in}%
\pgfsys@useobject{currentmarker}{}%
\end{pgfscope}%
\end{pgfscope}%
\begin{pgfscope}%
\definecolor{textcolor}{rgb}{0.000000,0.000000,0.000000}%
\pgfsetstrokecolor{textcolor}%
\pgfsetfillcolor{textcolor}%
\pgftext[x=1.736144in,y=0.363454in,,top]{\color{textcolor}\rmfamily\fontsize{8.330000}{9.996000}\selectfont \(\displaystyle 20\)}%
\end{pgfscope}%
\begin{pgfscope}%
\pgfsetbuttcap%
\pgfsetroundjoin%
\definecolor{currentfill}{rgb}{0.000000,0.000000,0.000000}%
\pgfsetfillcolor{currentfill}%
\pgfsetlinewidth{0.803000pt}%
\definecolor{currentstroke}{rgb}{0.000000,0.000000,0.000000}%
\pgfsetstrokecolor{currentstroke}%
\pgfsetdash{}{0pt}%
\pgfsys@defobject{currentmarker}{\pgfqpoint{0.000000in}{-0.048611in}}{\pgfqpoint{0.000000in}{0.000000in}}{%
\pgfpathmoveto{\pgfqpoint{0.000000in}{0.000000in}}%
\pgfpathlineto{\pgfqpoint{0.000000in}{-0.048611in}}%
\pgfusepath{stroke,fill}%
}%
\begin{pgfscope}%
\pgfsys@transformshift{2.724881in}{0.460676in}%
\pgfsys@useobject{currentmarker}{}%
\end{pgfscope}%
\end{pgfscope}%
\begin{pgfscope}%
\definecolor{textcolor}{rgb}{0.000000,0.000000,0.000000}%
\pgfsetstrokecolor{textcolor}%
\pgfsetfillcolor{textcolor}%
\pgftext[x=2.724881in,y=0.363454in,,top]{\color{textcolor}\rmfamily\fontsize{8.330000}{9.996000}\selectfont \(\displaystyle 40\)}%
\end{pgfscope}%
\begin{pgfscope}%
\pgfsetbuttcap%
\pgfsetroundjoin%
\definecolor{currentfill}{rgb}{0.000000,0.000000,0.000000}%
\pgfsetfillcolor{currentfill}%
\pgfsetlinewidth{0.803000pt}%
\definecolor{currentstroke}{rgb}{0.000000,0.000000,0.000000}%
\pgfsetstrokecolor{currentstroke}%
\pgfsetdash{}{0pt}%
\pgfsys@defobject{currentmarker}{\pgfqpoint{0.000000in}{-0.048611in}}{\pgfqpoint{0.000000in}{0.000000in}}{%
\pgfpathmoveto{\pgfqpoint{0.000000in}{0.000000in}}%
\pgfpathlineto{\pgfqpoint{0.000000in}{-0.048611in}}%
\pgfusepath{stroke,fill}%
}%
\begin{pgfscope}%
\pgfsys@transformshift{3.713618in}{0.460676in}%
\pgfsys@useobject{currentmarker}{}%
\end{pgfscope}%
\end{pgfscope}%
\begin{pgfscope}%
\definecolor{textcolor}{rgb}{0.000000,0.000000,0.000000}%
\pgfsetstrokecolor{textcolor}%
\pgfsetfillcolor{textcolor}%
\pgftext[x=3.713618in,y=0.363454in,,top]{\color{textcolor}\rmfamily\fontsize{8.330000}{9.996000}\selectfont \(\displaystyle 60\)}%
\end{pgfscope}%
\begin{pgfscope}%
\pgfsetbuttcap%
\pgfsetroundjoin%
\definecolor{currentfill}{rgb}{0.000000,0.000000,0.000000}%
\pgfsetfillcolor{currentfill}%
\pgfsetlinewidth{0.803000pt}%
\definecolor{currentstroke}{rgb}{0.000000,0.000000,0.000000}%
\pgfsetstrokecolor{currentstroke}%
\pgfsetdash{}{0pt}%
\pgfsys@defobject{currentmarker}{\pgfqpoint{0.000000in}{-0.048611in}}{\pgfqpoint{0.000000in}{0.000000in}}{%
\pgfpathmoveto{\pgfqpoint{0.000000in}{0.000000in}}%
\pgfpathlineto{\pgfqpoint{0.000000in}{-0.048611in}}%
\pgfusepath{stroke,fill}%
}%
\begin{pgfscope}%
\pgfsys@transformshift{4.702355in}{0.460676in}%
\pgfsys@useobject{currentmarker}{}%
\end{pgfscope}%
\end{pgfscope}%
\begin{pgfscope}%
\definecolor{textcolor}{rgb}{0.000000,0.000000,0.000000}%
\pgfsetstrokecolor{textcolor}%
\pgfsetfillcolor{textcolor}%
\pgftext[x=4.702355in,y=0.363454in,,top]{\color{textcolor}\rmfamily\fontsize{8.330000}{9.996000}\selectfont \(\displaystyle 80\)}%
\end{pgfscope}%
\begin{pgfscope}%
\definecolor{textcolor}{rgb}{0.000000,0.000000,0.000000}%
\pgfsetstrokecolor{textcolor}%
\pgfsetfillcolor{textcolor}%
\pgftext[x=2.947347in,y=0.203949in,,top]{\color{textcolor}\rmfamily\fontsize{8.330000}{9.996000}\selectfont Instances}%
\end{pgfscope}%
\begin{pgfscope}%
\pgfsetbuttcap%
\pgfsetroundjoin%
\definecolor{currentfill}{rgb}{0.000000,0.000000,0.000000}%
\pgfsetfillcolor{currentfill}%
\pgfsetlinewidth{0.803000pt}%
\definecolor{currentstroke}{rgb}{0.000000,0.000000,0.000000}%
\pgfsetstrokecolor{currentstroke}%
\pgfsetdash{}{0pt}%
\pgfsys@defobject{currentmarker}{\pgfqpoint{-0.048611in}{0.000000in}}{\pgfqpoint{0.000000in}{0.000000in}}{%
\pgfpathmoveto{\pgfqpoint{0.000000in}{0.000000in}}%
\pgfpathlineto{\pgfqpoint{-0.048611in}{0.000000in}}%
\pgfusepath{stroke,fill}%
}%
\begin{pgfscope}%
\pgfsys@transformshift{0.581793in}{0.460676in}%
\pgfsys@useobject{currentmarker}{}%
\end{pgfscope}%
\end{pgfscope}%
\begin{pgfscope}%
\definecolor{textcolor}{rgb}{0.000000,0.000000,0.000000}%
\pgfsetstrokecolor{textcolor}%
\pgfsetfillcolor{textcolor}%
\pgftext[x=0.274692in, y=0.421225in, left, base]{\color{textcolor}\rmfamily\fontsize{8.330000}{9.996000}\selectfont \(\displaystyle 0.00\)}%
\end{pgfscope}%
\begin{pgfscope}%
\pgfsetbuttcap%
\pgfsetroundjoin%
\definecolor{currentfill}{rgb}{0.000000,0.000000,0.000000}%
\pgfsetfillcolor{currentfill}%
\pgfsetlinewidth{0.803000pt}%
\definecolor{currentstroke}{rgb}{0.000000,0.000000,0.000000}%
\pgfsetstrokecolor{currentstroke}%
\pgfsetdash{}{0pt}%
\pgfsys@defobject{currentmarker}{\pgfqpoint{-0.048611in}{0.000000in}}{\pgfqpoint{0.000000in}{0.000000in}}{%
\pgfpathmoveto{\pgfqpoint{0.000000in}{0.000000in}}%
\pgfpathlineto{\pgfqpoint{-0.048611in}{0.000000in}}%
\pgfusepath{stroke,fill}%
}%
\begin{pgfscope}%
\pgfsys@transformshift{0.581793in}{0.651048in}%
\pgfsys@useobject{currentmarker}{}%
\end{pgfscope}%
\end{pgfscope}%
\begin{pgfscope}%
\definecolor{textcolor}{rgb}{0.000000,0.000000,0.000000}%
\pgfsetstrokecolor{textcolor}%
\pgfsetfillcolor{textcolor}%
\pgftext[x=0.274692in, y=0.611597in, left, base]{\color{textcolor}\rmfamily\fontsize{8.330000}{9.996000}\selectfont \(\displaystyle 0.01\)}%
\end{pgfscope}%
\begin{pgfscope}%
\pgfsetbuttcap%
\pgfsetroundjoin%
\definecolor{currentfill}{rgb}{0.000000,0.000000,0.000000}%
\pgfsetfillcolor{currentfill}%
\pgfsetlinewidth{0.803000pt}%
\definecolor{currentstroke}{rgb}{0.000000,0.000000,0.000000}%
\pgfsetstrokecolor{currentstroke}%
\pgfsetdash{}{0pt}%
\pgfsys@defobject{currentmarker}{\pgfqpoint{-0.048611in}{0.000000in}}{\pgfqpoint{0.000000in}{0.000000in}}{%
\pgfpathmoveto{\pgfqpoint{0.000000in}{0.000000in}}%
\pgfpathlineto{\pgfqpoint{-0.048611in}{0.000000in}}%
\pgfusepath{stroke,fill}%
}%
\begin{pgfscope}%
\pgfsys@transformshift{0.581793in}{0.886361in}%
\pgfsys@useobject{currentmarker}{}%
\end{pgfscope}%
\end{pgfscope}%
\begin{pgfscope}%
\definecolor{textcolor}{rgb}{0.000000,0.000000,0.000000}%
\pgfsetstrokecolor{textcolor}%
\pgfsetfillcolor{textcolor}%
\pgftext[x=0.274692in, y=0.846910in, left, base]{\color{textcolor}\rmfamily\fontsize{8.330000}{9.996000}\selectfont \(\displaystyle 0.05\)}%
\end{pgfscope}%
\begin{pgfscope}%
\pgfsetbuttcap%
\pgfsetroundjoin%
\definecolor{currentfill}{rgb}{0.000000,0.000000,0.000000}%
\pgfsetfillcolor{currentfill}%
\pgfsetlinewidth{0.803000pt}%
\definecolor{currentstroke}{rgb}{0.000000,0.000000,0.000000}%
\pgfsetstrokecolor{currentstroke}%
\pgfsetdash{}{0pt}%
\pgfsys@defobject{currentmarker}{\pgfqpoint{-0.048611in}{0.000000in}}{\pgfqpoint{0.000000in}{0.000000in}}{%
\pgfpathmoveto{\pgfqpoint{0.000000in}{0.000000in}}%
\pgfpathlineto{\pgfqpoint{-0.048611in}{0.000000in}}%
\pgfusepath{stroke,fill}%
}%
\begin{pgfscope}%
\pgfsys@transformshift{0.581793in}{1.062685in}%
\pgfsys@useobject{currentmarker}{}%
\end{pgfscope}%
\end{pgfscope}%
\begin{pgfscope}%
\definecolor{textcolor}{rgb}{0.000000,0.000000,0.000000}%
\pgfsetstrokecolor{textcolor}%
\pgfsetfillcolor{textcolor}%
\pgftext[x=0.274692in, y=1.023234in, left, base]{\color{textcolor}\rmfamily\fontsize{8.330000}{9.996000}\selectfont \(\displaystyle 0.10\)}%
\end{pgfscope}%
\begin{pgfscope}%
\pgfsetbuttcap%
\pgfsetroundjoin%
\definecolor{currentfill}{rgb}{0.000000,0.000000,0.000000}%
\pgfsetfillcolor{currentfill}%
\pgfsetlinewidth{0.803000pt}%
\definecolor{currentstroke}{rgb}{0.000000,0.000000,0.000000}%
\pgfsetstrokecolor{currentstroke}%
\pgfsetdash{}{0pt}%
\pgfsys@defobject{currentmarker}{\pgfqpoint{-0.048611in}{0.000000in}}{\pgfqpoint{0.000000in}{0.000000in}}{%
\pgfpathmoveto{\pgfqpoint{0.000000in}{0.000000in}}%
\pgfpathlineto{\pgfqpoint{-0.048611in}{0.000000in}}%
\pgfusepath{stroke,fill}%
}%
\begin{pgfscope}%
\pgfsys@transformshift{0.581793in}{1.312046in}%
\pgfsys@useobject{currentmarker}{}%
\end{pgfscope}%
\end{pgfscope}%
\begin{pgfscope}%
\definecolor{textcolor}{rgb}{0.000000,0.000000,0.000000}%
\pgfsetstrokecolor{textcolor}%
\pgfsetfillcolor{textcolor}%
\pgftext[x=0.274692in, y=1.272595in, left, base]{\color{textcolor}\rmfamily\fontsize{8.330000}{9.996000}\selectfont \(\displaystyle 0.20\)}%
\end{pgfscope}%
\begin{pgfscope}%
\pgfsetbuttcap%
\pgfsetroundjoin%
\definecolor{currentfill}{rgb}{0.000000,0.000000,0.000000}%
\pgfsetfillcolor{currentfill}%
\pgfsetlinewidth{0.803000pt}%
\definecolor{currentstroke}{rgb}{0.000000,0.000000,0.000000}%
\pgfsetstrokecolor{currentstroke}%
\pgfsetdash{}{0pt}%
\pgfsys@defobject{currentmarker}{\pgfqpoint{-0.048611in}{0.000000in}}{\pgfqpoint{0.000000in}{0.000000in}}{%
\pgfpathmoveto{\pgfqpoint{0.000000in}{0.000000in}}%
\pgfpathlineto{\pgfqpoint{-0.048611in}{0.000000in}}%
\pgfusepath{stroke,fill}%
}%
\begin{pgfscope}%
\pgfsys@transformshift{0.581793in}{1.664695in}%
\pgfsys@useobject{currentmarker}{}%
\end{pgfscope}%
\end{pgfscope}%
\begin{pgfscope}%
\definecolor{textcolor}{rgb}{0.000000,0.000000,0.000000}%
\pgfsetstrokecolor{textcolor}%
\pgfsetfillcolor{textcolor}%
\pgftext[x=0.274692in, y=1.625243in, left, base]{\color{textcolor}\rmfamily\fontsize{8.330000}{9.996000}\selectfont \(\displaystyle 0.40\)}%
\end{pgfscope}%
\begin{pgfscope}%
\pgfsetbuttcap%
\pgfsetroundjoin%
\definecolor{currentfill}{rgb}{0.000000,0.000000,0.000000}%
\pgfsetfillcolor{currentfill}%
\pgfsetlinewidth{0.803000pt}%
\definecolor{currentstroke}{rgb}{0.000000,0.000000,0.000000}%
\pgfsetstrokecolor{currentstroke}%
\pgfsetdash{}{0pt}%
\pgfsys@defobject{currentmarker}{\pgfqpoint{-0.048611in}{0.000000in}}{\pgfqpoint{0.000000in}{0.000000in}}{%
\pgfpathmoveto{\pgfqpoint{0.000000in}{0.000000in}}%
\pgfpathlineto{\pgfqpoint{-0.048611in}{0.000000in}}%
\pgfusepath{stroke,fill}%
}%
\begin{pgfscope}%
\pgfsys@transformshift{0.581793in}{1.935292in}%
\pgfsys@useobject{currentmarker}{}%
\end{pgfscope}%
\end{pgfscope}%
\begin{pgfscope}%
\definecolor{textcolor}{rgb}{0.000000,0.000000,0.000000}%
\pgfsetstrokecolor{textcolor}%
\pgfsetfillcolor{textcolor}%
\pgftext[x=0.274692in, y=1.895840in, left, base]{\color{textcolor}\rmfamily\fontsize{8.330000}{9.996000}\selectfont \(\displaystyle 0.60\)}%
\end{pgfscope}%
\begin{pgfscope}%
\pgfsetbuttcap%
\pgfsetroundjoin%
\definecolor{currentfill}{rgb}{0.000000,0.000000,0.000000}%
\pgfsetfillcolor{currentfill}%
\pgfsetlinewidth{0.803000pt}%
\definecolor{currentstroke}{rgb}{0.000000,0.000000,0.000000}%
\pgfsetstrokecolor{currentstroke}%
\pgfsetdash{}{0pt}%
\pgfsys@defobject{currentmarker}{\pgfqpoint{-0.048611in}{0.000000in}}{\pgfqpoint{0.000000in}{0.000000in}}{%
\pgfpathmoveto{\pgfqpoint{0.000000in}{0.000000in}}%
\pgfpathlineto{\pgfqpoint{-0.048611in}{0.000000in}}%
\pgfusepath{stroke,fill}%
}%
\begin{pgfscope}%
\pgfsys@transformshift{0.581793in}{2.163415in}%
\pgfsys@useobject{currentmarker}{}%
\end{pgfscope}%
\end{pgfscope}%
\begin{pgfscope}%
\definecolor{textcolor}{rgb}{0.000000,0.000000,0.000000}%
\pgfsetstrokecolor{textcolor}%
\pgfsetfillcolor{textcolor}%
\pgftext[x=0.274692in, y=2.123964in, left, base]{\color{textcolor}\rmfamily\fontsize{8.330000}{9.996000}\selectfont \(\displaystyle 0.80\)}%
\end{pgfscope}%
\begin{pgfscope}%
\pgfsetbuttcap%
\pgfsetroundjoin%
\definecolor{currentfill}{rgb}{0.000000,0.000000,0.000000}%
\pgfsetfillcolor{currentfill}%
\pgfsetlinewidth{0.803000pt}%
\definecolor{currentstroke}{rgb}{0.000000,0.000000,0.000000}%
\pgfsetstrokecolor{currentstroke}%
\pgfsetdash{}{0pt}%
\pgfsys@defobject{currentmarker}{\pgfqpoint{-0.048611in}{0.000000in}}{\pgfqpoint{0.000000in}{0.000000in}}{%
\pgfpathmoveto{\pgfqpoint{0.000000in}{0.000000in}}%
\pgfpathlineto{\pgfqpoint{-0.048611in}{0.000000in}}%
\pgfusepath{stroke,fill}%
}%
\begin{pgfscope}%
\pgfsys@transformshift{0.581793in}{2.364397in}%
\pgfsys@useobject{currentmarker}{}%
\end{pgfscope}%
\end{pgfscope}%
\begin{pgfscope}%
\definecolor{textcolor}{rgb}{0.000000,0.000000,0.000000}%
\pgfsetstrokecolor{textcolor}%
\pgfsetfillcolor{textcolor}%
\pgftext[x=0.274692in, y=2.324945in, left, base]{\color{textcolor}\rmfamily\fontsize{8.330000}{9.996000}\selectfont \(\displaystyle 1.00\)}%
\end{pgfscope}%
\begin{pgfscope}%
\definecolor{textcolor}{rgb}{0.000000,0.000000,0.000000}%
\pgfsetstrokecolor{textcolor}%
\pgfsetfillcolor{textcolor}%
\pgftext[x=0.219136in,y=1.527800in,,bottom,rotate=90.000000]{\color{textcolor}\rmfamily\fontsize{8.330000}{9.996000}\selectfont \(\displaystyle 1-(t_{\text{best}}/t_{\text{alg}})\)}%
\end{pgfscope}%
\begin{pgfscope}%
\definecolor{textcolor}{rgb}{0.000000,0.000000,0.000000}%
\pgfsetstrokecolor{textcolor}%
\pgfsetfillcolor{textcolor}%
\pgftext[x=2.947347in,y=2.678257in,,base]{\color{textcolor}\rmfamily\fontsize{8.330000}{9.996000}\selectfont all instances - runtime}%
\end{pgfscope}%
\begin{pgfscope}%
\pgfsetbuttcap%
\pgfsetroundjoin%
\definecolor{currentfill}{rgb}{0.121569,0.466667,0.705882}%
\pgfsetfillcolor{currentfill}%
\pgfsetlinewidth{1.003750pt}%
\definecolor{currentstroke}{rgb}{0.121569,0.466667,0.705882}%
\pgfsetstrokecolor{currentstroke}%
\pgfsetdash{}{0pt}%
\pgfsys@defobject{currentmarker}{\pgfqpoint{-0.028527in}{-0.028527in}}{\pgfqpoint{0.028527in}{0.028527in}}{%
\pgfpathmoveto{\pgfqpoint{0.000000in}{-0.028527in}}%
\pgfpathcurveto{\pgfqpoint{0.007566in}{-0.028527in}}{\pgfqpoint{0.014822in}{-0.025521in}}{\pgfqpoint{0.020172in}{-0.020172in}}%
\pgfpathcurveto{\pgfqpoint{0.025521in}{-0.014822in}}{\pgfqpoint{0.028527in}{-0.007566in}}{\pgfqpoint{0.028527in}{0.000000in}}%
\pgfpathcurveto{\pgfqpoint{0.028527in}{0.007566in}}{\pgfqpoint{0.025521in}{0.014822in}}{\pgfqpoint{0.020172in}{0.020172in}}%
\pgfpathcurveto{\pgfqpoint{0.014822in}{0.025521in}}{\pgfqpoint{0.007566in}{0.028527in}}{\pgfqpoint{0.000000in}{0.028527in}}%
\pgfpathcurveto{\pgfqpoint{-0.007566in}{0.028527in}}{\pgfqpoint{-0.014822in}{0.025521in}}{\pgfqpoint{-0.020172in}{0.020172in}}%
\pgfpathcurveto{\pgfqpoint{-0.025521in}{0.014822in}}{\pgfqpoint{-0.028527in}{0.007566in}}{\pgfqpoint{-0.028527in}{0.000000in}}%
\pgfpathcurveto{\pgfqpoint{-0.028527in}{-0.007566in}}{\pgfqpoint{-0.025521in}{-0.014822in}}{\pgfqpoint{-0.020172in}{-0.020172in}}%
\pgfpathcurveto{\pgfqpoint{-0.014822in}{-0.025521in}}{\pgfqpoint{-0.007566in}{-0.028527in}}{\pgfqpoint{0.000000in}{-0.028527in}}%
\pgfpathclose%
\pgfusepath{stroke,fill}%
}%
\begin{pgfscope}%
\pgfsys@transformshift{3.595646in}{2.284915in}%
\pgfsys@useobject{currentmarker}{}%
\end{pgfscope}%
\end{pgfscope}%
\begin{pgfscope}%
\definecolor{textcolor}{rgb}{0.000000,0.000000,0.000000}%
\pgfsetstrokecolor{textcolor}%
\pgfsetfillcolor{textcolor}%
\pgftext[x=3.722910in,y=2.254546in,left,base]{\color{textcolor}\rmfamily\fontsize{8.330000}{9.996000}\selectfont maximum degree (default)}%
\end{pgfscope}%
\begin{pgfscope}%
\pgfsetbuttcap%
\pgfsetroundjoin%
\definecolor{currentfill}{rgb}{1.000000,0.498039,0.054902}%
\pgfsetfillcolor{currentfill}%
\pgfsetfillopacity{0.750000}%
\pgfsetlinewidth{1.003750pt}%
\definecolor{currentstroke}{rgb}{1.000000,0.498039,0.054902}%
\pgfsetstrokecolor{currentstroke}%
\pgfsetstrokeopacity{0.750000}%
\pgfsetdash{}{0pt}%
\pgfsys@defobject{currentmarker}{\pgfqpoint{-0.028527in}{-0.028527in}}{\pgfqpoint{0.028527in}{0.028527in}}{%
\pgfpathmoveto{\pgfqpoint{-0.000000in}{-0.028527in}}%
\pgfpathlineto{\pgfqpoint{0.028527in}{0.028527in}}%
\pgfpathlineto{\pgfqpoint{-0.028527in}{0.028527in}}%
\pgfpathclose%
\pgfusepath{stroke,fill}%
}%
\begin{pgfscope}%
\pgfsys@transformshift{3.595646in}{2.155743in}%
\pgfsys@useobject{currentmarker}{}%
\end{pgfscope}%
\end{pgfscope}%
\begin{pgfscope}%
\definecolor{textcolor}{rgb}{0.000000,0.000000,0.000000}%
\pgfsetstrokecolor{textcolor}%
\pgfsetfillcolor{textcolor}%
\pgftext[x=3.722910in,y=2.125374in,left,base]{\color{textcolor}\rmfamily\fontsize{8.330000}{9.996000}\selectfont combined}%
\end{pgfscope}%
\begin{pgfscope}%
\pgfsetbuttcap%
\pgfsetroundjoin%
\definecolor{currentfill}{rgb}{0.172549,0.627451,0.172549}%
\pgfsetfillcolor{currentfill}%
\pgfsetfillopacity{0.750000}%
\pgfsetlinewidth{1.003750pt}%
\definecolor{currentstroke}{rgb}{0.172549,0.627451,0.172549}%
\pgfsetstrokecolor{currentstroke}%
\pgfsetstrokeopacity{0.750000}%
\pgfsetdash{}{0pt}%
\pgfsys@defobject{currentmarker}{\pgfqpoint{-0.028527in}{-0.028527in}}{\pgfqpoint{0.028527in}{0.028527in}}{%
\pgfpathmoveto{\pgfqpoint{-0.028527in}{-0.028527in}}%
\pgfpathlineto{\pgfqpoint{0.028527in}{-0.028527in}}%
\pgfpathlineto{\pgfqpoint{0.028527in}{0.028527in}}%
\pgfpathlineto{\pgfqpoint{-0.028527in}{0.028527in}}%
\pgfpathclose%
\pgfusepath{stroke,fill}%
}%
\begin{pgfscope}%
\pgfsys@transformshift{3.595646in}{2.026571in}%
\pgfsys@useobject{currentmarker}{}%
\end{pgfscope}%
\end{pgfscope}%
\begin{pgfscope}%
\definecolor{textcolor}{rgb}{0.000000,0.000000,0.000000}%
\pgfsetstrokecolor{textcolor}%
\pgfsetfillcolor{textcolor}%
\pgftext[x=3.722910in,y=1.996202in,left,base]{\color{textcolor}\rmfamily\fontsize{8.330000}{9.996000}\selectfont twin reduction based}%
\end{pgfscope}%
\begin{pgfscope}%
\pgfsetbuttcap%
\pgfsetroundjoin%
\definecolor{currentfill}{rgb}{0.839216,0.152941,0.156863}%
\pgfsetfillcolor{currentfill}%
\pgfsetfillopacity{0.750000}%
\pgfsetlinewidth{1.003750pt}%
\definecolor{currentstroke}{rgb}{0.839216,0.152941,0.156863}%
\pgfsetstrokecolor{currentstroke}%
\pgfsetstrokeopacity{0.750000}%
\pgfsetdash{}{0pt}%
\pgfsys@defobject{currentmarker}{\pgfqpoint{-0.028527in}{-0.028527in}}{\pgfqpoint{0.028527in}{0.028527in}}{%
\pgfpathmoveto{\pgfqpoint{0.000000in}{0.028527in}}%
\pgfpathlineto{\pgfqpoint{-0.028527in}{-0.028527in}}%
\pgfpathlineto{\pgfqpoint{0.028527in}{-0.028527in}}%
\pgfpathclose%
\pgfusepath{stroke,fill}%
}%
\begin{pgfscope}%
\pgfsys@transformshift{3.595646in}{1.897399in}%
\pgfsys@useobject{currentmarker}{}%
\end{pgfscope}%
\end{pgfscope}%
\begin{pgfscope}%
\definecolor{textcolor}{rgb}{0.000000,0.000000,0.000000}%
\pgfsetstrokecolor{textcolor}%
\pgfsetfillcolor{textcolor}%
\pgftext[x=3.722910in,y=1.867030in,left,base]{\color{textcolor}\rmfamily\fontsize{8.330000}{9.996000}\selectfont funnel reduction based}%
\end{pgfscope}%
\begin{pgfscope}%
\pgfsetbuttcap%
\pgfsetroundjoin%
\definecolor{currentfill}{rgb}{0.580392,0.403922,0.741176}%
\pgfsetfillcolor{currentfill}%
\pgfsetfillopacity{0.750000}%
\pgfsetlinewidth{1.003750pt}%
\definecolor{currentstroke}{rgb}{0.580392,0.403922,0.741176}%
\pgfsetstrokecolor{currentstroke}%
\pgfsetstrokeopacity{0.750000}%
\pgfsetdash{}{0pt}%
\pgfsys@defobject{currentmarker}{\pgfqpoint{-0.040344in}{-0.040344in}}{\pgfqpoint{0.040344in}{0.040344in}}{%
\pgfpathmoveto{\pgfqpoint{-0.000000in}{-0.040344in}}%
\pgfpathlineto{\pgfqpoint{0.040344in}{0.000000in}}%
\pgfpathlineto{\pgfqpoint{0.000000in}{0.040344in}}%
\pgfpathlineto{\pgfqpoint{-0.040344in}{0.000000in}}%
\pgfpathclose%
\pgfusepath{stroke,fill}%
}%
\begin{pgfscope}%
\pgfsys@transformshift{3.595646in}{1.768228in}%
\pgfsys@useobject{currentmarker}{}%
\end{pgfscope}%
\end{pgfscope}%
\begin{pgfscope}%
\definecolor{textcolor}{rgb}{0.000000,0.000000,0.000000}%
\pgfsetstrokecolor{textcolor}%
\pgfsetfillcolor{textcolor}%
\pgftext[x=3.722910in,y=1.737858in,left,base]{\color{textcolor}\rmfamily\fontsize{8.330000}{9.996000}\selectfont unconfined reduction based}%
\end{pgfscope}%
\end{pgfpicture}%
\makeatother%
\endgroup%

	\caption{Performance plot comparing running time of reduction-based branching strategies.}\label{fig:all_reduction_time}
\end{figure}

\begin{figure}[tb!]
	\centering
	%% Creator: Matplotlib, PGF backend
%%
%% To include the figure in your LaTeX document, write
%%   \input{<filename>.pgf}
%%
%% Make sure the required packages are loaded in your preamble
%%   \usepackage{pgf}
%%
%% and, on pdftex
%%   \usepackage[utf8]{inputenc}\DeclareUnicodeCharacter{2212}{-}
%%
%% or, on luatex and xetex
%%   \usepackage{unicode-math}
%%
%% Figures using additional raster images can only be included by \input if
%% they are in the same directory as the main LaTeX file. For loading figures
%% from other directories you can use the `import` package
%%   \usepackage{import}
%%
%% and then include the figures with
%%   \import{<path to file>}{<filename>.pgf}
%%
%% Matplotlib used the following preamble
%%   \renewcommand{\sfdefault}{phv}
%%   \renewcommand{\rmdefault}{ptm}
%%   \renewcommand{\ttdefault}{pcr}
%%   \normalfont\selectfont
%%
\begingroup%
\makeatletter%
\begin{pgfpicture}%
\pgfpathrectangle{\pgfpointorigin}{\pgfqpoint{5.412900in}{2.857160in}}%
\pgfusepath{use as bounding box, clip}%
\begin{pgfscope}%
\pgfsetbuttcap%
\pgfsetmiterjoin%
\definecolor{currentfill}{rgb}{1.000000,1.000000,1.000000}%
\pgfsetfillcolor{currentfill}%
\pgfsetlinewidth{0.000000pt}%
\definecolor{currentstroke}{rgb}{1.000000,1.000000,1.000000}%
\pgfsetstrokecolor{currentstroke}%
\pgfsetdash{}{0pt}%
\pgfpathmoveto{\pgfqpoint{0.000000in}{0.000000in}}%
\pgfpathlineto{\pgfqpoint{5.412900in}{0.000000in}}%
\pgfpathlineto{\pgfqpoint{5.412900in}{2.857160in}}%
\pgfpathlineto{\pgfqpoint{0.000000in}{2.857160in}}%
\pgfpathclose%
\pgfusepath{fill}%
\end{pgfscope}%
\begin{pgfscope}%
\pgfsetbuttcap%
\pgfsetmiterjoin%
\definecolor{currentfill}{rgb}{1.000000,1.000000,1.000000}%
\pgfsetfillcolor{currentfill}%
\pgfsetlinewidth{0.000000pt}%
\definecolor{currentstroke}{rgb}{0.000000,0.000000,0.000000}%
\pgfsetstrokecolor{currentstroke}%
\pgfsetstrokeopacity{0.000000}%
\pgfsetdash{}{0pt}%
\pgfpathmoveto{\pgfqpoint{0.581793in}{0.460676in}}%
\pgfpathlineto{\pgfqpoint{5.312900in}{0.460676in}}%
\pgfpathlineto{\pgfqpoint{5.312900in}{2.594924in}}%
\pgfpathlineto{\pgfqpoint{0.581793in}{2.594924in}}%
\pgfpathclose%
\pgfusepath{fill}%
\end{pgfscope}%
\begin{pgfscope}%
\pgfpathrectangle{\pgfqpoint{0.581793in}{0.460676in}}{\pgfqpoint{4.731107in}{2.134248in}}%
\pgfusepath{clip}%
\pgfsetbuttcap%
\pgfsetroundjoin%
\definecolor{currentfill}{rgb}{0.121569,0.466667,0.705882}%
\pgfsetfillcolor{currentfill}%
\pgfsetlinewidth{1.003750pt}%
\definecolor{currentstroke}{rgb}{0.121569,0.466667,0.705882}%
\pgfsetstrokecolor{currentstroke}%
\pgfsetdash{}{0pt}%
\pgfsys@defobject{currentmarker}{\pgfqpoint{-0.019018in}{-0.019018in}}{\pgfqpoint{0.019018in}{0.019018in}}{%
\pgfpathmoveto{\pgfqpoint{0.000000in}{-0.019018in}}%
\pgfpathcurveto{\pgfqpoint{0.005044in}{-0.019018in}}{\pgfqpoint{0.009881in}{-0.017014in}}{\pgfqpoint{0.013448in}{-0.013448in}}%
\pgfpathcurveto{\pgfqpoint{0.017014in}{-0.009881in}}{\pgfqpoint{0.019018in}{-0.005044in}}{\pgfqpoint{0.019018in}{0.000000in}}%
\pgfpathcurveto{\pgfqpoint{0.019018in}{0.005044in}}{\pgfqpoint{0.017014in}{0.009881in}}{\pgfqpoint{0.013448in}{0.013448in}}%
\pgfpathcurveto{\pgfqpoint{0.009881in}{0.017014in}}{\pgfqpoint{0.005044in}{0.019018in}}{\pgfqpoint{0.000000in}{0.019018in}}%
\pgfpathcurveto{\pgfqpoint{-0.005044in}{0.019018in}}{\pgfqpoint{-0.009881in}{0.017014in}}{\pgfqpoint{-0.013448in}{0.013448in}}%
\pgfpathcurveto{\pgfqpoint{-0.017014in}{0.009881in}}{\pgfqpoint{-0.019018in}{0.005044in}}{\pgfqpoint{-0.019018in}{0.000000in}}%
\pgfpathcurveto{\pgfqpoint{-0.019018in}{-0.005044in}}{\pgfqpoint{-0.017014in}{-0.009881in}}{\pgfqpoint{-0.013448in}{-0.013448in}}%
\pgfpathcurveto{\pgfqpoint{-0.009881in}{-0.017014in}}{\pgfqpoint{-0.005044in}{-0.019018in}}{\pgfqpoint{0.000000in}{-0.019018in}}%
\pgfpathclose%
\pgfusepath{stroke,fill}%
}%
\begin{pgfscope}%
\pgfsys@transformshift{0.796844in}{2.457315in}%
\pgfsys@useobject{currentmarker}{}%
\end{pgfscope}%
\begin{pgfscope}%
\pgfsys@transformshift{0.846280in}{2.001832in}%
\pgfsys@useobject{currentmarker}{}%
\end{pgfscope}%
\begin{pgfscope}%
\pgfsys@transformshift{0.895717in}{1.794814in}%
\pgfsys@useobject{currentmarker}{}%
\end{pgfscope}%
\begin{pgfscope}%
\pgfsys@transformshift{0.945154in}{1.551935in}%
\pgfsys@useobject{currentmarker}{}%
\end{pgfscope}%
\begin{pgfscope}%
\pgfsys@transformshift{0.994591in}{1.366795in}%
\pgfsys@useobject{currentmarker}{}%
\end{pgfscope}%
\begin{pgfscope}%
\pgfsys@transformshift{1.044028in}{1.302934in}%
\pgfsys@useobject{currentmarker}{}%
\end{pgfscope}%
\begin{pgfscope}%
\pgfsys@transformshift{1.093465in}{1.292789in}%
\pgfsys@useobject{currentmarker}{}%
\end{pgfscope}%
\begin{pgfscope}%
\pgfsys@transformshift{1.142902in}{1.281692in}%
\pgfsys@useobject{currentmarker}{}%
\end{pgfscope}%
\begin{pgfscope}%
\pgfsys@transformshift{1.192338in}{1.259944in}%
\pgfsys@useobject{currentmarker}{}%
\end{pgfscope}%
\begin{pgfscope}%
\pgfsys@transformshift{1.241775in}{1.243107in}%
\pgfsys@useobject{currentmarker}{}%
\end{pgfscope}%
\begin{pgfscope}%
\pgfsys@transformshift{1.291212in}{1.233168in}%
\pgfsys@useobject{currentmarker}{}%
\end{pgfscope}%
\begin{pgfscope}%
\pgfsys@transformshift{1.340649in}{1.210334in}%
\pgfsys@useobject{currentmarker}{}%
\end{pgfscope}%
\begin{pgfscope}%
\pgfsys@transformshift{1.390086in}{1.208482in}%
\pgfsys@useobject{currentmarker}{}%
\end{pgfscope}%
\begin{pgfscope}%
\pgfsys@transformshift{1.439523in}{1.207972in}%
\pgfsys@useobject{currentmarker}{}%
\end{pgfscope}%
\begin{pgfscope}%
\pgfsys@transformshift{1.488960in}{1.194499in}%
\pgfsys@useobject{currentmarker}{}%
\end{pgfscope}%
\begin{pgfscope}%
\pgfsys@transformshift{1.538396in}{1.181380in}%
\pgfsys@useobject{currentmarker}{}%
\end{pgfscope}%
\begin{pgfscope}%
\pgfsys@transformshift{1.587833in}{1.175323in}%
\pgfsys@useobject{currentmarker}{}%
\end{pgfscope}%
\begin{pgfscope}%
\pgfsys@transformshift{1.637270in}{1.166327in}%
\pgfsys@useobject{currentmarker}{}%
\end{pgfscope}%
\begin{pgfscope}%
\pgfsys@transformshift{1.686707in}{1.157029in}%
\pgfsys@useobject{currentmarker}{}%
\end{pgfscope}%
\begin{pgfscope}%
\pgfsys@transformshift{1.736144in}{1.152530in}%
\pgfsys@useobject{currentmarker}{}%
\end{pgfscope}%
\begin{pgfscope}%
\pgfsys@transformshift{1.785581in}{1.147808in}%
\pgfsys@useobject{currentmarker}{}%
\end{pgfscope}%
\begin{pgfscope}%
\pgfsys@transformshift{1.835018in}{1.147295in}%
\pgfsys@useobject{currentmarker}{}%
\end{pgfscope}%
\begin{pgfscope}%
\pgfsys@transformshift{1.884454in}{1.147192in}%
\pgfsys@useobject{currentmarker}{}%
\end{pgfscope}%
\begin{pgfscope}%
\pgfsys@transformshift{1.933891in}{1.129899in}%
\pgfsys@useobject{currentmarker}{}%
\end{pgfscope}%
\begin{pgfscope}%
\pgfsys@transformshift{1.983328in}{1.129404in}%
\pgfsys@useobject{currentmarker}{}%
\end{pgfscope}%
\begin{pgfscope}%
\pgfsys@transformshift{2.032765in}{1.118420in}%
\pgfsys@useobject{currentmarker}{}%
\end{pgfscope}%
\begin{pgfscope}%
\pgfsys@transformshift{2.082202in}{1.117713in}%
\pgfsys@useobject{currentmarker}{}%
\end{pgfscope}%
\begin{pgfscope}%
\pgfsys@transformshift{2.131639in}{1.114350in}%
\pgfsys@useobject{currentmarker}{}%
\end{pgfscope}%
\begin{pgfscope}%
\pgfsys@transformshift{2.181075in}{1.112259in}%
\pgfsys@useobject{currentmarker}{}%
\end{pgfscope}%
\begin{pgfscope}%
\pgfsys@transformshift{2.230512in}{1.102219in}%
\pgfsys@useobject{currentmarker}{}%
\end{pgfscope}%
\begin{pgfscope}%
\pgfsys@transformshift{2.279949in}{1.101388in}%
\pgfsys@useobject{currentmarker}{}%
\end{pgfscope}%
\begin{pgfscope}%
\pgfsys@transformshift{2.329386in}{1.098747in}%
\pgfsys@useobject{currentmarker}{}%
\end{pgfscope}%
\begin{pgfscope}%
\pgfsys@transformshift{2.378823in}{1.093031in}%
\pgfsys@useobject{currentmarker}{}%
\end{pgfscope}%
\begin{pgfscope}%
\pgfsys@transformshift{2.428260in}{1.091813in}%
\pgfsys@useobject{currentmarker}{}%
\end{pgfscope}%
\begin{pgfscope}%
\pgfsys@transformshift{2.477697in}{1.091813in}%
\pgfsys@useobject{currentmarker}{}%
\end{pgfscope}%
\begin{pgfscope}%
\pgfsys@transformshift{2.527133in}{1.091406in}%
\pgfsys@useobject{currentmarker}{}%
\end{pgfscope}%
\begin{pgfscope}%
\pgfsys@transformshift{2.576570in}{1.086909in}%
\pgfsys@useobject{currentmarker}{}%
\end{pgfscope}%
\begin{pgfscope}%
\pgfsys@transformshift{2.626007in}{1.083325in}%
\pgfsys@useobject{currentmarker}{}%
\end{pgfscope}%
\begin{pgfscope}%
\pgfsys@transformshift{2.675444in}{1.077301in}%
\pgfsys@useobject{currentmarker}{}%
\end{pgfscope}%
\begin{pgfscope}%
\pgfsys@transformshift{2.724881in}{1.070907in}%
\pgfsys@useobject{currentmarker}{}%
\end{pgfscope}%
\begin{pgfscope}%
\pgfsys@transformshift{2.774318in}{1.067075in}%
\pgfsys@useobject{currentmarker}{}%
\end{pgfscope}%
\begin{pgfscope}%
\pgfsys@transformshift{2.823755in}{1.063510in}%
\pgfsys@useobject{currentmarker}{}%
\end{pgfscope}%
\begin{pgfscope}%
\pgfsys@transformshift{2.873191in}{1.061350in}%
\pgfsys@useobject{currentmarker}{}%
\end{pgfscope}%
\begin{pgfscope}%
\pgfsys@transformshift{2.922628in}{1.022061in}%
\pgfsys@useobject{currentmarker}{}%
\end{pgfscope}%
\begin{pgfscope}%
\pgfsys@transformshift{2.972065in}{1.011118in}%
\pgfsys@useobject{currentmarker}{}%
\end{pgfscope}%
\begin{pgfscope}%
\pgfsys@transformshift{3.021502in}{1.010263in}%
\pgfsys@useobject{currentmarker}{}%
\end{pgfscope}%
\begin{pgfscope}%
\pgfsys@transformshift{3.070939in}{0.992219in}%
\pgfsys@useobject{currentmarker}{}%
\end{pgfscope}%
\begin{pgfscope}%
\pgfsys@transformshift{3.120376in}{0.988210in}%
\pgfsys@useobject{currentmarker}{}%
\end{pgfscope}%
\begin{pgfscope}%
\pgfsys@transformshift{3.169812in}{0.977184in}%
\pgfsys@useobject{currentmarker}{}%
\end{pgfscope}%
\begin{pgfscope}%
\pgfsys@transformshift{3.219249in}{0.955524in}%
\pgfsys@useobject{currentmarker}{}%
\end{pgfscope}%
\begin{pgfscope}%
\pgfsys@transformshift{3.268686in}{0.949148in}%
\pgfsys@useobject{currentmarker}{}%
\end{pgfscope}%
\begin{pgfscope}%
\pgfsys@transformshift{3.318123in}{0.948588in}%
\pgfsys@useobject{currentmarker}{}%
\end{pgfscope}%
\begin{pgfscope}%
\pgfsys@transformshift{3.367560in}{0.918988in}%
\pgfsys@useobject{currentmarker}{}%
\end{pgfscope}%
\begin{pgfscope}%
\pgfsys@transformshift{3.416997in}{0.858814in}%
\pgfsys@useobject{currentmarker}{}%
\end{pgfscope}%
\begin{pgfscope}%
\pgfsys@transformshift{3.466434in}{0.855193in}%
\pgfsys@useobject{currentmarker}{}%
\end{pgfscope}%
\begin{pgfscope}%
\pgfsys@transformshift{3.515870in}{0.847777in}%
\pgfsys@useobject{currentmarker}{}%
\end{pgfscope}%
\begin{pgfscope}%
\pgfsys@transformshift{3.565307in}{0.829544in}%
\pgfsys@useobject{currentmarker}{}%
\end{pgfscope}%
\begin{pgfscope}%
\pgfsys@transformshift{3.614744in}{0.824203in}%
\pgfsys@useobject{currentmarker}{}%
\end{pgfscope}%
\begin{pgfscope}%
\pgfsys@transformshift{3.664181in}{0.820704in}%
\pgfsys@useobject{currentmarker}{}%
\end{pgfscope}%
\begin{pgfscope}%
\pgfsys@transformshift{3.713618in}{0.811450in}%
\pgfsys@useobject{currentmarker}{}%
\end{pgfscope}%
\begin{pgfscope}%
\pgfsys@transformshift{3.763055in}{0.800674in}%
\pgfsys@useobject{currentmarker}{}%
\end{pgfscope}%
\begin{pgfscope}%
\pgfsys@transformshift{3.812492in}{0.784601in}%
\pgfsys@useobject{currentmarker}{}%
\end{pgfscope}%
\begin{pgfscope}%
\pgfsys@transformshift{3.861928in}{0.778369in}%
\pgfsys@useobject{currentmarker}{}%
\end{pgfscope}%
\begin{pgfscope}%
\pgfsys@transformshift{3.911365in}{0.774651in}%
\pgfsys@useobject{currentmarker}{}%
\end{pgfscope}%
\begin{pgfscope}%
\pgfsys@transformshift{3.960802in}{0.770663in}%
\pgfsys@useobject{currentmarker}{}%
\end{pgfscope}%
\begin{pgfscope}%
\pgfsys@transformshift{4.010239in}{0.769500in}%
\pgfsys@useobject{currentmarker}{}%
\end{pgfscope}%
\begin{pgfscope}%
\pgfsys@transformshift{4.059676in}{0.737879in}%
\pgfsys@useobject{currentmarker}{}%
\end{pgfscope}%
\begin{pgfscope}%
\pgfsys@transformshift{4.109113in}{0.703053in}%
\pgfsys@useobject{currentmarker}{}%
\end{pgfscope}%
\begin{pgfscope}%
\pgfsys@transformshift{4.158550in}{0.698963in}%
\pgfsys@useobject{currentmarker}{}%
\end{pgfscope}%
\begin{pgfscope}%
\pgfsys@transformshift{4.207986in}{0.685894in}%
\pgfsys@useobject{currentmarker}{}%
\end{pgfscope}%
\begin{pgfscope}%
\pgfsys@transformshift{4.257423in}{0.673197in}%
\pgfsys@useobject{currentmarker}{}%
\end{pgfscope}%
\begin{pgfscope}%
\pgfsys@transformshift{4.306860in}{0.631306in}%
\pgfsys@useobject{currentmarker}{}%
\end{pgfscope}%
\begin{pgfscope}%
\pgfsys@transformshift{4.356297in}{0.600368in}%
\pgfsys@useobject{currentmarker}{}%
\end{pgfscope}%
\begin{pgfscope}%
\pgfsys@transformshift{4.405734in}{0.580113in}%
\pgfsys@useobject{currentmarker}{}%
\end{pgfscope}%
\begin{pgfscope}%
\pgfsys@transformshift{4.455171in}{0.556803in}%
\pgfsys@useobject{currentmarker}{}%
\end{pgfscope}%
\begin{pgfscope}%
\pgfsys@transformshift{4.504607in}{0.526955in}%
\pgfsys@useobject{currentmarker}{}%
\end{pgfscope}%
\begin{pgfscope}%
\pgfsys@transformshift{4.554044in}{0.460676in}%
\pgfsys@useobject{currentmarker}{}%
\end{pgfscope}%
\begin{pgfscope}%
\pgfsys@transformshift{4.603481in}{0.460676in}%
\pgfsys@useobject{currentmarker}{}%
\end{pgfscope}%
\begin{pgfscope}%
\pgfsys@transformshift{4.652918in}{0.460676in}%
\pgfsys@useobject{currentmarker}{}%
\end{pgfscope}%
\begin{pgfscope}%
\pgfsys@transformshift{4.702355in}{0.460676in}%
\pgfsys@useobject{currentmarker}{}%
\end{pgfscope}%
\begin{pgfscope}%
\pgfsys@transformshift{4.751792in}{0.460676in}%
\pgfsys@useobject{currentmarker}{}%
\end{pgfscope}%
\begin{pgfscope}%
\pgfsys@transformshift{4.801229in}{0.460676in}%
\pgfsys@useobject{currentmarker}{}%
\end{pgfscope}%
\begin{pgfscope}%
\pgfsys@transformshift{4.850665in}{0.460676in}%
\pgfsys@useobject{currentmarker}{}%
\end{pgfscope}%
\begin{pgfscope}%
\pgfsys@transformshift{4.900102in}{0.460676in}%
\pgfsys@useobject{currentmarker}{}%
\end{pgfscope}%
\begin{pgfscope}%
\pgfsys@transformshift{4.949539in}{0.460676in}%
\pgfsys@useobject{currentmarker}{}%
\end{pgfscope}%
\begin{pgfscope}%
\pgfsys@transformshift{4.998976in}{0.460676in}%
\pgfsys@useobject{currentmarker}{}%
\end{pgfscope}%
\begin{pgfscope}%
\pgfsys@transformshift{5.048413in}{0.460676in}%
\pgfsys@useobject{currentmarker}{}%
\end{pgfscope}%
\begin{pgfscope}%
\pgfsys@transformshift{5.097850in}{0.460676in}%
\pgfsys@useobject{currentmarker}{}%
\end{pgfscope}%
\end{pgfscope}%
\begin{pgfscope}%
\pgfpathrectangle{\pgfqpoint{0.581793in}{0.460676in}}{\pgfqpoint{4.731107in}{2.134248in}}%
\pgfusepath{clip}%
\pgfsetbuttcap%
\pgfsetroundjoin%
\definecolor{currentfill}{rgb}{1.000000,0.498039,0.054902}%
\pgfsetfillcolor{currentfill}%
\pgfsetfillopacity{0.750000}%
\pgfsetlinewidth{1.003750pt}%
\definecolor{currentstroke}{rgb}{1.000000,0.498039,0.054902}%
\pgfsetstrokecolor{currentstroke}%
\pgfsetstrokeopacity{0.750000}%
\pgfsetdash{}{0pt}%
\pgfsys@defobject{currentmarker}{\pgfqpoint{-0.019018in}{-0.019018in}}{\pgfqpoint{0.019018in}{0.019018in}}{%
\pgfpathmoveto{\pgfqpoint{-0.000000in}{-0.019018in}}%
\pgfpathlineto{\pgfqpoint{0.019018in}{0.019018in}}%
\pgfpathlineto{\pgfqpoint{-0.019018in}{0.019018in}}%
\pgfpathclose%
\pgfusepath{stroke,fill}%
}%
\begin{pgfscope}%
\pgfsys@transformshift{0.796844in}{2.466370in}%
\pgfsys@useobject{currentmarker}{}%
\end{pgfscope}%
\begin{pgfscope}%
\pgfsys@transformshift{0.846280in}{2.466370in}%
\pgfsys@useobject{currentmarker}{}%
\end{pgfscope}%
\begin{pgfscope}%
\pgfsys@transformshift{0.895717in}{2.228128in}%
\pgfsys@useobject{currentmarker}{}%
\end{pgfscope}%
\begin{pgfscope}%
\pgfsys@transformshift{0.945154in}{2.168127in}%
\pgfsys@useobject{currentmarker}{}%
\end{pgfscope}%
\begin{pgfscope}%
\pgfsys@transformshift{0.994591in}{1.957501in}%
\pgfsys@useobject{currentmarker}{}%
\end{pgfscope}%
\begin{pgfscope}%
\pgfsys@transformshift{1.044028in}{1.700288in}%
\pgfsys@useobject{currentmarker}{}%
\end{pgfscope}%
\begin{pgfscope}%
\pgfsys@transformshift{1.093465in}{1.559789in}%
\pgfsys@useobject{currentmarker}{}%
\end{pgfscope}%
\begin{pgfscope}%
\pgfsys@transformshift{1.142902in}{1.549804in}%
\pgfsys@useobject{currentmarker}{}%
\end{pgfscope}%
\begin{pgfscope}%
\pgfsys@transformshift{1.192338in}{1.473385in}%
\pgfsys@useobject{currentmarker}{}%
\end{pgfscope}%
\begin{pgfscope}%
\pgfsys@transformshift{1.241775in}{1.412536in}%
\pgfsys@useobject{currentmarker}{}%
\end{pgfscope}%
\begin{pgfscope}%
\pgfsys@transformshift{1.291212in}{1.401347in}%
\pgfsys@useobject{currentmarker}{}%
\end{pgfscope}%
\begin{pgfscope}%
\pgfsys@transformshift{1.340649in}{1.351374in}%
\pgfsys@useobject{currentmarker}{}%
\end{pgfscope}%
\begin{pgfscope}%
\pgfsys@transformshift{1.390086in}{1.068625in}%
\pgfsys@useobject{currentmarker}{}%
\end{pgfscope}%
\begin{pgfscope}%
\pgfsys@transformshift{1.439523in}{1.003864in}%
\pgfsys@useobject{currentmarker}{}%
\end{pgfscope}%
\begin{pgfscope}%
\pgfsys@transformshift{1.488960in}{1.002778in}%
\pgfsys@useobject{currentmarker}{}%
\end{pgfscope}%
\begin{pgfscope}%
\pgfsys@transformshift{1.538396in}{0.965816in}%
\pgfsys@useobject{currentmarker}{}%
\end{pgfscope}%
\begin{pgfscope}%
\pgfsys@transformshift{1.587833in}{0.963477in}%
\pgfsys@useobject{currentmarker}{}%
\end{pgfscope}%
\begin{pgfscope}%
\pgfsys@transformshift{1.637270in}{0.863308in}%
\pgfsys@useobject{currentmarker}{}%
\end{pgfscope}%
\begin{pgfscope}%
\pgfsys@transformshift{1.686707in}{0.822289in}%
\pgfsys@useobject{currentmarker}{}%
\end{pgfscope}%
\begin{pgfscope}%
\pgfsys@transformshift{1.736144in}{0.789635in}%
\pgfsys@useobject{currentmarker}{}%
\end{pgfscope}%
\begin{pgfscope}%
\pgfsys@transformshift{1.785581in}{0.735543in}%
\pgfsys@useobject{currentmarker}{}%
\end{pgfscope}%
\begin{pgfscope}%
\pgfsys@transformshift{1.835018in}{0.726412in}%
\pgfsys@useobject{currentmarker}{}%
\end{pgfscope}%
\begin{pgfscope}%
\pgfsys@transformshift{1.884454in}{0.723197in}%
\pgfsys@useobject{currentmarker}{}%
\end{pgfscope}%
\begin{pgfscope}%
\pgfsys@transformshift{1.933891in}{0.716612in}%
\pgfsys@useobject{currentmarker}{}%
\end{pgfscope}%
\begin{pgfscope}%
\pgfsys@transformshift{1.983328in}{0.701147in}%
\pgfsys@useobject{currentmarker}{}%
\end{pgfscope}%
\begin{pgfscope}%
\pgfsys@transformshift{2.032765in}{0.697403in}%
\pgfsys@useobject{currentmarker}{}%
\end{pgfscope}%
\begin{pgfscope}%
\pgfsys@transformshift{2.082202in}{0.688493in}%
\pgfsys@useobject{currentmarker}{}%
\end{pgfscope}%
\begin{pgfscope}%
\pgfsys@transformshift{2.131639in}{0.688214in}%
\pgfsys@useobject{currentmarker}{}%
\end{pgfscope}%
\begin{pgfscope}%
\pgfsys@transformshift{2.181075in}{0.680753in}%
\pgfsys@useobject{currentmarker}{}%
\end{pgfscope}%
\begin{pgfscope}%
\pgfsys@transformshift{2.230512in}{0.662362in}%
\pgfsys@useobject{currentmarker}{}%
\end{pgfscope}%
\begin{pgfscope}%
\pgfsys@transformshift{2.279949in}{0.627076in}%
\pgfsys@useobject{currentmarker}{}%
\end{pgfscope}%
\begin{pgfscope}%
\pgfsys@transformshift{2.329386in}{0.597825in}%
\pgfsys@useobject{currentmarker}{}%
\end{pgfscope}%
\begin{pgfscope}%
\pgfsys@transformshift{2.378823in}{0.579370in}%
\pgfsys@useobject{currentmarker}{}%
\end{pgfscope}%
\begin{pgfscope}%
\pgfsys@transformshift{2.428260in}{0.565642in}%
\pgfsys@useobject{currentmarker}{}%
\end{pgfscope}%
\begin{pgfscope}%
\pgfsys@transformshift{2.477697in}{0.561485in}%
\pgfsys@useobject{currentmarker}{}%
\end{pgfscope}%
\begin{pgfscope}%
\pgfsys@transformshift{2.527133in}{0.560255in}%
\pgfsys@useobject{currentmarker}{}%
\end{pgfscope}%
\begin{pgfscope}%
\pgfsys@transformshift{2.576570in}{0.548117in}%
\pgfsys@useobject{currentmarker}{}%
\end{pgfscope}%
\begin{pgfscope}%
\pgfsys@transformshift{2.626007in}{0.538743in}%
\pgfsys@useobject{currentmarker}{}%
\end{pgfscope}%
\begin{pgfscope}%
\pgfsys@transformshift{2.675444in}{0.519613in}%
\pgfsys@useobject{currentmarker}{}%
\end{pgfscope}%
\begin{pgfscope}%
\pgfsys@transformshift{2.724881in}{0.460676in}%
\pgfsys@useobject{currentmarker}{}%
\end{pgfscope}%
\begin{pgfscope}%
\pgfsys@transformshift{2.774318in}{0.460676in}%
\pgfsys@useobject{currentmarker}{}%
\end{pgfscope}%
\begin{pgfscope}%
\pgfsys@transformshift{2.823755in}{0.460676in}%
\pgfsys@useobject{currentmarker}{}%
\end{pgfscope}%
\begin{pgfscope}%
\pgfsys@transformshift{2.873191in}{0.460676in}%
\pgfsys@useobject{currentmarker}{}%
\end{pgfscope}%
\begin{pgfscope}%
\pgfsys@transformshift{2.922628in}{0.460676in}%
\pgfsys@useobject{currentmarker}{}%
\end{pgfscope}%
\begin{pgfscope}%
\pgfsys@transformshift{2.972065in}{0.460676in}%
\pgfsys@useobject{currentmarker}{}%
\end{pgfscope}%
\begin{pgfscope}%
\pgfsys@transformshift{3.021502in}{0.460676in}%
\pgfsys@useobject{currentmarker}{}%
\end{pgfscope}%
\begin{pgfscope}%
\pgfsys@transformshift{3.070939in}{0.460676in}%
\pgfsys@useobject{currentmarker}{}%
\end{pgfscope}%
\begin{pgfscope}%
\pgfsys@transformshift{3.120376in}{0.460676in}%
\pgfsys@useobject{currentmarker}{}%
\end{pgfscope}%
\begin{pgfscope}%
\pgfsys@transformshift{3.169812in}{0.460676in}%
\pgfsys@useobject{currentmarker}{}%
\end{pgfscope}%
\begin{pgfscope}%
\pgfsys@transformshift{3.219249in}{0.460676in}%
\pgfsys@useobject{currentmarker}{}%
\end{pgfscope}%
\begin{pgfscope}%
\pgfsys@transformshift{3.268686in}{0.460676in}%
\pgfsys@useobject{currentmarker}{}%
\end{pgfscope}%
\begin{pgfscope}%
\pgfsys@transformshift{3.318123in}{0.460676in}%
\pgfsys@useobject{currentmarker}{}%
\end{pgfscope}%
\begin{pgfscope}%
\pgfsys@transformshift{3.367560in}{0.460676in}%
\pgfsys@useobject{currentmarker}{}%
\end{pgfscope}%
\begin{pgfscope}%
\pgfsys@transformshift{3.416997in}{0.460676in}%
\pgfsys@useobject{currentmarker}{}%
\end{pgfscope}%
\begin{pgfscope}%
\pgfsys@transformshift{3.466434in}{0.460676in}%
\pgfsys@useobject{currentmarker}{}%
\end{pgfscope}%
\begin{pgfscope}%
\pgfsys@transformshift{3.515870in}{0.460676in}%
\pgfsys@useobject{currentmarker}{}%
\end{pgfscope}%
\begin{pgfscope}%
\pgfsys@transformshift{3.565307in}{0.460676in}%
\pgfsys@useobject{currentmarker}{}%
\end{pgfscope}%
\begin{pgfscope}%
\pgfsys@transformshift{3.614744in}{0.460676in}%
\pgfsys@useobject{currentmarker}{}%
\end{pgfscope}%
\begin{pgfscope}%
\pgfsys@transformshift{3.664181in}{0.460676in}%
\pgfsys@useobject{currentmarker}{}%
\end{pgfscope}%
\begin{pgfscope}%
\pgfsys@transformshift{3.713618in}{0.460676in}%
\pgfsys@useobject{currentmarker}{}%
\end{pgfscope}%
\begin{pgfscope}%
\pgfsys@transformshift{3.763055in}{0.460676in}%
\pgfsys@useobject{currentmarker}{}%
\end{pgfscope}%
\begin{pgfscope}%
\pgfsys@transformshift{3.812492in}{0.460676in}%
\pgfsys@useobject{currentmarker}{}%
\end{pgfscope}%
\begin{pgfscope}%
\pgfsys@transformshift{3.861928in}{0.460676in}%
\pgfsys@useobject{currentmarker}{}%
\end{pgfscope}%
\begin{pgfscope}%
\pgfsys@transformshift{3.911365in}{0.460676in}%
\pgfsys@useobject{currentmarker}{}%
\end{pgfscope}%
\begin{pgfscope}%
\pgfsys@transformshift{3.960802in}{0.460676in}%
\pgfsys@useobject{currentmarker}{}%
\end{pgfscope}%
\begin{pgfscope}%
\pgfsys@transformshift{4.010239in}{0.460676in}%
\pgfsys@useobject{currentmarker}{}%
\end{pgfscope}%
\begin{pgfscope}%
\pgfsys@transformshift{4.059676in}{0.460676in}%
\pgfsys@useobject{currentmarker}{}%
\end{pgfscope}%
\begin{pgfscope}%
\pgfsys@transformshift{4.109113in}{0.460676in}%
\pgfsys@useobject{currentmarker}{}%
\end{pgfscope}%
\begin{pgfscope}%
\pgfsys@transformshift{4.158550in}{0.460676in}%
\pgfsys@useobject{currentmarker}{}%
\end{pgfscope}%
\begin{pgfscope}%
\pgfsys@transformshift{4.207986in}{0.460676in}%
\pgfsys@useobject{currentmarker}{}%
\end{pgfscope}%
\begin{pgfscope}%
\pgfsys@transformshift{4.257423in}{0.460676in}%
\pgfsys@useobject{currentmarker}{}%
\end{pgfscope}%
\begin{pgfscope}%
\pgfsys@transformshift{4.306860in}{0.460676in}%
\pgfsys@useobject{currentmarker}{}%
\end{pgfscope}%
\begin{pgfscope}%
\pgfsys@transformshift{4.356297in}{0.460676in}%
\pgfsys@useobject{currentmarker}{}%
\end{pgfscope}%
\begin{pgfscope}%
\pgfsys@transformshift{4.405734in}{0.460676in}%
\pgfsys@useobject{currentmarker}{}%
\end{pgfscope}%
\begin{pgfscope}%
\pgfsys@transformshift{4.455171in}{0.460676in}%
\pgfsys@useobject{currentmarker}{}%
\end{pgfscope}%
\begin{pgfscope}%
\pgfsys@transformshift{4.504607in}{0.460676in}%
\pgfsys@useobject{currentmarker}{}%
\end{pgfscope}%
\begin{pgfscope}%
\pgfsys@transformshift{4.554044in}{0.460676in}%
\pgfsys@useobject{currentmarker}{}%
\end{pgfscope}%
\begin{pgfscope}%
\pgfsys@transformshift{4.603481in}{0.460676in}%
\pgfsys@useobject{currentmarker}{}%
\end{pgfscope}%
\begin{pgfscope}%
\pgfsys@transformshift{4.652918in}{0.460676in}%
\pgfsys@useobject{currentmarker}{}%
\end{pgfscope}%
\begin{pgfscope}%
\pgfsys@transformshift{4.702355in}{0.460676in}%
\pgfsys@useobject{currentmarker}{}%
\end{pgfscope}%
\begin{pgfscope}%
\pgfsys@transformshift{4.751792in}{0.460676in}%
\pgfsys@useobject{currentmarker}{}%
\end{pgfscope}%
\begin{pgfscope}%
\pgfsys@transformshift{4.801229in}{0.460676in}%
\pgfsys@useobject{currentmarker}{}%
\end{pgfscope}%
\begin{pgfscope}%
\pgfsys@transformshift{4.850665in}{0.460676in}%
\pgfsys@useobject{currentmarker}{}%
\end{pgfscope}%
\begin{pgfscope}%
\pgfsys@transformshift{4.900102in}{0.460676in}%
\pgfsys@useobject{currentmarker}{}%
\end{pgfscope}%
\begin{pgfscope}%
\pgfsys@transformshift{4.949539in}{0.460676in}%
\pgfsys@useobject{currentmarker}{}%
\end{pgfscope}%
\begin{pgfscope}%
\pgfsys@transformshift{4.998976in}{0.460676in}%
\pgfsys@useobject{currentmarker}{}%
\end{pgfscope}%
\begin{pgfscope}%
\pgfsys@transformshift{5.048413in}{0.460676in}%
\pgfsys@useobject{currentmarker}{}%
\end{pgfscope}%
\begin{pgfscope}%
\pgfsys@transformshift{5.097850in}{0.460676in}%
\pgfsys@useobject{currentmarker}{}%
\end{pgfscope}%
\end{pgfscope}%
\begin{pgfscope}%
\pgfpathrectangle{\pgfqpoint{0.581793in}{0.460676in}}{\pgfqpoint{4.731107in}{2.134248in}}%
\pgfusepath{clip}%
\pgfsetbuttcap%
\pgfsetroundjoin%
\definecolor{currentfill}{rgb}{0.172549,0.627451,0.172549}%
\pgfsetfillcolor{currentfill}%
\pgfsetfillopacity{0.750000}%
\pgfsetlinewidth{1.003750pt}%
\definecolor{currentstroke}{rgb}{0.172549,0.627451,0.172549}%
\pgfsetstrokecolor{currentstroke}%
\pgfsetstrokeopacity{0.750000}%
\pgfsetdash{}{0pt}%
\pgfsys@defobject{currentmarker}{\pgfqpoint{-0.019018in}{-0.019018in}}{\pgfqpoint{0.019018in}{0.019018in}}{%
\pgfpathmoveto{\pgfqpoint{-0.019018in}{-0.019018in}}%
\pgfpathlineto{\pgfqpoint{0.019018in}{-0.019018in}}%
\pgfpathlineto{\pgfqpoint{0.019018in}{0.019018in}}%
\pgfpathlineto{\pgfqpoint{-0.019018in}{0.019018in}}%
\pgfpathclose%
\pgfusepath{stroke,fill}%
}%
\begin{pgfscope}%
\pgfsys@transformshift{0.796844in}{2.475384in}%
\pgfsys@useobject{currentmarker}{}%
\end{pgfscope}%
\begin{pgfscope}%
\pgfsys@transformshift{0.846280in}{2.475384in}%
\pgfsys@useobject{currentmarker}{}%
\end{pgfscope}%
\begin{pgfscope}%
\pgfsys@transformshift{0.895717in}{2.002782in}%
\pgfsys@useobject{currentmarker}{}%
\end{pgfscope}%
\begin{pgfscope}%
\pgfsys@transformshift{0.945154in}{1.794749in}%
\pgfsys@useobject{currentmarker}{}%
\end{pgfscope}%
\begin{pgfscope}%
\pgfsys@transformshift{0.994591in}{1.552620in}%
\pgfsys@useobject{currentmarker}{}%
\end{pgfscope}%
\begin{pgfscope}%
\pgfsys@transformshift{1.044028in}{1.340160in}%
\pgfsys@useobject{currentmarker}{}%
\end{pgfscope}%
\begin{pgfscope}%
\pgfsys@transformshift{1.093465in}{1.300345in}%
\pgfsys@useobject{currentmarker}{}%
\end{pgfscope}%
\begin{pgfscope}%
\pgfsys@transformshift{1.142902in}{1.292349in}%
\pgfsys@useobject{currentmarker}{}%
\end{pgfscope}%
\begin{pgfscope}%
\pgfsys@transformshift{1.192338in}{1.279235in}%
\pgfsys@useobject{currentmarker}{}%
\end{pgfscope}%
\begin{pgfscope}%
\pgfsys@transformshift{1.241775in}{1.260126in}%
\pgfsys@useobject{currentmarker}{}%
\end{pgfscope}%
\begin{pgfscope}%
\pgfsys@transformshift{1.291212in}{1.242317in}%
\pgfsys@useobject{currentmarker}{}%
\end{pgfscope}%
\begin{pgfscope}%
\pgfsys@transformshift{1.340649in}{1.231734in}%
\pgfsys@useobject{currentmarker}{}%
\end{pgfscope}%
\begin{pgfscope}%
\pgfsys@transformshift{1.390086in}{1.212502in}%
\pgfsys@useobject{currentmarker}{}%
\end{pgfscope}%
\begin{pgfscope}%
\pgfsys@transformshift{1.439523in}{1.207537in}%
\pgfsys@useobject{currentmarker}{}%
\end{pgfscope}%
\begin{pgfscope}%
\pgfsys@transformshift{1.488960in}{1.202206in}%
\pgfsys@useobject{currentmarker}{}%
\end{pgfscope}%
\begin{pgfscope}%
\pgfsys@transformshift{1.538396in}{1.185650in}%
\pgfsys@useobject{currentmarker}{}%
\end{pgfscope}%
\begin{pgfscope}%
\pgfsys@transformshift{1.587833in}{1.181366in}%
\pgfsys@useobject{currentmarker}{}%
\end{pgfscope}%
\begin{pgfscope}%
\pgfsys@transformshift{1.637270in}{1.175323in}%
\pgfsys@useobject{currentmarker}{}%
\end{pgfscope}%
\begin{pgfscope}%
\pgfsys@transformshift{1.686707in}{1.164809in}%
\pgfsys@useobject{currentmarker}{}%
\end{pgfscope}%
\begin{pgfscope}%
\pgfsys@transformshift{1.736144in}{1.157039in}%
\pgfsys@useobject{currentmarker}{}%
\end{pgfscope}%
\begin{pgfscope}%
\pgfsys@transformshift{1.785581in}{1.152372in}%
\pgfsys@useobject{currentmarker}{}%
\end{pgfscope}%
\begin{pgfscope}%
\pgfsys@transformshift{1.835018in}{1.151132in}%
\pgfsys@useobject{currentmarker}{}%
\end{pgfscope}%
\begin{pgfscope}%
\pgfsys@transformshift{1.884454in}{1.143541in}%
\pgfsys@useobject{currentmarker}{}%
\end{pgfscope}%
\begin{pgfscope}%
\pgfsys@transformshift{1.933891in}{1.129113in}%
\pgfsys@useobject{currentmarker}{}%
\end{pgfscope}%
\begin{pgfscope}%
\pgfsys@transformshift{1.983328in}{1.128790in}%
\pgfsys@useobject{currentmarker}{}%
\end{pgfscope}%
\begin{pgfscope}%
\pgfsys@transformshift{2.032765in}{1.127004in}%
\pgfsys@useobject{currentmarker}{}%
\end{pgfscope}%
\begin{pgfscope}%
\pgfsys@transformshift{2.082202in}{1.122808in}%
\pgfsys@useobject{currentmarker}{}%
\end{pgfscope}%
\begin{pgfscope}%
\pgfsys@transformshift{2.131639in}{1.120472in}%
\pgfsys@useobject{currentmarker}{}%
\end{pgfscope}%
\begin{pgfscope}%
\pgfsys@transformshift{2.181075in}{1.119254in}%
\pgfsys@useobject{currentmarker}{}%
\end{pgfscope}%
\begin{pgfscope}%
\pgfsys@transformshift{2.230512in}{1.112822in}%
\pgfsys@useobject{currentmarker}{}%
\end{pgfscope}%
\begin{pgfscope}%
\pgfsys@transformshift{2.279949in}{1.107358in}%
\pgfsys@useobject{currentmarker}{}%
\end{pgfscope}%
\begin{pgfscope}%
\pgfsys@transformshift{2.329386in}{1.101387in}%
\pgfsys@useobject{currentmarker}{}%
\end{pgfscope}%
\begin{pgfscope}%
\pgfsys@transformshift{2.378823in}{1.100066in}%
\pgfsys@useobject{currentmarker}{}%
\end{pgfscope}%
\begin{pgfscope}%
\pgfsys@transformshift{2.428260in}{1.095708in}%
\pgfsys@useobject{currentmarker}{}%
\end{pgfscope}%
\begin{pgfscope}%
\pgfsys@transformshift{2.477697in}{1.091531in}%
\pgfsys@useobject{currentmarker}{}%
\end{pgfscope}%
\begin{pgfscope}%
\pgfsys@transformshift{2.527133in}{1.091531in}%
\pgfsys@useobject{currentmarker}{}%
\end{pgfscope}%
\begin{pgfscope}%
\pgfsys@transformshift{2.576570in}{1.090758in}%
\pgfsys@useobject{currentmarker}{}%
\end{pgfscope}%
\begin{pgfscope}%
\pgfsys@transformshift{2.626007in}{1.088771in}%
\pgfsys@useobject{currentmarker}{}%
\end{pgfscope}%
\begin{pgfscope}%
\pgfsys@transformshift{2.675444in}{1.081568in}%
\pgfsys@useobject{currentmarker}{}%
\end{pgfscope}%
\begin{pgfscope}%
\pgfsys@transformshift{2.724881in}{1.077001in}%
\pgfsys@useobject{currentmarker}{}%
\end{pgfscope}%
\begin{pgfscope}%
\pgfsys@transformshift{2.774318in}{1.066926in}%
\pgfsys@useobject{currentmarker}{}%
\end{pgfscope}%
\begin{pgfscope}%
\pgfsys@transformshift{2.823755in}{1.065422in}%
\pgfsys@useobject{currentmarker}{}%
\end{pgfscope}%
\begin{pgfscope}%
\pgfsys@transformshift{2.873191in}{1.064733in}%
\pgfsys@useobject{currentmarker}{}%
\end{pgfscope}%
\begin{pgfscope}%
\pgfsys@transformshift{2.922628in}{1.059479in}%
\pgfsys@useobject{currentmarker}{}%
\end{pgfscope}%
\begin{pgfscope}%
\pgfsys@transformshift{2.972065in}{1.022417in}%
\pgfsys@useobject{currentmarker}{}%
\end{pgfscope}%
\begin{pgfscope}%
\pgfsys@transformshift{3.021502in}{1.014591in}%
\pgfsys@useobject{currentmarker}{}%
\end{pgfscope}%
\begin{pgfscope}%
\pgfsys@transformshift{3.070939in}{1.009503in}%
\pgfsys@useobject{currentmarker}{}%
\end{pgfscope}%
\begin{pgfscope}%
\pgfsys@transformshift{3.120376in}{0.998672in}%
\pgfsys@useobject{currentmarker}{}%
\end{pgfscope}%
\begin{pgfscope}%
\pgfsys@transformshift{3.169812in}{0.988539in}%
\pgfsys@useobject{currentmarker}{}%
\end{pgfscope}%
\begin{pgfscope}%
\pgfsys@transformshift{3.219249in}{0.970946in}%
\pgfsys@useobject{currentmarker}{}%
\end{pgfscope}%
\begin{pgfscope}%
\pgfsys@transformshift{3.268686in}{0.955524in}%
\pgfsys@useobject{currentmarker}{}%
\end{pgfscope}%
\begin{pgfscope}%
\pgfsys@transformshift{3.318123in}{0.949148in}%
\pgfsys@useobject{currentmarker}{}%
\end{pgfscope}%
\begin{pgfscope}%
\pgfsys@transformshift{3.367560in}{0.948518in}%
\pgfsys@useobject{currentmarker}{}%
\end{pgfscope}%
\begin{pgfscope}%
\pgfsys@transformshift{3.416997in}{0.941528in}%
\pgfsys@useobject{currentmarker}{}%
\end{pgfscope}%
\begin{pgfscope}%
\pgfsys@transformshift{3.466434in}{0.925472in}%
\pgfsys@useobject{currentmarker}{}%
\end{pgfscope}%
\begin{pgfscope}%
\pgfsys@transformshift{3.515870in}{0.876531in}%
\pgfsys@useobject{currentmarker}{}%
\end{pgfscope}%
\begin{pgfscope}%
\pgfsys@transformshift{3.565307in}{0.859466in}%
\pgfsys@useobject{currentmarker}{}%
\end{pgfscope}%
\begin{pgfscope}%
\pgfsys@transformshift{3.614744in}{0.857109in}%
\pgfsys@useobject{currentmarker}{}%
\end{pgfscope}%
\begin{pgfscope}%
\pgfsys@transformshift{3.664181in}{0.851884in}%
\pgfsys@useobject{currentmarker}{}%
\end{pgfscope}%
\begin{pgfscope}%
\pgfsys@transformshift{3.713618in}{0.829161in}%
\pgfsys@useobject{currentmarker}{}%
\end{pgfscope}%
\begin{pgfscope}%
\pgfsys@transformshift{3.763055in}{0.823586in}%
\pgfsys@useobject{currentmarker}{}%
\end{pgfscope}%
\begin{pgfscope}%
\pgfsys@transformshift{3.812492in}{0.820704in}%
\pgfsys@useobject{currentmarker}{}%
\end{pgfscope}%
\begin{pgfscope}%
\pgfsys@transformshift{3.861928in}{0.811450in}%
\pgfsys@useobject{currentmarker}{}%
\end{pgfscope}%
\begin{pgfscope}%
\pgfsys@transformshift{3.911365in}{0.800674in}%
\pgfsys@useobject{currentmarker}{}%
\end{pgfscope}%
\begin{pgfscope}%
\pgfsys@transformshift{3.960802in}{0.784633in}%
\pgfsys@useobject{currentmarker}{}%
\end{pgfscope}%
\begin{pgfscope}%
\pgfsys@transformshift{4.010239in}{0.778249in}%
\pgfsys@useobject{currentmarker}{}%
\end{pgfscope}%
\begin{pgfscope}%
\pgfsys@transformshift{4.059676in}{0.770798in}%
\pgfsys@useobject{currentmarker}{}%
\end{pgfscope}%
\begin{pgfscope}%
\pgfsys@transformshift{4.109113in}{0.770506in}%
\pgfsys@useobject{currentmarker}{}%
\end{pgfscope}%
\begin{pgfscope}%
\pgfsys@transformshift{4.158550in}{0.769500in}%
\pgfsys@useobject{currentmarker}{}%
\end{pgfscope}%
\begin{pgfscope}%
\pgfsys@transformshift{4.207986in}{0.768660in}%
\pgfsys@useobject{currentmarker}{}%
\end{pgfscope}%
\begin{pgfscope}%
\pgfsys@transformshift{4.257423in}{0.749993in}%
\pgfsys@useobject{currentmarker}{}%
\end{pgfscope}%
\begin{pgfscope}%
\pgfsys@transformshift{4.306860in}{0.703053in}%
\pgfsys@useobject{currentmarker}{}%
\end{pgfscope}%
\begin{pgfscope}%
\pgfsys@transformshift{4.356297in}{0.700999in}%
\pgfsys@useobject{currentmarker}{}%
\end{pgfscope}%
\begin{pgfscope}%
\pgfsys@transformshift{4.405734in}{0.685869in}%
\pgfsys@useobject{currentmarker}{}%
\end{pgfscope}%
\begin{pgfscope}%
\pgfsys@transformshift{4.455171in}{0.673197in}%
\pgfsys@useobject{currentmarker}{}%
\end{pgfscope}%
\begin{pgfscope}%
\pgfsys@transformshift{4.504607in}{0.645732in}%
\pgfsys@useobject{currentmarker}{}%
\end{pgfscope}%
\begin{pgfscope}%
\pgfsys@transformshift{4.554044in}{0.631306in}%
\pgfsys@useobject{currentmarker}{}%
\end{pgfscope}%
\begin{pgfscope}%
\pgfsys@transformshift{4.603481in}{0.600368in}%
\pgfsys@useobject{currentmarker}{}%
\end{pgfscope}%
\begin{pgfscope}%
\pgfsys@transformshift{4.652918in}{0.580113in}%
\pgfsys@useobject{currentmarker}{}%
\end{pgfscope}%
\begin{pgfscope}%
\pgfsys@transformshift{4.702355in}{0.556327in}%
\pgfsys@useobject{currentmarker}{}%
\end{pgfscope}%
\begin{pgfscope}%
\pgfsys@transformshift{4.751792in}{0.526955in}%
\pgfsys@useobject{currentmarker}{}%
\end{pgfscope}%
\begin{pgfscope}%
\pgfsys@transformshift{4.801229in}{0.460676in}%
\pgfsys@useobject{currentmarker}{}%
\end{pgfscope}%
\begin{pgfscope}%
\pgfsys@transformshift{4.850665in}{0.460676in}%
\pgfsys@useobject{currentmarker}{}%
\end{pgfscope}%
\begin{pgfscope}%
\pgfsys@transformshift{4.900102in}{0.460676in}%
\pgfsys@useobject{currentmarker}{}%
\end{pgfscope}%
\begin{pgfscope}%
\pgfsys@transformshift{4.949539in}{0.460676in}%
\pgfsys@useobject{currentmarker}{}%
\end{pgfscope}%
\begin{pgfscope}%
\pgfsys@transformshift{4.998976in}{0.460676in}%
\pgfsys@useobject{currentmarker}{}%
\end{pgfscope}%
\begin{pgfscope}%
\pgfsys@transformshift{5.048413in}{0.460676in}%
\pgfsys@useobject{currentmarker}{}%
\end{pgfscope}%
\begin{pgfscope}%
\pgfsys@transformshift{5.097850in}{0.460676in}%
\pgfsys@useobject{currentmarker}{}%
\end{pgfscope}%
\end{pgfscope}%
\begin{pgfscope}%
\pgfpathrectangle{\pgfqpoint{0.581793in}{0.460676in}}{\pgfqpoint{4.731107in}{2.134248in}}%
\pgfusepath{clip}%
\pgfsetbuttcap%
\pgfsetroundjoin%
\definecolor{currentfill}{rgb}{0.839216,0.152941,0.156863}%
\pgfsetfillcolor{currentfill}%
\pgfsetfillopacity{0.750000}%
\pgfsetlinewidth{1.003750pt}%
\definecolor{currentstroke}{rgb}{0.839216,0.152941,0.156863}%
\pgfsetstrokecolor{currentstroke}%
\pgfsetstrokeopacity{0.750000}%
\pgfsetdash{}{0pt}%
\pgfsys@defobject{currentmarker}{\pgfqpoint{-0.019018in}{-0.019018in}}{\pgfqpoint{0.019018in}{0.019018in}}{%
\pgfpathmoveto{\pgfqpoint{0.000000in}{0.019018in}}%
\pgfpathlineto{\pgfqpoint{-0.019018in}{-0.019018in}}%
\pgfpathlineto{\pgfqpoint{0.019018in}{-0.019018in}}%
\pgfpathclose%
\pgfusepath{stroke,fill}%
}%
\begin{pgfscope}%
\pgfsys@transformshift{0.796844in}{2.484359in}%
\pgfsys@useobject{currentmarker}{}%
\end{pgfscope}%
\begin{pgfscope}%
\pgfsys@transformshift{0.846280in}{2.484359in}%
\pgfsys@useobject{currentmarker}{}%
\end{pgfscope}%
\begin{pgfscope}%
\pgfsys@transformshift{0.895717in}{2.212229in}%
\pgfsys@useobject{currentmarker}{}%
\end{pgfscope}%
\begin{pgfscope}%
\pgfsys@transformshift{0.945154in}{2.179912in}%
\pgfsys@useobject{currentmarker}{}%
\end{pgfscope}%
\begin{pgfscope}%
\pgfsys@transformshift{0.994591in}{1.806088in}%
\pgfsys@useobject{currentmarker}{}%
\end{pgfscope}%
\begin{pgfscope}%
\pgfsys@transformshift{1.044028in}{1.705082in}%
\pgfsys@useobject{currentmarker}{}%
\end{pgfscope}%
\begin{pgfscope}%
\pgfsys@transformshift{1.093465in}{1.562716in}%
\pgfsys@useobject{currentmarker}{}%
\end{pgfscope}%
\begin{pgfscope}%
\pgfsys@transformshift{1.142902in}{1.559789in}%
\pgfsys@useobject{currentmarker}{}%
\end{pgfscope}%
\begin{pgfscope}%
\pgfsys@transformshift{1.192338in}{1.412536in}%
\pgfsys@useobject{currentmarker}{}%
\end{pgfscope}%
\begin{pgfscope}%
\pgfsys@transformshift{1.241775in}{1.358672in}%
\pgfsys@useobject{currentmarker}{}%
\end{pgfscope}%
\begin{pgfscope}%
\pgfsys@transformshift{1.291212in}{1.220142in}%
\pgfsys@useobject{currentmarker}{}%
\end{pgfscope}%
\begin{pgfscope}%
\pgfsys@transformshift{1.340649in}{1.065890in}%
\pgfsys@useobject{currentmarker}{}%
\end{pgfscope}%
\begin{pgfscope}%
\pgfsys@transformshift{1.390086in}{1.014218in}%
\pgfsys@useobject{currentmarker}{}%
\end{pgfscope}%
\begin{pgfscope}%
\pgfsys@transformshift{1.439523in}{0.999129in}%
\pgfsys@useobject{currentmarker}{}%
\end{pgfscope}%
\begin{pgfscope}%
\pgfsys@transformshift{1.488960in}{0.964065in}%
\pgfsys@useobject{currentmarker}{}%
\end{pgfscope}%
\begin{pgfscope}%
\pgfsys@transformshift{1.538396in}{0.959393in}%
\pgfsys@useobject{currentmarker}{}%
\end{pgfscope}%
\begin{pgfscope}%
\pgfsys@transformshift{1.587833in}{0.849647in}%
\pgfsys@useobject{currentmarker}{}%
\end{pgfscope}%
\begin{pgfscope}%
\pgfsys@transformshift{1.637270in}{0.819932in}%
\pgfsys@useobject{currentmarker}{}%
\end{pgfscope}%
\begin{pgfscope}%
\pgfsys@transformshift{1.686707in}{0.809296in}%
\pgfsys@useobject{currentmarker}{}%
\end{pgfscope}%
\begin{pgfscope}%
\pgfsys@transformshift{1.736144in}{0.781029in}%
\pgfsys@useobject{currentmarker}{}%
\end{pgfscope}%
\begin{pgfscope}%
\pgfsys@transformshift{1.785581in}{0.771801in}%
\pgfsys@useobject{currentmarker}{}%
\end{pgfscope}%
\begin{pgfscope}%
\pgfsys@transformshift{1.835018in}{0.769500in}%
\pgfsys@useobject{currentmarker}{}%
\end{pgfscope}%
\begin{pgfscope}%
\pgfsys@transformshift{1.884454in}{0.756539in}%
\pgfsys@useobject{currentmarker}{}%
\end{pgfscope}%
\begin{pgfscope}%
\pgfsys@transformshift{1.933891in}{0.741208in}%
\pgfsys@useobject{currentmarker}{}%
\end{pgfscope}%
\begin{pgfscope}%
\pgfsys@transformshift{1.983328in}{0.713456in}%
\pgfsys@useobject{currentmarker}{}%
\end{pgfscope}%
\begin{pgfscope}%
\pgfsys@transformshift{2.032765in}{0.702993in}%
\pgfsys@useobject{currentmarker}{}%
\end{pgfscope}%
\begin{pgfscope}%
\pgfsys@transformshift{2.082202in}{0.702993in}%
\pgfsys@useobject{currentmarker}{}%
\end{pgfscope}%
\begin{pgfscope}%
\pgfsys@transformshift{2.131639in}{0.696900in}%
\pgfsys@useobject{currentmarker}{}%
\end{pgfscope}%
\begin{pgfscope}%
\pgfsys@transformshift{2.181075in}{0.682711in}%
\pgfsys@useobject{currentmarker}{}%
\end{pgfscope}%
\begin{pgfscope}%
\pgfsys@transformshift{2.230512in}{0.679672in}%
\pgfsys@useobject{currentmarker}{}%
\end{pgfscope}%
\begin{pgfscope}%
\pgfsys@transformshift{2.279949in}{0.659842in}%
\pgfsys@useobject{currentmarker}{}%
\end{pgfscope}%
\begin{pgfscope}%
\pgfsys@transformshift{2.329386in}{0.653332in}%
\pgfsys@useobject{currentmarker}{}%
\end{pgfscope}%
\begin{pgfscope}%
\pgfsys@transformshift{2.378823in}{0.649564in}%
\pgfsys@useobject{currentmarker}{}%
\end{pgfscope}%
\begin{pgfscope}%
\pgfsys@transformshift{2.428260in}{0.643878in}%
\pgfsys@useobject{currentmarker}{}%
\end{pgfscope}%
\begin{pgfscope}%
\pgfsys@transformshift{2.477697in}{0.639729in}%
\pgfsys@useobject{currentmarker}{}%
\end{pgfscope}%
\begin{pgfscope}%
\pgfsys@transformshift{2.527133in}{0.635735in}%
\pgfsys@useobject{currentmarker}{}%
\end{pgfscope}%
\begin{pgfscope}%
\pgfsys@transformshift{2.576570in}{0.635232in}%
\pgfsys@useobject{currentmarker}{}%
\end{pgfscope}%
\begin{pgfscope}%
\pgfsys@transformshift{2.626007in}{0.628621in}%
\pgfsys@useobject{currentmarker}{}%
\end{pgfscope}%
\begin{pgfscope}%
\pgfsys@transformshift{2.675444in}{0.626404in}%
\pgfsys@useobject{currentmarker}{}%
\end{pgfscope}%
\begin{pgfscope}%
\pgfsys@transformshift{2.724881in}{0.620896in}%
\pgfsys@useobject{currentmarker}{}%
\end{pgfscope}%
\begin{pgfscope}%
\pgfsys@transformshift{2.774318in}{0.613766in}%
\pgfsys@useobject{currentmarker}{}%
\end{pgfscope}%
\begin{pgfscope}%
\pgfsys@transformshift{2.823755in}{0.611756in}%
\pgfsys@useobject{currentmarker}{}%
\end{pgfscope}%
\begin{pgfscope}%
\pgfsys@transformshift{2.873191in}{0.608754in}%
\pgfsys@useobject{currentmarker}{}%
\end{pgfscope}%
\begin{pgfscope}%
\pgfsys@transformshift{2.922628in}{0.607257in}%
\pgfsys@useobject{currentmarker}{}%
\end{pgfscope}%
\begin{pgfscope}%
\pgfsys@transformshift{2.972065in}{0.605668in}%
\pgfsys@useobject{currentmarker}{}%
\end{pgfscope}%
\begin{pgfscope}%
\pgfsys@transformshift{3.021502in}{0.604063in}%
\pgfsys@useobject{currentmarker}{}%
\end{pgfscope}%
\begin{pgfscope}%
\pgfsys@transformshift{3.070939in}{0.602430in}%
\pgfsys@useobject{currentmarker}{}%
\end{pgfscope}%
\begin{pgfscope}%
\pgfsys@transformshift{3.120376in}{0.602118in}%
\pgfsys@useobject{currentmarker}{}%
\end{pgfscope}%
\begin{pgfscope}%
\pgfsys@transformshift{3.169812in}{0.590172in}%
\pgfsys@useobject{currentmarker}{}%
\end{pgfscope}%
\begin{pgfscope}%
\pgfsys@transformshift{3.219249in}{0.579822in}%
\pgfsys@useobject{currentmarker}{}%
\end{pgfscope}%
\begin{pgfscope}%
\pgfsys@transformshift{3.268686in}{0.577601in}%
\pgfsys@useobject{currentmarker}{}%
\end{pgfscope}%
\begin{pgfscope}%
\pgfsys@transformshift{3.318123in}{0.570384in}%
\pgfsys@useobject{currentmarker}{}%
\end{pgfscope}%
\begin{pgfscope}%
\pgfsys@transformshift{3.367560in}{0.563453in}%
\pgfsys@useobject{currentmarker}{}%
\end{pgfscope}%
\begin{pgfscope}%
\pgfsys@transformshift{3.416997in}{0.559944in}%
\pgfsys@useobject{currentmarker}{}%
\end{pgfscope}%
\begin{pgfscope}%
\pgfsys@transformshift{3.466434in}{0.548117in}%
\pgfsys@useobject{currentmarker}{}%
\end{pgfscope}%
\begin{pgfscope}%
\pgfsys@transformshift{3.515870in}{0.545893in}%
\pgfsys@useobject{currentmarker}{}%
\end{pgfscope}%
\begin{pgfscope}%
\pgfsys@transformshift{3.565307in}{0.545688in}%
\pgfsys@useobject{currentmarker}{}%
\end{pgfscope}%
\begin{pgfscope}%
\pgfsys@transformshift{3.614744in}{0.540276in}%
\pgfsys@useobject{currentmarker}{}%
\end{pgfscope}%
\begin{pgfscope}%
\pgfsys@transformshift{3.664181in}{0.528645in}%
\pgfsys@useobject{currentmarker}{}%
\end{pgfscope}%
\begin{pgfscope}%
\pgfsys@transformshift{3.713618in}{0.517861in}%
\pgfsys@useobject{currentmarker}{}%
\end{pgfscope}%
\begin{pgfscope}%
\pgfsys@transformshift{3.763055in}{0.504202in}%
\pgfsys@useobject{currentmarker}{}%
\end{pgfscope}%
\begin{pgfscope}%
\pgfsys@transformshift{3.812492in}{0.460676in}%
\pgfsys@useobject{currentmarker}{}%
\end{pgfscope}%
\begin{pgfscope}%
\pgfsys@transformshift{3.861928in}{0.460676in}%
\pgfsys@useobject{currentmarker}{}%
\end{pgfscope}%
\begin{pgfscope}%
\pgfsys@transformshift{3.911365in}{0.460676in}%
\pgfsys@useobject{currentmarker}{}%
\end{pgfscope}%
\begin{pgfscope}%
\pgfsys@transformshift{3.960802in}{0.460676in}%
\pgfsys@useobject{currentmarker}{}%
\end{pgfscope}%
\begin{pgfscope}%
\pgfsys@transformshift{4.010239in}{0.460676in}%
\pgfsys@useobject{currentmarker}{}%
\end{pgfscope}%
\begin{pgfscope}%
\pgfsys@transformshift{4.059676in}{0.460676in}%
\pgfsys@useobject{currentmarker}{}%
\end{pgfscope}%
\begin{pgfscope}%
\pgfsys@transformshift{4.109113in}{0.460676in}%
\pgfsys@useobject{currentmarker}{}%
\end{pgfscope}%
\begin{pgfscope}%
\pgfsys@transformshift{4.158550in}{0.460676in}%
\pgfsys@useobject{currentmarker}{}%
\end{pgfscope}%
\begin{pgfscope}%
\pgfsys@transformshift{4.207986in}{0.460676in}%
\pgfsys@useobject{currentmarker}{}%
\end{pgfscope}%
\begin{pgfscope}%
\pgfsys@transformshift{4.257423in}{0.460676in}%
\pgfsys@useobject{currentmarker}{}%
\end{pgfscope}%
\begin{pgfscope}%
\pgfsys@transformshift{4.306860in}{0.460676in}%
\pgfsys@useobject{currentmarker}{}%
\end{pgfscope}%
\begin{pgfscope}%
\pgfsys@transformshift{4.356297in}{0.460676in}%
\pgfsys@useobject{currentmarker}{}%
\end{pgfscope}%
\begin{pgfscope}%
\pgfsys@transformshift{4.405734in}{0.460676in}%
\pgfsys@useobject{currentmarker}{}%
\end{pgfscope}%
\begin{pgfscope}%
\pgfsys@transformshift{4.455171in}{0.460676in}%
\pgfsys@useobject{currentmarker}{}%
\end{pgfscope}%
\begin{pgfscope}%
\pgfsys@transformshift{4.504607in}{0.460676in}%
\pgfsys@useobject{currentmarker}{}%
\end{pgfscope}%
\begin{pgfscope}%
\pgfsys@transformshift{4.554044in}{0.460676in}%
\pgfsys@useobject{currentmarker}{}%
\end{pgfscope}%
\begin{pgfscope}%
\pgfsys@transformshift{4.603481in}{0.460676in}%
\pgfsys@useobject{currentmarker}{}%
\end{pgfscope}%
\begin{pgfscope}%
\pgfsys@transformshift{4.652918in}{0.460676in}%
\pgfsys@useobject{currentmarker}{}%
\end{pgfscope}%
\begin{pgfscope}%
\pgfsys@transformshift{4.702355in}{0.460676in}%
\pgfsys@useobject{currentmarker}{}%
\end{pgfscope}%
\begin{pgfscope}%
\pgfsys@transformshift{4.751792in}{0.460676in}%
\pgfsys@useobject{currentmarker}{}%
\end{pgfscope}%
\begin{pgfscope}%
\pgfsys@transformshift{4.801229in}{0.460676in}%
\pgfsys@useobject{currentmarker}{}%
\end{pgfscope}%
\begin{pgfscope}%
\pgfsys@transformshift{4.850665in}{0.460676in}%
\pgfsys@useobject{currentmarker}{}%
\end{pgfscope}%
\begin{pgfscope}%
\pgfsys@transformshift{4.900102in}{0.460676in}%
\pgfsys@useobject{currentmarker}{}%
\end{pgfscope}%
\begin{pgfscope}%
\pgfsys@transformshift{4.949539in}{0.460676in}%
\pgfsys@useobject{currentmarker}{}%
\end{pgfscope}%
\begin{pgfscope}%
\pgfsys@transformshift{4.998976in}{0.460676in}%
\pgfsys@useobject{currentmarker}{}%
\end{pgfscope}%
\begin{pgfscope}%
\pgfsys@transformshift{5.048413in}{0.460676in}%
\pgfsys@useobject{currentmarker}{}%
\end{pgfscope}%
\begin{pgfscope}%
\pgfsys@transformshift{5.097850in}{0.460676in}%
\pgfsys@useobject{currentmarker}{}%
\end{pgfscope}%
\end{pgfscope}%
\begin{pgfscope}%
\pgfpathrectangle{\pgfqpoint{0.581793in}{0.460676in}}{\pgfqpoint{4.731107in}{2.134248in}}%
\pgfusepath{clip}%
\pgfsetbuttcap%
\pgfsetroundjoin%
\definecolor{currentfill}{rgb}{0.580392,0.403922,0.741176}%
\pgfsetfillcolor{currentfill}%
\pgfsetfillopacity{0.750000}%
\pgfsetlinewidth{1.003750pt}%
\definecolor{currentstroke}{rgb}{0.580392,0.403922,0.741176}%
\pgfsetstrokecolor{currentstroke}%
\pgfsetstrokeopacity{0.750000}%
\pgfsetdash{}{0pt}%
\pgfsys@defobject{currentmarker}{\pgfqpoint{-0.026896in}{-0.026896in}}{\pgfqpoint{0.026896in}{0.026896in}}{%
\pgfpathmoveto{\pgfqpoint{-0.000000in}{-0.026896in}}%
\pgfpathlineto{\pgfqpoint{0.026896in}{0.000000in}}%
\pgfpathlineto{\pgfqpoint{0.000000in}{0.026896in}}%
\pgfpathlineto{\pgfqpoint{-0.026896in}{0.000000in}}%
\pgfpathclose%
\pgfusepath{stroke,fill}%
}%
\begin{pgfscope}%
\pgfsys@transformshift{0.796844in}{2.493293in}%
\pgfsys@useobject{currentmarker}{}%
\end{pgfscope}%
\begin{pgfscope}%
\pgfsys@transformshift{0.846280in}{2.493293in}%
\pgfsys@useobject{currentmarker}{}%
\end{pgfscope}%
\begin{pgfscope}%
\pgfsys@transformshift{0.895717in}{2.355631in}%
\pgfsys@useobject{currentmarker}{}%
\end{pgfscope}%
\begin{pgfscope}%
\pgfsys@transformshift{0.945154in}{2.119248in}%
\pgfsys@useobject{currentmarker}{}%
\end{pgfscope}%
\begin{pgfscope}%
\pgfsys@transformshift{0.994591in}{2.113667in}%
\pgfsys@useobject{currentmarker}{}%
\end{pgfscope}%
\begin{pgfscope}%
\pgfsys@transformshift{1.044028in}{2.110283in}%
\pgfsys@useobject{currentmarker}{}%
\end{pgfscope}%
\begin{pgfscope}%
\pgfsys@transformshift{1.093465in}{1.609868in}%
\pgfsys@useobject{currentmarker}{}%
\end{pgfscope}%
\begin{pgfscope}%
\pgfsys@transformshift{1.142902in}{1.509323in}%
\pgfsys@useobject{currentmarker}{}%
\end{pgfscope}%
\begin{pgfscope}%
\pgfsys@transformshift{1.192338in}{1.322828in}%
\pgfsys@useobject{currentmarker}{}%
\end{pgfscope}%
\begin{pgfscope}%
\pgfsys@transformshift{1.241775in}{1.310177in}%
\pgfsys@useobject{currentmarker}{}%
\end{pgfscope}%
\begin{pgfscope}%
\pgfsys@transformshift{1.291212in}{1.279737in}%
\pgfsys@useobject{currentmarker}{}%
\end{pgfscope}%
\begin{pgfscope}%
\pgfsys@transformshift{1.340649in}{1.262517in}%
\pgfsys@useobject{currentmarker}{}%
\end{pgfscope}%
\begin{pgfscope}%
\pgfsys@transformshift{1.390086in}{1.227626in}%
\pgfsys@useobject{currentmarker}{}%
\end{pgfscope}%
\begin{pgfscope}%
\pgfsys@transformshift{1.439523in}{1.220496in}%
\pgfsys@useobject{currentmarker}{}%
\end{pgfscope}%
\begin{pgfscope}%
\pgfsys@transformshift{1.488960in}{1.220007in}%
\pgfsys@useobject{currentmarker}{}%
\end{pgfscope}%
\begin{pgfscope}%
\pgfsys@transformshift{1.538396in}{1.215589in}%
\pgfsys@useobject{currentmarker}{}%
\end{pgfscope}%
\begin{pgfscope}%
\pgfsys@transformshift{1.587833in}{1.211874in}%
\pgfsys@useobject{currentmarker}{}%
\end{pgfscope}%
\begin{pgfscope}%
\pgfsys@transformshift{1.637270in}{1.190494in}%
\pgfsys@useobject{currentmarker}{}%
\end{pgfscope}%
\begin{pgfscope}%
\pgfsys@transformshift{1.686707in}{1.179802in}%
\pgfsys@useobject{currentmarker}{}%
\end{pgfscope}%
\begin{pgfscope}%
\pgfsys@transformshift{1.736144in}{1.161058in}%
\pgfsys@useobject{currentmarker}{}%
\end{pgfscope}%
\begin{pgfscope}%
\pgfsys@transformshift{1.785581in}{1.160763in}%
\pgfsys@useobject{currentmarker}{}%
\end{pgfscope}%
\begin{pgfscope}%
\pgfsys@transformshift{1.835018in}{1.152226in}%
\pgfsys@useobject{currentmarker}{}%
\end{pgfscope}%
\begin{pgfscope}%
\pgfsys@transformshift{1.884454in}{1.148172in}%
\pgfsys@useobject{currentmarker}{}%
\end{pgfscope}%
\begin{pgfscope}%
\pgfsys@transformshift{1.933891in}{1.143636in}%
\pgfsys@useobject{currentmarker}{}%
\end{pgfscope}%
\begin{pgfscope}%
\pgfsys@transformshift{1.983328in}{1.137728in}%
\pgfsys@useobject{currentmarker}{}%
\end{pgfscope}%
\begin{pgfscope}%
\pgfsys@transformshift{2.032765in}{1.137105in}%
\pgfsys@useobject{currentmarker}{}%
\end{pgfscope}%
\begin{pgfscope}%
\pgfsys@transformshift{2.082202in}{1.132582in}%
\pgfsys@useobject{currentmarker}{}%
\end{pgfscope}%
\begin{pgfscope}%
\pgfsys@transformshift{2.131639in}{1.123046in}%
\pgfsys@useobject{currentmarker}{}%
\end{pgfscope}%
\begin{pgfscope}%
\pgfsys@transformshift{2.181075in}{1.116541in}%
\pgfsys@useobject{currentmarker}{}%
\end{pgfscope}%
\begin{pgfscope}%
\pgfsys@transformshift{2.230512in}{1.102183in}%
\pgfsys@useobject{currentmarker}{}%
\end{pgfscope}%
\begin{pgfscope}%
\pgfsys@transformshift{2.279949in}{1.097345in}%
\pgfsys@useobject{currentmarker}{}%
\end{pgfscope}%
\begin{pgfscope}%
\pgfsys@transformshift{2.329386in}{1.094915in}%
\pgfsys@useobject{currentmarker}{}%
\end{pgfscope}%
\begin{pgfscope}%
\pgfsys@transformshift{2.378823in}{1.093079in}%
\pgfsys@useobject{currentmarker}{}%
\end{pgfscope}%
\begin{pgfscope}%
\pgfsys@transformshift{2.428260in}{1.093079in}%
\pgfsys@useobject{currentmarker}{}%
\end{pgfscope}%
\begin{pgfscope}%
\pgfsys@transformshift{2.477697in}{1.090653in}%
\pgfsys@useobject{currentmarker}{}%
\end{pgfscope}%
\begin{pgfscope}%
\pgfsys@transformshift{2.527133in}{1.088516in}%
\pgfsys@useobject{currentmarker}{}%
\end{pgfscope}%
\begin{pgfscope}%
\pgfsys@transformshift{2.576570in}{1.087385in}%
\pgfsys@useobject{currentmarker}{}%
\end{pgfscope}%
\begin{pgfscope}%
\pgfsys@transformshift{2.626007in}{1.074551in}%
\pgfsys@useobject{currentmarker}{}%
\end{pgfscope}%
\begin{pgfscope}%
\pgfsys@transformshift{2.675444in}{1.073640in}%
\pgfsys@useobject{currentmarker}{}%
\end{pgfscope}%
\begin{pgfscope}%
\pgfsys@transformshift{2.724881in}{1.072829in}%
\pgfsys@useobject{currentmarker}{}%
\end{pgfscope}%
\begin{pgfscope}%
\pgfsys@transformshift{2.774318in}{1.070032in}%
\pgfsys@useobject{currentmarker}{}%
\end{pgfscope}%
\begin{pgfscope}%
\pgfsys@transformshift{2.823755in}{1.057909in}%
\pgfsys@useobject{currentmarker}{}%
\end{pgfscope}%
\begin{pgfscope}%
\pgfsys@transformshift{2.873191in}{1.034187in}%
\pgfsys@useobject{currentmarker}{}%
\end{pgfscope}%
\begin{pgfscope}%
\pgfsys@transformshift{2.922628in}{1.032996in}%
\pgfsys@useobject{currentmarker}{}%
\end{pgfscope}%
\begin{pgfscope}%
\pgfsys@transformshift{2.972065in}{1.031712in}%
\pgfsys@useobject{currentmarker}{}%
\end{pgfscope}%
\begin{pgfscope}%
\pgfsys@transformshift{3.021502in}{1.030259in}%
\pgfsys@useobject{currentmarker}{}%
\end{pgfscope}%
\begin{pgfscope}%
\pgfsys@transformshift{3.070939in}{1.022571in}%
\pgfsys@useobject{currentmarker}{}%
\end{pgfscope}%
\begin{pgfscope}%
\pgfsys@transformshift{3.120376in}{1.014413in}%
\pgfsys@useobject{currentmarker}{}%
\end{pgfscope}%
\begin{pgfscope}%
\pgfsys@transformshift{3.169812in}{0.997046in}%
\pgfsys@useobject{currentmarker}{}%
\end{pgfscope}%
\begin{pgfscope}%
\pgfsys@transformshift{3.219249in}{0.995683in}%
\pgfsys@useobject{currentmarker}{}%
\end{pgfscope}%
\begin{pgfscope}%
\pgfsys@transformshift{3.268686in}{0.989487in}%
\pgfsys@useobject{currentmarker}{}%
\end{pgfscope}%
\begin{pgfscope}%
\pgfsys@transformshift{3.318123in}{0.986185in}%
\pgfsys@useobject{currentmarker}{}%
\end{pgfscope}%
\begin{pgfscope}%
\pgfsys@transformshift{3.367560in}{0.969810in}%
\pgfsys@useobject{currentmarker}{}%
\end{pgfscope}%
\begin{pgfscope}%
\pgfsys@transformshift{3.416997in}{0.945653in}%
\pgfsys@useobject{currentmarker}{}%
\end{pgfscope}%
\begin{pgfscope}%
\pgfsys@transformshift{3.466434in}{0.939657in}%
\pgfsys@useobject{currentmarker}{}%
\end{pgfscope}%
\begin{pgfscope}%
\pgfsys@transformshift{3.515870in}{0.891784in}%
\pgfsys@useobject{currentmarker}{}%
\end{pgfscope}%
\begin{pgfscope}%
\pgfsys@transformshift{3.565307in}{0.851366in}%
\pgfsys@useobject{currentmarker}{}%
\end{pgfscope}%
\begin{pgfscope}%
\pgfsys@transformshift{3.614744in}{0.830193in}%
\pgfsys@useobject{currentmarker}{}%
\end{pgfscope}%
\begin{pgfscope}%
\pgfsys@transformshift{3.664181in}{0.815059in}%
\pgfsys@useobject{currentmarker}{}%
\end{pgfscope}%
\begin{pgfscope}%
\pgfsys@transformshift{3.713618in}{0.767334in}%
\pgfsys@useobject{currentmarker}{}%
\end{pgfscope}%
\begin{pgfscope}%
\pgfsys@transformshift{3.763055in}{0.766052in}%
\pgfsys@useobject{currentmarker}{}%
\end{pgfscope}%
\begin{pgfscope}%
\pgfsys@transformshift{3.812492in}{0.765349in}%
\pgfsys@useobject{currentmarker}{}%
\end{pgfscope}%
\begin{pgfscope}%
\pgfsys@transformshift{3.861928in}{0.739039in}%
\pgfsys@useobject{currentmarker}{}%
\end{pgfscope}%
\begin{pgfscope}%
\pgfsys@transformshift{3.911365in}{0.735283in}%
\pgfsys@useobject{currentmarker}{}%
\end{pgfscope}%
\begin{pgfscope}%
\pgfsys@transformshift{3.960802in}{0.707889in}%
\pgfsys@useobject{currentmarker}{}%
\end{pgfscope}%
\begin{pgfscope}%
\pgfsys@transformshift{4.010239in}{0.707537in}%
\pgfsys@useobject{currentmarker}{}%
\end{pgfscope}%
\begin{pgfscope}%
\pgfsys@transformshift{4.059676in}{0.697076in}%
\pgfsys@useobject{currentmarker}{}%
\end{pgfscope}%
\begin{pgfscope}%
\pgfsys@transformshift{4.109113in}{0.648368in}%
\pgfsys@useobject{currentmarker}{}%
\end{pgfscope}%
\begin{pgfscope}%
\pgfsys@transformshift{4.158550in}{0.608161in}%
\pgfsys@useobject{currentmarker}{}%
\end{pgfscope}%
\begin{pgfscope}%
\pgfsys@transformshift{4.207986in}{0.600280in}%
\pgfsys@useobject{currentmarker}{}%
\end{pgfscope}%
\begin{pgfscope}%
\pgfsys@transformshift{4.257423in}{0.559040in}%
\pgfsys@useobject{currentmarker}{}%
\end{pgfscope}%
\begin{pgfscope}%
\pgfsys@transformshift{4.306860in}{0.552815in}%
\pgfsys@useobject{currentmarker}{}%
\end{pgfscope}%
\begin{pgfscope}%
\pgfsys@transformshift{4.356297in}{0.516678in}%
\pgfsys@useobject{currentmarker}{}%
\end{pgfscope}%
\begin{pgfscope}%
\pgfsys@transformshift{4.405734in}{0.501073in}%
\pgfsys@useobject{currentmarker}{}%
\end{pgfscope}%
\begin{pgfscope}%
\pgfsys@transformshift{4.455171in}{0.496392in}%
\pgfsys@useobject{currentmarker}{}%
\end{pgfscope}%
\begin{pgfscope}%
\pgfsys@transformshift{4.504607in}{0.495704in}%
\pgfsys@useobject{currentmarker}{}%
\end{pgfscope}%
\begin{pgfscope}%
\pgfsys@transformshift{4.554044in}{0.483184in}%
\pgfsys@useobject{currentmarker}{}%
\end{pgfscope}%
\begin{pgfscope}%
\pgfsys@transformshift{4.603481in}{0.460676in}%
\pgfsys@useobject{currentmarker}{}%
\end{pgfscope}%
\begin{pgfscope}%
\pgfsys@transformshift{4.652918in}{0.460676in}%
\pgfsys@useobject{currentmarker}{}%
\end{pgfscope}%
\begin{pgfscope}%
\pgfsys@transformshift{4.702355in}{0.460676in}%
\pgfsys@useobject{currentmarker}{}%
\end{pgfscope}%
\begin{pgfscope}%
\pgfsys@transformshift{4.751792in}{0.460676in}%
\pgfsys@useobject{currentmarker}{}%
\end{pgfscope}%
\begin{pgfscope}%
\pgfsys@transformshift{4.801229in}{0.460676in}%
\pgfsys@useobject{currentmarker}{}%
\end{pgfscope}%
\begin{pgfscope}%
\pgfsys@transformshift{4.850665in}{0.460676in}%
\pgfsys@useobject{currentmarker}{}%
\end{pgfscope}%
\begin{pgfscope}%
\pgfsys@transformshift{4.900102in}{0.460676in}%
\pgfsys@useobject{currentmarker}{}%
\end{pgfscope}%
\begin{pgfscope}%
\pgfsys@transformshift{4.949539in}{0.460676in}%
\pgfsys@useobject{currentmarker}{}%
\end{pgfscope}%
\begin{pgfscope}%
\pgfsys@transformshift{4.998976in}{0.460676in}%
\pgfsys@useobject{currentmarker}{}%
\end{pgfscope}%
\begin{pgfscope}%
\pgfsys@transformshift{5.048413in}{0.460676in}%
\pgfsys@useobject{currentmarker}{}%
\end{pgfscope}%
\begin{pgfscope}%
\pgfsys@transformshift{5.097850in}{0.460676in}%
\pgfsys@useobject{currentmarker}{}%
\end{pgfscope}%
\end{pgfscope}%
\begin{pgfscope}%
\pgfpathrectangle{\pgfqpoint{0.581793in}{0.460676in}}{\pgfqpoint{4.731107in}{2.134248in}}%
\pgfusepath{clip}%
\pgfsetrectcap%
\pgfsetroundjoin%
\pgfsetlinewidth{1.003750pt}%
\definecolor{currentstroke}{rgb}{0.000000,0.000000,0.000000}%
\pgfsetstrokecolor{currentstroke}%
\pgfsetdash{}{0pt}%
\pgfpathmoveto{\pgfqpoint{0.581793in}{2.364397in}}%
\pgfpathlineto{\pgfqpoint{5.312900in}{2.364397in}}%
\pgfusepath{stroke}%
\end{pgfscope}%
\begin{pgfscope}%
\pgfsetrectcap%
\pgfsetmiterjoin%
\pgfsetlinewidth{0.803000pt}%
\definecolor{currentstroke}{rgb}{0.000000,0.000000,0.000000}%
\pgfsetstrokecolor{currentstroke}%
\pgfsetdash{}{0pt}%
\pgfpathmoveto{\pgfqpoint{0.581793in}{0.460676in}}%
\pgfpathlineto{\pgfqpoint{0.581793in}{2.594924in}}%
\pgfusepath{stroke}%
\end{pgfscope}%
\begin{pgfscope}%
\pgfsetrectcap%
\pgfsetmiterjoin%
\pgfsetlinewidth{0.803000pt}%
\definecolor{currentstroke}{rgb}{0.000000,0.000000,0.000000}%
\pgfsetstrokecolor{currentstroke}%
\pgfsetdash{}{0pt}%
\pgfpathmoveto{\pgfqpoint{5.312900in}{0.460676in}}%
\pgfpathlineto{\pgfqpoint{5.312900in}{2.594924in}}%
\pgfusepath{stroke}%
\end{pgfscope}%
\begin{pgfscope}%
\pgfsetrectcap%
\pgfsetmiterjoin%
\pgfsetlinewidth{0.803000pt}%
\definecolor{currentstroke}{rgb}{0.000000,0.000000,0.000000}%
\pgfsetstrokecolor{currentstroke}%
\pgfsetdash{}{0pt}%
\pgfpathmoveto{\pgfqpoint{0.581793in}{0.460676in}}%
\pgfpathlineto{\pgfqpoint{5.312900in}{0.460676in}}%
\pgfusepath{stroke}%
\end{pgfscope}%
\begin{pgfscope}%
\pgfsetrectcap%
\pgfsetmiterjoin%
\pgfsetlinewidth{0.803000pt}%
\definecolor{currentstroke}{rgb}{0.000000,0.000000,0.000000}%
\pgfsetstrokecolor{currentstroke}%
\pgfsetdash{}{0pt}%
\pgfpathmoveto{\pgfqpoint{0.581793in}{2.594924in}}%
\pgfpathlineto{\pgfqpoint{5.312900in}{2.594924in}}%
\pgfusepath{stroke}%
\end{pgfscope}%
\begin{pgfscope}%
\pgfsetbuttcap%
\pgfsetroundjoin%
\definecolor{currentfill}{rgb}{0.000000,0.000000,0.000000}%
\pgfsetfillcolor{currentfill}%
\pgfsetlinewidth{0.803000pt}%
\definecolor{currentstroke}{rgb}{0.000000,0.000000,0.000000}%
\pgfsetstrokecolor{currentstroke}%
\pgfsetdash{}{0pt}%
\pgfsys@defobject{currentmarker}{\pgfqpoint{0.000000in}{-0.048611in}}{\pgfqpoint{0.000000in}{0.000000in}}{%
\pgfpathmoveto{\pgfqpoint{0.000000in}{0.000000in}}%
\pgfpathlineto{\pgfqpoint{0.000000in}{-0.048611in}}%
\pgfusepath{stroke,fill}%
}%
\begin{pgfscope}%
\pgfsys@transformshift{0.747407in}{0.460676in}%
\pgfsys@useobject{currentmarker}{}%
\end{pgfscope}%
\end{pgfscope}%
\begin{pgfscope}%
\definecolor{textcolor}{rgb}{0.000000,0.000000,0.000000}%
\pgfsetstrokecolor{textcolor}%
\pgfsetfillcolor{textcolor}%
\pgftext[x=0.747407in,y=0.363454in,,top]{\color{textcolor}\rmfamily\fontsize{8.330000}{9.996000}\selectfont \(\displaystyle 0\)}%
\end{pgfscope}%
\begin{pgfscope}%
\pgfsetbuttcap%
\pgfsetroundjoin%
\definecolor{currentfill}{rgb}{0.000000,0.000000,0.000000}%
\pgfsetfillcolor{currentfill}%
\pgfsetlinewidth{0.803000pt}%
\definecolor{currentstroke}{rgb}{0.000000,0.000000,0.000000}%
\pgfsetstrokecolor{currentstroke}%
\pgfsetdash{}{0pt}%
\pgfsys@defobject{currentmarker}{\pgfqpoint{0.000000in}{-0.048611in}}{\pgfqpoint{0.000000in}{0.000000in}}{%
\pgfpathmoveto{\pgfqpoint{0.000000in}{0.000000in}}%
\pgfpathlineto{\pgfqpoint{0.000000in}{-0.048611in}}%
\pgfusepath{stroke,fill}%
}%
\begin{pgfscope}%
\pgfsys@transformshift{1.736144in}{0.460676in}%
\pgfsys@useobject{currentmarker}{}%
\end{pgfscope}%
\end{pgfscope}%
\begin{pgfscope}%
\definecolor{textcolor}{rgb}{0.000000,0.000000,0.000000}%
\pgfsetstrokecolor{textcolor}%
\pgfsetfillcolor{textcolor}%
\pgftext[x=1.736144in,y=0.363454in,,top]{\color{textcolor}\rmfamily\fontsize{8.330000}{9.996000}\selectfont \(\displaystyle 20\)}%
\end{pgfscope}%
\begin{pgfscope}%
\pgfsetbuttcap%
\pgfsetroundjoin%
\definecolor{currentfill}{rgb}{0.000000,0.000000,0.000000}%
\pgfsetfillcolor{currentfill}%
\pgfsetlinewidth{0.803000pt}%
\definecolor{currentstroke}{rgb}{0.000000,0.000000,0.000000}%
\pgfsetstrokecolor{currentstroke}%
\pgfsetdash{}{0pt}%
\pgfsys@defobject{currentmarker}{\pgfqpoint{0.000000in}{-0.048611in}}{\pgfqpoint{0.000000in}{0.000000in}}{%
\pgfpathmoveto{\pgfqpoint{0.000000in}{0.000000in}}%
\pgfpathlineto{\pgfqpoint{0.000000in}{-0.048611in}}%
\pgfusepath{stroke,fill}%
}%
\begin{pgfscope}%
\pgfsys@transformshift{2.724881in}{0.460676in}%
\pgfsys@useobject{currentmarker}{}%
\end{pgfscope}%
\end{pgfscope}%
\begin{pgfscope}%
\definecolor{textcolor}{rgb}{0.000000,0.000000,0.000000}%
\pgfsetstrokecolor{textcolor}%
\pgfsetfillcolor{textcolor}%
\pgftext[x=2.724881in,y=0.363454in,,top]{\color{textcolor}\rmfamily\fontsize{8.330000}{9.996000}\selectfont \(\displaystyle 40\)}%
\end{pgfscope}%
\begin{pgfscope}%
\pgfsetbuttcap%
\pgfsetroundjoin%
\definecolor{currentfill}{rgb}{0.000000,0.000000,0.000000}%
\pgfsetfillcolor{currentfill}%
\pgfsetlinewidth{0.803000pt}%
\definecolor{currentstroke}{rgb}{0.000000,0.000000,0.000000}%
\pgfsetstrokecolor{currentstroke}%
\pgfsetdash{}{0pt}%
\pgfsys@defobject{currentmarker}{\pgfqpoint{0.000000in}{-0.048611in}}{\pgfqpoint{0.000000in}{0.000000in}}{%
\pgfpathmoveto{\pgfqpoint{0.000000in}{0.000000in}}%
\pgfpathlineto{\pgfqpoint{0.000000in}{-0.048611in}}%
\pgfusepath{stroke,fill}%
}%
\begin{pgfscope}%
\pgfsys@transformshift{3.713618in}{0.460676in}%
\pgfsys@useobject{currentmarker}{}%
\end{pgfscope}%
\end{pgfscope}%
\begin{pgfscope}%
\definecolor{textcolor}{rgb}{0.000000,0.000000,0.000000}%
\pgfsetstrokecolor{textcolor}%
\pgfsetfillcolor{textcolor}%
\pgftext[x=3.713618in,y=0.363454in,,top]{\color{textcolor}\rmfamily\fontsize{8.330000}{9.996000}\selectfont \(\displaystyle 60\)}%
\end{pgfscope}%
\begin{pgfscope}%
\pgfsetbuttcap%
\pgfsetroundjoin%
\definecolor{currentfill}{rgb}{0.000000,0.000000,0.000000}%
\pgfsetfillcolor{currentfill}%
\pgfsetlinewidth{0.803000pt}%
\definecolor{currentstroke}{rgb}{0.000000,0.000000,0.000000}%
\pgfsetstrokecolor{currentstroke}%
\pgfsetdash{}{0pt}%
\pgfsys@defobject{currentmarker}{\pgfqpoint{0.000000in}{-0.048611in}}{\pgfqpoint{0.000000in}{0.000000in}}{%
\pgfpathmoveto{\pgfqpoint{0.000000in}{0.000000in}}%
\pgfpathlineto{\pgfqpoint{0.000000in}{-0.048611in}}%
\pgfusepath{stroke,fill}%
}%
\begin{pgfscope}%
\pgfsys@transformshift{4.702355in}{0.460676in}%
\pgfsys@useobject{currentmarker}{}%
\end{pgfscope}%
\end{pgfscope}%
\begin{pgfscope}%
\definecolor{textcolor}{rgb}{0.000000,0.000000,0.000000}%
\pgfsetstrokecolor{textcolor}%
\pgfsetfillcolor{textcolor}%
\pgftext[x=4.702355in,y=0.363454in,,top]{\color{textcolor}\rmfamily\fontsize{8.330000}{9.996000}\selectfont \(\displaystyle 80\)}%
\end{pgfscope}%
\begin{pgfscope}%
\definecolor{textcolor}{rgb}{0.000000,0.000000,0.000000}%
\pgfsetstrokecolor{textcolor}%
\pgfsetfillcolor{textcolor}%
\pgftext[x=2.947347in,y=0.203949in,,top]{\color{textcolor}\rmfamily\fontsize{8.330000}{9.996000}\selectfont Instances}%
\end{pgfscope}%
\begin{pgfscope}%
\pgfsetbuttcap%
\pgfsetroundjoin%
\definecolor{currentfill}{rgb}{0.000000,0.000000,0.000000}%
\pgfsetfillcolor{currentfill}%
\pgfsetlinewidth{0.803000pt}%
\definecolor{currentstroke}{rgb}{0.000000,0.000000,0.000000}%
\pgfsetstrokecolor{currentstroke}%
\pgfsetdash{}{0pt}%
\pgfsys@defobject{currentmarker}{\pgfqpoint{-0.048611in}{0.000000in}}{\pgfqpoint{0.000000in}{0.000000in}}{%
\pgfpathmoveto{\pgfqpoint{0.000000in}{0.000000in}}%
\pgfpathlineto{\pgfqpoint{-0.048611in}{0.000000in}}%
\pgfusepath{stroke,fill}%
}%
\begin{pgfscope}%
\pgfsys@transformshift{0.581793in}{0.460676in}%
\pgfsys@useobject{currentmarker}{}%
\end{pgfscope}%
\end{pgfscope}%
\begin{pgfscope}%
\definecolor{textcolor}{rgb}{0.000000,0.000000,0.000000}%
\pgfsetstrokecolor{textcolor}%
\pgfsetfillcolor{textcolor}%
\pgftext[x=0.274692in, y=0.421225in, left, base]{\color{textcolor}\rmfamily\fontsize{8.330000}{9.996000}\selectfont \(\displaystyle 0.00\)}%
\end{pgfscope}%
\begin{pgfscope}%
\pgfsetbuttcap%
\pgfsetroundjoin%
\definecolor{currentfill}{rgb}{0.000000,0.000000,0.000000}%
\pgfsetfillcolor{currentfill}%
\pgfsetlinewidth{0.803000pt}%
\definecolor{currentstroke}{rgb}{0.000000,0.000000,0.000000}%
\pgfsetstrokecolor{currentstroke}%
\pgfsetdash{}{0pt}%
\pgfsys@defobject{currentmarker}{\pgfqpoint{-0.048611in}{0.000000in}}{\pgfqpoint{0.000000in}{0.000000in}}{%
\pgfpathmoveto{\pgfqpoint{0.000000in}{0.000000in}}%
\pgfpathlineto{\pgfqpoint{-0.048611in}{0.000000in}}%
\pgfusepath{stroke,fill}%
}%
\begin{pgfscope}%
\pgfsys@transformshift{0.581793in}{0.651048in}%
\pgfsys@useobject{currentmarker}{}%
\end{pgfscope}%
\end{pgfscope}%
\begin{pgfscope}%
\definecolor{textcolor}{rgb}{0.000000,0.000000,0.000000}%
\pgfsetstrokecolor{textcolor}%
\pgfsetfillcolor{textcolor}%
\pgftext[x=0.274692in, y=0.611597in, left, base]{\color{textcolor}\rmfamily\fontsize{8.330000}{9.996000}\selectfont \(\displaystyle 0.01\)}%
\end{pgfscope}%
\begin{pgfscope}%
\pgfsetbuttcap%
\pgfsetroundjoin%
\definecolor{currentfill}{rgb}{0.000000,0.000000,0.000000}%
\pgfsetfillcolor{currentfill}%
\pgfsetlinewidth{0.803000pt}%
\definecolor{currentstroke}{rgb}{0.000000,0.000000,0.000000}%
\pgfsetstrokecolor{currentstroke}%
\pgfsetdash{}{0pt}%
\pgfsys@defobject{currentmarker}{\pgfqpoint{-0.048611in}{0.000000in}}{\pgfqpoint{0.000000in}{0.000000in}}{%
\pgfpathmoveto{\pgfqpoint{0.000000in}{0.000000in}}%
\pgfpathlineto{\pgfqpoint{-0.048611in}{0.000000in}}%
\pgfusepath{stroke,fill}%
}%
\begin{pgfscope}%
\pgfsys@transformshift{0.581793in}{0.886361in}%
\pgfsys@useobject{currentmarker}{}%
\end{pgfscope}%
\end{pgfscope}%
\begin{pgfscope}%
\definecolor{textcolor}{rgb}{0.000000,0.000000,0.000000}%
\pgfsetstrokecolor{textcolor}%
\pgfsetfillcolor{textcolor}%
\pgftext[x=0.274692in, y=0.846910in, left, base]{\color{textcolor}\rmfamily\fontsize{8.330000}{9.996000}\selectfont \(\displaystyle 0.05\)}%
\end{pgfscope}%
\begin{pgfscope}%
\pgfsetbuttcap%
\pgfsetroundjoin%
\definecolor{currentfill}{rgb}{0.000000,0.000000,0.000000}%
\pgfsetfillcolor{currentfill}%
\pgfsetlinewidth{0.803000pt}%
\definecolor{currentstroke}{rgb}{0.000000,0.000000,0.000000}%
\pgfsetstrokecolor{currentstroke}%
\pgfsetdash{}{0pt}%
\pgfsys@defobject{currentmarker}{\pgfqpoint{-0.048611in}{0.000000in}}{\pgfqpoint{0.000000in}{0.000000in}}{%
\pgfpathmoveto{\pgfqpoint{0.000000in}{0.000000in}}%
\pgfpathlineto{\pgfqpoint{-0.048611in}{0.000000in}}%
\pgfusepath{stroke,fill}%
}%
\begin{pgfscope}%
\pgfsys@transformshift{0.581793in}{1.062685in}%
\pgfsys@useobject{currentmarker}{}%
\end{pgfscope}%
\end{pgfscope}%
\begin{pgfscope}%
\definecolor{textcolor}{rgb}{0.000000,0.000000,0.000000}%
\pgfsetstrokecolor{textcolor}%
\pgfsetfillcolor{textcolor}%
\pgftext[x=0.274692in, y=1.023234in, left, base]{\color{textcolor}\rmfamily\fontsize{8.330000}{9.996000}\selectfont \(\displaystyle 0.10\)}%
\end{pgfscope}%
\begin{pgfscope}%
\pgfsetbuttcap%
\pgfsetroundjoin%
\definecolor{currentfill}{rgb}{0.000000,0.000000,0.000000}%
\pgfsetfillcolor{currentfill}%
\pgfsetlinewidth{0.803000pt}%
\definecolor{currentstroke}{rgb}{0.000000,0.000000,0.000000}%
\pgfsetstrokecolor{currentstroke}%
\pgfsetdash{}{0pt}%
\pgfsys@defobject{currentmarker}{\pgfqpoint{-0.048611in}{0.000000in}}{\pgfqpoint{0.000000in}{0.000000in}}{%
\pgfpathmoveto{\pgfqpoint{0.000000in}{0.000000in}}%
\pgfpathlineto{\pgfqpoint{-0.048611in}{0.000000in}}%
\pgfusepath{stroke,fill}%
}%
\begin{pgfscope}%
\pgfsys@transformshift{0.581793in}{1.312046in}%
\pgfsys@useobject{currentmarker}{}%
\end{pgfscope}%
\end{pgfscope}%
\begin{pgfscope}%
\definecolor{textcolor}{rgb}{0.000000,0.000000,0.000000}%
\pgfsetstrokecolor{textcolor}%
\pgfsetfillcolor{textcolor}%
\pgftext[x=0.274692in, y=1.272595in, left, base]{\color{textcolor}\rmfamily\fontsize{8.330000}{9.996000}\selectfont \(\displaystyle 0.20\)}%
\end{pgfscope}%
\begin{pgfscope}%
\pgfsetbuttcap%
\pgfsetroundjoin%
\definecolor{currentfill}{rgb}{0.000000,0.000000,0.000000}%
\pgfsetfillcolor{currentfill}%
\pgfsetlinewidth{0.803000pt}%
\definecolor{currentstroke}{rgb}{0.000000,0.000000,0.000000}%
\pgfsetstrokecolor{currentstroke}%
\pgfsetdash{}{0pt}%
\pgfsys@defobject{currentmarker}{\pgfqpoint{-0.048611in}{0.000000in}}{\pgfqpoint{0.000000in}{0.000000in}}{%
\pgfpathmoveto{\pgfqpoint{0.000000in}{0.000000in}}%
\pgfpathlineto{\pgfqpoint{-0.048611in}{0.000000in}}%
\pgfusepath{stroke,fill}%
}%
\begin{pgfscope}%
\pgfsys@transformshift{0.581793in}{1.664695in}%
\pgfsys@useobject{currentmarker}{}%
\end{pgfscope}%
\end{pgfscope}%
\begin{pgfscope}%
\definecolor{textcolor}{rgb}{0.000000,0.000000,0.000000}%
\pgfsetstrokecolor{textcolor}%
\pgfsetfillcolor{textcolor}%
\pgftext[x=0.274692in, y=1.625243in, left, base]{\color{textcolor}\rmfamily\fontsize{8.330000}{9.996000}\selectfont \(\displaystyle 0.40\)}%
\end{pgfscope}%
\begin{pgfscope}%
\pgfsetbuttcap%
\pgfsetroundjoin%
\definecolor{currentfill}{rgb}{0.000000,0.000000,0.000000}%
\pgfsetfillcolor{currentfill}%
\pgfsetlinewidth{0.803000pt}%
\definecolor{currentstroke}{rgb}{0.000000,0.000000,0.000000}%
\pgfsetstrokecolor{currentstroke}%
\pgfsetdash{}{0pt}%
\pgfsys@defobject{currentmarker}{\pgfqpoint{-0.048611in}{0.000000in}}{\pgfqpoint{0.000000in}{0.000000in}}{%
\pgfpathmoveto{\pgfqpoint{0.000000in}{0.000000in}}%
\pgfpathlineto{\pgfqpoint{-0.048611in}{0.000000in}}%
\pgfusepath{stroke,fill}%
}%
\begin{pgfscope}%
\pgfsys@transformshift{0.581793in}{1.935292in}%
\pgfsys@useobject{currentmarker}{}%
\end{pgfscope}%
\end{pgfscope}%
\begin{pgfscope}%
\definecolor{textcolor}{rgb}{0.000000,0.000000,0.000000}%
\pgfsetstrokecolor{textcolor}%
\pgfsetfillcolor{textcolor}%
\pgftext[x=0.274692in, y=1.895840in, left, base]{\color{textcolor}\rmfamily\fontsize{8.330000}{9.996000}\selectfont \(\displaystyle 0.60\)}%
\end{pgfscope}%
\begin{pgfscope}%
\pgfsetbuttcap%
\pgfsetroundjoin%
\definecolor{currentfill}{rgb}{0.000000,0.000000,0.000000}%
\pgfsetfillcolor{currentfill}%
\pgfsetlinewidth{0.803000pt}%
\definecolor{currentstroke}{rgb}{0.000000,0.000000,0.000000}%
\pgfsetstrokecolor{currentstroke}%
\pgfsetdash{}{0pt}%
\pgfsys@defobject{currentmarker}{\pgfqpoint{-0.048611in}{0.000000in}}{\pgfqpoint{0.000000in}{0.000000in}}{%
\pgfpathmoveto{\pgfqpoint{0.000000in}{0.000000in}}%
\pgfpathlineto{\pgfqpoint{-0.048611in}{0.000000in}}%
\pgfusepath{stroke,fill}%
}%
\begin{pgfscope}%
\pgfsys@transformshift{0.581793in}{2.163415in}%
\pgfsys@useobject{currentmarker}{}%
\end{pgfscope}%
\end{pgfscope}%
\begin{pgfscope}%
\definecolor{textcolor}{rgb}{0.000000,0.000000,0.000000}%
\pgfsetstrokecolor{textcolor}%
\pgfsetfillcolor{textcolor}%
\pgftext[x=0.274692in, y=2.123964in, left, base]{\color{textcolor}\rmfamily\fontsize{8.330000}{9.996000}\selectfont \(\displaystyle 0.80\)}%
\end{pgfscope}%
\begin{pgfscope}%
\pgfsetbuttcap%
\pgfsetroundjoin%
\definecolor{currentfill}{rgb}{0.000000,0.000000,0.000000}%
\pgfsetfillcolor{currentfill}%
\pgfsetlinewidth{0.803000pt}%
\definecolor{currentstroke}{rgb}{0.000000,0.000000,0.000000}%
\pgfsetstrokecolor{currentstroke}%
\pgfsetdash{}{0pt}%
\pgfsys@defobject{currentmarker}{\pgfqpoint{-0.048611in}{0.000000in}}{\pgfqpoint{0.000000in}{0.000000in}}{%
\pgfpathmoveto{\pgfqpoint{0.000000in}{0.000000in}}%
\pgfpathlineto{\pgfqpoint{-0.048611in}{0.000000in}}%
\pgfusepath{stroke,fill}%
}%
\begin{pgfscope}%
\pgfsys@transformshift{0.581793in}{2.364397in}%
\pgfsys@useobject{currentmarker}{}%
\end{pgfscope}%
\end{pgfscope}%
\begin{pgfscope}%
\definecolor{textcolor}{rgb}{0.000000,0.000000,0.000000}%
\pgfsetstrokecolor{textcolor}%
\pgfsetfillcolor{textcolor}%
\pgftext[x=0.274692in, y=2.324945in, left, base]{\color{textcolor}\rmfamily\fontsize{8.330000}{9.996000}\selectfont \(\displaystyle 1.00\)}%
\end{pgfscope}%
\begin{pgfscope}%
\definecolor{textcolor}{rgb}{0.000000,0.000000,0.000000}%
\pgfsetstrokecolor{textcolor}%
\pgfsetfillcolor{textcolor}%
\pgftext[x=0.219136in,y=1.527800in,,bottom,rotate=90.000000]{\color{textcolor}\rmfamily\fontsize{8.330000}{9.996000}\selectfont \(\displaystyle 1-(b_{\text{best}}/b_{\text{alg}})\)}%
\end{pgfscope}%
\begin{pgfscope}%
\definecolor{textcolor}{rgb}{0.000000,0.000000,0.000000}%
\pgfsetstrokecolor{textcolor}%
\pgfsetfillcolor{textcolor}%
\pgftext[x=2.947347in,y=2.678257in,,base]{\color{textcolor}\rmfamily\fontsize{8.330000}{9.996000}\selectfont all instances - number of branches}%
\end{pgfscope}%
\begin{pgfscope}%
\pgfsetbuttcap%
\pgfsetroundjoin%
\definecolor{currentfill}{rgb}{0.121569,0.466667,0.705882}%
\pgfsetfillcolor{currentfill}%
\pgfsetlinewidth{1.003750pt}%
\definecolor{currentstroke}{rgb}{0.121569,0.466667,0.705882}%
\pgfsetstrokecolor{currentstroke}%
\pgfsetdash{}{0pt}%
\pgfsys@defobject{currentmarker}{\pgfqpoint{-0.028527in}{-0.028527in}}{\pgfqpoint{0.028527in}{0.028527in}}{%
\pgfpathmoveto{\pgfqpoint{0.000000in}{-0.028527in}}%
\pgfpathcurveto{\pgfqpoint{0.007566in}{-0.028527in}}{\pgfqpoint{0.014822in}{-0.025521in}}{\pgfqpoint{0.020172in}{-0.020172in}}%
\pgfpathcurveto{\pgfqpoint{0.025521in}{-0.014822in}}{\pgfqpoint{0.028527in}{-0.007566in}}{\pgfqpoint{0.028527in}{0.000000in}}%
\pgfpathcurveto{\pgfqpoint{0.028527in}{0.007566in}}{\pgfqpoint{0.025521in}{0.014822in}}{\pgfqpoint{0.020172in}{0.020172in}}%
\pgfpathcurveto{\pgfqpoint{0.014822in}{0.025521in}}{\pgfqpoint{0.007566in}{0.028527in}}{\pgfqpoint{0.000000in}{0.028527in}}%
\pgfpathcurveto{\pgfqpoint{-0.007566in}{0.028527in}}{\pgfqpoint{-0.014822in}{0.025521in}}{\pgfqpoint{-0.020172in}{0.020172in}}%
\pgfpathcurveto{\pgfqpoint{-0.025521in}{0.014822in}}{\pgfqpoint{-0.028527in}{0.007566in}}{\pgfqpoint{-0.028527in}{0.000000in}}%
\pgfpathcurveto{\pgfqpoint{-0.028527in}{-0.007566in}}{\pgfqpoint{-0.025521in}{-0.014822in}}{\pgfqpoint{-0.020172in}{-0.020172in}}%
\pgfpathcurveto{\pgfqpoint{-0.014822in}{-0.025521in}}{\pgfqpoint{-0.007566in}{-0.028527in}}{\pgfqpoint{0.000000in}{-0.028527in}}%
\pgfpathclose%
\pgfusepath{stroke,fill}%
}%
\begin{pgfscope}%
\pgfsys@transformshift{3.595646in}{2.284915in}%
\pgfsys@useobject{currentmarker}{}%
\end{pgfscope}%
\end{pgfscope}%
\begin{pgfscope}%
\definecolor{textcolor}{rgb}{0.000000,0.000000,0.000000}%
\pgfsetstrokecolor{textcolor}%
\pgfsetfillcolor{textcolor}%
\pgftext[x=3.722910in,y=2.254546in,left,base]{\color{textcolor}\rmfamily\fontsize{8.330000}{9.996000}\selectfont maximum degree (default)}%
\end{pgfscope}%
\begin{pgfscope}%
\pgfsetbuttcap%
\pgfsetroundjoin%
\definecolor{currentfill}{rgb}{1.000000,0.498039,0.054902}%
\pgfsetfillcolor{currentfill}%
\pgfsetfillopacity{0.750000}%
\pgfsetlinewidth{1.003750pt}%
\definecolor{currentstroke}{rgb}{1.000000,0.498039,0.054902}%
\pgfsetstrokecolor{currentstroke}%
\pgfsetstrokeopacity{0.750000}%
\pgfsetdash{}{0pt}%
\pgfsys@defobject{currentmarker}{\pgfqpoint{-0.028527in}{-0.028527in}}{\pgfqpoint{0.028527in}{0.028527in}}{%
\pgfpathmoveto{\pgfqpoint{-0.000000in}{-0.028527in}}%
\pgfpathlineto{\pgfqpoint{0.028527in}{0.028527in}}%
\pgfpathlineto{\pgfqpoint{-0.028527in}{0.028527in}}%
\pgfpathclose%
\pgfusepath{stroke,fill}%
}%
\begin{pgfscope}%
\pgfsys@transformshift{3.595646in}{2.155743in}%
\pgfsys@useobject{currentmarker}{}%
\end{pgfscope}%
\end{pgfscope}%
\begin{pgfscope}%
\definecolor{textcolor}{rgb}{0.000000,0.000000,0.000000}%
\pgfsetstrokecolor{textcolor}%
\pgfsetfillcolor{textcolor}%
\pgftext[x=3.722910in,y=2.125374in,left,base]{\color{textcolor}\rmfamily\fontsize{8.330000}{9.996000}\selectfont combined}%
\end{pgfscope}%
\begin{pgfscope}%
\pgfsetbuttcap%
\pgfsetroundjoin%
\definecolor{currentfill}{rgb}{0.172549,0.627451,0.172549}%
\pgfsetfillcolor{currentfill}%
\pgfsetfillopacity{0.750000}%
\pgfsetlinewidth{1.003750pt}%
\definecolor{currentstroke}{rgb}{0.172549,0.627451,0.172549}%
\pgfsetstrokecolor{currentstroke}%
\pgfsetstrokeopacity{0.750000}%
\pgfsetdash{}{0pt}%
\pgfsys@defobject{currentmarker}{\pgfqpoint{-0.028527in}{-0.028527in}}{\pgfqpoint{0.028527in}{0.028527in}}{%
\pgfpathmoveto{\pgfqpoint{-0.028527in}{-0.028527in}}%
\pgfpathlineto{\pgfqpoint{0.028527in}{-0.028527in}}%
\pgfpathlineto{\pgfqpoint{0.028527in}{0.028527in}}%
\pgfpathlineto{\pgfqpoint{-0.028527in}{0.028527in}}%
\pgfpathclose%
\pgfusepath{stroke,fill}%
}%
\begin{pgfscope}%
\pgfsys@transformshift{3.595646in}{2.026571in}%
\pgfsys@useobject{currentmarker}{}%
\end{pgfscope}%
\end{pgfscope}%
\begin{pgfscope}%
\definecolor{textcolor}{rgb}{0.000000,0.000000,0.000000}%
\pgfsetstrokecolor{textcolor}%
\pgfsetfillcolor{textcolor}%
\pgftext[x=3.722910in,y=1.996202in,left,base]{\color{textcolor}\rmfamily\fontsize{8.330000}{9.996000}\selectfont twin-reduction-based}%
\end{pgfscope}%
\begin{pgfscope}%
\pgfsetbuttcap%
\pgfsetroundjoin%
\definecolor{currentfill}{rgb}{0.839216,0.152941,0.156863}%
\pgfsetfillcolor{currentfill}%
\pgfsetfillopacity{0.750000}%
\pgfsetlinewidth{1.003750pt}%
\definecolor{currentstroke}{rgb}{0.839216,0.152941,0.156863}%
\pgfsetstrokecolor{currentstroke}%
\pgfsetstrokeopacity{0.750000}%
\pgfsetdash{}{0pt}%
\pgfsys@defobject{currentmarker}{\pgfqpoint{-0.028527in}{-0.028527in}}{\pgfqpoint{0.028527in}{0.028527in}}{%
\pgfpathmoveto{\pgfqpoint{0.000000in}{0.028527in}}%
\pgfpathlineto{\pgfqpoint{-0.028527in}{-0.028527in}}%
\pgfpathlineto{\pgfqpoint{0.028527in}{-0.028527in}}%
\pgfpathclose%
\pgfusepath{stroke,fill}%
}%
\begin{pgfscope}%
\pgfsys@transformshift{3.595646in}{1.897399in}%
\pgfsys@useobject{currentmarker}{}%
\end{pgfscope}%
\end{pgfscope}%
\begin{pgfscope}%
\definecolor{textcolor}{rgb}{0.000000,0.000000,0.000000}%
\pgfsetstrokecolor{textcolor}%
\pgfsetfillcolor{textcolor}%
\pgftext[x=3.722910in,y=1.867030in,left,base]{\color{textcolor}\rmfamily\fontsize{8.330000}{9.996000}\selectfont funnel-reduction-based}%
\end{pgfscope}%
\begin{pgfscope}%
\pgfsetbuttcap%
\pgfsetroundjoin%
\definecolor{currentfill}{rgb}{0.580392,0.403922,0.741176}%
\pgfsetfillcolor{currentfill}%
\pgfsetfillopacity{0.750000}%
\pgfsetlinewidth{1.003750pt}%
\definecolor{currentstroke}{rgb}{0.580392,0.403922,0.741176}%
\pgfsetstrokecolor{currentstroke}%
\pgfsetstrokeopacity{0.750000}%
\pgfsetdash{}{0pt}%
\pgfsys@defobject{currentmarker}{\pgfqpoint{-0.040344in}{-0.040344in}}{\pgfqpoint{0.040344in}{0.040344in}}{%
\pgfpathmoveto{\pgfqpoint{-0.000000in}{-0.040344in}}%
\pgfpathlineto{\pgfqpoint{0.040344in}{0.000000in}}%
\pgfpathlineto{\pgfqpoint{0.000000in}{0.040344in}}%
\pgfpathlineto{\pgfqpoint{-0.040344in}{0.000000in}}%
\pgfpathclose%
\pgfusepath{stroke,fill}%
}%
\begin{pgfscope}%
\pgfsys@transformshift{3.595646in}{1.768228in}%
\pgfsys@useobject{currentmarker}{}%
\end{pgfscope}%
\end{pgfscope}%
\begin{pgfscope}%
\definecolor{textcolor}{rgb}{0.000000,0.000000,0.000000}%
\pgfsetstrokecolor{textcolor}%
\pgfsetfillcolor{textcolor}%
\pgftext[x=3.722910in,y=1.737858in,left,base]{\color{textcolor}\rmfamily\fontsize{8.330000}{9.996000}\selectfont unconfined-reduction-based}%
\end{pgfscope}%
\end{pgfpicture}%
\makeatother%
\endgroup%

	\caption{Performance plot comparing branches of reduction-based branching strategies.}\label{fig:all_reduction_branch}
\end{figure}


\section{Conclusion and Future Work}
In this work we presented several novel branching strategies for the maximum independent set problem.
Our strategies either follow a decomposition-based or reduction-rule-based approach.
The decomposition-based strategies make use of increasingly sophisticated methods of finding vertices that are likely to decompose the graph into two or more connected components.
Even though these strategies often come with a non negligible overhead, they work well for graphs that have a suitable structure, such as social networks.
For instances that still favor the default branching strategy of branching on the vertex of highest degree, our reduction-rule-based strategies provide a small but consistent speedup.
These rules aim to facilitate the application of reduction rules which leads to smaller graphs that can be solved more quickly.

Overall, using one of our proposed strategies allows us to find the optimal solution the fastest for most instances tested.
However, deciding which particular strategy to use for a given instance still remains an open problem.
Finding suitable graph characteristics to do so provides an interesting opportunity for future work.
Furthermore, our experimental evaluation on a combined approach that tries to use all reduction-rule-based strategies at the same time achieves the least number of branches for a large set of instances.
However, the running time of this approach still suffers from frequent checks if a particular vertex is a potential branching vertex.
A more sophisticated and incremental way of tracking when a vertex becomes a branching vertex might provide significant performance benefits.
In turn, this might lead to a branching strategy that consistently outperforms branching on the vertex of highest degree independent of the instance type.

\FloatBarrier
\newpage

%%
%% Bibliography
%%

%% Please use bibtex, 
\todo{TODO: Before submitting check for all references whether they are actually used in the text}
\bibliography{references}

\newpage

\appendix

\section{Detailed Instance Information}
\label{app:instances}
\begin{table}[htb!]	
	\scriptsize
	\caption{Number of vertices $|V|$ and edges $|E|$ for each graph.}\label{table:instance}
	\begin{center}
    \begin{minipage}{0.29\textwidth}
    \centering
		\begin{tabular}{|l|r|r|}
			\hline
          \multicolumn{3}{|l|}{PACE \cite{dzulfikar_et_al:LIPIcs:2019:11486} instances:}                                                            \\
			\hline
			Graph                 & $|V|$              & $|E|$                                               \\
			\hline
			36                    & \numprint{26300}   & \numprint{41500}                                    \\
			37                    & \numprint{198}     & \numprint{808}                                      \\
			38                    & \numprint{786}     & \numprint{14024}                                    \\
			39                    & \numprint{6795}    & \numprint{10620}                                    \\
			40                    & \numprint{210}     & \numprint{625}                                      \\
			41                    & \numprint{200}     & \numprint{1023}                                     \\
			42                    & \numprint{200}     & \numprint{952}                                      \\
			43                    & \numprint{200}     & \numprint{841}                                      \\
			44                    & \numprint{200}     & \numprint{1147}                                     \\
			45                    & \numprint{200}     & \numprint{1020}                                     \\
			46                    & \numprint{200}     & \numprint{812}                                      \\
			47                    & \numprint{200}     & \numprint{1093}                                     \\
			48                    & \numprint{200}     & \numprint{1025}                                     \\
			49                    & \numprint{200}     & \numprint{933}                                      \\
			50                    & \numprint{200}     & \numprint{1025}                                     \\
			51                    & \numprint{200}     & \numprint{1098}                                     \\
			52                    & \numprint{200}     & \numprint{992}                                      \\
			53                    & \numprint{200}     & \numprint{1026}                                     \\
			54                    & \numprint{200}     & \numprint{961}                                      \\
			55                    & \numprint{200}     & \numprint{938}                                      \\
			56                    & \numprint{200}     & \numprint{1089}                                     \\
			57                    & \numprint{200}     & \numprint{1160}                                     \\
			58                    & \numprint{200}     & \numprint{1171}                                     \\
			59                    & \numprint{200}     & \numprint{961}                                      \\
			60                    & \numprint{200}     & \numprint{1118}                                     \\
			61                    & \numprint{200}     & \numprint{931}                                      \\
			62                    & \numprint{199}     & \numprint{1128}                                     \\
			63                    & \numprint{200}     & \numprint{1011}                                     \\
			64                    & \numprint{200}     & \numprint{1042}                                     \\
			65                    & \numprint{200}     & \numprint{1011}                                     \\
			66                    & \numprint{200}     & \numprint{866}                                      \\
			67                    & \numprint{200}     & \numprint{1174}                                     \\
			68                    & \numprint{200}     & \numprint{961}                                      \\
			69                    & \numprint{200}     & \numprint{1083}                                     \\
			70                    & \numprint{200}     & \numprint{860}                                      \\
			71                    & \numprint{200}     & \numprint{952}                                      \\
			72                    & \numprint{200}     & \numprint{1167}                                     \\
			73                    & \numprint{200}     & \numprint{1078}                                     \\
			74                    & \numprint{200}     & \numprint{805}                                      \\
			75                    & \numprint{26300}   & \numprint{41500}                                    \\
			\hline
		\end{tabular}
    \end{minipage}
    \begin{minipage}{0.7\textwidth}
    \centering
		\begin{tabular}{|l|r|r|}
			\hline
          \multicolumn{3}{|l|}{DIMACS \cite{johnson1993cliques} instances:}                                                          \\
			\hline
			Graph                 & $|V|$              & $|E|$                                               \\
			\hline
			C125                  & \numprint{125}     & \numprint{787}                                      \\
			C250                  & \numprint{250}     & \numprint{3141}                                     \\
			MANN\_a27             & \numprint{378}     & \numprint{702}                                      \\
			MANN\_a45             & \numprint{1035}    & \numprint{1980}                                     \\
			brock200\_1           & \numprint{200}     & \numprint{5066}                                     \\
			brock200\_2           & \numprint{200}     & \numprint{10024}                                    \\
			brock200\_3           & \numprint{200}     & \numprint{7852}                                     \\
			brock200\_4           & \numprint{200}     & \numprint{6811}                                     \\
			c-fat200-1            & \numprint{200}     & \numprint{18366}                                    \\
			gen200\_p0            & \numprint{200}     & \numprint{1990}                                     \\
			hamming6-4            & \numprint{64}      & \numprint{1312}                                     \\
			hamming8-4            & \numprint{256}     & \numprint{11776}                                    \\
			johnson16-2-4         & \numprint{120}     & \numprint{1680}                                     \\
			johnson8-4-4          & \numprint{70}      & \numprint{560}                                      \\
			keller4               & \numprint{171}     & \numprint{5100}                                     \\
			p\_hat1000-1          & \numprint{1000}    & \numprint{377247}                                   \\
			p\_hat1000-2          & \numprint{1000}    & \numprint{254701}                                   \\
			p\_hat1500-1          & \numprint{1500}    & \numprint{839327}                                   \\
			p\_hat300-1           & \numprint{300}     & \numprint{33917}                                    \\
			p\_hat300-2           & \numprint{300}     & \numprint{22922}                                    \\
			p\_hat300-3           & \numprint{300}     & \numprint{11460}                                    \\
			p\_hat500-1           & \numprint{500}     & \numprint{93181}                                    \\
			p\_hat500-2           & \numprint{500}     & \numprint{61804}                                    \\
			p\_hat500-3           & \numprint{500}     & \numprint{30950}                                    \\
			p\_hat700-1           & \numprint{700}     & \numprint{183651}                                   \\
			p\_hat700-2           & \numprint{700}     & \numprint{122922}                                   \\
			san1000               & \numprint{1000}    & \numprint{249000}                                   \\
			san200\_0             & \numprint{200}     & \numprint{1990}                                     \\
			san400\_0             & \numprint{400}     & \numprint{23940}                                    \\
			sanr200\_0            & \numprint{200}     & \numprint{6032}                                     \\
			\hline
		\end{tabular}
    \vspace{2em}
    \newline
		\begin{tabular}{|l|r|r|c|}
			\hline			
			\multicolumn{4}{|l|}{Sparse networks:}                                                           \\
			\hline
			Graph                 & $|V|$              & $|E|$               & source                        \\			
			\hline
			as-skitter            & \numprint{1696415} & \numprint{11095298} & \cite{snapnets}               \\
          bay                     & \numprint{321270}  & \numprint{397415}   & \cite{demetrescu2009shortest} \\
			col                   & \numprint{435666}  & \numprint{521200}   & \cite{demetrescu2009shortest} \\
			feather-deezer-social & \numprint{28281}   & \numprint{92752}    & \cite{snapnets}               \\
			fla                   & \numprint{1070376} & \numprint{1343951}  & \cite{demetrescu2009shortest} \\
			libimseti             & \numprint{220970}  & \numprint{17233144} & \cite{nr}                     \\
			musae-twitch\_DE      & \numprint{9498}    & \numprint{153138}   & \cite{snapnets}               \\
			musae-twitch\_FR      & \numprint{6549}    & \numprint{112666}   & \cite{snapnets}               \\
			ny                    & \numprint{264346}  & \numprint{365050}   & \cite{demetrescu2009shortest} \\
			web-BerkStan          & \numprint{685230}  & \numprint{6649470}  & \cite{snapnets}               \\
			web-Google            & \numprint{875713}  & \numprint{4322051}  & \cite{snapnets}               \\
			web-NotreDame         & \numprint{325730}  & \numprint{1090108}  & \cite{snapnets}               \\
			web-Stanford          & \numprint{281903}  & \numprint{1992636}  & \cite{snapnets}               \\
			\hline
		\end{tabular}
    \end{minipage}
		
	\end{center}
\end{table}
\FloatBarrier
% \newpage
\section{Detailed Experimental Results}
\label{app:detailed_results}
We now present detailed results of our experimental evaluation. Detailed tables
show running times $t(s)$ (in seconds) and speedups $s$.
Aggregates average running time $t_{avg}$ in seconds and geometric mean speedup
$s_{geom}$  over maximum degree branching.
We show two types of aggregates: ``all'': all instances (timeouts are assigned a running time of ten hours), and ``solved'': only instances that have been solved by all techniques.
A value is highlighted in bold if its the best one within a row.

\begin{table}[htb!]	
	\scriptsize
	\setlength{\tabcolsep}{2pt}
	\caption{Aggregated results for our decomposition-based strategies.}
	\label{tab:summary_reduction}
	\begin{center}
		\begin{tabular}{|l|r|r|r|r|}\hline
			& max. deg. & \multicolumn{1}{c|}{articulation} & \multicolumn{1}{c|}{edge cuts} & \multicolumn{1}{c|}{nested dis.} \\
			\hline
			$t_{\text{avg}}$ (all) & \numprint{6900.32} & \numprint{6552.92} & \textbf{\numprint{6486.11}} & \numprint{6586.06} \\
			$s_{\text{geom}}$ (all) & \numprint{1.00} & \numprint{1.04} & \textbf{\numprint{1.06}} & \numprint{0.91} \\
			$t_{\text{avg}}$ (solved) & \textbf{\numprint{1166.84}} & \numprint{1193.75} & \numprint{1177.06} & \numprint{1188.98} \\
			$s_{\text{geom}}$ (solved) & \textbf{\numprint{1.00}} & \numprint{0.97} & \numprint{1.00} & \numprint{0.85} \\
			\hline
		\end{tabular}
	\end{center}
	\label{table:another_table}
\end{table}

\begin{table}[htb!]	
	\scriptsize
	\setlength{\tabcolsep}{2pt}
	\caption{Detailed results for our decomposition-based strategies on the PACE instances.}
	\begin{center}
		\begin{tabular}{|l|r|r|r|r|}\hline
			Graph & max. deg. & \multicolumn{1}{c|}{articulation} & \multicolumn{1}{c|}{edge cuts} & \multicolumn{1}{c|}{nested dis.} \\
			\hline
			PACE & t (s) & t (s) & t (s) & t (s) \\
			\hline
			05 & \textbf{\numprint{11.46}} & \numprint{11.84} (\numprint{0.97}) & \numprint{11.72} (\numprint{0.98}) & \numprint{11.94} (\numprint{0.96}) \\
			06 & \textbf{\numprint{5.01}} & \numprint{5.19} (\numprint{0.97}) & \numprint{5.16} (\numprint{0.97}) & \numprint{5.52} (\numprint{0.91}) \\
			10 & \textbf{\numprint{13.02}} & \numprint{13.46} (\numprint{0.97}) & \numprint{13.36} (\numprint{0.97}) & \numprint{13.49} (\numprint{0.97}) \\
			16 & - & - & - & - \\
			19 & \textbf{\numprint{18.17}} & \numprint{18.76} (\numprint{0.97}) & \numprint{18.50} (\numprint{0.98}) & \numprint{18.66} (\numprint{0.97}) \\
			31 & \textbf{\numprint{428.74}} & \numprint{442.20} (\numprint{0.97}) & \numprint{438.92} (\numprint{0.98}) & \numprint{429.11} (\numprint{1.00}) \\
			32 & - & - & - & - \\
			33 & \textbf{\numprint{5.79}} & \numprint{5.97} (\numprint{0.97}) & \numprint{5.93} (\numprint{0.98}) & \numprint{146.05} (\numprint{0.04}) \\
			34 & - & - & - & - \\
			35 & \numprint{44.74} & \numprint{45.72} (\numprint{0.98}) & \numprint{45.54} (\numprint{0.98}) & \textbf{\numprint{43.54} (\numprint{1.03})} \\
			36 & \textbf{\numprint{10.67}} & \numprint{10.81} (\numprint{0.99}) & \numprint{10.72} (\numprint{1.00}) & \numprint{13.02} (\numprint{0.82}) \\
			37 & \textbf{\numprint{60.62}} & \numprint{61.64} (\numprint{0.98}) & \numprint{61.31} (\numprint{0.99}) & \numprint{63.09} (\numprint{0.96}) \\
			38 & \numprint{70.79} & \numprint{64.41} (\numprint{1.10}) & \textbf{\numprint{18.80} (\numprint{3.77})} & \numprint{76.09} (\numprint{0.93}) \\
			39 & \textbf{\numprint{559.85}} & \numprint{567.92} (\numprint{0.99}) & \numprint{563.61} (\numprint{0.99}) & \numprint{578.62} (\numprint{0.97}) \\
			40 & \textbf{\numprint{27278.42}} & \numprint{27624.74} (\numprint{0.99}) & \numprint{27385.12} (\numprint{1.00}) & \numprint{28154.28} (\numprint{0.97}) \\
			41 & \textbf{\numprint{282.83}} & \numprint{288.96} (\numprint{0.98}) & \numprint{286.14} (\numprint{0.99}) & \numprint{292.02} (\numprint{0.97}) \\
			42 & \textbf{\numprint{215.43}} & \numprint{220.05} (\numprint{0.98}) & \numprint{218.00} (\numprint{0.99}) & \numprint{222.76} (\numprint{0.97}) \\
			43 & \textbf{\numprint{1015.97}} & \numprint{1036.92} (\numprint{0.98}) & \numprint{1030.05} (\numprint{0.99}) & \numprint{1048.52} (\numprint{0.97}) \\
			44 & \textbf{\numprint{545.48}} & \numprint{557.95} (\numprint{0.98}) & \numprint{554.30} (\numprint{0.98}) & \numprint{563.41} (\numprint{0.97}) \\
			45 & \textbf{\numprint{147.99}} & \numprint{151.41} (\numprint{0.98}) & \numprint{150.04} (\numprint{0.99}) & \numprint{153.29} (\numprint{0.97}) \\
			46 & \textbf{\numprint{634.70}} & \numprint{643.85} (\numprint{0.99}) & \numprint{640.22} (\numprint{0.99}) & \numprint{655.75} (\numprint{0.97}) \\
			47 & \textbf{\numprint{342.88}} & \numprint{348.28} (\numprint{0.98}) & \numprint{346.03} (\numprint{0.99}) & \numprint{354.46} (\numprint{0.97}) \\
			48 & \textbf{\numprint{147.66}} & \numprint{149.99} (\numprint{0.98}) & \numprint{148.81} (\numprint{0.99}) & \numprint{152.95} (\numprint{0.97}) \\
			49 & \textbf{\numprint{104.01}} & \numprint{105.73} (\numprint{0.98}) & \numprint{105.03} (\numprint{0.99}) & \numprint{107.91} (\numprint{0.96}) \\
			50 & \textbf{\numprint{285.75}} & \numprint{290.25} (\numprint{0.98}) & \numprint{288.28} (\numprint{0.99}) & \numprint{295.53} (\numprint{0.97}) \\
			51 & \textbf{\numprint{330.87}} & \numprint{338.28} (\numprint{0.98}) & \numprint{335.72} (\numprint{0.99}) & \numprint{341.83} (\numprint{0.97}) \\
			52 & \textbf{\numprint{129.84}} & \numprint{132.89} (\numprint{0.98}) & \numprint{131.71} (\numprint{0.99}) & \numprint{134.53} (\numprint{0.97}) \\
			53 & \textbf{\numprint{351.55}} & \numprint{359.29} (\numprint{0.98}) & \numprint{356.87} (\numprint{0.99}) & \numprint{363.08} (\numprint{0.97}) \\
			54 & \textbf{\numprint{189.21}} & \numprint{193.49} (\numprint{0.98}) & \numprint{192.39} (\numprint{0.98}) & \numprint{195.56} (\numprint{0.97}) \\
			55 & \textbf{\numprint{40.08}} & \numprint{41.02} (\numprint{0.98}) & \numprint{40.70} (\numprint{0.98}) & \numprint{41.87} (\numprint{0.96}) \\
			56 & \textbf{\numprint{563.85}} & \numprint{572.37} (\numprint{0.99}) & \numprint{568.09} (\numprint{0.99}) & \numprint{582.32} (\numprint{0.97}) \\
			57 & \textbf{\numprint{391.92}} & \numprint{398.06} (\numprint{0.98}) & \numprint{395.36} (\numprint{0.99}) & \numprint{404.95} (\numprint{0.97}) \\
			58 & \textbf{\numprint{280.57}} & \numprint{285.11} (\numprint{0.98}) & \numprint{283.15} (\numprint{0.99}) & \numprint{290.23} (\numprint{0.97}) \\
			59 & \textbf{\numprint{78.11}} & \numprint{79.44} (\numprint{0.98}) & \numprint{78.90} (\numprint{0.99}) & \numprint{81.13} (\numprint{0.96}) \\
			60 & \textbf{\numprint{466.65}} & \numprint{474.19} (\numprint{0.98}) & \numprint{471.02} (\numprint{0.99}) & \numprint{482.30} (\numprint{0.97}) \\
			61 & \textbf{\numprint{128.58}} & \numprint{131.53} (\numprint{0.98}) & \numprint{130.71} (\numprint{0.98}) & \numprint{133.14} (\numprint{0.97}) \\
			62 & \textbf{\numprint{386.85}} & \numprint{395.27} (\numprint{0.98}) & \numprint{392.36} (\numprint{0.99}) & \numprint{399.27} (\numprint{0.97}) \\
			63 & \textbf{\numprint{400.81}} & \numprint{409.70} (\numprint{0.98}) & \numprint{406.70} (\numprint{0.99}) & \numprint{413.82} (\numprint{0.97}) \\
			64 & \textbf{\numprint{173.83}} & \numprint{177.79} (\numprint{0.98}) & \numprint{176.40} (\numprint{0.99}) & \numprint{179.85} (\numprint{0.97}) \\
			65 & \textbf{\numprint{212.66}} & \numprint{217.21} (\numprint{0.98}) & \numprint{215.30} (\numprint{0.99}) & \numprint{219.83} (\numprint{0.97}) \\
			66 & \textbf{\numprint{47.47}} & \numprint{48.29} (\numprint{0.98}) & \numprint{47.97} (\numprint{0.99}) & \numprint{49.64} (\numprint{0.96}) \\
			67 & \textbf{\numprint{708.92}} & \numprint{718.20} (\numprint{0.99}) & \numprint{713.74} (\numprint{0.99}) & \numprint{732.34} (\numprint{0.97}) \\
			68 & \textbf{\numprint{50.67}} & \numprint{51.40} (\numprint{0.99}) & \numprint{50.94} (\numprint{0.99}) & \numprint{52.87} (\numprint{0.96}) \\
			69 & \textbf{\numprint{252.39}} & \numprint{255.92} (\numprint{0.99}) & \numprint{254.63} (\numprint{0.99}) & \numprint{261.30} (\numprint{0.97}) \\
			70 & \textbf{\numprint{68.24}} & \numprint{69.29} (\numprint{0.98}) & \numprint{68.93} (\numprint{0.99}) & \numprint{71.01} (\numprint{0.96}) \\
			71 & \textbf{\numprint{208.63}} & \numprint{213.05} (\numprint{0.98}) & \numprint{210.80} (\numprint{0.99}) & \numprint{215.74} (\numprint{0.97}) \\
			72 & \textbf{\numprint{269.53}} & \numprint{275.26} (\numprint{0.98}) & \numprint{272.72} (\numprint{0.99}) & \numprint{278.66} (\numprint{0.97}) \\
			73 & \textbf{\numprint{251.79}} & \numprint{257.52} (\numprint{0.98}) & \numprint{255.57} (\numprint{0.99}) & \numprint{260.38} (\numprint{0.97}) \\
			74 & \textbf{\numprint{41.27}} & \numprint{42.22} (\numprint{0.98}) & \numprint{41.96} (\numprint{0.98}) & \numprint{43.06} (\numprint{0.96}) \\
			75 & - & - & - & - \\
			76 & - & - & - & - \\
			77 & \textbf{\numprint{77.46}} & \numprint{79.98} (\numprint{0.97}) & \numprint{79.48} (\numprint{0.97}) & \numprint{77.87} (\numprint{0.99}) \\
			78 & - & - & - & - \\
			79 & - & - & - & - \\
			80 & - & - & - & - \\
			\hline
			$t_{\text{avg}}$ (all) & \textbf{\numprint{5933.30}} & \numprint{5943.32} & \numprint{5936.69} & \numprint{5958.30} \\
			$s_{\text{geom}}$ (all) & \numprint{1.00} & \numprint{0.98} & \textbf{\numprint{1.02}} & \numprint{0.90} \\
			$t_{\text{avg}}$ (solved) & \textbf{\numprint{815.57}} & \numprint{827.29} & \numprint{819.53} & \numprint{844.82} \\
			$s_{\text{geom}}$ (solved) & \numprint{1.00} & \numprint{0.98} & \textbf{\numprint{1.02}} & \numprint{0.90} \\
			
			\hline
		\end{tabular}
	\end{center}
	\label{table:another_table}
\end{table}
\begin{table}
	\scriptsize
	\setlength{\tabcolsep}{2pt}
	\caption{Detailed results for our decomposition-based strategies on the DIMACS instances.}
	\begin{center}
		\begin{tabular}{|l|r|r|r|r|}\hline
			Graph & max. deg. & \multicolumn{1}{c|}{articulation} & \multicolumn{1}{c|}{edge cuts} & \multicolumn{1}{c|}{nested dis.} \\
			\hline
			DIMACS & t (s) & t (s) & t (s) & t (s) \\
			\hline
			C125 & \textbf{\numprint{5.49}} & \numprint{5.62} (\numprint{0.98}) & \numprint{5.68} (\numprint{0.97}) & \numprint{6.03} (\numprint{0.91}) \\
			C250 & - & - & - & - \\
			MANN\_a27 & \textbf{\numprint{3.05}} & \numprint{3.11} (\numprint{0.98}) & \numprint{3.17} (\numprint{0.96}) & \numprint{3.58} (\numprint{0.85}) \\
			MANN\_a45 & \textbf{\numprint{473.07}} & \numprint{480.19} (\numprint{0.99}) & \numprint{488.14} (\numprint{0.97}) & \numprint{479.79} (\numprint{0.99}) \\
			brock200\_1 & \textbf{\numprint{790.54}} & \numprint{805.51} (\numprint{0.98}) & \numprint{803.13} (\numprint{0.98}) & \numprint{802.48} (\numprint{0.99}) \\
			brock200\_2 & \textbf{\numprint{27.00}} & \numprint{28.39} (\numprint{0.95}) & \numprint{27.92} (\numprint{0.97}) & \numprint{32.82} (\numprint{0.82}) \\
			brock200\_3 & \textbf{\numprint{134.07}} & \numprint{136.48} (\numprint{0.98}) & \numprint{136.43} (\numprint{0.98}) & \numprint{140.52} (\numprint{0.95}) \\
			brock200\_4 & \textbf{\numprint{167.00}} & \numprint{169.84} (\numprint{0.98}) & \numprint{170.28} (\numprint{0.98}) & \numprint{173.32} (\numprint{0.96}) \\
			c-fat200-1 & \textbf{\numprint{0.75}} & \numprint{0.79} (\numprint{0.95}) & \numprint{0.90} (\numprint{0.84}) & \numprint{5.19} (\numprint{0.15}) \\
			gen200\_p0 & \textbf{\numprint{822.73}} & \numprint{852.78} (\numprint{0.96}) & \numprint{846.15} (\numprint{0.97}) & \numprint{833.19} (\numprint{0.99}) \\
			hamming6-4 & \textbf{\numprint{0.14}} & \numprint{0.14} (\numprint{0.96}) & \numprint{0.15} (\numprint{0.90}) & \numprint{0.48} (\numprint{0.29}) \\
			hamming8-4 & \textbf{\numprint{129.50}} & \numprint{135.30} (\numprint{0.96}) & \numprint{133.14} (\numprint{0.97}) & \numprint{137.20} (\numprint{0.94}) \\
			johnson16-2-4 & \textbf{\numprint{236.42}} & \numprint{242.22} (\numprint{0.98}) & \numprint{240.87} (\numprint{0.98}) & \numprint{239.57} (\numprint{0.99}) \\
			johnson8-4-4 & \textbf{\numprint{0.37}} & \numprint{0.39} (\numprint{0.94}) & \numprint{0.38} (\numprint{0.97}) & \numprint{1.38} (\numprint{0.27}) \\
			keller4 & \textbf{\numprint{15.94}} & \numprint{16.75} (\numprint{0.95}) & \numprint{16.40} (\numprint{0.97}) & \numprint{17.94} (\numprint{0.89}) \\
			p\_hat1000-1 & \textbf{\numprint{5498.84}} & \numprint{5517.79} (\numprint{1.00}) & \numprint{5640.30} (\numprint{0.97}) & \numprint{5605.71} (\numprint{0.98}) \\
			p\_hat1000-2 & - & - & - & - \\
			p\_hat1500-1 & - & - & - & - \\
			p\_hat300-1 & \textbf{\numprint{25.36}} & \numprint{25.94} (\numprint{0.98}) & \numprint{26.52} (\numprint{0.96}) & \numprint{47.59} (\numprint{0.53}) \\
			p\_hat300-2 & \textbf{\numprint{34.20}} & \numprint{34.77} (\numprint{0.98}) & \numprint{35.31} (\numprint{0.97}) & \numprint{50.36} (\numprint{0.68}) \\
			p\_hat300-3 & \textbf{\numprint{1124.48}} & \numprint{1138.15} (\numprint{0.99}) & \numprint{1147.00} (\numprint{0.98}) & \numprint{1146.45} (\numprint{0.98}) \\
			p\_hat500-1 & \textbf{\numprint{251.75}} & \numprint{255.29} (\numprint{0.99}) & \numprint{261.24} (\numprint{0.96}) & \numprint{276.61} (\numprint{0.91}) \\
			p\_hat500-2 & \textbf{\numprint{593.16}} & \numprint{613.28} (\numprint{0.97}) & \numprint{608.09} (\numprint{0.98}) & \numprint{611.76} (\numprint{0.97}) \\
			p\_hat500-3 & - & - & - & - \\
			p\_hat700-1 & \textbf{\numprint{1035.72}} & \numprint{1044.02} (\numprint{0.99}) & \numprint{1066.74} (\numprint{0.97}) & \numprint{1065.20} (\numprint{0.97}) \\
			p\_hat700-2 & \textbf{\numprint{5675.42}} & \numprint{5707.71} (\numprint{0.99}) & \numprint{5775.82} (\numprint{0.98}) & \numprint{5745.93} (\numprint{0.99}) \\
			san1000 & \textbf{\numprint{6738.26}} & \numprint{6947.83} (\numprint{0.97}) & \numprint{6864.69} (\numprint{0.98}) & \numprint{6843.53} (\numprint{0.98}) \\
			san200\_0 & \textbf{\numprint{1290.60}} & \numprint{1316.67} (\numprint{0.98}) & \numprint{1319.54} (\numprint{0.98}) & \numprint{1307.43} (\numprint{0.99}) \\
			san400\_0 & \textbf{\numprint{12826.58}} & \numprint{12964.68} (\numprint{0.99}) & \numprint{13112.91} (\numprint{0.98}) & \numprint{12975.06} (\numprint{0.99}) \\
			sanr200\_0 & \textbf{\numprint{2144.20}} & \numprint{2187.94} (\numprint{0.98}) & \numprint{2198.09} (\numprint{0.98}) & \numprint{2170.73} (\numprint{0.99}) \\
			\hline
			$t_{\text{avg}}$ (all) & \textbf{\numprint{6134.82}} & \numprint{6154.39} & \numprint{6164.30} & \numprint{6157.33} \\
			$s_{\text{geom}}$ (all) & \textbf{\numprint{1.00}} & \numprint{0.98} & \numprint{0.97} & \numprint{0.78} \\
			$t_{\text{avg}}$ (solved) & \textbf{\numprint{1540.18}} & \numprint{1562.75} & \numprint{1574.19} & \numprint{1566.15} \\
			$s_{\text{geom}}$ (solved) & \textbf{\numprint{1.00}} & \numprint{0.98} & \numprint{0.97} & \numprint{0.78} \\
			
			\hline
		\end{tabular}
	\end{center}
	\label{table:another_table}
\end{table}

\begin{table}
	\scriptsize
	\setlength{\tabcolsep}{2pt}
	\caption{Detailed results for our decomposition-based strategies on sparse networks.}
	\begin{center}
		\begin{tabular}{|l|r|r|r|r|}\hline
			Graph & max. deg. & \multicolumn{1}{c|}{articulation} & \multicolumn{1}{c|}{edge cuts} & \multicolumn{1}{c|}{nested dis.} \\
			\hline
			Sparse networks & t (s) & t (s) & t (s)  & t (s)  \\
			\hline
			as-skitter & \numprint{11977.45} & \numprint{12088.82} (\numprint{0.99}) & \numprint{11931.14} (\numprint{1.00}) & \textbf{\numprint{11795.50} (\numprint{1.02})} \\
			baidu & \textbf{\numprint{5.26}} & \numprint{5.63} (\numprint{0.94}) & \numprint{5.51} (\numprint{0.95}) & \numprint{11.83} (\numprint{0.44}) \\
			bay-sorted & \numprint{10.82} & \numprint{12.13} (\numprint{0.89}) & \textbf{\numprint{7.90} (\numprint{1.37})} & \numprint{25.25} (\numprint{0.43}) \\
			col-sorted & \numprint{34384.77} & \numprint{32240.53} (\numprint{1.07}) & \numprint{26677.97} (\numprint{1.29}) & - \\
			deezer\_europe & - & - & - & - \\
			fla-sorted & \numprint{157.50} & \numprint{139.68} (\numprint{1.13}) & \numprint{144.67} (\numprint{1.09}) & \textbf{\numprint{137.02} (\numprint{1.15})} \\
			hudong-internallink & \textbf{\numprint{3.38}} & \numprint{4.15} (\numprint{0.81}) & \numprint{4.33} (\numprint{0.78}) & \numprint{3.97} (\numprint{0.85}) \\
			in-2004 & \textbf{\numprint{37.76}} & \numprint{39.00} (\numprint{0.97}) & \numprint{38.13} (\numprint{0.99}) & \numprint{47.73} (\numprint{0.79}) \\
			libimseti & \numprint{8579.32} & \numprint{8428.66} (\numprint{1.02}) & \textbf{\numprint{8427.01} (\numprint{1.02})} & \numprint{8510.58} (\numprint{1.01}) \\
			musae\_DE & - & - & - & - \\
			musae\_FR & \textbf{\numprint{211.72}} & \numprint{219.32} (\numprint{0.97}) & \numprint{217.71} (\numprint{0.97}) & \numprint{212.38} (\numprint{1.00}) \\
			ny-sorted & - & - & - & - \\
			petster-friendships-dog & \numprint{38.66} & \numprint{45.73} (\numprint{0.85}) & \numprint{40.28} (\numprint{0.96}) & \textbf{\numprint{36.61} (\numprint{1.06})} \\
			soc-LiveJournal1 & \textbf{\numprint{31.54}} & \numprint{44.11} (\numprint{0.72}) & \numprint{37.10} (\numprint{0.85}) & \numprint{40.66} (\numprint{0.78}) \\
			web-BerkStan & \textbf{\numprint{793.05}} & \numprint{2023.01} (\numprint{0.39}) & \numprint{802.19} (\numprint{0.99}) & \numprint{840.56} (\numprint{0.94}) \\
			web-Google & \numprint{2.91} & \numprint{2.85} (\numprint{1.02}) & \textbf{\numprint{2.68} (\numprint{1.09})} & \numprint{3.66} (\numprint{0.80}) \\
			web-NotreDame & \numprint{122.57} & \textbf{\numprint{94.59} (\numprint{1.30})} & \numprint{122.09} (\numprint{1.00}) & \numprint{159.76} (\numprint{0.77}) \\
			web-Stanford & - & \numprint{47.94} ($\infty$) & \numprint{163.79} ($\infty$) & \numprint{112.18} ($\infty$) \\
			\hline
			$t_{\text{avg}}$ (all) & \numprint{11130.93} & \numprint{9079.79} & \textbf{\numprint{8701.25}} & \numprint{9218.76} \\
			$s_{\text{geom}}$ (all) & \numprint{1.00} & \numprint{1.41} & \textbf{\numprint{1.45}} & \numprint{1.25} \\
			$t_{\text{avg}}$ (solved) & \numprint{1690.15} & \numprint{1780.59} & \textbf{\numprint{1675.44}} & \numprint{1678.88} \\
			$s_{\text{geom}}$ (solved) & \textbf{\numprint{1.00}} & \numprint{0.89} & \numprint{1.00} & \numprint{0.82} \\
			
			\hline
		\end{tabular}
	\end{center}
	\label{table:another_table}
\end{table}

\begin{table}
	\scriptsize
	\setlength{\tabcolsep}{2pt}
	\caption{Aggregated results for our reduction-based strategies.}
	\begin{center}
		\begin{tabular}{|l|r|r|r|r|r|}\hline
			& max. deg. & \multicolumn{1}{c|}{Twin} & \multicolumn{1}{c|}{Funnel} & \multicolumn{1}{c|}{Unconfined} & \multicolumn{1}{c|}{Combined}  \\
			\hline
			$t_{\text{avg}}$ (all) & \textbf{\numprint{7251.19}} & \numprint{7257.51} & \numprint{7256.41} & \numprint{7286.47} & \numprint{7290.51} \\
			$s_{\text{geom}}$ (all) & \numprint{1.00} & \textbf{\numprint{1.01}} & \numprint{0.97} & \numprint{0.89} & \numprint{0.92} \\
			$t_{\text{avg}}$ (solved) & \numprint{1237.43} & \numprint{1225.14} & \textbf{\numprint{1223.81}} & \numprint{1260.17} & \numprint{1265.06} \\
			$s_{\text{geom}}$ (solved) & \numprint{1.00} & \textbf{\numprint{1.01}} & \numprint{0.97} & \numprint{0.89} & \numprint{0.92} \\
			
			\hline
		\end{tabular}
	\end{center}
	\label{table:another_table}
\end{table}

\begin{table}
	\scriptsize
	\setlength{\tabcolsep}{2pt}
	\caption{Detailed results for our reduction-based strategies on the PACE instances.}
	\begin{center}
		\begin{tabular}{|l|r|r|r|r|r|}\hline
			Graph & \multicolumn{1}{c|}{max. deg.} & \multicolumn{1}{c|}{Twin} & \multicolumn{1}{c|}{Funnel} & \multicolumn{1}{c|}{Unconfined} & \multicolumn{1}{c|}{Combined}  \\
			\hline
			PACE & t (s) & t (s) & t (s) & t (s) & t (s) \\
			\hline
			05 & \textbf{\numprint{11.46}} & \numprint{11.48} (\numprint{1.00}) & \numprint{12.06} (\numprint{0.95}) & \numprint{12.26} (\numprint{0.93}) & \numprint{13.37} (\numprint{0.86}) \\
			06 & \numprint{5.01} & \numprint{5.01} (\numprint{1.00}) & \textbf{\numprint{4.56} (\numprint{1.10})} & \numprint{5.61} (\numprint{0.89}) & \numprint{5.11} (\numprint{0.98}) \\
			10 & \numprint{13.02} & \textbf{\numprint{13.01} (\numprint{1.00})} & \numprint{13.46} (\numprint{0.97}) & \numprint{13.94} (\numprint{0.93}) & \numprint{13.22} (\numprint{0.99}) \\
			16 & - & - & - & - & - \\
			19 & \numprint{18.17} & \numprint{18.11} (\numprint{1.00}) & \textbf{\numprint{17.34} (\numprint{1.05})} & \numprint{19.13} (\numprint{0.95}) & \numprint{18.00} (\numprint{1.01}) \\
			31 & \numprint{428.74} & \numprint{428.74} (\numprint{1.00}) & \numprint{351.96} (\numprint{1.22}) & \numprint{434.53} (\numprint{0.99}) & \textbf{\numprint{346.98} (\numprint{1.24})} \\
			32 & - & - & - & - & - \\
			33 & \textbf{\numprint{5.79}} & \numprint{5.84} (\numprint{0.99}) & \numprint{7.00} (\numprint{0.83}) & \numprint{6.80} (\numprint{0.85}) & \numprint{7.75} (\numprint{0.75}) \\
			34 & - & - & - & - & - \\
			35 & \textbf{\numprint{44.74}} & \numprint{44.81} (\numprint{1.00}) & \numprint{44.82} (\numprint{1.00}) & \numprint{47.11} (\numprint{0.95}) & \numprint{47.68} (\numprint{0.94}) \\
			36 & \textbf{\numprint{10.67}} & \numprint{10.82} (\numprint{0.99}) & \numprint{67.92} (\numprint{0.16}) & \numprint{975.40} (\numprint{0.01}) & \numprint{74.73} (\numprint{0.14}) \\
			37 & \textbf{\numprint{60.62}} & \numprint{60.73} (\numprint{1.00}) & \numprint{63.42} (\numprint{0.96}) & \numprint{63.38} (\numprint{0.96}) & \numprint{65.95} (\numprint{0.92}) \\
			38 & \numprint{70.79} & \numprint{70.76} (\numprint{1.00}) & \textbf{\numprint{65.55} (\numprint{1.08})} & \numprint{66.96} (\numprint{1.06}) & \numprint{67.73} (\numprint{1.05}) \\
			39 & \numprint{559.85} & \numprint{560.56} (\numprint{1.00}) & \textbf{\numprint{201.00} (\numprint{2.79})} & \numprint{780.71} (\numprint{0.72}) & \numprint{564.12} (\numprint{0.99}) \\
			40 & \numprint{27278.42} & \numprint{27235.60} (\numprint{1.00}) & \textbf{\numprint{26066.83} (\numprint{1.05})} & \numprint{27500.50} (\numprint{0.99}) & \numprint{27218.27} (\numprint{1.00}) \\
			41 & \numprint{282.83} & \numprint{282.81} (\numprint{1.00}) & \textbf{\numprint{254.90} (\numprint{1.11})} & \numprint{290.35} (\numprint{0.97}) & \numprint{263.18} (\numprint{1.07}) \\
			42 & \textbf{\numprint{215.43}} & \numprint{215.80} (\numprint{1.00}) & \numprint{215.73} (\numprint{1.00}) & \numprint{222.44} (\numprint{0.97}) & \numprint{220.72} (\numprint{0.98}) \\
			43 & \numprint{1015.97} & \numprint{1015.94} (\numprint{1.00}) & \textbf{\numprint{959.03} (\numprint{1.06})} & \numprint{1032.41} (\numprint{0.98}) & \numprint{988.41} (\numprint{1.03}) \\
			44 & \numprint{545.48} & \numprint{545.20} (\numprint{1.00}) & \textbf{\numprint{504.69} (\numprint{1.08})} & \numprint{567.84} (\numprint{0.96}) & \numprint{526.25} (\numprint{1.04}) \\
			45 & \textbf{\numprint{147.99}} & \numprint{148.01} (\numprint{1.00}) & \numprint{155.06} (\numprint{0.95}) & \numprint{155.50} (\numprint{0.95}) & \numprint{162.30} (\numprint{0.91}) \\
			46 & \numprint{634.70} & \numprint{632.60} (\numprint{1.00}) & \textbf{\numprint{560.56} (\numprint{1.13})} & \numprint{645.63} (\numprint{0.98}) & \numprint{571.84} (\numprint{1.11}) \\
			47 & \numprint{342.88} & \numprint{341.82} (\numprint{1.00}) & \textbf{\numprint{322.93} (\numprint{1.06})} & \numprint{354.86} (\numprint{0.97}) & \numprint{334.53} (\numprint{1.02}) \\
			48 & \numprint{147.66} & \numprint{147.31} (\numprint{1.00}) & \textbf{\numprint{138.04} (\numprint{1.07})} & \numprint{153.72} (\numprint{0.96}) & \numprint{141.82} (\numprint{1.04}) \\
			49 & \numprint{104.01} & \numprint{103.71} (\numprint{1.00}) & \textbf{\numprint{99.41} (\numprint{1.05})} & \numprint{113.83} (\numprint{0.91}) & \numprint{105.30} (\numprint{0.99}) \\
			50 & \numprint{285.75} & \numprint{285.05} (\numprint{1.00}) & \textbf{\numprint{279.88} (\numprint{1.02})} & \numprint{297.58} (\numprint{0.96}) & \numprint{290.71} (\numprint{0.98}) \\
			51 & \numprint{330.87} & \numprint{330.51} (\numprint{1.00}) & \textbf{\numprint{312.10} (\numprint{1.06})} & \numprint{342.94} (\numprint{0.96}) & \numprint{324.68} (\numprint{1.02}) \\
			52 & \numprint{129.84} & \numprint{129.44} (\numprint{1.00}) & \textbf{\numprint{125.31} (\numprint{1.04})} & \numprint{137.82} (\numprint{0.94}) & \numprint{134.13} (\numprint{0.97}) \\
			53 & \numprint{351.55} & \numprint{351.14} (\numprint{1.00}) & \textbf{\numprint{334.23} (\numprint{1.05})} & \numprint{363.54} (\numprint{0.97}) & \numprint{345.66} (\numprint{1.02}) \\
			54 & \numprint{189.21} & \numprint{188.61} (\numprint{1.00}) & \textbf{\numprint{179.16} (\numprint{1.06})} & \numprint{198.46} (\numprint{0.95}) & \numprint{187.24} (\numprint{1.01}) \\
			55 & \numprint{40.08} & \numprint{39.97} (\numprint{1.00}) & \textbf{\numprint{39.54} (\numprint{1.01})} & \numprint{42.06} (\numprint{0.95}) & \numprint{41.52} (\numprint{0.97}) \\
			56 & \numprint{563.85} & \numprint{562.77} (\numprint{1.00}) & \textbf{\numprint{538.68} (\numprint{1.05})} & \numprint{582.59} (\numprint{0.97}) & \numprint{554.00} (\numprint{1.02}) \\
			57 & \numprint{391.92} & \numprint{391.16} (\numprint{1.00}) & \textbf{\numprint{353.28} (\numprint{1.11})} & \numprint{398.09} (\numprint{0.98}) & \numprint{359.49} (\numprint{1.09}) \\
			58 & \numprint{280.57} & \numprint{279.59} (\numprint{1.00}) & \textbf{\numprint{277.96} (\numprint{1.01})} & \numprint{290.47} (\numprint{0.97}) & \numprint{295.63} (\numprint{0.95}) \\
			59 & \numprint{78.11} & \numprint{77.81} (\numprint{1.00}) & \textbf{\numprint{77.53} (\numprint{1.01})} & \numprint{81.95} (\numprint{0.95}) & \numprint{79.48} (\numprint{0.98}) \\
			60 & \numprint{466.65} & \numprint{464.83} (\numprint{1.00}) & \textbf{\numprint{439.16} (\numprint{1.06})} & \numprint{482.55} (\numprint{0.97}) & \numprint{457.52} (\numprint{1.02}) \\
			61 & \numprint{128.58} & \numprint{128.63} (\numprint{1.00}) & \textbf{\numprint{124.98} (\numprint{1.03})} & \numprint{134.11} (\numprint{0.96}) & \numprint{127.63} (\numprint{1.01}) \\
			62 & \numprint{386.85} & \numprint{385.80} (\numprint{1.00}) & \textbf{\numprint{359.66} (\numprint{1.08})} & \numprint{403.67} (\numprint{0.96}) & \numprint{376.14} (\numprint{1.03}) \\
			63 & \numprint{400.81} & \numprint{399.60} (\numprint{1.00}) & \textbf{\numprint{371.92} (\numprint{1.08})} & \numprint{420.44} (\numprint{0.95}) & \numprint{386.58} (\numprint{1.04}) \\
			64 & \numprint{173.83} & \numprint{173.54} (\numprint{1.00}) & \textbf{\numprint{164.74} (\numprint{1.06})} & \numprint{177.58} (\numprint{0.98}) & \numprint{171.06} (\numprint{1.02}) \\
			65 & \numprint{212.66} & \numprint{212.62} (\numprint{1.00}) & \textbf{\numprint{200.84} (\numprint{1.06})} & \numprint{224.66} (\numprint{0.95}) & \numprint{209.17} (\numprint{1.02}) \\
			66 & \numprint{47.47} & \numprint{47.40} (\numprint{1.00}) & \textbf{\numprint{45.69} (\numprint{1.04})} & \numprint{49.27} (\numprint{0.96}) & \numprint{47.36} (\numprint{1.00}) \\
			67 & \numprint{708.92} & \numprint{706.97} (\numprint{1.00}) & \textbf{\numprint{681.84} (\numprint{1.04})} & \numprint{727.18} (\numprint{0.97}) & \numprint{715.19} (\numprint{0.99}) \\
			68 & \numprint{50.67} & \textbf{\numprint{50.45} (\numprint{1.00})} & \numprint{53.65} (\numprint{0.94}) & \numprint{52.82} (\numprint{0.96}) & \numprint{55.57} (\numprint{0.91}) \\
			69 & \numprint{252.39} & \numprint{251.75} (\numprint{1.00}) & \textbf{\numprint{234.74} (\numprint{1.08})} & \numprint{262.70} (\numprint{0.96}) & \numprint{246.46} (\numprint{1.02}) \\
			70 & \numprint{68.24} & \numprint{67.83} (\numprint{1.01}) & \textbf{\numprint{61.13} (\numprint{1.12})} & \numprint{72.52} (\numprint{0.94}) & \numprint{64.01} (\numprint{1.07}) \\
			71 & \numprint{208.63} & \numprint{208.06} (\numprint{1.00}) & \textbf{\numprint{193.27} (\numprint{1.08})} & \numprint{209.53} (\numprint{1.00}) & \numprint{203.51} (\numprint{1.03}) \\
			72 & \numprint{269.53} & \numprint{269.31} (\numprint{1.00}) & \textbf{\numprint{253.55} (\numprint{1.06})} & \numprint{277.97} (\numprint{0.97}) & \numprint{261.98} (\numprint{1.03}) \\
			73 & \numprint{251.79} & \numprint{251.31} (\numprint{1.00}) & \textbf{\numprint{246.16} (\numprint{1.02})} & \numprint{261.86} (\numprint{0.96}) & \numprint{268.31} (\numprint{0.94}) \\
			74 & \numprint{41.27} & \numprint{41.21} (\numprint{1.00}) & \textbf{\numprint{41.01} (\numprint{1.01})} & \numprint{47.59} (\numprint{0.87}) & \numprint{43.03} (\numprint{0.96}) \\
			75 & - & - & - & - & - \\
			76 & - & - & - & - & - \\
			77 & \numprint{77.46} & \numprint{77.43} (\numprint{1.00}) & \textbf{\numprint{76.80} (\numprint{1.01})} & \numprint{81.62} (\numprint{0.95}) & \numprint{78.81} (\numprint{0.98}) \\
			78 & - & - & - & - & - \\
			79 & - & - & - & - & - \\
			80 & - & - & - & - & - \\
			\hline
			$t_{\text{avg}}$ (all) & \numprint{5933.30} & \numprint{5932.21} & \textbf{\numprint{5894.42}} & \numprint{5965.17} & \numprint{5928.77} \\
			$s_{\text{geom}}$ (all) & \numprint{1.00} & \numprint{1.00} & \textbf{\numprint{1.02}} & \numprint{0.86} & \numprint{0.96} \\
			$t_{\text{avg}}$ (solved) & \numprint{815.57} & \numprint{814.29} & \textbf{\numprint{770.07}} & \numprint{852.86} & \numprint{810.26} \\
			$s_{\text{geom}}$ (solved) & \numprint{1.00} & \numprint{1.00} & \textbf{\numprint{1.02}} & \numprint{0.86} & \numprint{0.96} \\
			
			
			\hline
		\end{tabular}
	\end{center}
	\label{table:another_table}
\end{table}

\begin{table}
	\scriptsize
	\setlength{\tabcolsep}{2pt}
	\caption{Detailed results for our reduction-based strategies on the DIMACS instances.}
	\begin{center}
		\begin{tabular}{|l|r|r|r|r|r|}\hline
			Graph & \multicolumn{1}{c|}{max. deg.} & \multicolumn{1}{c|}{Twin} & \multicolumn{1}{c|}{Funnel} & \multicolumn{1}{c|}{Unconfined} & \multicolumn{1}{c|}{Combined}  \\
			\hline
			DIMACS & t (s) & t (s) & t (s) & t (s) & t (s) \\
			\hline
			C125 & \numprint{5.49} & \numprint{5.34} (\numprint{1.03}) & \textbf{\numprint{5.24} (\numprint{1.05})} & \numprint{5.57} (\numprint{0.99}) & \numprint{5.60} (\numprint{0.98}) \\
			C250 & - & - & - & - & - \\
			MANN\_a27 & \numprint{3.05} & \textbf{\numprint{2.98} (\numprint{1.03})} & \numprint{4.08} (\numprint{0.75}) & \numprint{3.30} (\numprint{0.93}) & \numprint{4.29} (\numprint{0.71}) \\
			MANN\_a45 & \numprint{473.07} & \textbf{\numprint{461.60} (\numprint{1.02})} & \numprint{602.88} (\numprint{0.78}) & \numprint{500.84} (\numprint{0.94}) & \numprint{620.94} (\numprint{0.76}) \\
			brock200\_1 & \numprint{790.54} & \textbf{\numprint{771.78} (\numprint{1.02})} & \numprint{803.96} (\numprint{0.98}) & \numprint{786.84} (\numprint{1.00}) & \numprint{832.33} (\numprint{0.95}) \\
			brock200\_2 & \numprint{27.00} & \textbf{\numprint{26.97} (\numprint{1.00})} & \numprint{28.24} (\numprint{0.96}) & \numprint{27.39} (\numprint{0.99}) & \numprint{28.94} (\numprint{0.93}) \\
			brock200\_3 & \numprint{134.07} & \textbf{\numprint{130.76} (\numprint{1.03})} & \numprint{137.64} (\numprint{0.97}) & \numprint{133.19} (\numprint{1.01}) & \numprint{142.11} (\numprint{0.94}) \\
			brock200\_4 & \numprint{167.00} & \textbf{\numprint{162.98} (\numprint{1.02})} & \numprint{173.06} (\numprint{0.97}) & \numprint{165.49} (\numprint{1.01}) & \numprint{177.95} (\numprint{0.94}) \\
			c-fat200-1 & \numprint{0.75} & \textbf{\numprint{0.74} (\numprint{1.03})} & \numprint{0.79} (\numprint{0.95}) & \numprint{0.74} (\numprint{1.01}) & \numprint{0.81} (\numprint{0.93}) \\
			gen200\_p0 & \numprint{822.73} & \numprint{807.96} (\numprint{1.02}) & \textbf{\numprint{779.23} (\numprint{1.06})} & \numprint{868.49} (\numprint{0.95}) & \numprint{824.22} (\numprint{1.00}) \\
			hamming6-4 & \numprint{0.14} & \textbf{\numprint{0.14} (\numprint{1.02})} & \numprint{0.15} (\numprint{0.95}) & \numprint{0.14} (\numprint{1.02}) & \numprint{0.15} (\numprint{0.94}) \\
			hamming8-4 & \numprint{129.50} & \textbf{\numprint{129.42} (\numprint{1.00})} & \numprint{135.48} (\numprint{0.96}) & \numprint{130.23} (\numprint{0.99}) & \numprint{135.31} (\numprint{0.96}) \\
			johnson16-2-4 & \numprint{236.42} & \textbf{\numprint{230.70} (\numprint{1.02})} & \numprint{243.17} (\numprint{0.97}) & \numprint{239.50} (\numprint{0.99}) & \numprint{257.80} (\numprint{0.92}) \\
			johnson8-4-4 & \numprint{0.37} & \textbf{\numprint{0.37} (\numprint{1.01})} & \numprint{0.40} (\numprint{0.92}) & \numprint{0.38} (\numprint{0.98}) & \numprint{0.42} (\numprint{0.89}) \\
			keller4 & \numprint{15.94} & \textbf{\numprint{15.73} (\numprint{1.01})} & \numprint{16.76} (\numprint{0.95}) & \numprint{16.19} (\numprint{0.98}) & \numprint{17.27} (\numprint{0.92}) \\
			p\_hat1000-1 & \numprint{5498.84} & \textbf{\numprint{5362.44} (\numprint{1.03})} & \numprint{5560.49} (\numprint{0.99}) & \numprint{5437.00} (\numprint{1.01}) & \numprint{5697.92} (\numprint{0.97}) \\
			p\_hat1000-2 & - & - & - & - & - \\
			p\_hat1500-1 & - & - & - & - & - \\
			p\_hat300-1 & \numprint{25.36} & \textbf{\numprint{24.75} (\numprint{1.02})} & \numprint{26.25} (\numprint{0.97}) & \numprint{25.09} (\numprint{1.01}) & \numprint{26.92} (\numprint{0.94}) \\
			p\_hat300-2 & \numprint{34.20} & \textbf{\numprint{33.35} (\numprint{1.03})} & \numprint{35.24} (\numprint{0.97}) & \numprint{33.82} (\numprint{1.01}) & \numprint{36.20} (\numprint{0.94}) \\
			p\_hat300-3 & \numprint{1124.48} & \textbf{\numprint{1097.77} (\numprint{1.02})} & \numprint{1133.11} (\numprint{0.99}) & \numprint{1102.21} (\numprint{1.02}) & \numprint{1160.75} (\numprint{0.97}) \\
			p\_hat500-1 & \numprint{251.75} & \textbf{\numprint{245.70} (\numprint{1.02})} & \numprint{257.28} (\numprint{0.98}) & \numprint{249.54} (\numprint{1.01}) & \numprint{264.20} (\numprint{0.95}) \\
			p\_hat500-2 & \numprint{593.16} & \textbf{\numprint{585.37} (\numprint{1.01})} & \numprint{615.66} (\numprint{0.96}) & \numprint{592.60} (\numprint{1.00}) & \numprint{629.68} (\numprint{0.94}) \\
			p\_hat500-3 & - & - & - & - & - \\
			p\_hat700-1 & \numprint{1035.72} & \textbf{\numprint{1010.70} (\numprint{1.02})} & \numprint{1052.85} (\numprint{0.98}) & \numprint{1024.93} (\numprint{1.01}) & \numprint{1079.55} (\numprint{0.96}) \\
			p\_hat700-2 & \numprint{5675.42} & \textbf{\numprint{5537.37} (\numprint{1.02})} & \numprint{5736.46} (\numprint{0.99}) & \numprint{5590.47} (\numprint{1.02}) & \numprint{5869.41} (\numprint{0.97}) \\
			san1000 & \numprint{6738.26} & \textbf{\numprint{6653.61} (\numprint{1.01})} & \numprint{6918.81} (\numprint{0.97}) & \numprint{6727.94} (\numprint{1.00}) & \numprint{6997.43} (\numprint{0.96}) \\
			san200\_0 & \numprint{1290.60} & \numprint{1258.57} (\numprint{1.03}) & \numprint{1226.54} (\numprint{1.05}) & \textbf{\numprint{783.37} (\numprint{1.65})} & \numprint{1025.87} (\numprint{1.26}) \\
			san400\_0 & \numprint{12826.58} & \textbf{\numprint{12540.22} (\numprint{1.02})} & \numprint{13073.50} (\numprint{0.98}) & \numprint{13240.40} (\numprint{0.97}) & \numprint{13528.34} (\numprint{0.95}) \\
			sanr200\_0 & \numprint{2144.20} & \numprint{2092.56} (\numprint{1.02}) & \textbf{\numprint{2027.34} (\numprint{1.06})} & \numprint{2104.37} (\numprint{1.02}) & \numprint{2106.70} (\numprint{1.02}) \\
			\hline
			$t_{\text{avg}}$ (all) & \numprint{6134.82} & \textbf{\numprint{6106.33}} & \numprint{6153.15} & \numprint{6126.33} & \numprint{6182.37} \\
			$s_{\text{geom}}$ (all) & \numprint{1.00} & \textbf{\numprint{1.02}} & \numprint{0.96} & \numprint{1.01} & \numprint{0.94} \\
			$t_{\text{avg}}$ (solved) & \numprint{1540.18} & \textbf{\numprint{1507.30}} & \numprint{1561.33} & \numprint{1530.39} & \numprint{1595.04} \\
			$s_{\text{geom}}$ (solved) & \numprint{1.00} & \textbf{\numprint{1.02}} & \numprint{0.96} & \numprint{1.01} & \numprint{0.94} \\
			
			\hline
		\end{tabular}
	\end{center}
	\label{table:another_tablee}
\end{table}

\begin{table}
	\scriptsize
	\setlength{\tabcolsep}{2pt}
	\caption{Detailed results for our reduction-based strategies on sparse networks.}
	\begin{center}
		\begin{tabular}{|l|r|r|r|r|r|}\hline
			Graph & \multicolumn{1}{c|}{max. deg.} & \multicolumn{1}{c|}{Twin} & \multicolumn{1}{c|}{Funnel} & \multicolumn{1}{c|}{Unconfined} & \multicolumn{1}{c|}{Combined}  \\
			\hline
			Sparse networks & t (s) & t (s) & t (s) & t (s)  & t (s)  \\
			\hline
			as-skitter & \numprint{11977.45} & \numprint{11936.70} (\numprint{1.00}) & \textbf{\numprint{11128.03} (\numprint{1.08})} & \numprint{11795.90} (\numprint{1.02}) & \numprint{11403.20} (\numprint{1.05}) \\
			baidu & \textbf{\numprint{5.26}} & \numprint{5.34} (\numprint{0.98}) & \numprint{5.68} (\numprint{0.93}) & \numprint{5.68} (\numprint{0.93}) & \numprint{5.84} (\numprint{0.90}) \\
			bay-sorted & \textbf{\numprint{10.82}} & \numprint{10.95} (\numprint{0.99}) & \numprint{61.14} (\numprint{0.18}) & \numprint{33.21} (\numprint{0.33}) & \numprint{58.21} (\numprint{0.19}) \\
			col-sorted & \numprint{34384.77} & - & - & - & - \\
			deezer\_europe & - & - & - & - & - \\
			fla-sorted & \textbf{\numprint{157.50}} & \numprint{159.10} (\numprint{0.99}) & \numprint{303.30} (\numprint{0.52}) & \numprint{511.19} (\numprint{0.31}) & \numprint{292.22} (\numprint{0.54}) \\
			hudong-internallink & \textbf{\numprint{3.38}} & \numprint{3.74} (\numprint{0.90}) & \numprint{3.48} (\numprint{0.97}) & \numprint{3.58} (\numprint{0.94}) & \numprint{3.90} (\numprint{0.87}) \\
			in-2004 & \numprint{37.76} & \textbf{\numprint{37.66} (\numprint{1.00})} & \numprint{38.01} (\numprint{0.99}) & \numprint{104.12} (\numprint{0.36}) & \numprint{104.42} (\numprint{0.36}) \\
			libimseti & \numprint{8579.32} & \textbf{\numprint{8537.78} (\numprint{1.00})} & \numprint{9715.24} (\numprint{0.88}) & \numprint{8541.15} (\numprint{1.00}) & \numprint{9772.28} (\numprint{0.88}) \\
			musae\_DE & - & - & - & - & - \\
			musae\_FR & \textbf{\numprint{211.72}} & \numprint{215.57} (\numprint{0.98}) & \numprint{235.97} (\numprint{0.90}) & \numprint{247.27} (\numprint{0.86}) & \numprint{268.42} (\numprint{0.79}) \\
			ny-sorted & - & - & - & - & - \\
			petster-friendships-dog & \numprint{38.66} & \textbf{\numprint{34.69} (\numprint{1.11})} & \numprint{39.60} (\numprint{0.98}) & \numprint{36.02} (\numprint{1.07}) & \numprint{37.50} (\numprint{1.03}) \\
			soc-LiveJournal1 & \numprint{31.54} & \textbf{\numprint{28.46} (\numprint{1.11})} & \numprint{36.50} (\numprint{0.86}) & \numprint{39.52} (\numprint{0.80}) & \numprint{44.69} (\numprint{0.71}) \\
			web-BerkStan & \numprint{793.05} & \numprint{785.55} (\numprint{1.01}) & \textbf{\numprint{741.49} (\numprint{1.07})} & \numprint{939.90} (\numprint{0.84}) & \numprint{963.78} (\numprint{0.82}) \\
			web-Google & \numprint{2.91} & \textbf{\numprint{2.87} (\numprint{1.01})} & \numprint{2.92} (\numprint{1.00}) & \numprint{3.07} (\numprint{0.95}) & \numprint{3.03} (\numprint{0.96}) \\
			web-NotreDame & \textbf{\numprint{122.57}} & \numprint{126.49} (\numprint{0.97}) & \numprint{152.53} (\numprint{0.80}) & \numprint{127.48} (\numprint{0.96}) & \numprint{155.77} (\numprint{0.79}) \\
			web-Stanford & - & - & - & - & - \\\hline
			$t_{\text{avg}}$ (all) & \textbf{\numprint{11130.93}} & \numprint{11215.83} & \numprint{11247.99} & \numprint{11243.78} & \numprint{11284.07} \\
			$s_{\text{geom}}$ (all) & \numprint{1.00} & \textbf{\numprint{1.00}} & \numprint{0.79} & \numprint{0.73} & \numprint{0.70} \\
			$t_{\text{avg}}$ (solved) & \numprint{1690.15} & \textbf{\numprint{1683.45}} & \numprint{1727.99} & \numprint{1722.16} & \numprint{1777.94} \\
			$s_{\text{geom}}$ (solved) & \numprint{1.00} & \textbf{\numprint{1.00}} & \numprint{0.79} & \numprint{0.74} & \numprint{0.70} \\
			
			\hline
		\end{tabular}
	\end{center}
	\label{table:another_table}
\end{table}
\end{document}
