
\documentclass[a4paper,UKenglish,cleveref, autoref, thm-restate]{lipics-v2021}
%See lipics-v2021-authors-guidelines.pdf for further information.
%for A4 paper format use option "a4paper", for US-letter use option "letterpaper"
%for british hyphenation rules use option "UKenglish", for american hyphenation rules use option "USenglish"
%for section-numbered lemmas etc., use "numberwithinsect"
%for enabling cleveref support, use "cleveref"
%for enabling autoref support, use "autoref"
%for anonymousing the authors (e.g. for double-blind review), add "anonymous"
%for enabling thm-restate support, use "thm-restate"
%for enabling a two-column layout for the author/affilation part (only applicable for > 6 authors), use "authorcolumns"

\listfiles

%\graphicspath{{./graphics/}}%helpful if your graphic files are in another directory

\bibliographystyle{plainurl}% the mandatory bibstyle

\title{Faster Maximum Independent Sets using Targeted Branching} %TODO Please add

\titlerunning{Faster MIS using Targeted Branching} %TODO optional, please use if title is longer than one line

% \author{John Q. Public}{Dummy University Computing Laboratory, [optional: Address], Country \and My second affiliation, Country \and \url{http://www.myhomepage.edu} }{johnqpublic@dummyuni.org}{https://orcid.org/0000-0002-1825-0097}{(Optional) author-specific funding acknowledgements}%TODO mandatory, please use full name; only 1 author per \author macro; first two parameters are mandatory, other parameters can be empty. Please provide at least the name of the affiliation and the country. The full address is optional

\author{Demian Hespe}{Karlsruhe Institute of Technology, Institute for
  Theoretical Informatics, Germany}{hespe@kit.edu}{}{}

\author{Sebastian Lamm}{Karlsruhe Institute of Technology, Institute for
  Theoretical Informatics, Germany}{lamm@kit.edu}{}{\textcolor{red} {TODO: Add funding}}

\author{Christian Schorr}{Karlsruhe Institute of Technology, Institute for
  Theoretical Informatics, Germany}{uztfp@student.kit.edu}{}{}

\authorrunning{D. Hespe and S. Lamm and C. Schorr} %TODO mandatory. First: Use abbreviated first/middle names. Second (only in severe cases): Use first author plus 'et al.'

\Copyright{Demian Hespe and Sebastian Lamm and Christian Schorr} %TODO mandatory, please use full first names. LIPIcs license is "CC-BY";  http://creativecommons.org/licenses/by/3.0/
\begin{CCSXML}
  <ccs2012>
  <concept>
  <concept_id>10002950.10003624.10003633.10010917</concept_id>
  <concept_desc>Mathematics of computing~Graph algorithms</concept_desc>
  <concept_significance>500</concept_significance>
  </concept>
  <concept>
  <concept_id>10002950.10003624.10003625.10003630</concept_id>
  <concept_desc>Mathematics of computing~Combinatorial optimization</concept_desc>
  <concept_significance>500</concept_significance>
  </concept>
  </ccs2012>
\end{CCSXML}

\ccsdesc[500]{Mathematics of computing~Graph algorithms}
\ccsdesc[500]{Mathematics of computing~Combinatorial optimization}
% \ccsdesc[100]{\textcolor{red}{Replace ccsdesc macro with valid oll
% ne}} %TODO mandatory: Please choose ACM 2012 classifications from https://dl.acm.org/ccs/ccs_flat.cfm 

\keywords{Graphs, Combinatorial Optimization, Maximum Independent Set, Branch
  and Reduce, Kernelization}%TODO mandatory; please add comma-separated list of keywords

\category{} %optional, e.g. invited paper

\relatedversion{} %optional, e.g. full version hosted on arXiv, HAL, or other respository/website
%\relatedversiondetails[linktext={opt. text shown instead of the URL}, cite=DBLP:books/mk/GrayR93]{Classification (e.g. Full Version, Extended Version, Previous Version}{URL to related version} %linktext and cite are optional

%\supplement{}%optional, e.g. related research data, source code, ... hosted on a repository like zenodo, figshare, GitHub, ...
%\supplementdetails[linktext={opt. text shown instead of the URL}, cite=DBLP:books/mk/GrayR93, subcategory={Description, Subcategory}, swhid={Software Heritage Identifier}]{General Classification (e.g. Software, Dataset, Model, ...)}{URL to related version} %linktext, cite, and subcategory are optional

%\funding{(Optional) general funding statement \dots}%optional, to capture a funding statement, which applies to all authors. Please enter author specific funding statements as fifth argument of the \author macro.

\acknowledgements{}%optional

%\nolinenumbers %uncomment to disable line numbering

%\hideLIPIcs  %uncomment to remove references to LIPIcs series (logo, DOI, ...), e.g. when preparing a pre-final version to be uploaded to arXiv or another public repository

%Editor-only macros:: begin (do not touch as author)%%%%%%%%%%%%%%%%%%%%%%%%%%%%%%%%%%
\EventEditors{John Q. Open and Joan R. Access}
\EventNoEds{2}
\EventLongTitle{42nd Conference on Very Important Topics (CVIT 2016)}
\EventShortTitle{CVIT 2016}
\EventAcronym{CVIT}
\EventYear{2016}
\EventDate{December 24--27, 2016}
\EventLocation{Little Whinging, United Kingdom}
\EventLogo{}
\SeriesVolume{42}
\ArticleNo{23}
%%%%%%%%%%%%%%%%%%%%%%%%%%%%%%%%%%%%%%%%%%%%%%%%%%%%%%

% Custom
\newcommand{\ie}{i.\,e.,\xspace}
\newcommand{\eg}{e.\,g.,\xspace}
\newcommand{\etal}{et~al.\xspace}
\newcommand{\Wlog}{w.\,l.\,o.\,g.\ }
\newcommand{\wrt}{w.\,r.\,t.\xspace}

\begin{document}

\maketitle

%TODO mandatory: add short abstract of the document
\begin{abstract}
  \textcolor{red}{TODO: Write Abstract}
\end{abstract}

\newpage

\section{Introduction}
% Probably cite Schorr~\cite{schorr2020improved}.
An \emph{independent set} of a graph $G = (V,E)$ is a set of vertices $I \subseteq V$ of $G$ such that no two vertices in this set are adjacent. The problem of finding such an independent set of maximum cardinality, the \emph{maximum independent set problem}, is a fundamental NP-hard problem~\cite{Garey1974}. Its applications cover a wide variety of fields including computer graphics \cite{CG}, network analysis \cite{NW}, route planning \cite{RP} and computational biology \cite{BIO1, BIO2}. In computer graphics for instance, large independent sets can be used to optimize the traversal of mesh edges in a triangle mesh. Further applications stem from its complementary problems minimum vertex cover and maximum clique.

One of the best known techniques for finding maximum independent sets, both in theory~\cite{XiaoNagamochi, ChenXiaKanj} and practice \cite{AkibaIwata}, are \emph{kernelization algorithms}. These algorithms apply a set of reduction rules to decrease the complexity of an instance while maintaining the ability to compute an optimal solution afterwards. A recently successful type of kernelization algorithms are so-called \emph{branch-and-reduce algorithms}~\cite{AkibaIwata,WGYC}, which exhaustively apply a set of reduction rules to compute a \emph{kernel}. If no further rule can be applied, the algorithm branches into (at least) two subproblems of lower complexity, which are then solved recursively. To make them more efficient in practice, these algorithms also make use of problem specific upper and lower bounds to quickly prune the search space.

Due to the practical impact of kernelization, most of the research aimed at improving the performance of branch-and-reduce algorithms so far has been focused on either proposing more practically efficient special cases of already existing rules~\cite{ChangKern,dahlum2016accelerating}, or maintaining dependencies between reduction rules to reduce unnecessary checks~\cite{alsahafy2020computing,hespe2019scalable}. In comparison, efforts to improve other aspects of branch-and-reduce, in particular the branching strategy are still lacking. However, the branching strategy in particular has been shown to have a significant impact on the running time~\cite{AkibaIwata}. Up to now, the most frequently used branching strategy employed in many state-of-the-art solvers selects branching vertices solely based on their degree. Other factors, such as the actual reduction rules used during the algorithm are rarely taken into account.

\subsection{Contribution}
In this paper, we propose and examine several novel strategies for selecting branching vertices. These strategies follow two main approaches that are motivated by existing research: (1) Branching on vertices that decompose the graph into several connected components that can be solved independently. This has been shown to significantly improve the performance of branch-and-reduce in practice, especially when the side of the largest component is kept small~\cite{alsahafy2020computing}. (2) Branching on vertices whose removal leads to reduction rules becoming applicable again. In turn, this leads to a smaller reduced graph and thus improved performance. For each approach we present several concrete strategies that vary in their complexity. Finally, we evaluate their performance by comparing them to the aforementioned default strategy used in the state-of-the-art solver by Akiba and Iwata~\cite{AkibaIwata}. For this purpose we make of a wide spectrum of instances from different graph classes and applications. Our experiments indicate \ldots.

\section{Preliminaries}
Sebastian

Graph, Subgraph, Independent Set, MIS, Vertex Cover, Clique, Kernelization,
degeneracy order

\section{Related Work}
Demian

\subsection{Branch and Reduce Algorithm by Akiba and Iwata~\cite{DBLP:journals/tcs/AkibaI16}}

\section{Decomposition Branching}
Sebastian

\section{Reduction Branching}
Demian

\section{Experimental Evaluation}

\subsection{Experimental Environment}

\subsection{Algorithm Configuration}

\subsection{Instances}

\subsection{Decomposition Branching}

\subsection{Reduction Branching}

\section{Conclusion and Future Work}

%%
%% Bibliography
%%

%% Please use bibtex, 

\bibliography{references}

\appendix

\section{Detailed Instance Information}

\end{document}
